\chapter{Waveform Audio File Format}
The Waveform Audio File Format is the most common form of PCM container.
What that means is that the file is mostly PCM data with a
small amount of header data to tell applications what format the
PCM data is in.
Since RIFF WAVE originated on Intel processors, everything in it
is little-endian.
\section{the RIFF WAVE Stream}
\includegraphics{figures/wav_stream.pdf}
\par
\noindent
`Chunk Size' is the total size of the chunk, minus
8 bytes for the chunk header.
\section{the Classic `fmt' Chunk}
Wave files with 2 channels or less, and 16 bits-per-sample or less,
use a classic `fmt' chunk to indicate its PCM data format.
This chunk is required to appear before the `data' chunk.
\includegraphics{figures/wav_fmt.pdf}
\begin{align}
\text{Average Bytes per Second} &= \frac{\text{Sample Rate}
  \times \text{Channel Count} \times \text{Bits per Sample}}{8} \\
\text{Block Align} &= \frac{\text{Channel Count} \times \text{Bits per Sample}}{8}
\end{align}
\section{the WAVEFORMATEXTENSIBLE `fmt' Chunk}
Wave files with more than 2 channels or more than 16 bits-per-sample
should use a WAVEFORMATEXTENSIBLE `fmt' chunk which contains
additional fields for channel assignment.

\includegraphics{figures/wav_fmtext.pdf}
\noindent
Note that the `Average Bytes per Second' and `Block Align' fields
are calculated the same as a classic fmt chunk.
