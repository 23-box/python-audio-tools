%This work is licensed under the
%Creative Commons Attribution-Share Alike 3.0 United States License.
%To view a copy of this license, visit
%http://creativecommons.org/licenses/by-sa/3.0/us/ or send a letter to
%Creative Commons,
%171 Second Street, Suite 300,
%San Francisco, California, 94105, USA.

\chapter{Free Lossless Audio Codec}
FLAC compresses PCM audio data losslessly using predictors and a
residual.
FLACs contain checksumming to verify their integrity, contain comment tags for
metadata and are streamable.

Except for the contents of the VORBIS\_COMMENT metadata block, everything in FLAC is big-endian.

\section{the FLAC file Stream}
\begin{figure}[h]
\includegraphics{figures/flac_stream.pdf}
\end{figure}
\par
\noindent
``Last'' is 0 when there are no additional metadata blocks and 1 when
it is the final block before the the audio frames.
``Block Length'' is the size of the metadata block data to follow,
not including the header.
\begin{figure}[h]
\begin{tabular}{| r | l |}
\hline
Block Type & Block \\
\hline
\texttt{0} & STREAMINFO \\
\texttt{1} & PADDING \\
\texttt{2} & APPLICATION \\
\texttt{3} & SEEKTABLE \\
\texttt{4} & VORBIS\_COMMENT \\
\texttt{5} & CUESHEET \\
\texttt{6} & PICTURE \\
\texttt{7-126} & reserved \\
\texttt{127} & invalid \\
\hline
\end{tabular}
\end{figure}

\pagebreak

\section{FLAC Metadata Blocks}

\subsection{STREAMINFO}
\begin{figure}[h]
\includegraphics{figures/flac_streaminfo.pdf}
\end{figure}

\subsection{PADDING}

PADDING is simply a block full of NULL (\texttt{0x00}) bytes.
Its purpose is to provide extra metadata space within the FLAC file.
By having a padding block, other metadata blocks can be grown or
shrunk without having to rewrite the entire FLAC file by removing or
adding space to the padding.


\subsection{APPLICATION}
\begin{figure}[h]
\includegraphics{figures/flac_application.pdf}
\end{figure}
\noindent
APPLICATION is a general-purpose metadata block used by a variety of
different programs.
Its contents are defined by the ASCII Application ID value.

\subsection{SEEKTABLE}
\begin{figure}[h]
\includegraphics{figures/flac_seektable.pdf}
\end{figure}

\pagebreak

\subsection{VORBIS\_COMMENT}
\begin{figure}[h]
\includegraphics{figures/flac_vorbiscomment.pdf}
\end{figure}
\par
\noindent
The length fields are all little-endian.
The Vendor String and Comment Strings are all UTF-8 encoded.
Keys are not case-sensitive and may occur multiple times,
indicating multiple values for the same field.
For instance, a track with multiple artists may have
more than one \texttt{ARTIST}.

\begin{multicols}{2}
{\relsize{-2}
\begin{description}
\item[ALBUM] album name
\item[ARTIST] artist name, band name, composer, author, etc.
\item[CATALOGNUMBER*] CD spine number
\item[COMPOSER*] the work's author
\item[CONDUCTOR*] performing ensemble's leader
\item[COPYRIGHT] copyright attribution
\item[DATE] recording date
\item[DESCRIPTION] a short description
\item[DISCNUMBER*] disc number for multi-volume work
\item[ENGINEER*] the recording masterer
\item[ENSEMBLE*] performing group
\item[GENRE] a short music genre label
\item[GUEST ARTIST*] collaborating artist
\item[ISRC] ISRC number for the track
\item[LICENSE] license information
\item[LOCATION] recording location
\item[OPUS*] number of the work
\item[ORGANIZATION] record label
\item[PART*] track's movement title
\item[PERFORMER] performer name, orchestra, actor, etc.
\item[PRODUCER*] person responsible for the project
\item[PRODUCTNUMBER*] UPC, EAN, or JAN code
\item[PUBLISHER*] album's publisher
\item[RELEASE DATE*] date the album was published
\item[REMIXER*] person who created the remix
\item[SOURCE ARTIST*] artist of the work being performed
\item[SOURCE MEDIUM*] CD, radio, cassette, vinyl LP, etc.
\item[SOURCE WORK*] a soundtrack's original work
\item[SPARS*] DDD, ADD, AAD, etc.
\item[SUBTITLE*] for multiple track names in a single file
\item[TITLE] track name
\item[TRACKNUMBER] track number
\item[VERSION] track version
\end{description}
}
\end{multicols}
\par
\noindent
Fields marked with * are proposed extension fields and not part of the official Vorbis comment specification.

\pagebreak

\subsection{CUESHEET}
\begin{figure}[h]
\includegraphics{figures/flac_cuesheet.pdf}
\end{figure}

\subsection{PICTURE}
\begin{figure}[h]
\includegraphics{figures/flac_picture.pdf}
\end{figure}
\begin{tabular}{|r|l|}
\hline
Picture Type & Type \\
\hline
0 & Other \\
1 & 32x32 pixels `file icon' (PNG only) \\
2 & Other file icon \\
3 & Cover (front) \\
4 & Cover (back) \\
5 & Leaflet page \\
6 & Media (e.g. label side of CD) \\
7 & Lead artist / Lead performer / Soloist \\
8 & Artist / Performer \\
9 & Conductor \\
10 & Band / Orchestra \\
11 & Composer \\
12 & Lyricist / Text writer \\
13 & Recording location \\
14 & During recording \\
15 & During performance \\
16 & Movie / Video screen capture \\
17 & A bright coloured fish \\
18 & Illustration \\
19 & Band / Artist logotype \\
20 & Publisher / Studio logotype \\
\hline
\end{tabular}

\section{FLAC Decoding}
A FLAC stream is made up of individual FLAC frames, as follows:
\begin{figure}[h]
\includegraphics{figures/flac_frame.pdf}
\end{figure}

\label{flac_encoded_fields}
\begin{figure}[h]
\begin{tiny}
\begin{tabular}{|c|r|r|r|l|r|c|}
\hline
Value & Block Size & Sample Rate & Channels & Assignment & Bits per Sample & Value \\
\hline
\texttt{0000} & STREAMINFO & STREAMINFO & 1 & front center & STREAMINFO & \texttt{0000} \\
\texttt{0001} & 192 & 88200 & 2 & front left, front right & 8 & \texttt{0001} \\
\texttt{0010} & 576 & 176400 & 3 & f. left, f. right, f. center & 12 & \texttt{0010} \\
\texttt{0011} & 1152 & 192000 & 4 & f. left, f. right, back left, back right & reserved & \texttt{0011} \\
\texttt{0100} & 2304 & 8000 & 5 & f. L, f. R, f. C, b. L, b. R & 16 & \texttt{0100} \\
\texttt{0101} & 4608 & 16000 & 6 & f. L, f. R, f. C, LFE, b. L, b. R & 20 & \texttt{0101} \\
\texttt{0110} & 8 bits (+1) & 22050 & 7 & undefined & 24 & \texttt{0110} \\
\texttt{0111} & 16 bits (+1) & 24000 & 8 & undefined & reserved & \texttt{0111} \\
\texttt{1000} & 256 & 32000 & 2 & 0 left, 1 difference & & \texttt{1000} \\
\texttt{1001} & 512 & 44100 & 2 & 0 difference, 1 right & & \texttt{1001} \\
\texttt{1010} & 1024 & 48000 & 2 & 0 average, 1 difference & & \texttt{1010} \\
\texttt{1011} & 2048 & 96000 & & reserved & & \texttt{1011} \\
\texttt{1100} & 4096 & 8 bits (in kHz) & & reserved & & \texttt{1100} \\
\texttt{1101} & 8192 & 16 bits (in Hz) & & reserved & & \texttt{1101} \\
\texttt{1110} & 16384 & 16 bits (in 10s of Hz) & & reserved & & \texttt{1110} \\
\texttt{1111} & 32768 & invalid & & reserved & & \texttt{1111} \\
\hline
\end{tabular}
\end{tiny}
\end{figure}
\par
\parpic[r]{
\begin{tabular}{|c|l|}
\hline
Value & Subframe Type \\
\hline
\texttt{000000} & \texttt{SUBFRAME\_CONSTANT} \\
\texttt{000001} & \texttt{SUBFRAME\_VERBATIM} \\
\texttt{00001x} & reserved \\
\texttt{0001xx} & reserved \\
\texttt{001xxx} & \texttt{SUBFRAME\_FIXED} \\
& \texttt{xxx} = predictor order \\
\texttt{01xxxx} & reserved \\
\texttt{1xxxxx} & \texttt{SUBFRAME\_LPC} \\
& \texttt{xxxxx} = predictor order - 1 \\
\hline
\end{tabular}
}
\noindent
Sample/Frame Number is a UTF-8 coded value.
If the blocking strategy is 0, it decodes to a 32-bit frame number.
If the blocking strategy is 1, it decodes to a 36-bit sample number.

There is one Subframe per channel.

`Wasted Bits Per Sample' is typically a single bit set to 0, indicating
no wasted bits per sample.  If set to 1, a unary-encoded value
follows which indicates how many bits are wasted per sample.

Padding is added as needed between the final subframe and CRC-16
in order to byte-align frames.

\pagebreak

\subsection{CONSTANT subframe}

This is the simplest possible subframe.
It consists of a single value whose size is equal to the subframe's
`Bits per Sample'.
For instance, a 16-bit subframe would have CONSTANT subframes
16 bits in length.
The value of the subframe is the value of all samples the subframe
contains.
An obvious use of this subframe is to store an entire subframe's worth
of digital silence (samples with a value of 0) very efficiently.

\subsection{VERBATIM subframe}

\begin{figure}[h]
\includegraphics{figures/flac_verbatim.pdf}
\end{figure}
\noindent
This subframe's length is equal to the subframe's `Bits per Sample'
multiplied by the frame's `Block Size'.
Since it does no compression whatsoever and simply stores
audio samples as-is, this subframe is only suitable for especially
noisy portions of a track where no suitable predictor can be found.

\subsection{FIXED subframe}

\begin{figure}[h]
\includegraphics{figures/flac_fixed.pdf}
\end{figure}
\noindent
The number of warm-up samples equals the `Predictor Order'
(which is encoded in the `Subframe Type').
Each warm-up sample is the same size as the subframe's `Bits per Sample'.
These samples are sent out as-is; they are the subframe's
`starting point' upon which further samples build when decompressing
the stream.
Determining the value of the current sample is then a matter of
looking backwards at previously decoded samples (or warm-up samples),
applying a simple formula on their values (which depends on the
Predictor Order) and adding the residual.

\begin{figure}[h]
{\relsize{-1}
\begin{tabular}{|c| >{$}l<{$} |}
\hline
Order & \text{Calculation} \\
\hline
0 & Sample_i = Residual_i  \\
1 & Sample_i = Sample_{i - 1} + Residual_i \\
2 & Sample_i = ( 2 \times Sample_{i - 1} ) - Sample_{i - 2} + Residual_i \\
3 & Sample_i = ( 3 \times Sample_{i - 1} ) - ( 3 \times Sample_{i - 2} ) + Sample_{i - 3} + Residual_i \\
4 & Sample_i = ( 4 \times Sample_{i - 1} ) - ( 6 \times Sample_{i - 2} ) + (4 \times Sample_{i - 3} ) - Sample_{i - 4} + Residual_i \\
\hline
\end{tabular}
}
\end{figure}
\parpic[r]{
{\relsize{-3} %% need to make this small or it fudges the next page's text
\begin{tabular}{|c|r|>{$}r<{$}|}
\hline
Index & Residual & \text{Sample} \\
\hline
0 & & \text{(warm-up)} \thickspace \bf10 \\
1 & 1 & 10 + 1 = \bf11 \\
2 & 2 & 11 + 2 = \bf13 \\
3 & -2 & 13 - 2 = \bf11 \\
4 & 1 & 11 + 1 = \bf12 \\
5 & -1 & 12 - 1 = \bf11 \\
\hline
\end{tabular}
}
}
\par
Let's run through a simple example in which the Predictor Order is 1.
Note that residual does not apply to warm-up samples.
How to extract the encoded residual will be covered in a later section.

\subsection{LPC Subframe}
\begin{figure}[h]
\includegraphics{figures/flac_lpc.pdf}
\end{figure}
\par
\noindent
The number of warm-up samples equals the `LPC Order'
(which is encoded in the `Subframe Type').
The size of each QLP Coefficient is equal to
`QLP Precision' number of bits, plus 1.
`QLP Shift Needed' and the value of each Coefficient are
signed two's-complement integers.
The number of Coefficients equals the `LPC Order'.
\begin{equation}
Sample_i = \left\lfloor \frac{\overset{Order - 1}{\underset{j = 0}{\sum}}
  \text{QLP Coeffcient}_j \times \text{Sample}_{i - j - 1} } {2 ^ \text{QLP Shift Needed}}\right\rfloor + Residual_i
\end{equation}
This simply means we're taking the sum of the calculated values from
0 to Order - 1, bit-shifting that sum down and added the residual
when determining the current sample.
Much like the FIXED subframe, LPC subframes also contain warm-up
samples which serve as our calculation's starting point.

In this example, the LPC Order is 5, the QLP Shift Needed is 9
and the encoded Coefficients are as follows:

\begin{figure}[h]
{\relsize{-2}
\begin{tabular}{>{$}r<{$} r}
\text{QLP Coefficient}_0 & 1241 \\
\text{QLP Coefficient}_1 & -944 \\
\text{QLP Coefficient}_2 & 14 \\
\text{QLP Coefficient}_3 & 342 \\
\text{QLP Coefficient}_4 & -147 \\
\end{tabular}
}
\end{figure}

\begin{figure}[h]
{\relsize{-2}
\begin{tabular}{|c|r|>{$}r<{$}|}
\hline
Index & Residual & \text{Sample} \\
\hline
0 & & \text{(warm-up)} \thickspace \bf1053 \\
1 & & \text{(warm-up)} \thickspace \bf1116 \\
2 & & \text{(warm-up)} \thickspace \bf1257 \\
3 & & \text{(warm-up)} \thickspace \bf1423 \\
4 & & \text{(warm-up)} \thickspace \bf1529 \\
\hline
5 & 11 & (1241 \times 1529) + (-944 \times 1423) + (14 \times 1257) + (342 \times 1116) + (-147 \times 1053) = 798656 \\
& & \lfloor 798656 \div 2 ^ 9 \rfloor = 1559 + 11 = \bf1570 \\
\hline
6 & 79 & (1241 \times 1570) + (-944 \times 1529) + (14 \times 1423) + (342 \times 1257) + (-147 \times 1116) = 790758 \\
& & \lfloor 790758 \div 2 ^ 9 \rfloor = 1544 + 79 = \bf1623 \\
\hline
7 & 24 & (1241 \times 1623) + (-944 \times 1570) + (14 \times 1529) + (342 \times 1423) + (-147 \times 1257) = 855356 \\
& & \lfloor 855356 \div 2 ^ 9 \rfloor = 1670 + 24 = \bf1694 \\
\hline
8 & -81 & (1241 \times 1694) + (-944 \times 1623) + (14 \times 1570) + (342 \times 1529) + (-147 \times 1423) = 905859 \\
& & \lfloor 905859 \div 2 ^ 9 \rfloor = 1769 - 81 = \bf1688 \\
\hline
9 & -72 & (1241 \times 1688) + (-944 \times 1694) + (14 \times 1623) + (342 \times 1570) + (-147 \times 1529) = 830571 \\
& & \lfloor 830571 \div 2 ^ 9 \rfloor = 1622 - 72 = \bf1550 \\
\hline
\end{tabular}
}
\end{figure}
\noindent
In this instance, division should always round down and \textit{not} towards zero.

\subsection{the Residual}
Though the FLAC format allows for different forms of
residual coding, two forms of partitioned Rice are the only ones
currently supported.
The difference between the two is that when `Coding Method' is 0,
the Rice Parameter in each partition is 4 bits.
When the `Coding Method' is 1, that parameter is 5 bits.
\begin{figure}[h]
\includegraphics{figures/flac_residual.pdf}
\end{figure}
\par
\noindent

There are $2 ^ \text{Partition Order}$ number of Partitions.
The number of decoded samples in a Partition depends on the
its position in the subframe.
The first partition in the subframe contains:
\begin{align}
\text{Total Samples} &= \frac{\text{Frame's Block Size}}{2 ^ {\text{Partition Order}}} - \text{Predictor Order} \\
\intertext{Subsequent partitions contain:}
\text{Total Samples} &= \frac{\text{Frame's Block Size}}{2 ^ {\text{Partition Order}}} \\
\intertext{Unless the Partition Order is 0.  In that case:}
\text{Total Samples} &= \text{Frame's Block Size} - \text{Predictor Order}
\end{align}
\noindent
since there is only one partition which takes up the entire block.

If all of the bits in `Rice Parameter' are set,
the partition is unencoded binary using `Escape Code' number of bits per
sample.

\pagebreak

\subsubsection{Rice Encoding}
The residual uses Rice coding to compress lots of mostly small values
in a very small amount of space.
To decode it, one first needs the Rice parameter.
Take a unary-encoded value\footnote{In this instance, unary-encoding is a simple matter of
counting the number of 0 bits before the next 1 bit.
The resulting sum is the value.}
from the bit stream, which are our most
significant bits (MSB).
Then take `parameter' number of additional bits, which are our least
significant bits (LSB).
Combine the two sets into our new value,
making the MSB set as the high bits and the LSB set as the low bits.
Bit 0 of this new value is the sign bit.
If it is 0, the actual value is equal to the rest of the bits.
If it is 1, the actual value is equal to the rest of the bits,
multiplied by -1 and minus 1.

This is less complicated than it sounds, so let's run through an example
in which the Rice parameter is 1:
\begin{figure}[h]
\includegraphics{figures/flac_rice1.pdf}
\end{figure}

Now, let's run through another example in which the Rice parameter is 4:
\begin{figure}[h]
\includegraphics{figures/flac_rice2.pdf}
\end{figure}

\subsection{Channels}
Since most audio has more than one channel, it is important to
understand how FLAC handles putting it back together.
When channels are stored independently, one simply interleaves
them together in the proper order.
Let's take an example of 2 channel, 16-bit audio stored this way:

\begin{figure}[h]
\includegraphics{figures/flac_channels.pdf}
\end{figure}

This is the simplest case.
However, in the case of difference channels, one subframe will
contain actual channel data and the other channel will contain
signed difference data which is applied to that actual data
in order to reconstruct both channels.
It's very important to remember that the difference channel
has 1 additional bit per sample which will be consumed during
reconstruction.
Why 1 additional bit?
Let's take an example where the left sample's value is -30000
and the right sample's value is +30000.
Storing this pair as left + difference means the left
sample remains -30000 and the difference is -60000
($-30000 - -60000 = +30000$).
-60000 won't fit into a 16-bit signed integer.
Adding that 1 additional bit doubles our range of values
and that's just enough to cover any possible difference between
two samples.

\begin{figure}[h]
{\relsize{-2}
\begin{tabular}{|c||l|l||>{$}l<{$}|>{$}l<{$}|}
\hline
Assignment & Channel 0 & Channel 1 & \text{Left Channel} & \text{Right Channel} \\
\hline
\texttt{1000} & left & difference & \text{left} & \text{left} - \text{difference} \\
\texttt{1001} & difference & right & \text{right} + \text{difference} & \text{right} \\
\texttt{1010} & mid & side & (((\text{mid} \ll 1) | (\text{side} \& 1)) + \text{side}) \gg 1 & (((\text{mid} \ll 1) | (\text{side} \& 1)) - \text{side}) \gg 1 \\
\hline
\end{tabular}
}
\end{figure}

The mid channel case is another unusual exception.
We're prepending the mid channel with bit 0 from the
side channel, performing the addition/subtraction and then
discarding that bit before assigning the results to the left and
right channels.

\subsection{Wasted Bits per Sample}
Though rare in practice, FLAC subframes support `wasted bits per sample'.
Put simply, these wasted bits are removed during subframe calculation
and restored to the subframe's least significant bits as zero value bits
when it is returned.
For instance, a subframe with 1 wasted bit per sample in a 16-bit FLAC
stream is treated as having only 15 bits per sample when reading
warm-up samples and then all through the rest of the subframe
calculation.
That wasted zero bit is then prepended to each sample prior to returning
the subframe.

\section{FLAC Encoding}
For the purposes of discussing FLAC encoding,
we'll assume one has a stream of input PCM values along with the
stream's sample rate, number of channels and bits per sample.
Creating a valid FLAC file is then a matter of writing the proper
file header, metadata blocks and FLAC frames.
\begin{figure}[h]
\includegraphics{figures/flac_stream2.pdf}
\end{figure}

\begin{figure}[h]
\begin{tabular}{|r|l|}
\hline
bits & value \\
\hline
1 & 0 if additional metadata blocks follow, 1 if not \\
7 & 0 for STREAMINFO, 1 for PADDING, 4 for VORBIS\_COMMENT, etc. \\
24 & the length of the block data in bytes, not including the header \\
\hline
\end{tabular}
\caption{Metadata Header}
\end{figure}

\subsection{the STREAMINFO metadata block}
\begin{figure}[h]
\begin{tabular}{|r|l|}
\hline
bits & value \\
\hline
16 & the minimum FLAC frame size, in PCM frames \\
16 & the maximum FLAC frame size, in PCM frames \\
24 & the minimum FLAC frame size, in bytes \\
24 & the maximum FLAC frame size, in bytes \\
20 & the stream's sample rate, in Hz \\
3 & the stream's channel count, minus one \\
5 & the stream's bit-per-sample, minus one \\
36 & the stream's total number of PCM frames \\
128 & an MD5 sum of the PCM stream's bytes \\
\hline
\end{tabular}
\end{figure}
\par
\noindent
This metadata block must come first and is the only required block
in a FLAC file.

When encoding a FLAC file, many of these fields cannot be known in advance.
Instead, one must keep track of those values during encoding and then
rewrite the STREAMINFO block when finished.

\subsection{Frame header}
\begin{minipage}{\linewidth}
\renewcommand\thefootnote{\thempfootnote}
\begin{tabular}{|r|l|}
\hline
bits & value \\
\hline
14 & \texttt{0x3FFE} sync code \\
1 & \texttt{0} reserved \\
1 & \texttt{0} if the header encodes the frame number, \texttt{1} if it encodes the sample number \\
4 & this frame's block size, as encoded PCM frames\footnote{See table on page \pageref{flac_encoded_fields} } \\
4 & this frame's encoded sample rate\footnotemark[\value{mpfootnote}] \\
4 & this frame's encoded channel assignment\footnotemark[\value{mpfootnote}] \\
3 & this frame's encoded bits per sample\footnotemark[\value{mpfootnote}] \\
1 & \texttt{0} padding \\
8-56 & the frame number, or sample number, UTF-8 encoded and starting from 0 \\
0/8/16 & the number of PCM frames (minus one) in this FLAC frame \\
& if block size is \texttt{0x6} (8 bits) or \texttt{0x7} (16 bits) \\
0/8/16 & the sample rate of this FLAC frame \\
& if sample rate is \texttt{0xC} (8 bits), \texttt{0xD} (16 bits) or \texttt{0xE} (16 bits) \\
8 & the CRC-8 of all data from the beginning of the frame header \\
\hline
\end{tabular}
\end{minipage}

\vspace{2ex}
%% The FLAC frame's block size in PCM frames
%% (called ``channel independent samples'' in FLAC's documentation)
%% is typically encoded in the 4 bit `block size' field.
%% But for odd-sized frames - which often occur at the end of the stream -
%% that value is stored as an 8 or 16 bit integer following the UTF-8 encoded
%% frame number.

%% In addition, odd sample rate values are stored as 8 bit (in kHz),
%% 16 bit (in Hz) or 16 bit (in 10s of Hz) prior to the CRC-8,
%% should a predefined value not be available.

%% Up until this point, nearly all of these fields can be filled from
%% the PCM stream data.
%% Unless you're writing a variable block size encoder,
%% one should encode the frame number starting from 0 in the frame header
%% and choose a predefined block size for as many FLAC frames as possible.

\subsection{Channel assignment}

If the input stream has a number of channels other than 2,
one has no choice but to store them independently.
If the number of channels equals 2, one can try all four possible
assignments (left-difference, difference-right, mid-side and
independent) and use the one which takes the least amount of space.

\subsection{Subframe header}
\begin{figure}[h]
\begin{tabular}{|r|l|}
\hline
bits & value \\
\hline
1 & \texttt{0} padding \\
\texttt{000000} & \texttt{SUBFRAME\_CONSTANT} \\
\texttt{000001} & \texttt{SUBFRAME\_VERBATIM} \\
\texttt{001xxx} & \texttt{SUBFRAME\_FIXED} (\texttt{xxx} = Predictor Order) \\
\texttt{1xxxxx} & \texttt{SUBFRAME\_LPC} (\texttt{xxxxx} = Predictor Order - 1) \\
1 & \texttt{0} if no wasted bits per sample, \texttt{1} if a unary-encoded number follows \\
0+ & the number of wasted bits per sample (minus one) encoded as unary \\
\hline
\end{tabular}
\end{figure}

\pagebreak

\subsection{the CONSTANT subframe}

If all the samples in a subframe are identical, one can encode them
using a \texttt{CONSTANT} subframe, which is essentially a single sample value
that gets duplicated `block size' number of times when decoded.

\subsection{the VERBATIM subframe}

This subframe simply stores all the samples as-is,
with no compression whatsoever.
It is a fallback encoding method for when no other subframe makes one's
data any smaller.

\subsection{the FIXED subframe}

This subframe consists of `predictor order' number of unencoded
warm-up samples followed by a residual.
Determining which predictor order to use on a given set of input samples
depends on their minimum delta sum.
This process is best explained by example:


\begin{wrapfigure}{r}{3in}
{\relsize{-2}
\begin{tabular}{|c|r|r|r|r|r|r|}
\hline
index & sample & $\Delta ^ 0$ & $\Delta ^ 1$ & $\Delta ^ 2$ & $\Delta ^ 3$ & $\Delta ^ 4$ \\
\hline
0 & -40 & & & & & \\
1 & -41 & \textit{-41} & & & & \\
2 & -40 & \textit{-40} & \textit{-1} & & & \\
3 & -39 & \textit{-39} & \textit{-1} & \textit{0} & & \\
4 & -38 & -38 & \textit{-1} & \textit{0} & \textit{0} & \\
5 & -38 & -38 & 0 & -1 & 1 & -1 \\
6 & -35 & -35 & -3 & 3 & -4 & 5 \\
7 & -35 & -35 & 0 & -3 & 6 & -10 \\
8 & -39 & -39 & 4 & -4 & 1 & 5 \\
9 & -40 & -40 & 1 & 3 & -7 & 8 \\
10 & -40 & -40 & 0 & 1 & 2 & -9 \\
11 & -39 & -39 & -1 & 1 & 0 & 2 \\
12 & -38 & -38 & -1 & 0 & 1 & -1 \\
13 & -37 & -37 & -1 & 0 & 0 & 1 \\
14 & -33 & -33 & -4 & 3 & -3 & 3 \\
15 & -36 & -36 & 3 & -7 & 10 & -13 \\
16 & -35 & -35 & -1 & 4 & -11 & 21 \\
17 & -31 & -31 & -4 & 3 & 1 & -12 \\
18 & -32 & -32 & 1 & -5 & 8 & -7 \\
19 & -33 & -33 & 1 & 0 & -5 & 13 \\
\hline
$| sum |$ &  & 579 & 26 & 38 & 60 & 111 \\
\hline
\end{tabular}
}
\end{wrapfigure}
Note that the numbers in italics play a part in the delta
calculation to their right, but do \textbf{not} figure into the
delta's absolute value sum, below.

In this example, $\Delta ^ 1$'s value of 26 is the smallest.
Therefore, when compressing this set of samples in a FIXED subframe,
it's best to use a predictor order of 1.

The predictor order indicates how many warm-up samples to take from
the PCM stream.
Determining the residual values can then be done automatically
based on the current $\text{Sample}_i$ and previously encoded samples, or
warm-up samples.

In this example, warm-up sample is -40 and the residual values are: -1 1 1 1 0 3 0 -4 -1 0 1 1 1 4 -3 1 4 -1 -1
\begin{figure}[h]
{\relsize{-1}
\begin{tabular}{|c|>{$}l<{$}|}
\hline
Order & \text{Calculation} \\
\hline
0 & \text{Residual}_i = \text{Sample}_i \\
1 & \text{Residual}_i = \text{Sample}_i - \text{Sample}_{i - 1} \\
2 & \text{Residual}_i = \text{Sample}_i - ((2 \times \text{Sample}_{i - 1}) - \text{Sample}_{i - 2}) \\
3 & \text{Residual}_i = \text{Sample}_i - ((3 \times \text{Sample}_{i - 1}) - (3 \times \text{Sample}_{i - 2}) + \text{Sample}_{i - 3}) \\
4 & \text{Residual}_i = \text{Sample}_i - ((4 \times \text{Sample}_{i - 1}) - (6 \times \text{Sample}_{i - 2}) + (4 \times \text{Sample}_{i - 3}) - \text{Sample}_{i - 4}) \\
\hline
\end{tabular}
}
\end{figure}

\pagebreak

\subsection{the LPC subframe}

Unlike the FIXED subframe which required only input samples and a
predictor order, LPC subframes also require a list of QLP coefficients,
a QLP precision value of those coefficients, and a QLP shift needed
value.

\begin{figure}[h]
\includegraphics{figures/flac_lpc.pdf}
\end{figure}
\par
\noindent
Determining these values for a given input PCM signal is a somewhat
complicated process which depends on whether one is performing an
exhaustive LP coefficient order search or not:

\begin{figure}[h]
\subfloat[non-exhaustive search]{
\includegraphics{figures/flac_nonexhaustive.pdf}
}
\qquad
\subfloat[exhaustive search]{
\includegraphics{figures/flac_exhaustive.pdf}
}
\end{figure}

\pagebreak

\subsubsection{Windowing}

The first step in LPC subframe encoding is `windowing' the input signal.
Put simply, this is a process of multiplying each input sample by
an equivalent value from the window, which are floats from 0.0 to 1.0.
In this case, the default is a Tukey window with a ratio of 0.5.
A Tukey window is a combination of the Hann and Rectangular windows.
The ratio of 0.5 means there's 0.5 samples in the Hann window per
sample in the Rectangular window.

\begin{align}
\text{hann}(n) &= \frac{1}{2} \left ( 1 - \cos \left ( \frac{2 \pi n}{\text{sample count} - 1} \right ) \right ) \\
\text{rectangle}(n) &= 1.0
\end{align}

The Tukey window is defined by taking a Hann window, splitting it at
the halfway point, and inserting a Rectangular window between the
two.

\begin{figure}[h]
\subfloat{
\includegraphics{figures/flac_hann.pdf}
}
\subfloat{
\includegraphics{figures/flac_rectangular.pdf}
}
\subfloat{
\includegraphics{figures/flac_tukey.pdf}
}
\end{figure}

Let's run through a short example with 20 samples:

{\relsize{-1}
\begin{tabular}{|r|r >{$}c<{$} r >{$}c<{$} r|}
\hline
& input & & Tukey & & windowed \\
index & sample & & window & & signal \\
\hline
0 & -40 & \times & 0.0000 & = & 0.00 \\
1 & -41 & \times & 0.1464 & = & -6.00 \\
2 & -40 & \times & 0.5000 & = & -20.00 \\
3 & -39 & \times & 0.8536 & = & -33.29 \\
4 & -38 & \times & 1.0000 & = & -38.00 \\
5 & -38 & \times & 1.0000 & = & -38.00 \\
6 & -35 & \times & 1.0000 & = & -35.00 \\
7 & -35 & \times & 1.0000 & = & -35.00 \\
8 & -39 & \times & 1.0000 & = & -39.00 \\
9 & -40 & \times & 1.0000 & = & -40.00 \\
10 & -40 & \times & 1.0000 & = & -40.00 \\
11 & -39 & \times & 1.0000 & = & -39.00 \\
12 & -38 & \times & 1.0000 & = & -38.00 \\
13 & -37 & \times & 1.0000 & = & -37.00 \\
14 & -33 & \times & 1.0000 & = & -33.00 \\
15 & -36 & \times & 1.0000 & = & -36.00 \\
16 & -35 & \times & 0.8536 & = & -29.88 \\
17 & -31 & \times & 0.5000 & = & -15.50 \\
18 & -32 & \times & 0.1464 & = & -4.68 \\
19 & -33 & \times & 0.0000 & = & 0.00 \\
\hline
\end{tabular}
}

\pagebreak

\subsubsection{Computing autocorrelation}

Once our input samples have been converted to a windowed signal,
we then compute the autocorrelation values from that signal.
Each autocorrelation value is determined by multiplying the signal's
samples by the samples of a lagged version of that same signal,
and then taking the sum.
The lagged signal is simply the original signal with `lag' number of
samples removed from the beginning.

\begin{figure}[h]
\includegraphics{figures/flac_lag.pdf}
\end{figure}

The lagged sums from 0 to the maximum LPC order are our autocorrelation
values.  In this example, they are 14979.0, 13651.0 and 12405.0.

\subsubsection{LP coefficient calculation}


Calculating the LP coefficients uses the Levinson-Durbin recursive method.\footnote{This algorithm is taken from \LINK{http://www.engineer.tamuk.edu/SPark/chap7.pdf} }
Our inputs are $M$, the maximum LPC order minus 1, and $r$ autocorrelation
values, from $r(0)$ to $r(M - 1)$.
Our outputs are $a$, a list of LP coefficient lists from
$a_{11}$ to $a_{(M - 1)(M - 1)}$, and $E$, a list
of error values from $E_0$ to $E_{(M - 1)}$.
$q_m$ and $\kappa_m$ are temporary values.
\par
\noindent
Initial values:
\begin{align}
E_0 &= r(0) \\
a_{11} &= \kappa_1 = \frac{r(1)}{E_0} \\
E_1 &= E_0 {(1 - {{\kappa_1} ^ 2})}
\end{align}
\noindent
With $m \geq 2$, the following recurive algorithm is performed:
\begin{align}
\text{Step 1. } && q_m &= r(m) - \overset{m - 1}{\underset{i = 1}{\sum}} a_{i(m - 1)}r(m - i) \\
\text{Step 2. } && \kappa_m &= \frac{q_m}{E_{(m - 1)}} \\
\text{Step 3. } && a_{mm} &= \kappa_m \\
\text{Step 4. } && a_{im} &= a_{i(m - 1)} - \kappa_m a_{(m - i)(m - 1)} \text{ for $i = 1$, $i = 2$,...,$i = m - 1$} \\
\text{Step 5. } && E_m &= E_{m - 1}(1 - {\kappa_m} ^ 2) \\
\text{Step 6. } &&& \text{If $m < M$ then $m \leftarrow m + 1$ and goto step 1.  If $m = M$ then stop.}
\end{align}

Let's run through an example in which $M = 4$, $r(0) = 11018$, $r(1) = 9690$,
$r(2) = 8443$ and $r(3) = 7280$:
{\relsize{-2}
\begin{align}
E_0 &= r(0) = 11018 \\
a_{11} &= \kappa_1 = \frac{r(1)}{E_0} = \frac{9690}{11018} = 0.8795 \\
E_1 &= E_o(1 - {\kappa_1} ^ 2) = 11018(1 - 0.8795 ^ 2) = 2495 \\
q_2 &= r(2) - \overset{1}{\underset{i = 1}{\sum}} a_{i1}r(2 - i) = 8443 - (0.8795)(9690) = -79.35 \\
\kappa_2 &= \frac{q_2}{E_1} = \frac{-79.35}{2495} = -0.0318 \\
a_{22} &= \kappa_2 = -0.0318 \\
a_{12} &= a_{11} - \kappa_2 a_{11} = 0.8795 - (-0.0318)(0.8795) = 0.9074 \\
E_2 &= E_1 (1 - {\kappa_2} ^ 2 ) = 2495(1 - {-0.0318 ^ 2}) = 2492 \\
q_3 &= r(3) - \overset{2}{\underset{i = 1}{\sum}} a_{i2} r(3 - i) = 7280 - ((0.9074)(8443) + (-0.0318)(9690)) = -73.04 \\
\kappa_3 &= \frac{q_3}{E_2} = \frac{-73.04}{2492} = -0.0293 \\
a_{33} &= \kappa_3 = -0.0293 \\
a_{13} &= a_{12} - \kappa_3 a_{22} = 0.9074 - (-0.0293)(-0.0318) = 0.9065 \\
a_{23} &= a_{22} - \kappa_3 a_{12} = -0.0318 - (-0.0293)(0.9074) = -0.0052 \\
E_3 &= E_2 (1 - {\kappa_3} ^ 2 ) = 2492(1 - {-0.0293 ^ 2}) = 2490
\end{align}
}
Our final values are:
{\relsize{-2}
\begin{align}
a_{11} &= 0.8795 & & & & \\
a_{12} &= 0.9074 & a_{22} &= -0.0318 & & \\
a_{13} &= 0.9065 & a_{23} &= -0.0052 & a_{33} &= -0.0293 \\
E_1 &= 2495 & E_2 &= 2492 & E_3 &= 2490
\end{align}
}
\noindent
These values have been rounded to the nearest significant digit
and will not be an exact match to those generated by a computer.

\subsubsection{Best order estimation}

At this point, we have an array of prospective LP coefficient lists,
a list of error values and must decide which LPC order to use.
There are two ways to accomplish this:  we can either estimate
the total bits from the error values or perform an exhaustive search.
Making the estimation requires the total number of samples
in the subframe, the number of overhead bits per order (by default,
this is the number of bits per sample in the subframe, plus 5),
and an error scale constant in addition to the LPC error values:
\begin{equation}
\text{Error Scale} = \frac{\ln(2) ^ 2}{2 \times \text{Total Samples}}
\end{equation}
\noindent
Once the error scale has been calculated, one can generate a
`Bits per Residual' estimation function which, given an LPC Error value,
returns what its name implies:
\begin{equation}
\text{Bits per Residual}(\text{LPC Error}) = \frac{\ln(\text{Error Scale} \times \text{LPC Error})}{2 \times \ln(2)}
\end{equation}
With this function, we can estimate how many bits the entire LPC subframe
will take for each LPC Error value and its associated Order:
{\relsize{-2}
\begin{equation}
\text{Total Bits}(\text{LPC Error},\text{Order}) = \text{Bits per Residual}(\text{LPC Error}) \times (\text{Total Samples} - \text{Order}) + (\text{Order} \times \text{Overhead bits})
\end{equation}
}
Continuing with our example, we have 20 samples and
now have the error values of 2495, 2492 and 2490.
This gives us an error scale of:
$\frac{{\ln(2) ^ 2}}{2 \times 20} = \frac{.6931 ^ 2}{40} = .01201$
\begin{align}
\intertext{At LPC order 1, our bits per residual are:}
\frac{\ln(.01201 \times 2495)}{2 \times ln (2)} = \frac{\ln(29.96)}{1.386} &= 2.453 \\
\intertext{And our total bits are:}
(2.453 \times (20 - 1)) + (1 \times (16 + 5)) = 46.61 + 21 &= 67.61 \\
\intertext{At LPC order 2, our bits per residual are:}
\frac{\ln(.01201 \times 2492)}{2 \times \ln(2)} = \frac{\ln(29.92)}{1.386} &= 2.452 \\
\intertext{And our total bits are:}
(2.452 \times (20 - 2)) + (2 \times (16 + 5)) = 44.14 + 42 &= 86.14 \\
\intertext{At LPC order 3, our bits per residual are:}
\frac{\ln(.01201 \times 2490)}{2 \times \ln(2)} = \frac{\ln(29.90)}{1.386} &= 2.451
\intertext{And our total bits are:}
(2.451 \times (20 - 3)) + (3 \times (16 + 5)) = 41.67 + 63 &= 104.7
\end{align}
\par
\noindent
Therefore, since the total bits for order 1 are the smallest,
the best order for this group of samples is 1.

Though as you'll notice, the bits per residual for order 3 were the smallest.
So if this group of samples was very large, it's likely that order 3
would prevail since the residuals multiplied by a smaller bits per residual
would counteract the relatively fixed overhead bits per order value.

\subsubsection{Best order exhaustive search}

In a curious bit of recursion, finding the best order for an LPC subframe
via an exhaustive search requires taking each list of LP Coefficients
calculated previously, quantizing them into a list of QLP Coefficients
and a QLP Shift Needed value,\footnote{Quantizing coefficients will be covered in the next section.}
determining the total amount of bits each hypothetical LPC subframe
uses and using the LPC order which uses the fewest.

Remember that building an LPC subframe requires the following values:
LPC Order, QLP Precision, QLP Shift Needed and QLP Coefficients along
with the subframe's samples and bits-per-sample.
For each possible LPC Order, the QLP Shift Needed and the QLP Coefficient
list values can be calculated by quantizing the LP Coefficients.
QLP Precision is the size of each QLP Coefficient list value in the
subframe header.
Simply choose the field with the largest number of bits in the
QLP Coefficient list for the QLP Precision value.

Finally, instead of writing these hypothetical LPC subframes directly to disk,
one only has to capture how many bits they \textit{would} use.
The hypothetical LPC subframe that uses the fewest number of bits
is the one we should actually write to disk.

\pagebreak

\subsubsection{Quantizing coefficients}

Quantizing coefficients is a process of taking a list of LP Coefficients
along with a QLP Coefficients Precision value and
returning a list of QLP Coefficients and a QLP Shift Needed value.
The first step is determining the upper and lower limits of the
QLP Coefficients:
\begin{align}
\text{QLP coefficient maximum} &= 2 ^ {precision - 1} - 1 \\
\text{QLP coefficient mininum} &= - 2 ^ {precision - 1}
\end{align}
The QLP Coefficients Precision value is typically based on the encoder's
block size:
\begin{table}[h]
\begin{tabular}{|>{$}r<{$}|r||>{$}r<{$}|r|}
\hline
\text{Block Size} & Precision & \text{Block Size} & Precision \\
\hline
\text{Size} \leq 192 & 7 & \text{Size} \leq 384 & 8 \\
\text{Size} \leq 576 & 9 & \text{Size} \leq 1152 & 10 \\
\text{Size} \leq 2304 & 11 & \text{Size} \leq 4608 & 12 \\
\text{Size} > 4608 & 13 & & \\
\hline
\end{tabular}
\end{table}
\par
\noindent
So in our example of a block of 20 samples,
\begin{align}
\text{QLP Coefficient maximum} = 2 ^ {7 - 1} -1 = 64 - 1 &= 63 \\
\text{QLP Coefficient minimum} = - 2 ^ {7 - 1} &= -64
\end{align}
\par
\noindent
Now we determine the initial QLP Shift Needed value:
\begin{equation}
\text{shift} = \text{precision} - \left \lceil \frac{\log(\max(|\text{LP Coefficients}|))}{\log(2) } \right \rceil - 1
\end{equation}
\par
\noindent
where `shift' is adjusted if necessary such that:
$0 \leq \text{shift} \leq \texttt{0xF }$
, since it must fit into a 5-bit signed field and negative shifts
are no-ops in the FLAC decoder.

Continuing our ongoing example, let's assume we're quantizing the LP
coefficients 0.9065, -0.0052 and -0.0293.  So our shift should be:
\begin{equation}
\text{shift} = 7 - \left \lceil \frac{\log(0.9065)}{\log(2)} \right \rceil - 1
= 7 - \left \lceil \frac{-0.0981}{0.6931} \right \rceil - 1 = 7 - 0 - 1 = 6
\end{equation}
\par
\noindent
Finally, we determine the QLP Coefficient values themselves via a small
recursive routine:
\begin{align}
X(i) &= E(i - 1) + (\text{LP Coefficient}_i \times 2 ^ {shift}) \\
\text{QLP Coefficient}_i &= \text{round}(X(i)) \\
E(i) &= X(i) - \text{QLP Coefficient}_i
\end{align}
\par
\noindent
where $E(0) = 0$ and each QLP Coefficient is adjusted
prior to calculating the next $E(i)$ value such that:
$\text{QLP coefficient minimum} \leq \text{QLP Coefficient}_i \leq \text{QLP coefficient maximum}$

\pagebreak

So to finish our LPC example:
\begin{align}
X(1) &= E(0) + (0.9065 \times 2 ^ 6 ) = 0 + 58.016 = \textbf{58.016} \\
\text{QLP Coefficient}_1 &= \text{round}(58.016) = \textbf{58} \\
E(1) &= X(1) - \text{QLP Coefficient}_1 = 58.016 - 58 = \textbf{0.016} \\
X(2) &= E(1) + (-0.0052 \times 2 ^ 6 ) = 0.016 + -0.3328 = \textbf{0.3168} \\
\text{QLP Coefficient}_2 &= \text{round}(0.3168) = \textbf{0} \\
E(2) &= X(2) - \text{QLP Coefficient}_2 = 0.3168 - 0 = \textbf{0.3168} \\
X(3) &= E(2) + (-0.0293 \times 2 ^ 6 ) = 0.3168 + -1.875 = \textbf{-1.558} \\
\text{QLP Coefficient}_3 &= \text{round}(-1.558) = \textbf{-2} \\
E(3) &= X(3) - \text{QLP Coefficient}_3 = -1.558 - -2 = \textbf{0.4420}
\end{align}
\par
\noindent
Therefore, the LPC order is 3.
The QLP Coefficients are 58, 0 and -2.
The QLP Shift Needed value is 6.
And, the QLP precision value can be calculated from the bits required
for the largest absolute QLP Coefficient value.
In this case, 6 bits are required to hold the value 58 so
QLP precision can be 6.

%% The LPC Order, QLP Precision, QLP Shift Needed, and QLP Coefficients
%% make up the LPC subframe.

\subsubsection{Calculating LPC residual}

A number of warm-up samples equal to LPC Order are taken from the input PCM
and the subframe's residuals are calculated according to the following formula:
\begin{equation}
\text{Residual}_i = \text{Sample}_i -
\left \lfloor \frac{\overset{Order - 1}{\underset{j = 0}{\sum}}
  \text{QLP Coeffcient}_j \times \text{Sample}_{i - j - 1} }{
2 ^ \text{QLP Shift Needed}} \right \rfloor
\end{equation}
For example, given the samples
1053, 1116, 1257, 1423, 1529, 1570, 1623, 1694, 1688, 1550,
the coefficients: 1241, -944, 14, 342, -147
%% \begin{table}[h]
%% {\relsize{-2}
%% \begin{tabular}{>{$}r<{$} r}
%% \text{QLP Coefficient}_0 & 1241 \\
%% \text{QLP Coefficient}_1 & -944 \\
%% \text{QLP Coefficient}_2 & 14 \\
%% \text{QLP Coefficient}_3 & 342 \\
%% \text{QLP Coefficient}_4 & -147 \\
%% \end{tabular}
%% }
%% \end{table}
and a QLP Shift Needed value of 9, our residuals are as follows:
\begin{table}[h]
{\relsize{-2}
\begin{tabular}{|c|r|>{$}r<{$}|}
\hline
Index & Sample & \text{Residual} \\
\hline
0 & (warm-up) 1053 & \\
1 & (warm-up) 1116 & \\
2 & (warm-up) 1257 & \\
3 & (warm-up) 1423 & \\
4 & (warm-up) 1529 & \\
5 & 1570 & 1570 - \frac{(1241 \times 1529) + (-944 \times 1423) + (14 \times 1257) + (342 \times 1116) + (-147 \times 1053)}{2 ^ 9} = 1570 - \frac{798656}{512} = \textbf{11} \\
6 & 1623 & 1623 - \frac{(1241 \times 1570) + (-944 \times 1529) + (14 \times 1423) + (342 \times 1257) + (-147 \times 1116)}{2 ^ 9} = 1623 - \frac{790758}{512} = \textbf{79} \\
7 & 1694 & 1694 -  \frac{(1241 \times 1623) + (-944 \times 1570) + (14 \times 1529) + (342 \times 1423) + (-147 \times 1257)}{2 ^ 9} = 1694 - \frac{855356}{512} = \textbf{24} \\
8 & 1688 & 1688 - \frac{(1241 \times 1694) + (-944 \times 1623) + (14 \times 1570) + (342 \times 1529) + (-147 \times 1423)}{2 ^ 9} = 1688 - \frac{905859}{512} = \textbf{-81} \\
9 & 1550 & 1550 - \frac{(1241 \times 1688) + (-944 \times 1694) + (14 \times 1623) + (342 \times 1570) + (-147 \times 1529)}{2 ^ 9} = 1550 - \frac{830571}{512} = \textbf{-72} \\
\hline
\end{tabular}
}
\end{table}

\subsection{the Residual}

Given a stream of residual values, one must place them in one or more
partitions, each with its own Rice parameter, and prepended with a
small header:
\begin{figure}[h]
\includegraphics{figures/flac_residual.pdf}
\end{figure}


The residual's coding method is typically 0, unless one is encoding
audio with more than 16 bits-per-sample and one of the partitions
requests a Rice parameter higher than $2 ^ 4$.
The residual's partition order is chosen exhaustively, which means
trying all of them within a certain range (e.g. 0 to 5) such that
the residuals can be divided evenly between them and then the partition
order which uses the smallest estimated amount of space is chosen.

Choosing the best Rice parameter is a matter of selecting the smallest
value of `x' such that:
\begin{equation}
\text{sample count} \times 2 ^ x > \overset{\text{residual count} - 1}{\underset{i = 0}{\sum}} |\text{residual}_i|
\end{equation}
\par
\noindent
Again, this is easier to understand with a block of example residuals,
19 in total:
\begin{table}[h]
{\relsize{-2}
\begin{tabular}{|c|r|r|}
\hline
index & $\text{residual}_i$ & $|\text{residual}_i|$ \\
\hline
0 & -1 & 1 \\
1 & 1 & 1 \\
2 & 1 & 1 \\
3 & 1 & 1 \\
4 & 0 & 0 \\
5 & 3 & 3 \\
6 & 0 & 0 \\
7 & -4 & 4 \\
8 & -1 & 1 \\
9 & 0 & 0 \\
10 & 1 & 1 \\
11 & 1 & 1 \\
12 & 1 & 1 \\
13 & 4 & 4 \\
14 & -3 & 3 \\
15 & 1 & 1 \\
16 & 4 & 4 \\
17 & -1 & 1 \\
18 & -1 & 1 \\
\hline
$|\text{sum}|$ & & \textbf{29} \\
\hline
\end{tabular}
}
\end{table}
\par
\noindent
$19 \times {2 ^ 0}$ is not larger than 29.
$19 \times {2 ^ 1}$ is larger than 29, so the best Rice parameter
for this block of residuals is 1.
Remember that the Rice parameter's maximum value is limited to
$2 ^ 4$ using coding method 0, or $2 ^ 5$ using coding method 1.

\subsubsection{Residual values}


Encoding individual residual values to Rice coding requires only
the Rice parameter and the values themselves.
First, one must convert any negative values to positive by
multiplying it by -1, subtracting 1 and prepending a 1 bit.
If the value is already positive, prepend a 0 bit instead.
Next, we split out new value into most significant bits (MSB) and
least significant bits (LSB) where the length of the LSB is
equal to the Rice parameter and MSB contains the remaining bits.
The MSB value is written unary encoded, whereas the LSB is written as-is.

As with residual decoding, this process is not as difficult as it
sounds and is best explained by an example, in this case using
a parameter of 3 and encoding the residual values 18, -25 and 12:
\begin{figure}[h]
\includegraphics{figures/flac_rice3.pdf}
\end{figure}

\pagebreak

\subsection{Checksums}

Calculating the frame header's CRC-8 and frame footer's CRC-16 is necessary
both for FLAC encoders and decoders, but the process is the same for each.

\subsubsection{CRC-8}

Given a byte of input and the previous CRC-8 checksum,
or 0 as an initial value, the current checksum can be calculated as follows:
\begin{equation}
\text{checksum}_i = \text{CRC8}(byte\xor\text{checksum}_{i - 1})
\end{equation}
\begin{table}[h]
{\relsize{-3}\ttfamily
\begin{tabular}{|r||r|r|r|r|r|r|r|r|r|r|r|r|r|r|r|r|r|}
\hline
 & 0x?0 & 0x?1 & 0x?2 & 0x?3 & 0x?4 & 0x?5 & 0x?6 & 0x?7 & 0x?8 & 0x?9 & 0x?A & 0x?B & 0x?C & 0x?D & 0x?E & 0x?F \\
\hline
0x0? & 0x00 & 0x07 & 0x0E & 0x09 & 0x1C & 0x1B & 0x12 & 0x15 & 0x38 & 0x3F & 0x36 & 0x31 & 0x24 & 0x23 & 0x2A & 0x2D \\
0x1? & 0x70 & 0x77 & 0x7E & 0x79 & 0x6C & 0x6B & 0x62 & 0x65 & 0x48 & 0x4F & 0x46 & 0x41 & 0x54 & 0x53 & 0x5A & 0x5D \\
0x2? & 0xE0 & 0xE7 & 0xEE & 0xE9 & 0xFC & 0xFB & 0xF2 & 0xF5 & 0xD8 & 0xDF & 0xD6 & 0xD1 & 0xC4 & 0xC3 & 0xCA & 0xCD \\
0x3? & 0x90 & 0x97 & 0x9E & 0x99 & 0x8C & 0x8B & 0x82 & 0x85 & 0xA8 & 0xAF & 0xA6 & 0xA1 & 0xB4 & 0xB3 & 0xBA & 0xBD \\
0x4? & 0xC7 & 0xC0 & 0xC9 & 0xCE & 0xDB & 0xDC & 0xD5 & 0xD2 & 0xFF & 0xF8 & 0xF1 & 0xF6 & 0xE3 & 0xE4 & 0xED & 0xEA \\
0x5? & 0xB7 & 0xB0 & 0xB9 & 0xBE & 0xAB & 0xAC & 0xA5 & 0xA2 & 0x8F & 0x88 & 0x81 & 0x86 & 0x93 & 0x94 & 0x9D & 0x9A \\
0x6? & 0x27 & 0x20 & 0x29 & 0x2E & 0x3B & 0x3C & 0x35 & 0x32 & 0x1F & 0x18 & 0x11 & 0x16 & 0x03 & 0x04 & 0x0D & 0x0A \\
0x7? & 0x57 & 0x50 & 0x59 & 0x5E & 0x4B & 0x4C & 0x45 & 0x42 & 0x6F & 0x68 & 0x61 & 0x66 & 0x73 & 0x74 & 0x7D & 0x7A \\
0x8? & 0x89 & 0x8E & 0x87 & 0x80 & 0x95 & 0x92 & 0x9B & 0x9C & 0xB1 & 0xB6 & 0xBF & 0xB8 & 0xAD & 0xAA & 0xA3 & 0xA4 \\
0x9? & 0xF9 & 0xFE & 0xF7 & 0xF0 & 0xE5 & 0xE2 & 0xEB & 0xEC & 0xC1 & 0xC6 & 0xCF & 0xC8 & 0xDD & 0xDA & 0xD3 & 0xD4 \\
0xA? & 0x69 & 0x6E & 0x67 & 0x60 & 0x75 & 0x72 & 0x7B & 0x7C & 0x51 & 0x56 & 0x5F & 0x58 & 0x4D & 0x4A & 0x43 & 0x44 \\
0xB? & 0x19 & 0x1E & 0x17 & 0x10 & 0x05 & 0x02 & 0x0B & 0x0C & 0x21 & 0x26 & 0x2F & 0x28 & 0x3D & 0x3A & 0x33 & 0x34 \\
0xC? & 0x4E & 0x49 & 0x40 & 0x47 & 0x52 & 0x55 & 0x5C & 0x5B & 0x76 & 0x71 & 0x78 & 0x7F & 0x6A & 0x6D & 0x64 & 0x63 \\
0xD? & 0x3E & 0x39 & 0x30 & 0x37 & 0x22 & 0x25 & 0x2C & 0x2B & 0x06 & 0x01 & 0x08 & 0x0F & 0x1A & 0x1D & 0x14 & 0x13 \\
0xE? & 0xAE & 0xA9 & 0xA0 & 0xA7 & 0xB2 & 0xB5 & 0xBC & 0xBB & 0x96 & 0x91 & 0x98 & 0x9F & 0x8A & 0x8D & 0x84 & 0x83 \\
0xF? & 0xDE & 0xD9 & 0xD0 & 0xD7 & 0xC2 & 0xC5 & 0xCC & 0xCB & 0xE6 & 0xE1 & 0xE8 & 0xEF & 0xFA & 0xFD & 0xF4 & 0xF3 \\
\hline
\end{tabular}
}
\end{table}
\par
\noindent
For example, given the header bytes: 0xFF, 0xF8, 0xCC, 0x1C, 0x00 and 0xC0:
\begin{align}
\text{checksum}_0 &= \text{CRC8}(\texttt{0xFF}\xor\texttt{0x00}) = \text{CRC8}(\texttt{0xFF}) = \texttt{0xF3} \\
\text{checksum}_1 &= \text{CRC8}(\texttt{0xF8}\xor\texttt{0xF3}) = \text{CRC8}(\texttt{0x0B}) = \texttt{0x31} \\
\text{checksum}_2 &= \text{CRC8}(\texttt{0xCC}\xor\texttt{0x31}) = \text{CRC8}(\texttt{0xFD}) = \texttt{0xFD} \\
\text{checksum}_3 &= \text{CRC8}(\texttt{0x1C}\xor\texttt{0xFD}) = \text{CRC8}(\texttt{0xE1}) = \texttt{0xA9} \\
\text{checksum}_4 &= \text{CRC8}(\texttt{0x00}\xor\texttt{0xA9}) = \text{CRC8}(\texttt{0xA9}) = \texttt{0x56} \\
\text{checksum}_5 &= \text{CRC8}(\texttt{0xC0}\xor\texttt{0x56}) = \text{CRC8}(\texttt{0x96}) = \texttt{0xEB}
\intertext{Thus, the next byte after the header should be \texttt{0xEB}.
Furthermore, when the checksum byte itself is run through the
checksumming procedure:}
\text{checksum}_6 &= \text{CRC8}(\texttt{0xEB}\xor\texttt{0xEB}) = \text{CRC8}(\texttt{0x00}) = \texttt{0x00}
\end{align}
\par
\noindent
the result will always be 0.
This is a handy way to verify a frame header's checksum since the
checksum of the header's bytes along with the header's checksum itself will
always result in 0.

\subsubsection{CRC-16}

CRC-16 is used to checksum the entire FLAC frame, including the header
and any padding bits after the final subframe.
Given a byte of input and the previous CRC-16 checksum,
or 0 as an initial value, the current checksum can be calculated as follows:
\begin{equation}
\text{checksum}_i = \text{CRC16}(byte\xor(\text{checksum}_{i - 1} \gg 8 ))\xor(\text{checksum}_{i - 1} \ll 8)
\end{equation}
\par
\noindent
and the checksum is always truncated to 16-bits.
\begin{table}[h]
{\relsize{-3}\ttfamily
\begin{tabular}{|r||r|r|r|r|r|r|r|r|r|r|r|r|r|r|r|r|}
\hline
 & 0x?0 & 0x?1 & 0x?2 & 0x?3 & 0x?4 & 0x?5 & 0x?6 & 0x?7 & 0x?8 & 0x?9 & 0x?A & 0x?B & 0x?C & 0x?D & 0x?E & 0x?F \\
\hline
0x0? & 0000 & 8005 & 800f & 000a & 801b & 001e & 0014 & 8011 & 8033 & 0036 & 003c & 8039 & 0028 & 802d & 8027 & 0022 \\
0x1? & 8063 & 0066 & 006c & 8069 & 0078 & 807d & 8077 & 0072 & 0050 & 8055 & 805f & 005a & 804b & 004e & 0044 & 8041 \\
0x2? & 80c3 & 00c6 & 00cc & 80c9 & 00d8 & 80dd & 80d7 & 00d2 & 00f0 & 80f5 & 80ff & 00fa & 80eb & 00ee & 00e4 & 80e1 \\
0x3? & 00a0 & 80a5 & 80af & 00aa & 80bb & 00be & 00b4 & 80b1 & 8093 & 0096 & 009c & 8099 & 0088 & 808d & 8087 & 0082 \\
0x4? & 8183 & 0186 & 018c & 8189 & 0198 & 819d & 8197 & 0192 & 01b0 & 81b5 & 81bf & 01ba & 81ab & 01ae & 01a4 & 81a1 \\
0x5? & 01e0 & 81e5 & 81ef & 01ea & 81fb & 01fe & 01f4 & 81f1 & 81d3 & 01d6 & 01dc & 81d9 & 01c8 & 81cd & 81c7 & 01c2 \\
0x6? & 0140 & 8145 & 814f & 014a & 815b & 015e & 0154 & 8151 & 8173 & 0176 & 017c & 8179 & 0168 & 816d & 8167 & 0162 \\
0x7? & 8123 & 0126 & 012c & 8129 & 0138 & 813d & 8137 & 0132 & 0110 & 8115 & 811f & 011a & 810b & 010e & 0104 & 8101 \\
0x8? & 8303 & 0306 & 030c & 8309 & 0318 & 831d & 8317 & 0312 & 0330 & 8335 & 833f & 033a & 832b & 032e & 0324 & 8321 \\
0x9? & 0360 & 8365 & 836f & 036a & 837b & 037e & 0374 & 8371 & 8353 & 0356 & 035c & 8359 & 0348 & 834d & 8347 & 0342 \\
0xA? & 03c0 & 83c5 & 83cf & 03ca & 83db & 03de & 03d4 & 83d1 & 83f3 & 03f6 & 03fc & 83f9 & 03e8 & 83ed & 83e7 & 03e2 \\
0xB? & 83a3 & 03a6 & 03ac & 83a9 & 03b8 & 83bd & 83b7 & 03b2 & 0390 & 8395 & 839f & 039a & 838b & 038e & 0384 & 8381 \\
0xC? & 0280 & 8285 & 828f & 028a & 829b & 029e & 0294 & 8291 & 82b3 & 02b6 & 02bc & 82b9 & 02a8 & 82ad & 82a7 & 02a2 \\
0xD? & 82e3 & 02e6 & 02ec & 82e9 & 02f8 & 82fd & 82f7 & 02f2 & 02d0 & 82d5 & 82df & 02da & 82cb & 02ce & 02c4 & 82c1 \\
0xE? & 8243 & 0246 & 024c & 8249 & 0258 & 825d & 8257 & 0252 & 0270 & 8275 & 827f & 027a & 826b & 026e & 0264 & 8261 \\
0xF? & 0220 & 8225 & 822f & 022a & 823b & 023e & 0234 & 8231 & 8213 & 0216 & 021c & 8219 & 0208 & 820d & 8207 & 0202 \\
\hline
\end{tabular}
}
\end{table}
\par
\noindent
For example, given the frame bytes:
0xFF, 0xF8, 0xCC, 0x1C, 0x00, 0xC0, 0xEB, 0x00, 0x00, 0x00, 0x00,
0x00, 0x00, 0x00 and 0x00, the frame's CRC-16 can be calculated as
follows:
{\relsize{-2}
\begin{align}
\CRCSIXTEEN{0}{0xFF}{0x0000}{0xFF}{0x0000}{0x0202} \\
\CRCSIXTEEN{1}{0xF8}{0x0202}{0xFA}{0x0200}{0x001C} \\
\CRCSIXTEEN{2}{0xCC}{0x001C}{0xCC}{0x1C00}{0x1EA8} \\
\CRCSIXTEEN{3}{0x1C}{0x1EA8}{0x02}{0xA800}{0x280F} \\
\CRCSIXTEEN{4}{0x00}{0x280F}{0x28}{0x0F00}{0x0FF0} \\
\CRCSIXTEEN{5}{0xC0}{0x0FF0}{0xCF}{0xF000}{0xF2A2} \\
\CRCSIXTEEN{6}{0xEB}{0xF2A2}{0x19}{0xA200}{0x2255} \\
\CRCSIXTEEN{7}{0x00}{0x2255}{0x22}{0x5500}{0x55CC} \\
\CRCSIXTEEN{8}{0x00}{0x55CC}{0x55}{0xCC00}{0xCDFE} \\
\CRCSIXTEEN{9}{0x00}{0xCDFE}{0xCD}{0xFE00}{0x7CAD} \\
\CRCSIXTEEN{10}{0x00}{0x7CAD}{0x7C}{0xAD00}{0x2C0B} \\
\CRCSIXTEEN{11}{0x00}{0x2C0B}{0x2C}{0x0B00}{0x8BEB} \\
\CRCSIXTEEN{12}{0x00}{0x8BEB}{0x8B}{0xEB00}{0xE83A} \\
\CRCSIXTEEN{13}{0x00}{0xE83A}{0xE8}{0x3A00}{0x3870} \\
\CRCSIXTEEN{14}{0x00}{0x3870}{0x38}{0x7000}{0xF093} \\
\intertext{Thus, the next two bytes after the final subframe should be
\texttt{0xF0} and \texttt{0x93}.
Again, when the checksum bytes are run through the checksumming procedure:}
\CRCSIXTEEN{15}{0xF0}{0xF093}{0x00}{0x9300}{0x9300} \\
\CRCSIXTEEN{16}{0x93}{0x9300}{0x00}{0x0000}{0x0000}
\end{align}
the result will also always be 0, just as in the CRC-8.
}
