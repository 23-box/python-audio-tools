\chapter{Free Lossless Audio Codec}
FLAC compresses PCM audio data losslessly using predictors and a
residual.
FLACs contain checksumming to verify their integrity, contain comment tags for
metadata and are streamable.

Except for the contents of the VORBIS\_COMMENT metadata block, everything in FLAC is big-endian.

\section{the FLAC File Stream}
\begin{figure}[h]
\includegraphics{figures/flac_stream.pdf}
\end{figure}
\par
\noindent
``Last'' is 0 when there are no additional metadata blocks and 1 when
it is the final block before the the audio frames.
``Block Length'' is the size of the metadata block data to follow,
not including the header.
\begin{figure}[h]
\begin{tabular}{| r | l |}
\hline
Block Type & Block \\
\hline
\texttt{0} & STREAMINFO \\
\texttt{1} & PADDING \\
\texttt{2} & APPLICATION \\
\texttt{3} & SEEKTABLE \\
\texttt{4} & VORBIS\_COMMENT \\
\texttt{5} & CUESHEET \\
\texttt{6} & PICTURE \\
\texttt{7-126} & reserved \\
\texttt{127} & invalid \\
\hline
\end{tabular}
\end{figure}

\pagebreak

\section{FLAC Metadata Blocks}

\subsection{STREAMINFO}
\begin{figure}[h]
\includegraphics{figures/flac_streaminfo.pdf}
\end{figure}

\subsection{PADDING}

PADDING is simply a block full of NULL (\texttt{0x00}) bytes.
Its purpose is to provide extra metadata space within the FLAC file.
By having a padding block, other metadata blocks can be grown or
shrunk without having to rewrite the entire FLAC file by removing or
adding space to the padding.


\subsection{APPLICATION}
\begin{figure}[h]
\includegraphics{figures/flac_application.pdf}
\end{figure}
\noindent
APPLICATION is a general-purpose metadata block used by a variety of
different programs.
Its contents are defined by the ASCII Application ID value.

\subsection{SEEKTABLE}
\begin{figure}[h]
\includegraphics{figures/flac_seektable.pdf}
\end{figure}

\pagebreak

\subsection{VORBIS\_COMMENT}
\begin{figure}[h]
\includegraphics{figures/flac_vorbiscomment.pdf}
\end{figure}
\par
\noindent
The length fields are all little-endian.
The Vendor String and Comment Strings are all UTF-8 encoded.
Keys are not case-sensitive and may occur multiple times,
indicating multiple values for the same field.
For instance, a track with multiple artists may have
more than one \texttt{ARTIST}.

\begin{minipage}{\linewidth}
\renewcommand\thefootnote{\thempfootnote}
{\relsize{-1}
\begin{tabular}{|r|l|}
\hline
key & value \\
\hline
ALBUM & album name \\
ARTIST & artist name, band name, composer, author, etc. \\
CATALOGNUMBER\footnote{These are proposed extension fields and not part of the official Vorbis comment specification.} & CD spine number \\
COMPOSER\footnotemark[\value{mpfootnote}] & the work's author \\
CONDUCTOR\footnotemark[\value{mpfootnote}] & performing ensemble's leader \\
COPYRIGHT & copyright attribution \\
DATE & recording date \\
DESCRIPTION & a short description \\
DISCNUMBER\footnotemark[\value{mpfootnote}] & disc number for multi-volume work \\
ENGINEER\footnotemark[\value{mpfootnote}] & the recording masterer \\
ENSEMBLE\footnotemark[\value{mpfootnote}] & performing group \\
GENRE & a short music genre label \\
GUEST ARTIST\footnotemark[\value{mpfootnote}] & collaborating artist \\
ISRC & ISRC number for the track \\
LICENSE & license information \\
LOCATION & recording location \\
OPUS\footnotemark[\value{mpfootnote}] & number of the work \\
ORGANIZATION & record label \\
PART\footnotemark[\value{mpfootnote}] & track's movement title \\
PERFORMER & performer name, orchestra, actor, etc. \\
PRODUCER\footnotemark[\value{mpfootnote}] & person responsible for the project \\
PRODUCTNUMBER\footnotemark[\value{mpfootnote}] & UPC, EAN, or JAN code \\
PUBLISHER\footnotemark[\value{mpfootnote}] & album's publisher \\
RELEASE DATE\footnotemark[\value{mpfootnote}] & date the album was published \\
REMIXER\footnotemark[\value{mpfootnote}] & person who created the remix \\
SOURCE ARTIST\footnotemark[\value{mpfootnote}] & artist of the work being performed \\
SOURCE MEDIUM\footnotemark[\value{mpfootnote}] & CD, radio, cassette, vinyl LP, etc. \\
SOURCE WORK\footnotemark[\value{mpfootnote}] & a soundtrack's original work \\
SPARS\footnotemark[\value{mpfootnote}] & DDD, ADD, AAD, etc. \\
SUBTITLE\footnotemark[\value{mpfootnote}] & for multiple track names in a single file \\
TITLE & track name \\
TRACKNUMBER & track number \\
VERSION & track version \\
\hline
\end{tabular}
}
\end{minipage}

\pagebreak

\subsection{CUESHEET}

\subsection{PICTURE}
