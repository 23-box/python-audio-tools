%This work is licensed under the
%Creative Commons Attribution-Share Alike 3.0 United States License.
%To view a copy of this license, visit
%http://creativecommons.org/licenses/by-sa/3.0/us/ or send a letter to
%Creative Commons,
%171 Second Street, Suite 300,
%San Francisco, California, 94105, USA.

\chapter{Free Lossless Audio Codec}
FLAC compresses PCM audio data losslessly using predictors and a
residual.
FLACs contain checksumming to verify their integrity, contain comment tags for
metadata and are streamable.

Except for the contents of the VORBIS\_COMMENT metadata block, everything in FLAC is big-endian.

\section{the FLAC File Stream}
\begin{figure}[h]
\includegraphics{figures/flac/stream.pdf}
\end{figure}
\par
\noindent
\VAR{Last} is 0 when there are no additional metadata blocks and 1 when
it is the final block before the the audio frames.
\VAR{Block Length} is the size of the metadata block data to follow,
not including the header.
\begin{figure}[h]
\begin{tabular}{| r | l |}
\hline
Block Type & Block \\
\hline
\texttt{0} & STREAMINFO \\
\texttt{1} & PADDING \\
\texttt{2} & APPLICATION \\
\texttt{3} & SEEKTABLE \\
\texttt{4} & VORBIS\_COMMENT \\
\texttt{5} & CUESHEET \\
\texttt{6} & PICTURE \\
\texttt{7-126} & reserved \\
\texttt{127} & invalid \\
\hline
\end{tabular}
\end{figure}

\pagebreak

\section{FLAC Metadata Blocks}

\subsection{STREAMINFO}
\begin{figure}[h]
\includegraphics{figures/flac/streaminfo.pdf}
\end{figure}

\subsection{PADDING}

PADDING is simply a block full of NULL (\texttt{0x00}) bytes.
Its purpose is to provide extra metadata space within the FLAC file.
By having a padding block, other metadata blocks can be grown or
shrunk without having to rewrite the entire FLAC file by removing or
adding space to the padding.


\subsection{APPLICATION}
\begin{figure}[h]
\includegraphics{figures/flac/application.pdf}
\end{figure}
\noindent
APPLICATION is a general-purpose metadata block used by a variety of
different programs.
Its contents are defined by the ASCII Application ID value.

\subsection{SEEKTABLE}
\begin{figure}[h]
\includegraphics{figures/flac/seektable.pdf}
\end{figure}

\pagebreak

\subsection{VORBIS\_COMMENT}
\begin{figure}[h]
\includegraphics{figures/flac/vorbiscomment.pdf}
\end{figure}
\par
\noindent
The length fields are all little-endian.
The Vendor String and Comment Strings are all UTF-8 encoded.
Keys are not case-sensitive and may occur multiple times,
indicating multiple values for the same field.
For instance, a track with multiple artists may have
more than one \texttt{ARTIST}.

\begin{multicols}{2}
{\relsize{-2}
\begin{description}
\item[ALBUM] album name
\item[ARTIST] artist name, band name, composer, author, etc.
\item[CATALOGNUMBER*] CD spine number
\item[COMPOSER*] the work's author
\item[CONDUCTOR*] performing ensemble's leader
\item[COPYRIGHT] copyright attribution
\item[DATE] recording date
\item[DESCRIPTION] a short description
\item[DISCNUMBER*] disc number for multi-volume work
\item[ENGINEER*] the recording masterer
\item[ENSEMBLE*] performing group
\item[GENRE] a short music genre label
\item[GUEST ARTIST*] collaborating artist
\item[ISRC] ISRC number for the track
\item[LICENSE] license information
\item[LOCATION] recording location
\item[OPUS*] number of the work
\item[ORGANIZATION] record label
\item[PART*] track's movement title
\item[PERFORMER] performer name, orchestra, actor, etc.
\item[PRODUCER*] person responsible for the project
\item[PRODUCTNUMBER*] UPC, EAN, or JAN code
\item[PUBLISHER*] album's publisher
\item[RELEASE DATE*] date the album was published
\item[REMIXER*] person who created the remix
\item[SOURCE ARTIST*] artist of the work being performed
\item[SOURCE MEDIUM*] CD, radio, cassette, vinyl LP, etc.
\item[SOURCE WORK*] a soundtrack's original work
\item[SPARS*] DDD, ADD, AAD, etc.
\item[SUBTITLE*] for multiple track names in a single file
\item[TITLE] track name
\item[TRACKNUMBER] track number
\item[VERSION] track version
\end{description}
}
\end{multicols}
\par
\noindent
Fields marked with * are proposed extension fields and not part of the official Vorbis comment specification.

\pagebreak

\subsection{CUESHEET}
\begin{figure}[h]
\includegraphics{figures/flac/cuesheet.pdf}
\end{figure}

\subsection{PICTURE}
\begin{figure}[h]
\includegraphics{figures/flac/picture.pdf}
\end{figure}
\begin{tabular}{|r|l|}
\hline
Picture Type & Type \\
\hline
0 & Other \\
1 & 32x32 pixels `file icon' (PNG only) \\
2 & Other file icon \\
3 & Cover (front) \\
4 & Cover (back) \\
5 & Leaflet page \\
6 & Media (e.g. label side of CD) \\
7 & Lead artist / Lead performer / Soloist \\
8 & Artist / Performer \\
9 & Conductor \\
10 & Band / Orchestra \\
11 & Composer \\
12 & Lyricist / Text writer \\
13 & Recording location \\
14 & During recording \\
15 & During performance \\
16 & Movie / Video screen capture \\
17 & A bright colored fish \\
18 & Illustration \\
19 & Band / Artist logotype \\
20 & Publisher / Studio logotype \\
\hline
\end{tabular}

\section{FLAC Decoding}

The basic process for decoding a FLAC file is as follows:
\par
\noindent
\ALGORITHM{a FLAC encoded file}{PCM samples}
file header $\leftarrow$ \READ 4 bytes\;
\ASSERT $\text{file header} = \texttt{"fLaC"}$\;
get $PCM~frame~count$ and MD5 sum from STREAMINFO metadata block\;
skip remaining metadata blocks\;
initialize stream MD5\;
\While{$PCM~frame~count > 0$}{
  decode FLAC frame to 1 or more PCM frames\;
  deduct FLAC frame's block size from $PCM~frame~count$\;
  update stream MD5 sum with decoded PCM frame data\;
  return decoded PCM frames\;
}
\ASSERT STREAMINFO MD5 sum = stream MD5 sum
\EALGORITHM
\begin{figure}[h]
\includegraphics{figures/flac/stream3.pdf}
\end{figure}
\par
All of the fields in the FLAC stream are big-endian.\footnote{Except
for the length fields in the VORBIS\_COMMENT metadata block.
However, this block is not needed for decoding.
}

\clearpage

\subsection{Reading Metadata Blocks}
\ALGORITHM{the FLAC file stream}{nine STREAMINFO values used for decoding}
\Repeat{last = 1}{
  $last \leftarrow$ \READ 1 unsigned bit\;
  $type \leftarrow$ \READ 7 unsigned bits\;
  $size \leftarrow$ \READ 24 unsigned bits\;
\eIf(\tcc*[f]{read STREAMINFO metadata block}){type = 0}{
\begin{tabular}{rcl}
minimum block size & $\leftarrow$ & \READ 16 unsigned bits\; \\
maximum block size & $\leftarrow$ & \READ 16 unsigned bits\; \\
mimimum frame size & $\leftarrow$ & \READ 24 unsigned bits\; \\
maximum frame size & $\leftarrow$ & \READ 24 unsigned bits\; \\
sample rate & $\leftarrow$ & \READ 20 unsigned bits\; \\
channels & $\leftarrow$ & (\READ 3 unsigned bits) + 1\; \\
bits per sample & $\leftarrow$ & (\READ 5 unsigned bits) + 1\; \\
total PCM frames & $\leftarrow$ & \READ 36 unsigned bits\; \\
MD5 sum & $\leftarrow$ & \READ 16 bytes\;
\end{tabular}
}(\tcc*[f]{skip other metadata blocks}){
  \SKIP $size$ bytes\;
}
}
\EALGORITHM
\begin{figure}[h]
\includegraphics{figures/flac/metadata.pdf}
\end{figure}

\clearpage

For example, given the metadata bytes:
\begin{figure}[h]
\includegraphics{figures/flac/block_header.pdf}
\end{figure}
\par
\noindent
\begin{tabular}{rcrcl}
$last$ & $\leftarrow$ & \texttt{0x1} & = & is last metadata block \\
$type$ & $\leftarrow$ & \texttt{0x0} & = & METADATA block \\
$size$ & $\leftarrow$ & \texttt{0x22} & = & 34 bytes \\
minimum block size & $\leftarrow$ & \texttt{0x0100} & = & 4096 samples \\
maximum block size & $\leftarrow$ & \texttt{0x0100} & = & 4096 samples \\
minimum frame size & $\leftarrow$ & \texttt{0x00000C} & = & 12 bytes \\
maximum frame size & $\leftarrow$ & \texttt{0x00000C} & = & 12 bytes \\
sample rate & $\leftarrow$ & \texttt{0xAC44} & = & 44100Hz \\
channels & $\leftarrow$ & \texttt{0x1} & = & 1 (+ 1) = 2 \\
bits per sample & $\leftarrow$ & \texttt{0xF} & = & 15 (+ 1) = 16 \\
total PCM frames & $\leftarrow$ & \texttt{0x50} & = & 80
\end{tabular}

\clearpage

\subsection{Decoding a FLAC Frame}

\ALGORITHM{STREAMINFO values and the FLAC file stream}{decoded PCM samples}
read frame header to determine channel count, assignment and bits-per-sample\;
\ForEach{channel \IN channel count}{
  decode subframe to PCM samples based on its effective bits-per-sample\;
}
byte-align file stream\;
verify frame's CRC-16 checksum\;
recombine subframes based on the frame's channel assignment\;
\Return samples\;
\EALGORITHM
\begin{figure}[h]
\includegraphics{figures/flac/frames.pdf}
\end{figure}

\clearpage

\subsection{Reading a FLAC Frame Header}

\ALGORITHM{STREAMINFO values and the FLAC file stream}{stream information and subframe decoding parameters}
\begin{tabular}{rcl}
sync code & $\leftarrow$ & \READ 14 unsigned bits\; \\
& & \ASSERT $\text{sync code} = \texttt{0x3FFE}$\; \\
& & \SKIP 1 bit\; \\
$blocking~strategy$ & $\leftarrow$ & \READ 1 unsigned bit\; \\
$encoded~block~size$ & $\leftarrow$ & \READ 4 unsigned bits\; \\
$encoded~sample~rate$ & $\leftarrow$ & \READ 4 unsigned bits\; \\
$encoded~channels$ & $\leftarrow$ & \READ 4 unsigned bits\; \\
$encoded~bps$ & $\leftarrow$ & \READ 3 unsigned bits\; \\
& & \SKIP 1 bit\; \\
frame number & $\leftarrow$ & \READ UTF-8 value\; \\
block size & $\leftarrow$ & decode $encoded~block~size$\; \\
sample rate & $\leftarrow$ & decode $encoded~sample~rate$\; \\
bits per sample & $\leftarrow$ & decode $encoded~bps$\; \\
channel count & $\leftarrow$ & decode $encoded~channels$\; \\
$CRC8$ & $\leftarrow$ & \READ 8 unsigned bits\; \\
& & verify $CRC8$\;
\end{tabular}
\EALGORITHM

\subsubsection{Reading UTF-8 Frame Number}
\ALGORITHM{FLAC file stream}{UTF-8 value as unsigned integer}
$total~bytes \leftarrow$ \UNARY with stop bit 0\;
$value \leftarrow$ \READ (7 - $total~bytes$) unsigned bits\;
\While{$total~bytes > 0$}{
  $continuation~header \leftarrow$ \READ 2 unsigned bits\;
  \ASSERT $continuation~header = 2$\;
  $continuation~bits \leftarrow$ \READ 6 unsigned bits\;
  $value \leftarrow (value \times 2 ^ 6) + continuation~bits$\;
  $total~bytes \leftarrow total~bytes - 1$\;
}
\Return $value$\;
\EALGORITHM
For example, given the UTF-8 bytes \texttt{E1 82 84}:
\par
\begin{wrapfigure}[5]{l}{2.375in}
\includegraphics{figures/flac/utf8.pdf}
\end{wrapfigure}
\begin{align*}
\text{UTF-8 value} &= \texttt{0001 000010 000100} \\
&= \texttt{0001 0000 1000 0100} \\
&= \texttt{0x1084} \\
&= 4228
\end{align*}

\clearpage

\subsubsection{Decoding Block Size}
{\relsize{-1}
\begin{tabular}{rl||rl}
encoded & block size (in samples) &
encoded & block size \\
\hline
\texttt{0000} & maximum block size from STREAMINFO &
\texttt{1000} & 256 \\
\texttt{0001} & 192 &
\texttt{1001} & 512 \\
\texttt{0010} & 576 &
\texttt{1010} & 1024 \\
\texttt{0011} & 1152 &
\texttt{1011} & 2048 \\
\texttt{0100} & 2304 &
\texttt{1100} & 4096 \\
\texttt{0101} & 4608 &
\texttt{1101} & 8192 \\
\texttt{0110} & (\textbf{read} 8 unsigned bits) + 1 &
\texttt{1110} & 16384 \\
\texttt{0111} & (\textbf{read} 16 unsigned bits) + 1 &
\texttt{1111} & 32768 \\
\end{tabular}
}

\subsubsection{Decoding Sample Rate}
{\relsize{-1}
\begin{tabular}{rl||rl}
encoded & sample rate (in Hz) &
encoded & sample rate \\
\hline
\texttt{0000} & from STREAMINFO &
\texttt{1000} & 32000 \\
\texttt{0001} & 88200 &
\texttt{1001} & 44100 \\
\texttt{0010} & 176400 &
\texttt{1010} & 48000 \\
\texttt{0011} & 192000 &
\texttt{1011} & 96000 \\
\texttt{0100} & 8000 &
\texttt{1100} & (\textbf{read} 8 unsigned bits) $\times$ 1000 \\
\texttt{0101} & 16000 &
\texttt{1101} & \textbf{read} 16 unsigned bits \\
\texttt{0110} & 22050 &
\texttt{1110} & (\textbf{read} 16 unsigned bits) $\times$ 10 \\
\texttt{0111} & 24000 &
\texttt{1111} & invalid \\
\end{tabular}
}

\subsubsection{Decoding Bits per Sample}
{\relsize{-1}
\begin{tabular}{rl||rl}
encoded & bits-per-sample &
encoded & bits-per-sample \\
\hline
\texttt{000} & from STREAMINFO &
\texttt{100} & 16 \\
\texttt{001} & 8 &
\texttt{101} & 20 \\
\texttt{010} & 12 &
\texttt{110} & 24 \\
\texttt{011} & invalid &
\texttt{111} & invalid \\
\end{tabular}
}

\subsubsection{Decoding Channel Count and Assignment}
{\relsize{-1}
\begin{tabular}{rrl}
& channel & \\
encoded & count & channel assignment \\
\hline
\texttt{0000} & 1 & front Center \\
\texttt{0001} & 2 & front Left, front Right \\
\texttt{0010} & 3 & front Left, front Right, front Center \\
\texttt{0011} & 4 & front Left, front Right, back Left, back Right \\
\texttt{0100} & 5 & fL, fR, fC, back/surround left, back/surround right \\
\texttt{0101} & 6 & fL, fR, fC, LFE, back/surround left, back/surround right \\
\texttt{0110} & 7 & undefined \\
\texttt{0111} & 8 & undefined \\
\texttt{1000} & 2 & front Left, Difference \\
\texttt{1001} & 2 & Difference, front Right \\
\texttt{1010} & 2 & Mid, Side \\
\texttt{1011} & & reserved \\
\texttt{1100} & & reserved \\
\texttt{1101} & & reserved \\
\texttt{1110} & & reserved \\
\texttt{1111} & & reserved \\
\end{tabular}
}

\subsubsection{Frame Header Decoding Example}
\begin{figure}[h]
\includegraphics{figures/flac/header-example.pdf}
\end{figure}
{\relsize{-1}
\begin{tabular}{rcr}
$sync~code$ & = & \texttt{0x3FFE} \\
$encoded~block~size$ & = & \texttt{1100b} \\
$encoded~sample~rate$ & = & \texttt{1001b} \\
$encoded~channels$ & = & \texttt{0001b} \\
$encoded~bps$ & = & \texttt{100b} \\
frame number & = & 0 \\
block size & = & 4096 samples \\
sample rate & = & 44100Hz \\
bits per sample & = & 16 \\
channel count & = & 2 \\
channel assignment & = & front left, front right
\end{tabular}
}
\subsubsection{Calculating Frame Header CRC-8}
Given a header byte and previous CRC-8 checksum,
or 0 as an initial value:
\begin{equation*}
\text{checksum}_i = \text{CRC8}(byte\xor\text{checksum}_{i - 1})
\end{equation*}
\begin{table}[h]
{\relsize{-3}\ttfamily
\begin{tabular}{|r||r|r|r|r|r|r|r|r|r|r|r|r|r|r|r|r|r|}
\hline
 & 0x?0 & 0x?1 & 0x?2 & 0x?3 & 0x?4 & 0x?5 & 0x?6 & 0x?7 & 0x?8 & 0x?9 & 0x?A & 0x?B & 0x?C & 0x?D & 0x?E & 0x?F \\
\hline
0x0? & 0x00 & 0x07 & 0x0E & 0x09 & 0x1C & 0x1B & 0x12 & 0x15 & 0x38 & 0x3F & 0x36 & 0x31 & 0x24 & 0x23 & 0x2A & 0x2D \\
0x1? & 0x70 & 0x77 & 0x7E & 0x79 & 0x6C & 0x6B & 0x62 & 0x65 & 0x48 & 0x4F & 0x46 & 0x41 & 0x54 & 0x53 & 0x5A & 0x5D \\
0x2? & 0xE0 & 0xE7 & 0xEE & 0xE9 & 0xFC & 0xFB & 0xF2 & 0xF5 & 0xD8 & 0xDF & 0xD6 & 0xD1 & 0xC4 & 0xC3 & 0xCA & 0xCD \\
0x3? & 0x90 & 0x97 & 0x9E & 0x99 & 0x8C & 0x8B & 0x82 & 0x85 & 0xA8 & 0xAF & 0xA6 & 0xA1 & 0xB4 & 0xB3 & 0xBA & 0xBD \\
0x4? & 0xC7 & 0xC0 & 0xC9 & 0xCE & 0xDB & 0xDC & 0xD5 & 0xD2 & 0xFF & 0xF8 & 0xF1 & 0xF6 & 0xE3 & 0xE4 & 0xED & 0xEA \\
0x5? & 0xB7 & 0xB0 & 0xB9 & 0xBE & 0xAB & 0xAC & 0xA5 & 0xA2 & 0x8F & 0x88 & 0x81 & 0x86 & 0x93 & 0x94 & 0x9D & 0x9A \\
0x6? & 0x27 & 0x20 & 0x29 & 0x2E & 0x3B & 0x3C & 0x35 & 0x32 & 0x1F & 0x18 & 0x11 & 0x16 & 0x03 & 0x04 & 0x0D & 0x0A \\
0x7? & 0x57 & 0x50 & 0x59 & 0x5E & 0x4B & 0x4C & 0x45 & 0x42 & 0x6F & 0x68 & 0x61 & 0x66 & 0x73 & 0x74 & 0x7D & 0x7A \\
0x8? & 0x89 & 0x8E & 0x87 & 0x80 & 0x95 & 0x92 & 0x9B & 0x9C & 0xB1 & 0xB6 & 0xBF & 0xB8 & 0xAD & 0xAA & 0xA3 & 0xA4 \\
0x9? & 0xF9 & 0xFE & 0xF7 & 0xF0 & 0xE5 & 0xE2 & 0xEB & 0xEC & 0xC1 & 0xC6 & 0xCF & 0xC8 & 0xDD & 0xDA & 0xD3 & 0xD4 \\
0xA? & 0x69 & 0x6E & 0x67 & 0x60 & 0x75 & 0x72 & 0x7B & 0x7C & 0x51 & 0x56 & 0x5F & 0x58 & 0x4D & 0x4A & 0x43 & 0x44 \\
0xB? & 0x19 & 0x1E & 0x17 & 0x10 & 0x05 & 0x02 & 0x0B & 0x0C & 0x21 & 0x26 & 0x2F & 0x28 & 0x3D & 0x3A & 0x33 & 0x34 \\
0xC? & 0x4E & 0x49 & 0x40 & 0x47 & 0x52 & 0x55 & 0x5C & 0x5B & 0x76 & 0x71 & 0x78 & 0x7F & 0x6A & 0x6D & 0x64 & 0x63 \\
0xD? & 0x3E & 0x39 & 0x30 & 0x37 & 0x22 & 0x25 & 0x2C & 0x2B & 0x06 & 0x01 & 0x08 & 0x0F & 0x1A & 0x1D & 0x14 & 0x13 \\
0xE? & 0xAE & 0xA9 & 0xA0 & 0xA7 & 0xB2 & 0xB5 & 0xBC & 0xBB & 0x96 & 0x91 & 0x98 & 0x9F & 0x8A & 0x8D & 0x84 & 0x83 \\
0xF? & 0xDE & 0xD9 & 0xD0 & 0xD7 & 0xC2 & 0xC5 & 0xCC & 0xCB & 0xE6 & 0xE1 & 0xE8 & 0xEF & 0xFA & 0xFD & 0xF4 & 0xF3 \\
\hline
\end{tabular}
}
\end{table}
\begin{align*}
\text{checksum}_0 = \text{CRC8}(\texttt{FF}\xor\texttt{00}) = \texttt{F3} & &
\text{checksum}_3 = \text{CRC8}(\texttt{18}\xor\texttt{E6}) = \texttt{F4} \\
\text{checksum}_1 = \text{CRC8}(\texttt{F8}\xor\texttt{F3}) = \texttt{31} & &
\text{checksum}_4 = \text{CRC8}(\texttt{00}\xor\texttt{F4}) = \texttt{C2} \\
\text{checksum}_2 = \text{CRC8}(\texttt{C9}\xor\texttt{31}) = \texttt{E6} & &
\text{checksum}_5 = \text{CRC8}(\texttt{C2}\xor\texttt{C2}) = \texttt{00} \\
\end{align*}
Note that the final checksum (including the CRC-8 byte itself)
should always be 0.


\clearpage

\subsection{Decoding a FLAC Subframe}
\ALGORITHM{the frame's block size and bits per sample, the subframe's channel assignment and the FLAC file stream}{decoded signed PCM samples}
\SetKwData{DIFFERENCE}{Difference}
\SetKwData{SIDE}{Side}
\SetKwData{ORDER}{order}
\SetKwData{SAMPLES}{subframe samples}
\SKIP 1 bit\;
$type\_order \leftarrow$ \READ 6 unsigned bits\;
\eIf(\tcc*[f]{account for wasted bits}){(\READ 1 unsigned bit) $ = 1$}{
  $wasted~BPS \leftarrow$ (\UNARY stop bit 1) + 1\;
  subframe's bits per sample $\leftarrow \text{frame header's bits per sample} - wasted~BPS$\;
}{
  $wasted~BPS \leftarrow 0$\;
  subframe's bits per sample $\leftarrow \text{frame header's bits per sample}$\;
}
\If{subframe's channel assignment is \DIFFERENCE or \SIDE}{
  subframe's bits per sample $\leftarrow$ subframe's bits per sample + 1
}
\uIf{$type\_order = 0$}{
  \SAMPLES $\leftarrow$ decode CONSTANT subframe
}
\uElseIf{$type\_order = 1$} {
  \SAMPLES $\leftarrow$ decode VERBATIM subframe\;
}
\uElseIf{$8 \leq type\_order \leq 12$} {
  \SAMPLES $\leftarrow$ decode FIXED subframe with \ORDER ($type\_order - 8$)\;
}
\uElseIf{$32 \leq type\_order \leq 63$} {
  \SAMPLES $\leftarrow$ decode LPC subframe with \ORDER ($type\_order - 31$)\;
}
\Else {
  undefined subframe type error\;
}
\If(\tcc*[f]{prepend any wasted bits to each sample}){$wasted~BPS > 0$}{
  \ForEach{sample \IN \SAMPLES}{
    $sample \leftarrow sample \times 2 ^ {wasted~BPS}$\;
  }
}
\Return \SAMPLES\;
\EALGORITHM
\begin{figure}[h]
\includegraphics{figures/flac/subframes.pdf}
\end{figure}

\clearpage

\subsubsection{Decoding CONSTANT Subframe}
\ALGORITHM{the frame's block size, the subframe's bits per sample and the FLAC file stream}{decoded signed PCM samples}
\SetKwData{BPS}{bits per sample}
constant $\leftarrow$ \READ (\BPS) signed bits\;
\For{i = 0 \emph{\KwTo}block size}{
  $sample_i \leftarrow $ constant\;
}
\Return samples\;
\EALGORITHM

\subsubsection{Decoding VERBATIM Subframe}
\ALGORITHM{the frame's block size, the subframe's bits per sample and the FLAC file stream}{decoded signed PCM samples}
\SetKwData{BPS}{bits per sample}
\For{i = 0 \emph{\KwTo}block size}{
  $sample_i \leftarrow $ \READ (\BPS) signed bits\;
}
\Return samples\;
\EALGORITHM
\begin{figure}[h]
\includegraphics{figures/flac/verbatim.pdf}
\end{figure}

\clearpage

\subsubsection{Decoding FIXED Subframe}
\ALGORITHM{the frame's block size and predictor order, the subframe's bits per sample and the FLAC file stream}{decoded signed PCM samples}
\SetKwData{BPS}{bits per sample}
\SetKwData{ORDER}{order}
\SetKwData{RESIDUALS}{residuals}
\SetKwData{BLOCKSIZE}{block size}
\For(\tcc*[f]{warm-up samples}){i = 0 \emph{\KwTo}\ORDER}{
  $sample_i \leftarrow $ \READ (\BPS) signed bits\;
}
\RESIDUALS $\leftarrow$ read residual block with frame's \BLOCKSIZE and subframe's \ORDER\;
\Switch{\ORDER}{
  \uCase{0} {
    \For{i = 0 \emph{\KwTo}\BLOCKSIZE}{
      $sample_i \leftarrow residual_i$
    }
  }
  \uCase{1} {
    \For{i = 1 \emph{\KwTo}\BLOCKSIZE}{
      $sample_i \leftarrow sample_{i - 1} + residual_{i - 1}$
    }
  }
  \uCase{2} {
    \For{i = 2 \emph{\KwTo}\BLOCKSIZE}{
      $sample_i \leftarrow (2 \times sample_{i - 1}) - sample_{i - 2} + residual_{i - 2}$
    }
  }
  \uCase{3} {
    \For{i = 3 \emph{\KwTo}\BLOCKSIZE}{
      $sample_i \leftarrow (3 \times sample_{i - 1}) - (3 \times sample_{i - 2}) + sample_{i - 3} + residual_{i - 3}$
    }
  }
  \Case{4} {
    \For{i = 4 \emph{\KwTo}\BLOCKSIZE}{
      $sample_i \leftarrow (4 \times sample_{i - 1}) - (6 \times sample_{i - 2}) + (4 \times sample_{i - 3}) - sample_{i - 4} + residual_{i - 4}$
    }
  }
}
\Return samples\;
\EALGORITHM
\begin{figure}[h]
\includegraphics{figures/flac/fixed.pdf}
\end{figure}

\clearpage

\subsubsection{FIXED Subframe Decoding Example}

Given the subframe bytes of a 16 bits per sample stream\footnote{Decoding the residual block is explained on page \pageref{residual_decoding_example}}:
\begin{figure}[h]
\includegraphics{figures/flac/fixed-parse.pdf}
\end{figure}
\par
\noindent
\begin{tabular}{rcl|rcr|rcr}
subframe type & $\leftarrow$ & FIXED &
$\text{residual}_0$ & $\leftarrow$ & -2 &
$\text{residual}_5$ & $\leftarrow$ & -5 \\
subframe order & $\leftarrow$ & 1 &
$\text{residual}_1$ & $\leftarrow$ & 3 &
$\text{residual}_6$ & $\leftarrow$ & 4 \\
wasted BPS & $\leftarrow$ & 0 &
$\text{residual}_2$ & $\leftarrow$ & -1 &
$\text{residual}_7$ & $\leftarrow$ & -2 \\
$\text{sample}_0$ & $\leftarrow$ & 37 &
$\text{residual}_3$ & $\leftarrow$ & -5 &
$\text{residual}_8$ & $\leftarrow$ & -3 \\
& & &
$\text{residual}_4$ & $\leftarrow$ & 1 &
$\text{residual}_9$ & $\leftarrow$ & 1 \\
\end{tabular}
\par
\noindent
our calculated subframe samples are:
\par
\noindent
\begin{tabular}{rcr}
$\text{sample}_0$ & $\leftarrow$ & \textbf{37} \\
$\text{sample}_1$ & $\leftarrow$ & 37 - 2 = \textbf{35} \\
$\text{sample}_2$ & $\leftarrow$ & 35 + 3 = \textbf{38} \\
$\text{sample}_3$ & $\leftarrow$ & 38 - 1 = \textbf{37} \\
$\text{sample}_4$ & $\leftarrow$ & 37 - 5 = \textbf{32} \\
$\text{sample}_5$ & $\leftarrow$ & 32 + 1 = \textbf{33} \\
$\text{sample}_6$ & $\leftarrow$ & 33 - 5 = \textbf{27} \\
$\text{sample}_7$ & $\leftarrow$ & 27 + 4 = \textbf{31} \\
$\text{sample}_8$ & $\leftarrow$ & 31 - 2 = \textbf{29} \\
$\text{sample}_9$ & $\leftarrow$ & 29 - 3 = \textbf{26} \\
$\text{sample}_{10}$ & $\leftarrow$ & 26 + 1 = \textbf{27} \\
\end{tabular}

\clearpage

\subsubsection{Decoding LPC Subframe}
\ALGORITHM{the frame's block size and predictor order, the subframe's bits per sample and the FLAC file stream}{decoded signed PCM samples}
\SetKwData{BPS}{bits per sample}
\SetKwData{ORDER}{order}
\SetKwData{RESIDUALS}{residuals}
\SetKwData{BLOCKSIZE}{block size}
\SetKwFunction{MAX}{max}
\For(\tcc*[f]{warm-up samples}){i = 0 \emph{\KwTo}\ORDER}{
  $sample_i \leftarrow $ \READ (\BPS) signed bits\;
}
QLP precision $\leftarrow$ (\READ 4 unsigned bits) + 1\;
QLP shift needed $\leftarrow$ \MAX(\READ 5 signed bits, 0)\footnote{negative shifts are noops in the decoder}\;
\For{i = 0 \emph{\KwTo}\ORDER}{
  $QLP~coefficient_i \leftarrow$ \READ (QLP precision) signed bits\;
}
\RESIDUALS $\leftarrow$ read residual block with frame's \BLOCKSIZE and subframe's \ORDER\;
\For{i = \ORDER \emph{\KwTo}\BLOCKSIZE}{
  $sample_i \leftarrow \left\lfloor \frac{\overset{\ORDER - 1}{\underset{j = 0}{\sum}}
    QLP~coefficient_j \times sample_{i - j - 1} } {2 ^ \text{QLP shift needed}}\right\rfloor + residual_{i - \ORDER}$\;
}
\Return samples\;
\EALGORITHM
\begin{figure}[h]
\includegraphics{figures/flac/lpc.pdf}
\end{figure}

\subsubsection{LPC Subframe Decoding Example}

\begin{tabular}{rcl|rcr}
subframe type & $\leftarrow$ & LPC &
$\text{residual}_0$ & $\leftarrow$ & 4 \\
subframe order & $\leftarrow$ & 3 &
$\text{residual}_1$ & $\leftarrow$ & 0 \\
wasted BPS & $\leftarrow$ & 0 &
$\text{residual}_2$ & $\leftarrow$ & 1 \\
$\text{sample}_0$ & $\leftarrow$ & 43 &
$\text{residual}_3$ & $\leftarrow$ & -2 \\
$\text{sample}_1$ & $\leftarrow$ & 48 &
$\text{residual}_4$ & $\leftarrow$ & -3 \\
$\text{sample}_2$ & $\leftarrow$ & 50 & & & \\
QLP precision & $\leftarrow$ & 12 & & & \\
QLP shift needed & $\leftarrow$ & 10 & & & \\
$\text{QLP coefficient}_0$ & $\leftarrow$ & 1451 & & & \\
$\text{QLP coefficient}_1$ & $\leftarrow$ & -323 & & & \\
$\text{QLP coefficient}_2$ & $\leftarrow$ & -110 & & & \\
\end{tabular}

\clearpage

\begin{figure}[h]
\includegraphics{figures/flac/lpc-parse.pdf}
\end{figure}
\begin{align*}
\text{sample}_0 &\leftarrow 43 \\
\text{sample}_1 &\leftarrow 48 \\
\text{sample}_2 &\leftarrow 50 \\
\text{sample}_3 &\leftarrow \left\lfloor\frac{(1451 \times 50) + (-323 \times 48) + (-110 \times 43)}{2 ^ {10}}\right\rfloor + 4 = \left\lfloor\frac{52316}{1024}\right\rfloor + 4 = \textbf{55} \\
\text{sample}_4 &\leftarrow \left\lfloor\frac{(1451 \times 55) + (-323 \times 50) + (-110 \times 48)}{2 ^ {10}}\right\rfloor + 0 = \left\lfloor\frac{58375}{1024}\right\rfloor + 0 = \textbf{57} \\
\text{sample}_5 &\leftarrow \left\lfloor\frac{(1451 \times 57) + (-323 \times 55) + (-110 \times 50)}{2 ^ {10}}\right\rfloor + 1 = \left\lfloor\frac{59442}{1024}\right\rfloor + 1 = \textbf{59} \\
\text{sample}_6 &\leftarrow \left\lfloor\frac{(1451 \times 59) + (-323 \times 57) + (-110 \times 55)}{2 ^ {10}}\right\rfloor - 2 = \left\lfloor\frac{61148}{1024}\right\rfloor - 2 = \textbf{57} \\
\text{sample}_7 &\leftarrow \left\lfloor\frac{(1451 \times 57) + (-323 \times 59) + (-110 \times 57)}{2 ^ {10}}\right\rfloor - 3 = \left\lfloor\frac{57380}{1024}\right\rfloor - 3 = \textbf{53} \\
\end{align*}

\clearpage

\subsubsection{Decoding Residual Block}
\label{residual_decoding_example}
\ALGORITHM{the frame's block size and predictor order, the FLAC file stream}{decoded signed residual values}
\SetKwData{ORDER}{predictor order}
\SetKwData{RESIDUALS}{residuals}
coding method $\leftarrow$ \READ 2 unsigned bits\;
partition order $\leftarrow$ \READ 4 unsigned bits\;
\For{p = 0 \emph{\KwTo}$2 ^ {\text{partition order}}$}{
  \eIf{p = 0}{
    $partition~residual~count \leftarrow \lfloor\text{block size} \div 2 ^ {partition~order}\rfloor - \text{\ORDER}$
  }{
    $partition~residual~count \leftarrow \lfloor\text{block size} \div 2 ^ {partition~order}\rfloor$
  }
  partition's residuals $\leftarrow$ decode residual partition\;
  append partition's residuals to block's residual list\;
}
\Return block's residual list\;
\EALGORITHM

\subsubsection{Decoding Residual Partition}
{\relsize{-1}
\ALGORITHM{the residual block's coding method, the partition's residual count, the FLAC file stream}{decoded signed residual values}
  \uIf{coding method = 0}{
    Rice parameter $\leftarrow$ \READ 4 unsigned bits\;
    \If(\tcc*[f]{handle unencoded residuals}){Rice parameter = 15}{
      escape code $\leftarrow$ \READ 5 unsigned bits\;
      \For{i = 0 \emph{\KwTo}partition residual count}{
        $residual_i \leftarrow$ \READ (escape code) signed bits\;
      }
      \Return partition's residual list\;
    }
  }
  \uElseIf{coding method = 1}{
    Rice parameter $\leftarrow$ \READ 5 unsigned bits\;
    \If(\tcc*[f]{handle unencoded residuals}){Rice parameter = 31}{
      escape code $\leftarrow$ \READ 5 unsigned bits\;
      \For{i = 0 \emph{\KwTo}partition residual count}{
        $residual_i \leftarrow$ \READ (escape code) signed bits\;
      }
      \Return partition's residual list\;
    }
  }
  \Else{
    undefined residual coding method error\;
  }

  \For{i = 0 \emph{\KwTo}partition residual count} {
    MSB $\leftarrow$ \UNARY with stop bit 1\;
    LSB $\leftarrow$ \READ (Rice parameter) unsigned bits\;
    unsigned $\leftarrow \text{MSB} \times 2 ^ \text{Rice parameter} + \text{LSB}$\;
    \eIf{unsigned is even}{
      $residual_i \leftarrow unsigned \div 2$\;
    }{
      $residual_i \leftarrow -\lfloor unsigned \div 2 \rfloor - 1$\;
    }
  }
  \Return partition's residual list\;
\EALGORITHM
}

\clearpage

\begin{figure}[h]
\includegraphics{figures/flac/residual.pdf}
\end{figure}
\par
\noindent
As an example, we'll decode 10 residual values from the following bytes:
\begin{figure}[h]
\includegraphics{figures/flac/residual-parse.pdf}
\end{figure}
\par
\noindent
{\relsize{-1}
\begin{tabular}{rcl|rcl}
coding method & $\leftarrow$ & 0 \\
paritition order & $\leftarrow$ & 0 \\
Rice parameter & $\leftarrow$ & 2 \\
\hline
$\text{MSB}_0$ & $\leftarrow$ & 0 &
$\text{MSB}_5$ & $\leftarrow$ & 2 \\
$\text{LSB}_0$ & $\leftarrow$ & 3 &
$\text{LSB}_5$ & $\leftarrow$ & 1 \\
$\text{unsigned}_0$ & $\leftarrow$ & $0 \times 2 ^ 2 + 3 = 3$ &
$\text{unsigned}_5$ & $\leftarrow$ & $2 \times 2 ^ 2 + 1 = 9$ \\
$\text{residual}_0$ & $\leftarrow$ & $-\lfloor 3 \div 2\rfloor - 1 = -2$ &
$\text{residual}_5$ & $\leftarrow$ & $-\lfloor 9 \div 2\rfloor - 1 = -5$ \\
\hline
$\text{MSB}_1$ & $\leftarrow$ & 1 &
$\text{MSB}_6$ & $\leftarrow$ & 2 \\
$\text{LSB}_1$ & $\leftarrow$ & 2 &
$\text{LSB}_6$ & $\leftarrow$ & 0 \\
$\text{unsigned}_1$ & $\leftarrow$ & $1 \times 2 ^ 2 + 2 = 6$ &
$\text{unsigned}_6$ & $\leftarrow$ & $2 \times 2 ^ 2 + 0 = 8$ \\
$\text{residual}_1$ & $\leftarrow$ & $6 \div 2 = 3$ &
$\text{residual}_6$ & $\leftarrow$ & $8 \div 2 = 4$ \\
\hline
$\text{MSB}_2$ & $\leftarrow$ & 0 &
$\text{MSB}_7$ & $\leftarrow$ & 0 \\
$\text{LSB}_2$ & $\leftarrow$ & 1 &
$\text{LSB}_7$ & $\leftarrow$ & 3 \\
$\text{unsigned}_2$ & $\leftarrow$ & $0 \times 2 ^ 2 + 1 = 1$ &
$\text{unsigned}_7$ & $\leftarrow$ & $0 \times 2 ^ 2 + 3 = 3$ \\
$\text{residual}_2$ & $\leftarrow$ & $-\lfloor 1 \div 2\rfloor - 1 = -1$ &
$\text{residual}_7$ & $\leftarrow$ & $-\lfloor 3 \div 2\rfloor - 1 = -2$ \\
\hline
$\text{MSB}_3$ & $\leftarrow$ & 2 &
$\text{MSB}_8$ & $\leftarrow$ & 1 \\
$\text{LSB}_3$ & $\leftarrow$ & 1 &
$\text{LSB}_8$ & $\leftarrow$ & 1 \\
$\text{unsigned}_3$ & $\leftarrow$ & $2 \times 2 ^ 2 + 1 = 9$ &
$\text{unsigned}_8$ & $\leftarrow$ & $1 \times 2 ^ 2 + 1 = 5$ \\
$\text{residual}_3$ & $\leftarrow$ & $-\lfloor 9 \div 2\rfloor - 1 = -5$ &
$\text{residual}_8$ & $\leftarrow$ & $-\lfloor 5 \div 2\rfloor - 1 = -3$ \\
\hline
$\text{MSB}_4$ & $\leftarrow$ & 0 &
$\text{MSB}_9$ & $\leftarrow$ & 0 \\
$\text{LSB}_4$ & $\leftarrow$ & 2 &
$\text{LSB}_9$ & $\leftarrow$ & 2 \\
$\text{unsigned}_4$ & $\leftarrow$ & $0 \times 2 ^ 2 + 2 = 2$ &
$\text{unsigned}_9$ & $\leftarrow$ & $0 \times 2 ^ 2 + 2 = 2$ \\
$\text{residual}_4$ & $\leftarrow$ & $2 \div 2 = 1$ &
$\text{residual}_9$ & $\leftarrow$ & $2 \div 2 = 1$ \\
\end{tabular}
}
\par
\noindent
\vskip .25in
for a final set of residuals: -2, 3, -1, -5, 1, -5, 4, -2, -3 and 1.

\clearpage

\subsection{Calculating Frame CRC-16}

CRC-16 is used to checksum the entire FLAC frame, including the header
and any padding bits after the final subframe.
Given a byte of input and the previous CRC-16 checksum,
or 0 as an initial value, the current checksum can be calculated as follows:
\begin{equation}
\text{checksum}_i = \text{CRC16}(byte\xor(\text{checksum}_{i - 1} \gg 8 ))\xor(\text{checksum}_{i - 1} \ll 8)
\end{equation}
\par
\noindent
and the checksum is always truncated to 16-bits.
\begin{table}[h]
{\relsize{-3}\ttfamily
\begin{tabular}{|r||r|r|r|r|r|r|r|r|r|r|r|r|r|r|r|r|}
\hline
 & 0x?0 & 0x?1 & 0x?2 & 0x?3 & 0x?4 & 0x?5 & 0x?6 & 0x?7 & 0x?8 & 0x?9 & 0x?A & 0x?B & 0x?C & 0x?D & 0x?E & 0x?F \\
\hline
0x0? & 0000 & 8005 & 800f & 000a & 801b & 001e & 0014 & 8011 & 8033 & 0036 & 003c & 8039 & 0028 & 802d & 8027 & 0022 \\
0x1? & 8063 & 0066 & 006c & 8069 & 0078 & 807d & 8077 & 0072 & 0050 & 8055 & 805f & 005a & 804b & 004e & 0044 & 8041 \\
0x2? & 80c3 & 00c6 & 00cc & 80c9 & 00d8 & 80dd & 80d7 & 00d2 & 00f0 & 80f5 & 80ff & 00fa & 80eb & 00ee & 00e4 & 80e1 \\
0x3? & 00a0 & 80a5 & 80af & 00aa & 80bb & 00be & 00b4 & 80b1 & 8093 & 0096 & 009c & 8099 & 0088 & 808d & 8087 & 0082 \\
0x4? & 8183 & 0186 & 018c & 8189 & 0198 & 819d & 8197 & 0192 & 01b0 & 81b5 & 81bf & 01ba & 81ab & 01ae & 01a4 & 81a1 \\
0x5? & 01e0 & 81e5 & 81ef & 01ea & 81fb & 01fe & 01f4 & 81f1 & 81d3 & 01d6 & 01dc & 81d9 & 01c8 & 81cd & 81c7 & 01c2 \\
0x6? & 0140 & 8145 & 814f & 014a & 815b & 015e & 0154 & 8151 & 8173 & 0176 & 017c & 8179 & 0168 & 816d & 8167 & 0162 \\
0x7? & 8123 & 0126 & 012c & 8129 & 0138 & 813d & 8137 & 0132 & 0110 & 8115 & 811f & 011a & 810b & 010e & 0104 & 8101 \\
0x8? & 8303 & 0306 & 030c & 8309 & 0318 & 831d & 8317 & 0312 & 0330 & 8335 & 833f & 033a & 832b & 032e & 0324 & 8321 \\
0x9? & 0360 & 8365 & 836f & 036a & 837b & 037e & 0374 & 8371 & 8353 & 0356 & 035c & 8359 & 0348 & 834d & 8347 & 0342 \\
0xA? & 03c0 & 83c5 & 83cf & 03ca & 83db & 03de & 03d4 & 83d1 & 83f3 & 03f6 & 03fc & 83f9 & 03e8 & 83ed & 83e7 & 03e2 \\
0xB? & 83a3 & 03a6 & 03ac & 83a9 & 03b8 & 83bd & 83b7 & 03b2 & 0390 & 8395 & 839f & 039a & 838b & 038e & 0384 & 8381 \\
0xC? & 0280 & 8285 & 828f & 028a & 829b & 029e & 0294 & 8291 & 82b3 & 02b6 & 02bc & 82b9 & 02a8 & 82ad & 82a7 & 02a2 \\
0xD? & 82e3 & 02e6 & 02ec & 82e9 & 02f8 & 82fd & 82f7 & 02f2 & 02d0 & 82d5 & 82df & 02da & 82cb & 02ce & 02c4 & 82c1 \\
0xE? & 8243 & 0246 & 024c & 8249 & 0258 & 825d & 8257 & 0252 & 0270 & 8275 & 827f & 027a & 826b & 026e & 0264 & 8261 \\
0xF? & 0220 & 8225 & 822f & 022a & 823b & 023e & 0234 & 8231 & 8213 & 0216 & 021c & 8219 & 0208 & 820d & 8207 & 0202 \\
\hline
\end{tabular}
}
\end{table}
\par
\noindent
For example, given the frame bytes:
\texttt{FF F8 CC 1C 00 C0 EB 00 00 00 00 00 00 00 00},
the frame's CRC-16 can be calculated:
{\relsize{-2}
\begin{align*}
\CRCSIXTEEN{0}{0xFF}{0x0000}{0xFF}{0x0000}{0x0202} \\
\CRCSIXTEEN{1}{0xF8}{0x0202}{0xFA}{0x0200}{0x001C} \\
\CRCSIXTEEN{2}{0xCC}{0x001C}{0xCC}{0x1C00}{0x1EA8} \\
\CRCSIXTEEN{3}{0x1C}{0x1EA8}{0x02}{0xA800}{0x280F} \\
\CRCSIXTEEN{4}{0x00}{0x280F}{0x28}{0x0F00}{0x0FF0} \\
\CRCSIXTEEN{5}{0xC0}{0x0FF0}{0xCF}{0xF000}{0xF2A2} \\
\CRCSIXTEEN{6}{0xEB}{0xF2A2}{0x19}{0xA200}{0x2255} \\
\CRCSIXTEEN{7}{0x00}{0x2255}{0x22}{0x5500}{0x55CC} \\
\CRCSIXTEEN{8}{0x00}{0x55CC}{0x55}{0xCC00}{0xCDFE} \\
\CRCSIXTEEN{9}{0x00}{0xCDFE}{0xCD}{0xFE00}{0x7CAD} \\
\CRCSIXTEEN{10}{0x00}{0x7CAD}{0x7C}{0xAD00}{0x2C0B} \\
\CRCSIXTEEN{11}{0x00}{0x2C0B}{0x2C}{0x0B00}{0x8BEB} \\
\CRCSIXTEEN{12}{0x00}{0x8BEB}{0x8B}{0xEB00}{0xE83A} \\
\CRCSIXTEEN{13}{0x00}{0xE83A}{0xE8}{0x3A00}{0x3870} \\
\CRCSIXTEEN{14}{0x00}{0x3870}{0x38}{0x7000}{0xF093} \\
\intertext{Thus, the next two bytes after the final subframe should be
\texttt{0xF0} and \texttt{0x93}.
Again, when the checksum bytes are run through the checksumming procedure:}
\CRCSIXTEEN{15}{0xF0}{0xF093}{0x00}{0x9300}{0x9300} \\
\CRCSIXTEEN{16}{0x93}{0x9300}{0x00}{0x0000}{0x0000}
\end{align*}
the result will also always be 0, just as in the CRC-8.
}

\clearpage

\subsection{Recombining Subframes}
\label{flac_recombining_subframes}
\ALGORITHM{the frame's block size and channel assignment, a set of decoded subframe samples\footnote{$subframe_{x~y}$ indicates the $y$th sample in subframe $x$}}{a single list of signed PCM frames}
\uIf(\tcc*[f]{independent}){$0 \leq encoded~channels \leq 7$}{
 channel count $\leftarrow encoded~channels + 1$\;
  \For{i = 0 \emph{\KwTo}block size}{
    \For{j = 0 \emph{\KwTo}channel count}{
      $sample_{i \times channel~count + j} \leftarrow subframe_{j~i}$\;
    }
  }
}
\uElseIf(\tcc*[f]{left-difference}){$encoded~channels = 8$}{
  \For{i = 0 \emph{\KwTo}block size}{
    \begin{tabular}{ll}
      $sample_{i \times 2}$ & $\leftarrow subframe_{0~i}$ \\
      $sample_{i \times 2 + 1}$ & $\leftarrow subframe_{0~i} - subframe_{1~i}$ \\
    \end{tabular}
  }
}
\uElseIf(\tcc*[f]{difference-right}){$encoded~channels = 9$}{
  \For{i = 0 \emph{\KwTo}block size}{
    \begin{tabular}{ll}
      $sample_{i \times 2}$ & $\leftarrow subframe_{0~i} + subframe_{1~i}$ \\
      $sample_{i \times 2 + 1}$ & $\leftarrow subframe_{1~i}$ \\
    \end{tabular}
  }
}
\ElseIf(\tcc*[f]{mid-side}){$encoded~channels = 10$}{
  \For{i = 0 \emph{\KwTo}block size}{
    \begin{tabular}{ll}
      $sample_{i \times 2}$ & $\leftarrow \lfloor(((subframe_{0~i} \times 2) + (subframe_{1~i} \bmod 2)) + subframe_{1~i}) \div 2\rfloor$ \\
      $sample_{i \times 2 + 1}$ & $\leftarrow \lfloor(((subframe_{0~i} \times 2) + (subframe_{1~i} \bmod 2)) - subframe_{1~i}) \div 2\rfloor$ \\
    \end{tabular}
  }
}
\Return samples\;
\EALGORITHM

\clearpage

\subsection{Updating Stream MD5 Sum}

\ALGORITHM{the frame's signed PCM samples\footnote{$sample_{x~y}$ indicates the $y$th sample in channel $x$}}{the stream's updated MD5 sum}
\For{i = 0 \emph{\KwTo}block size}{
  \For{j = 0 \emph{\KwTo}channel count}{
    bytes $\leftarrow sample_{j~i}$ as signed, little-endian bytes\;
    update stream's MD5 sum with bytes\;
  }
}
\EALGORITHM
\vskip .25in
\par
\noindent
For example, given a 16 bits per sample stream with the signed sample values:
\begin{table}[h]
\begin{tabular}{r|rr}
& $channel_0$ & $channel_1$ \\
\hline
$sample_0$ & 1 & -1 \\
$sample_1$ & 2 & -2 \\
$sample_2$ & 3 & -3 \\
\end{tabular}
\end{table}
\par
\noindent
are translated to the bytes:
\begin{table}[h]
\begin{tabular}{r|rr}
& $channel_0$ & $channel_1$ \\
\hline
$sample_0$ & \texttt{01 00} & \texttt{FF FF} \\
$sample_1$ & \texttt{02 00} & \texttt{FE FF} \\
$sample_2$ & \texttt{03 00} & \texttt{FD FF} \\
\end{tabular}
\end{table}
\par
\noindent
and combined as:
\vskip .15in
\par
\noindent
\texttt{01 00 FF FF 02 00 FE FF 03 00 FD FF}
\vskip .15in
\par
\noindent
whose MD5 sum is:
\vskip .15in
\par
\noindent
\texttt{E7482f6462B27EE04EADC079291C79E9}

\clearpage

\section{FLAC Encoding}

The basic process for encoding a FLAC file is as follows:
\par
\noindent
\ALGORITHM{PCM frames, a default block size and various encoding parameters:
\newline
\begin{tabular}{rll}
parameter & possible values & typical values \\
\hline
block size & a positive number of PCM frames & 1152 or 4096 \\
maximum LPC order & integer between 0 and 32, inclusive & 0, 6, 8 or 12 \\
minimum partition order & integer between 0 and 16, inclusive & 0 \\
maximum partition order & integer between 0 and 16, inclusive & 3, 4, 5 or 6 \\
try mid-side & true or false & \\
try adaptive mid-side & true or false & \\
exhaustive model search & true or false & \\
\end{tabular}
}{an encoded FLAC file}
\SetKwData{BLOCKSIZE}{block size}
\WRITE \texttt{"fLaC"} in 4 bytes\;
write placeholder STREAMINFO metadata block\;
write PADDING metadata block\;
initialize stream's MD5 sum\;
\While{PCM frames remain}{
  take \BLOCKSIZE PCM frames from the input\;
  update the stream's MD5 sum with that PCM data\;
  encode a FLAC frame from PCM frames using the given encoding parameters\;
  update STREAMINFO's values from the FLAC frame\;
}
return to the start of the file and rewrite the STREAMINFO metadata block\;
\EALGORITHM
\begin{figure}[h]
\includegraphics{figures/flac/stream3.pdf}
\end{figure}
\par
\noindent
All of the fields in the FLAC stream are big-endian.

\clearpage

\subsection{Writing Placeholder Metadata Blocks}
\ALGORITHM{input stream's attributes, a default block size}{1 or more metadata blocks to the FLAC file stream}
\WRITE 0 as 1 unsigned bit\tcc*[r]{is last block}
\WRITE 0 as 7 unsigned bits\tcc*[r]{STREAMINFO type}
\WRITE 34 as 24 unsigned bits\tcc*[r]{STREAMINFO size}
\WRITE default block size as 16 unsigned bits\tcc*[r]{minimum block size}
\WRITE default block size as 16 unsigned bits\tcc*[r]{maximum block size}
\WRITE 0 as 24 unsigned bits\tcc*[r]{minimum frame size}
\WRITE 0 as 24 unsigned bits\tcc*[r]{maximum frame size}
\WRITE stream's sample rate as 20 unsigned bits\;
\WRITE (stream's channels - 1) as 3 unsigned bits\;
\WRITE (streams bits per sample - 1) as 5 unsigned bits\;
\WRITE 0 as 36 unsigned bits\tcc*[r]{total PCM frames}
\WRITE 0 as 16 bytes\tcc*[r]{stream's MD5 sum}
\BlankLine
\BlankLine
\WRITE 1 as 1 unsigned bit\tcc*[r]{is last block}
\WRITE 1 as 7 unsigned bits\tcc*[r]{PADDING type}
\WRITE 4096 as 24 unsigned bits\tcc*[r]{PADDING size}
\WRITE 0 as 4096 bytes\tcc*[r]{PADDING's data}
\EALGORITHM
\par
\noindent
PADDING can be some size other than 4096 bytes.
One simply wants to leave enough room for a VORBIS\_COMMENT block,
SEEKTABLE and so forth.
Other fields such as the minimum/maximum frame size
and the stream's final MD5 sum can't be known in advance;
we'll need to return to this block once encoding is finished
in order to populate them.
\begin{figure}[h]
\includegraphics{figures/flac/metadata.pdf}
\end{figure}


\clearpage

\subsection{Updating Stream MD5 Sum}
\ALGORITHM{the frame's signed PCM input samples\footnote{$sample_{x~y}$ indicates the $y$th sample in channel $x$}}{the stream's updated MD5 sum}
\For{i = 0 \emph{\KwTo}block size}{
  \For{j = 0 \emph{\KwTo}channel count}{
    bytes $\leftarrow sample_{j~i}$ as signed, little-endian bytes\;
    update stream's MD5 sum with bytes\;
  }
}
\EALGORITHM
\par
\noindent
For example, given a 16 bits per sample stream with the signed sample values:
\begin{table}[h]
\begin{tabular}{r|rr}
& $channel_0$ & $channel_1$ \\
\hline
$sample_0$ & 1 & -1 \\
$sample_1$ & 2 & -2 \\
$sample_2$ & 3 & -3 \\
\end{tabular}
\end{table}
\par
\noindent
are translated to the bytes:
\begin{table}[h]
\begin{tabular}{r|rr}
& $channel_0$ & $channel_1$ \\
\hline
$sample_0$ & \texttt{01 00} & \texttt{FF FF} \\
$sample_1$ & \texttt{02 00} & \texttt{FE FF} \\
$sample_2$ & \texttt{03 00} & \texttt{FD FF} \\
\end{tabular}
\end{table}
\par
\noindent
and combined as:
\vskip .15in
\par
\noindent
\texttt{01 00 FF FF 02 00 FE FF 03 00 FD FF}
\vskip .15in
\par
\noindent
whose MD5 sum is:
\vskip .15in
\par
\noindent
\texttt{E7482f6462B27EE04EADC079291C79E9}
\vskip .25in
\par
This process is identical to the MD5 sum calculation performed
during FLAC decoding, but performed in the opposite order.

\clearpage

\subsection{Encoding a FLAC Frame}
{\relsize{-1}
\ALGORITHM{up to ``block size'' number of PCM frames, encoding parameters}{a single FLAC frame}
\SetKw{AND}{and}
\SetKw{OR}{or}
\SetKwFunction{MIN}{min}
\SetKwFunction{BUILDSUBFRAME}{build subframe}
\SetKwFunction{CALCMIDSIDE}{calculate mid-side}
\eIf{channels = 2 \AND (try mid-side \OR try adaptive mid-side)}{
  $average$, $difference$ $\leftarrow$ \CALCMIDSIDE($channel_0$, $channel_1$)\;
  left subframe $\leftarrow$ \BUILDSUBFRAME($channel_0$, stream's bits per sample)\;
  right subframe $\leftarrow$ \BUILDSUBFRAME($channel_1$, stream's bits per sample)\;
  average subframe $\leftarrow$ \BUILDSUBFRAME($average$, stream's bits per sample)\;
  difference subframe $\leftarrow$ \BUILDSUBFRAME($difference$, stream's bits per sample + 1)\;
  left bits $\leftarrow$ left subframe length, in bits\;
  right bits $\leftarrow$ right subframe length, in bits\;
  avg bits $\leftarrow$ average subframe length, in bits\;
  diff bits $\leftarrow$ difference subframe length, in bits\;
  \BlankLine
  \uIf{try mid-side}{
    \uIf{left bits + right bits $<$ \MIN(left bits + diff bits, diff bits + right bits, avg bits + diff bits)}{
      write frame header with channel assignment \texttt{0x1}\;
      write left subframe\;
      write right subframe\;
    }
    \uElseIf{left bits $<$ \MIN(right bits, avg bits)}{
      write frame header with channel assignment \texttt{0x8}\;
      write left subframe\;
      write difference subframe\;
    }
    \uElseIf{right bits $<$ avg bits}{
      write frame header with channel assignment \texttt{0x9}\;
      write difference subframe\;
      write right subframe\;
    }
    \Else{
      write frame header with channel assignment \texttt{0xA}\;
      write average subframe\;
      write difference subframe\;
    }
  }\uElseIf{left bits + right bits $<$ avg bits + diff bits}{
    write frame header with channel assignment \texttt{0x1}\;
    write left subframe\;
    write right subframe\;
  }
  \Else{
    write frame header with channel assignment \texttt{0xA}\;
    write average subframe\;
    write difference subframe\;
  }
}(\tcc*[f]{store subframes independently}){
  write frame header with channel assigment $channels - 1$\;
  \ForEach{channel \IN channels}{
    subframe $\leftarrow$ \BUILDSUBFRAME($channel$, stream's bits per sample)\;
    write subframe\;
  }
}
byte align the stream\;
write frame's CRC-16 checksum\;
\EALGORITHM
}
\clearpage

\subsubsection{Calculating Mid-Side}
For each sample in $channel_0$ and $channel_1$:
\begin{align*}
average_i &\leftarrow \left\lfloor\frac{channel_{0~i} + channel_{1~i}}{2}\right\rfloor \\
difference_i &\leftarrow channel_{0~i} - channel_{1~i}
\intertext{For example, given the input samples:}
channel_{0~0} &\leftarrow 10 \\
channel_{1~0} &\leftarrow 15
\intertext{Our average and difference samples are:}
average_0 &\leftarrow \left\lfloor\frac{10 + 15}{2}\right\rfloor = 12 \\
difference_0 &\leftarrow 10 - 15 = -5
\intertext{Note that the $difference$ channel is identical for left-difference,
difference-right and mid-side channel assignments.
For example, when recombined from left-difference\footnotemark:}
sample_0 &\leftarrow 10 \\
sample_1 &\leftarrow 10 - (-5) = 15
\intertext{difference-right:}
sample_0 &\leftarrow -5 + 15 = 10 \\
sample_1 &\leftarrow 15
\intertext{and mid-side:}
sample_0 &\leftarrow \lfloor(((12 \times 2) + (-5 \bmod 2)) + -5) \div 2\rfloor  = \lfloor((24 + 1 - 5) \div 2\rfloor = 10 \\
sample_1 &\leftarrow \lfloor(((12 \times 2) + (-5 \bmod 2)) - -5) \div 2\rfloor =  \lfloor((24 + 1 + 5) \div 2\rfloor = 15
\end{align*}
\footnotetext{See the recombining subframes algorithms on page
\pageref{flac_recombining_subframes}.}

\clearpage

\subsubsection{Writing Frame Header}
\ALGORITHM{the frame's channel assignment, the input stream's parameters}{a FLAC frame header}
\SetKw{OR}{or}
\WRITE \texttt{0x3FFE} in 14 unsigned bits\tcc*[r]{sync code}
\WRITE 0 in 1 unsigned bit\;
\WRITE 0 in 1 unsigned bit\tcc*[r]{blocking strategy}
\WRITE $encoded~block~size$ in 4 unsigned bits\;
\WRITE $encoded~sample~rate$ in 4 unsigned bits\;
\WRITE frame's channel assignment in 4 unsigned bits\;
\WRITE $encoded~bps$ in 3 unsigned bits\;
\WRITE 0 in 1 unsigned bit\;
\WRITE frame number as a UTF-8 encoded value\;
\uIf{encoded block size = 6}{
  \WRITE (block size - 1) in 8 unsigned bits\;
}
\ElseIf{encoded block size = 7}{
  \WRITE (block size - 1) in 16 unsigned bits\;
}
\uIf{encoded sample rate = 12}{
  \WRITE ($\text{sample rate} \div 1000$) in 8 unsigned bits\;
}
\uElseIf{encoded sample rate = 13}{
  \WRITE sample rate in 16 unsigned bits\;
}
\ElseIf{encoded sample rate = 14}{
  \WRITE ($\text{sample rate} \div 10$) in 16 unsigned bits\;
}
\WRITE header CRC8 in 8 unsigned bits\;
\EALGORITHM

\subsubsection{Encoding Block Size}
{\relsize{-1}
\ALGORITHM{block size in samples}{encoded block size as 4 bit value}
\uIf{block size \IN}{
  \begin{tabular}{rr|rr|rr}
    block size & encoded & block size & encoded & block size & encoded \\
    \hline
    192 & 1 &
    1152 & 3 &
    4608 & 5 \\
    256 & 8 &
    2048 & 11 &
    8192 & 13 \\
    512 & 9 &
    2304 & 4 &
    16384 & 14 \\
    576 & 2 &
    4096 & 12 &
    32768 & 15 \\
    1024 & 10 \\
  \end{tabular}
}
\uElseIf{block size $\leq 256$}{
  $encoded~block~size \leftarrow 6$
}
\uElseIf{block size $\leq 65536$}{
  $encoded~block~size \leftarrow 7$
}
\Else{
  $encoded~block~size \leftarrow 0$
}
\EALGORITHM
}

\clearpage

\subsubsection{Encoding Sample Rate}
{\relsize{-1}
\ALGORITHM{sample rate in Hz}{encoded sample rate as 4 bit value}
\SetKw{AND}{and}
\uIf{sample rate \IN}{
  \begin{tabular}{rr|rr|rr}
    sample rate & encoded & sample rate & encoded & sample rate & encoded \\
    \hline
    8000 & 4 &
    32000 & 8 &
    96000 & 11 \\
    16000 & 5 &
    44100 & 9 &
    176400 & 2 \\
    22050 & 6 &
    48000 & 10 &
    192000 & 3 \\
    24000 & 7 &
    88200 & 1 \\
  \end{tabular}
}
\uElseIf{$sample~rate \bmod 1000 = 0$ \AND $sample~rate \leq 255000$}{
  $encoded~sample~rate \leftarrow 12$\;
}
\uElseIf{$sample~rate \bmod 10 = 0$ \AND $sample~rate \leq 655350$}{
  $encoded~sample~rate \leftarrow 14$\;
}
\uElseIf{$sample~rate \leq 65535$}{
  $encoded~sample~rate \leftarrow 13$\;
}
\Else{$encoded~sample~rate \leftarrow 0$}
\EALGORITHM
}
\subsubsection{Encoding Bits Per Sample}
{\relsize{-1}
\ALGORITHM{bits per sample}{encoded bits per sample as 3 bit value}
\eIf{bits per sample \IN}{
  \begin{tabular}{rr|rr}
    bits per sample & encoded & bits per sample & encoded \\
    \hline
    8 & 1 &
    20 & 5 \\
    12 & 2 &
    24 & 6 \\
    16 & 4 \\
  \end{tabular}
}{
  $encoded~bps \leftarrow 0$\;
}
\EALGORITHM
}
\begin{figure}[h]
\includegraphics{figures/flac/frames.pdf}
\end{figure}

\clearpage

\subsubsection{Encoding UTF-8 Frame Number}
{\relsize{-1}
\ALGORITHM{value as unsigned integer}{1 or more UTF-8 bytes}
\eIf{value $\leq 127$}{
  \WRITE value in 8 unsigned bits\;
}{
  \uIf{value $\leq 2047$}{
    total bytes $\leftarrow 2$\;
  }
  \uElseIf{value $\leq 65535$}{
    total bytes $\leftarrow 3$\;
  }
  \uElseIf{value $\leq 2097151$}{
    total bytes $\leftarrow 4$\;
  }
  \uElseIf{value $\leq 67108863$}{
    total bytes $\leftarrow 5$\;
  }
  \ElseIf{value $\leq 2147483647$}{
    total bytes $\leftarrow 6$\;
  }
  shift $\leftarrow (total~bytes - 1) \times 6$\;
  \WUNARY total bytes with stop bit 0\;
  \WRITE $\lfloor \text{value} \div 2 ^ \text{shift} \rfloor$ in (7 - total bytes) unsigned bits\tcc*[r]{initial value}
  shift $\leftarrow$ shift - 6\;
  \While{$shift \geq 0$}{
    \WRITE 2 in 2 unsigned bits\tcc*[r]{continuation header}
    \WRITE $(\lfloor \text{value} \div 2 ^ \text{shift} \rfloor \bmod 64)$ in 6 unsigned bits\tcc*[r]{continuation bits}
    shift $\leftarrow$ shift - 6\;
  }
}
\EALGORITHM
}
\par
\noindent
For example, encoding the frame number 4228 in UTF-8:
\par
\noindent
\begin{wrapfigure}[10]{r}{2.375in}
\includegraphics{figures/flac/utf8.pdf}
\end{wrapfigure}
\begin{align*}
\text{total bytes} &\leftarrow 3 \\
\text{shift} &\leftarrow 12 \\
& \textbf{write unary } 3 \text{ with stop bit 1} \\
& \textbf{write } 1 \text{ in 4 unsigned bits} \\
\text{shift} &\leftarrow 12 - 6 = 6 \\
& \textbf{write } 2 \text{ in 2 unsigned bits} \\
& \textbf{write } 2 \text{ in 6 unsigned bits} \\
\text{shift} &\leftarrow 6 - 6 = 0 \\
& \textbf{write } 2 \text{ in 2 unsigned bits} \\
& \textbf{write } 4 \text{ in 6 unsigned bits}
\end{align*}

\clearpage

\subsubsection{Calculating CRC-8}
Given a header byte and previous CRC-8 checksum,
or 0 as an initial value:
\begin{equation*}
\text{checksum}_i = \text{CRC8}(byte\xor\text{checksum}_{i - 1})
\end{equation*}
\begin{table}[h]
{\relsize{-3}\ttfamily
\begin{tabular}{|r||r|r|r|r|r|r|r|r|r|r|r|r|r|r|r|r|r|}
\hline
 & 0x?0 & 0x?1 & 0x?2 & 0x?3 & 0x?4 & 0x?5 & 0x?6 & 0x?7 & 0x?8 & 0x?9 & 0x?A & 0x?B & 0x?C & 0x?D & 0x?E & 0x?F \\
\hline
0x0? & 0x00 & 0x07 & 0x0E & 0x09 & 0x1C & 0x1B & 0x12 & 0x15 & 0x38 & 0x3F & 0x36 & 0x31 & 0x24 & 0x23 & 0x2A & 0x2D \\
0x1? & 0x70 & 0x77 & 0x7E & 0x79 & 0x6C & 0x6B & 0x62 & 0x65 & 0x48 & 0x4F & 0x46 & 0x41 & 0x54 & 0x53 & 0x5A & 0x5D \\
0x2? & 0xE0 & 0xE7 & 0xEE & 0xE9 & 0xFC & 0xFB & 0xF2 & 0xF5 & 0xD8 & 0xDF & 0xD6 & 0xD1 & 0xC4 & 0xC3 & 0xCA & 0xCD \\
0x3? & 0x90 & 0x97 & 0x9E & 0x99 & 0x8C & 0x8B & 0x82 & 0x85 & 0xA8 & 0xAF & 0xA6 & 0xA1 & 0xB4 & 0xB3 & 0xBA & 0xBD \\
0x4? & 0xC7 & 0xC0 & 0xC9 & 0xCE & 0xDB & 0xDC & 0xD5 & 0xD2 & 0xFF & 0xF8 & 0xF1 & 0xF6 & 0xE3 & 0xE4 & 0xED & 0xEA \\
0x5? & 0xB7 & 0xB0 & 0xB9 & 0xBE & 0xAB & 0xAC & 0xA5 & 0xA2 & 0x8F & 0x88 & 0x81 & 0x86 & 0x93 & 0x94 & 0x9D & 0x9A \\
0x6? & 0x27 & 0x20 & 0x29 & 0x2E & 0x3B & 0x3C & 0x35 & 0x32 & 0x1F & 0x18 & 0x11 & 0x16 & 0x03 & 0x04 & 0x0D & 0x0A \\
0x7? & 0x57 & 0x50 & 0x59 & 0x5E & 0x4B & 0x4C & 0x45 & 0x42 & 0x6F & 0x68 & 0x61 & 0x66 & 0x73 & 0x74 & 0x7D & 0x7A \\
0x8? & 0x89 & 0x8E & 0x87 & 0x80 & 0x95 & 0x92 & 0x9B & 0x9C & 0xB1 & 0xB6 & 0xBF & 0xB8 & 0xAD & 0xAA & 0xA3 & 0xA4 \\
0x9? & 0xF9 & 0xFE & 0xF7 & 0xF0 & 0xE5 & 0xE2 & 0xEB & 0xEC & 0xC1 & 0xC6 & 0xCF & 0xC8 & 0xDD & 0xDA & 0xD3 & 0xD4 \\
0xA? & 0x69 & 0x6E & 0x67 & 0x60 & 0x75 & 0x72 & 0x7B & 0x7C & 0x51 & 0x56 & 0x5F & 0x58 & 0x4D & 0x4A & 0x43 & 0x44 \\
0xB? & 0x19 & 0x1E & 0x17 & 0x10 & 0x05 & 0x02 & 0x0B & 0x0C & 0x21 & 0x26 & 0x2F & 0x28 & 0x3D & 0x3A & 0x33 & 0x34 \\
0xC? & 0x4E & 0x49 & 0x40 & 0x47 & 0x52 & 0x55 & 0x5C & 0x5B & 0x76 & 0x71 & 0x78 & 0x7F & 0x6A & 0x6D & 0x64 & 0x63 \\
0xD? & 0x3E & 0x39 & 0x30 & 0x37 & 0x22 & 0x25 & 0x2C & 0x2B & 0x06 & 0x01 & 0x08 & 0x0F & 0x1A & 0x1D & 0x14 & 0x13 \\
0xE? & 0xAE & 0xA9 & 0xA0 & 0xA7 & 0xB2 & 0xB5 & 0xBC & 0xBB & 0x96 & 0x91 & 0x98 & 0x9F & 0x8A & 0x8D & 0x84 & 0x83 \\
0xF? & 0xDE & 0xD9 & 0xD0 & 0xD7 & 0xC2 & 0xC5 & 0xCC & 0xCB & 0xE6 & 0xE1 & 0xE8 & 0xEF & 0xFA & 0xFD & 0xF4 & 0xF3 \\
\hline
\end{tabular}
}
\end{table}

\subsubsection{Frame Header Encoding Example}
Given a frame header with the following attributes:
\begin{table}[h]
\begin{tabular}{rl}
block size : & 4096 PCM frames \\
sample rate : & 44100 Hz \\
channel assignment : & 1 (2 channels stored independently) \\
bits per sample : & 16 \\
frame number : & 0
\end{tabular}
\end{table}
\par
\noindent
we generate the following frame header bytes:
\begin{figure}[h]
\includegraphics{figures/flac/header-example.pdf}
\end{figure}
\par
\noindent
Note how the CRC-8 is calculated from the preceding 5 header bytes:
\begin{align*}
\text{checksum}_0 = \text{CRC8}(\texttt{FF}\xor\texttt{00}) = \texttt{F3} & &
\text{checksum}_3 = \text{CRC8}(\texttt{18}\xor\texttt{E6}) = \texttt{F4} \\
\text{checksum}_1 = \text{CRC8}(\texttt{F8}\xor\texttt{F3}) = \texttt{31} & &
\text{checksum}_4 = \text{CRC8}(\texttt{00}\xor\texttt{F4}) = \texttt{C2} \\
\text{checksum}_2 = \text{CRC8}(\texttt{C9}\xor\texttt{31}) = \texttt{E6} \\
\end{align*}

\clearpage

\subsection{Encoding a FLAC Subframe}
{\relsize{-1}
\ALGORITHM{signed subframe samples, subframe's bits per sample}{a FLAC subframe}
\SetKwFunction{LEN}{len}
\SetKwFunction{MIN}{min}
\eIf{all samples are the same}{
  \Return CONSTANT subframe using $sample_0$\;
}{
  wasted BPS $\leftarrow$ calculate wasted bits per sample\;
  \If{wasted BPS $> 0$}{
    \ForEach{sample \IN subframe samples}{
      sample $\leftarrow \lfloor \text{sample} \div 2 ^ \text{wasted BPS} \rfloor$
    }
    subframes's bits per sample $\leftarrow$ subframe's bits per sample - wasted BPS\;
  }
  \BlankLine
  best FIXED order $\leftarrow$ calculate FIXED subframe order for samples\;
  FIXED subframe data $\leftarrow$ build FIXED subframe with best FIXED order\;
  \BlankLine
  \eIf(\tcc*[f]{from encoding parameters}){maximum LPC order $ > 0$}{
    LPC subframe parameters $\leftarrow$ compute best LPC coefficients\;
    LPC subframe data $\leftarrow$ build LPC subframe with LPC parameters\;
    \BlankLine
    \uIf{$(\text{bits per sample} \times \text{sample count}) < $ \newline$\MIN(\LEN(\text{LPC subframe data}) , \LEN(\text{FIXED subframe data}))$}{
      \Return VERBATIM subframe using subframe's samples and bits per sample\;
    }
    \uElseIf{$\LEN(\text{FIXED subframe data}) < \LEN(\text{LPC subframe data})$}{
      \Return FIXED subframe data\;
    }
    \Else{
      \Return LPC subframe data\;
    }
  }{
    \eIf{\LEN(FIXED subframe data) $< (\text{bits per sample} \times \text{sample count})$}{
      \Return FIXED subframe data\;
    }{
      \Return VERBATIM subframe using subframe's sample and bits per sample\;
    }
  }
}
\EALGORITHM
\par
\noindent
Note that FIXED and LPC subframes aren't typically written to the stream
outright.
Instead, they're written to temporary spaces.
The size of those spaces are compared to each other, and to a hypothetical
VERBATIM subframe.
The smallest subframe is the one written to disk.
}
\begin{figure}[h]
\includegraphics{figures/flac/subframes.pdf}
\end{figure}

\clearpage

\subsubsection{Calculating Wasted Bits Per Sample}
\ALGORITHM{signed subframe samples}{the amount of wasted bits per sample as an unsigned integer}
\SetKw{AND}{and}
\SetKwFunction{MIN}{min}
wasted BPS $\leftarrow$ maximum unsigned integer\;
\ForEach{sample \IN subframe samples}{
  \If{sample $\neq 0$}{
    wasted $\leftarrow$ 0\;
    \While{sample $\bmod~2 = 0$}{
      wasted $\leftarrow$ wasted + 1\;
      sample $\leftarrow \text{sample} \div 2$\;
    }
    \eIf{wasted $= 0$}{
      \Return 0\tcc*[r]{stop once a sample has no wasted bits}
    }{
      wasted BPS $\leftarrow$ \MIN(wasted BPS, wasted)\;
    }
  }
}
\tcc{if all samples are 0, we should return a CONSTANT subframe}
\Return wasted BPS\;
\EALGORITHM

\subsection{Encoding a CONSTANT Subframe}
\ALGORITHM{signed subframe sample, subframe's bits per sample}{a CONSTANT subframe}
\WRITE 0 in 1 unsigned bit\tcc*[r]{pad}
\WRITE 0 in 6 unsigned bits\tcc*[r]{subframe type}
\WRITE 0 in 1 unsigned bit\tcc*[r]{no wasted BPS}
\WRITE sample in (bits per sample) signed bits\;
\Return a CONSTANT subframe\;
\EALGORITHM

\subsection{Encoding a VERBATIM Subframe}
\ALGORITHM{signed subframe samples, subframe's bits per sample}{a VERBATIM subframe}
\WRITE 0 in 1 unsigned bit\tcc*[r]{pad}
\WRITE 1 in 6 unsigned bits\tcc*[r]{subframe type}
\WRITE 0 in 1 unsigned bit\tcc*[r]{no wasted BPS}
\ForEach{sample \IN subframe samples}{
  \WRITE sample in (bits per sample) signed bits\;
}
\Return a VERBATIM subframe\;
\EALGORITHM

\clearpage

\subsection{Calculating FIXED Subframe Order}
\ALGORITHM{signed subframe samples}{the FIXED subframe order, between 0 and 4}
\SetKwFunction{MIN}{min}
\eIf{sample count $\geq 4$}{
  $\text{last error}_1 \leftarrow \text{sample}_3 - \text{sample}_2$\;
  $\text{last error}_2 \leftarrow \text{last error}_1 - (\text{sample}_2 - \text{sample}_1)$\;
  $\text{last error}_3 \leftarrow \text{last error}_2 - (\text{sample}_2 - (\text{sample}_1 \times 2) + \text{sample}_0)$\;
  \BlankLine
  $\text{total error}_0 \leftarrow 0$\;
  $\text{total error}_1 \leftarrow 0$\;
  $\text{total error}_2 \leftarrow 0$\;
  $\text{total error}_3 \leftarrow 0$\;
  $\text{total error}_4 \leftarrow 0$\;
  \BlankLine
  \For{i = 4 \emph{\KwTo}sample count}{
    $\text{error}_0 \leftarrow \text{sample}_i - \text{sample}_{i - 1}$\;
    $\text{error}_1 \leftarrow \text{error}_0 - \text{last error}_1$\;
    $\text{error}_2 \leftarrow \text{error}_1 - \text{last error}_2$\;
    $\text{error}_3 \leftarrow \text{error}_2 - \text{last error}_3$\;
    \BlankLine
    $\text{total error}_0 \leftarrow \text{total error}_0 + |\text{sample}_i|$\;
    $\text{total error}_1 \leftarrow \text{total error}_1 + |\text{error}_0|$\;
    $\text{total error}_2 \leftarrow \text{total error}_2 + |\text{error}_1|$\;
    $\text{total error}_3 \leftarrow \text{total error}_3 + |\text{error}_2|$\;
    $\text{total error}_4 \leftarrow \text{total error}_4 + |\text{error}_3|$\;
    \BlankLine
    $\text{last error}_1 \leftarrow \text{error}_0$\;
    $\text{last error}_2 \leftarrow \text{error}_1$\;
    $\text{last error}_3 \leftarrow \text{error}_2$\;
  }
  \uIf{$\text{total error}_0 < \MIN(\text{total error}_1 , \text{total error}_2 , \text{total error}_3 , \text{total error}_4)$}{
    \Return 0\;
  }
  \uElseIf{$\text{total error}_1 < \MIN(\text{total error}_2 , \text{total error}_3 , \text{total error}_4)$}{
    \Return 1\;
  }
  \uElseIf{$\text{total error}_2 < \MIN(\text{total error}_3 , \text{total error}_4)$}{
    \Return 2\;
  }
  \uElseIf{$\text{total error}_3 < \text{total error}_4$}{
    \Return 3\;
  }
  \Else{
    \Return 4\;
  }
}{
  \Return 0\;
}
\EALGORITHM

\clearpage

\subsubsection{FIXED Subframe Order Calculation Example}
Given the subframe samples: \texttt{18, 20, 26, 24, 24, 23, 21, 24, 23, 20}:
\begin{align*}
\text{initial last error}_1 &\leftarrow 24 - 26 = -2 \\
\text{initial last error}_2 &\leftarrow -2 - (26 - 20) = -8 \\
\text{initial last error}_3 &\leftarrow -8 - (26 - (20 \times 2) + 18) = -12
\end{align*}

\begin{table}[h]
\begin{tabular}{r|r|r|r}
\hline
\hline
$i$ & 4 & 5 & 6 \\
$\text{sample}_i$ & $24$ & $23$ & $21$ \\
\hline
$\text{error}_0$ & $24 - 24 = 0$ & $23 - 24 = -1$ & $21 - 23 = -2$ \\
$\text{error}_1$ & $0 - -2 = 2$ & $-1 - 0 = -1$ & $-2 - -1 = -1$ \\
$\text{error}_2$ & $2 - -8 = 10$ & $-1 - 2 = -3$ & $-1 - -1 = 0$ \\
$\text{error}_3$ & $6 - -8 = 22$ & $-3 - 10 = -13$ & $0 - -3 = 3$ \\
\hline
$\text{total error}_0$ & $0 + |24| = 24$ & $24 + |23| = 47$ & $47 + |21| = 68$ \\
$\text{total error}_1$ & $0 + 0 = 0$ & $0 + |-1| = 1$ & $1 + |-2| = 3$ \\
$\text{total error}_2$ & $0 + 2 = 2$ & $2 + |-1| = 3$ & $3 + |-1| = 4$ \\
$\text{total error}_3$ & $0 + 10 = 10$ & $10 + |-3| = 13$ & $13 + |0| = 13$ \\
$\text{total error}_4$ & $0 + 22 = 22$ & $22 + |-13| = 35$ & $35 + |3| = 38$ \\
\hline
$\text{last error}_1$ & $0$ & $-1$ & $-2$ \\
$\text{last error}_2$ & $2$ & $-1$ & $-1$ \\
$\text{last error}_3$ & $10$ & $-3$ & $0$ \\
\hline
\hline
$i$ & 7 & 8 & 9 \\
$\text{sample}_i$ & $24$ & $23$ & $20$ \\
\hline
$\text{error}_0$ & $24 - 21 = 3$ & $23 - 24 = -1$ & $20 - 23 = -3$ \\
$\text{error}_1$ & $3 - -2 = 5$ & $-1 - 3 = -4$ & $-3 - -1 = -2$ \\
$\text{error}_2$ & $5 - -1 = 6$ & $-4 - 5 = -9$ & $-2 - -4 = 2$ \\
$\text{error}_3$ & $6 - 0 = 6$ & $-9 - 6 = -15$ & $2 - -9 = 11$ \\
\hline
$\text{total error}_0$ & $68 + |24| = 92$ & $92 + |23| = 115$ & $115 + |20| = 135$ \\
$\text{total error}_1$ & $3 + |3| = 6$ & $6 + |-1| = 7$ & $7 + |-3| = 10$ \\
$\text{total error}_2$ & $4 + |5| = 9$ & $9 + |-4| = 13$ & $13 + |-2| = 15$ \\
$\text{total error}_3$ & $13 + |6| = 19$ & $19 + |-9| = 28$ & $28 + |2| = 30$ \\
$\text{total error}_4$ & $38 + |6| = 44$ & $44 + |15| = 59$ & $59 + |11| = 70$ \\
\hline
$\text{last error}_1$ & $3$ & $-1$ & $-3$ \\
$\text{last error}_2$ & $5$ & $-4$ & $-2$ \\
$\text{last error}_3$ & $6$ & $-9$ & $2$ \\
\hline
\hline
\end{tabular}
\end{table}
\par
\noindent
Since $\text{total error}_1$'s value of 10 is the smallest of the five
total error values,
the best FIXED predictor order for these samples is 1.

\clearpage

\subsection{Encoding a FIXED Subframe}
\ALGORITHM{signed subframe samples, subframe order, subframe's bits per sample, wasted BPS}{a FIXED subframe}
\WRITE 0 in 1 unsigned bit\tcc*[r]{pad}
\WRITE 1 in 3 unsigned bits\tcc*[r]{subframe type}
\WRITE subframe order in 3 unsigned bits\;
\eIf{$wasted~BPS > 0$}{
  \WRITE 1 in 1 unsigned bit\;
  \WUNARY (wasted BPS - 1) with stop bit 1\;
}{
  \WRITE 0 in 1 unsigned bit\;
}
\For(\tcc*[f]{warm-up samples}){i = 0 \emph{\KwTo}subframe order}{
  \WRITE $sample_i$ in (bits per sample) signed bits\;
}
calculate FIXED subframe's residuals based on signed samples and subframe order\;
write encoded residual block based on signed residual values\;
\Return a FIXED subframe\;
\EALGORITHM

\begin{figure}[h]
\includegraphics{figures/flac/fixed2.pdf}
\end{figure}
\par
\noindent
Building a residual block from signed residual values is explained
on page \pageref{flac_residual_encoding}.

\clearpage

\subsubsection{Calculating FIXED Subframe Residuals}
\ALGORITHM{signed subframe samples, subframe order}{signed residual values}
\Switch{subframe order}{
  \uCase{0}{
    \For{i = 0 \emph{\KwTo}sample count}{
      $residual_i \leftarrow sample_i$\;
    }
  }
  \uCase{1}{
    \For{i = 0 \emph{\KwTo}sample count - 1}{
      $residual_i \leftarrow sample_{i + 1} - sample_i$
    }
  }
  \uCase{2}{
    \For{i = 0 \emph{\KwTo}sample count - 2}{
      $residual_i \leftarrow sample_{i + 2} - ((2 \times sample_{i + 1}) - sample_i)$
    }
  }
  \uCase{3}{
    \For{i = 3 \emph{\KwTo}sample count}{
      $residual_i \leftarrow sample_{i + 3} - ((3 \times sample_{i + 2}) - (3 \times sample_{i + 1}) + sample_i)$
    }
  }
  \Case{4}{
    \For{i = 4 \emph{\KwTo}sample count}{
      $residual_i \leftarrow sample_{i + 4} - ((4 \times sample_{i + 3}) - (6 \times sample_{i + 2}) + (4 \times sample_{i + 1}) - sample_i)$
    }
  }
}
\EALGORITHM

\subsubsection{FIXED Subframe Residual Calculation Example}

Given the subframe samples: \texttt{18, 20, 26, 24, 24, 23, 21, 24, 23, 20}:
\begin{table}[h]
{\relsize{-1}
\begin{tabular}{r|r|r|r|r|r}
& order 0 & order 1 & order 2 & order 3 & order 4 \\
\hline
$residual_0$ & 18 & 2 & 4 & -12 & 22 \\
$residual_1$ & 20 & 6 & -8 & 10 & -13 \\
$residual_2$ & 26 & -2 & 2 & -3 & 3 \\
$residual_3$ & 24 & 0 & -1 & 0 & 6 \\
$residual_4$ & 24 & -1 & -1 & 6 & -15 \\
$residual_5$ & 23 & -2 & 5 & -9 & 11 \\
$residual_6$ & 21 & 3 & -4 & 2 \\
$residual_7$ & 24 & -1 & -2 \\
$residual_8$ & 23 & -3 \\
$residual_9$ & 20 \\
\hline
total error & 135 & 10 & 15 & 30 & 70 \\
\end{tabular}
}
\end{table}
\par
\noindent
Note how the total number of residuals equals the
total number of samples minus the subframe's order,
to account for the warm-up samples.
Also note that if you remove the first $4 - order$ residuals
and sum the absolute value of the remainding residuals,
the result is the \VAR{total error} value
used when calculating the best FIXED subframe order.

\subsection{Residual Encoding}
\label{flac_residual_encoding}
{\relsize{-1}
\ALGORITHM{a set of signed residual values, the subframe's block size and order, minimum and maximum partition order from encoding parameters}{an encoded block of residuals}
\SetKwData{PORDER}{porder}
\SetKwData{BPORDER}{best porder}
\SetKwData{BSIZE}{best size}
\SetKwData{PARTITION}{partition}
\SetKwFunction{LEN}{len}
\tcc{calculate best Rice parameters, partition order and coding method}
\BPORDER $\leftarrow 0$\;
\BSIZE $\leftarrow$ maximum integer\;
\For{\PORDER = minimum partition order \emph{\KwTo}(maximum partition order + 1)}{
  \If{block size equally divisible into $2^{\PORDER}$ partitions}{
    $\text{partition size}_{\PORDER} \leftarrow 0$\;
    \For{p = 0 \emph{\KwTo}$2 ^ {\PORDER}$} {
      $\text{partition size}_{\PORDER} \leftarrow \text{partition size}_{\PORDER} + 4$\tcc*[r]{partition header bits}
      \eIf{p = 0}{
        $partition_{\PORDER~0} \leftarrow block~size \div 2 ^ {\PORDER} - subframe~order$ residuals
      }{
        $partition_{\PORDER~p} \leftarrow block~size \div 2 ^ {\PORDER}$ residuals
      }
      \BlankLine
      $Rice_{\PORDER~p} \leftarrow$ compute best Rice parameter for $partition_{\PORDER~p}$\;
      \BlankLine
      $encoded~residuals_{\PORDER~p} \leftarrow$ encode residual partition with $Rice_{\PORDER~p}$\;
      \BlankLine
      $\text{partition size}_{\PORDER} \leftarrow \text{partition size}_{\PORDER} + \LEN(encoded~residuals_{\PORDER~p})$\;
    }
    \If{$\text{partition size}_{\PORDER} < \BSIZE$}{
      \BPORDER $\leftarrow$ \PORDER\;
      \BSIZE $\leftarrow \text{partition size}_{\PORDER}$\;
    }
  }
}
\BlankLine
\tcc{write best residual partition(s) to residual block}
\eIf{any $Rice_{\BPORDER} > 14$}{
  coding method $\leftarrow 1$\;
}{
  coding method $\leftarrow 0$\;
}
\WRITE coding method in 2 unsigned bits\;
\WRITE \BPORDER in 4 unsigned bits\;
\For{p = 0 \emph{\KwTo}$2 ^ {\BPORDER}$} {
  \eIf{coding method = 0}{
    \WRITE $Rice_{\BPORDER~p}$ in 4 unsigned bits\;
  }{
    \WRITE $Rice_{\BPORDER~p}$ in 5 unsigned bits\;
  }
  write $encoded~residuals_{\PORDER~p}$\;
}
\Return encoded residual block\;
\EALGORITHM
}

\clearpage

\subsubsection{Computing Best Rice Parameter}
\ALGORITHM{a set of signed residual values, the stream's bits per sample}{the Rice parameter as an unsigned integer}
\SetKwData{PARAMETER}{Rice parameter}
\SetKwData{PSUM}{partition sum}
\SetKwFunction{MIN}{min}
\PSUM $\leftarrow \overset{partition~size - 1}{\underset{i = 0}{\sum}} |residual_i|$\;
$\PARAMETER \leftarrow 0$\;
\While{$partition~size \times 2 ^ \text{\PARAMETER} < \PSUM$}{
  $\PARAMETER \leftarrow \PARAMETER + 1$\;
}
\eIf{bits per sample $\leq 16$}{
  \Return \MIN(\PARAMETER , 14)\tcc*[r]{coding method 0}
}{
  \Return \MIN(\PARAMETER , 30)\tcc*[r]{coding method 1}
}
\Return \PARAMETER\;
\EALGORITHM

\subsubsection{Encoding Residual Partition}
\ALGORITHM{a set of signed residual values, the partition's Rice parameter}{an encoded residual partition}
\ForEach{residual \emph{\KwTo}residuals}{
  \eIf{$residual \geq 0$}{
    $unsigned \leftarrow residual \times 2$\;
  }{
    $unsigned \leftarrow ((-residual - 1) \times 2) + 1$\;
  }
  MSB $\leftarrow \lfloor unsigned \div 2 ^ \text{Rice parameter} \rfloor$\;
  LSB $\leftarrow unsigned - (\text{MSB} \times 2 ^ \text{Rice parameter})$\;
  \WUNARY MSB with stop bit 1\;
  \WRITE LSB in (Rice parameter) unsigned bits\;
}
\Return encoded partition\;
\EALGORITHM

\begin{figure}[h]
\includegraphics{figures/flac/residual.pdf}
\end{figure}

\clearpage

\subsubsection{Residual Encoding Example}
Given a block size of 10 and the residuals \texttt{2, 6, -2, 0, -1, -2, 3, -1, -3}:
\begin{align*}
\intertext{for $\text{partition order (porder)} = 0$:}
partition_{0~0} &\leftarrow \texttt{[2, 6, -2, 0, -1, -2, 3, -1, -3]} \\
\text{partition sum}_{0~0} &\leftarrow 2 + 6 + 2 + 0 + 1 + 2 + 3 + 1 + 3 = 20 \\
Rice_{0~0} &\leftarrow 1~~(9 \times 2 ^ 0 < 20~,~9 \times 2 ^ 1 < 20~,~9 \times 2 ^ 2 > 20) \\
\intertext{which is encoded to $encoded~residuals_{0~0}$:
\newline
\includegraphics{figures/flac/residual-example1.pdf}}
\intertext{for partition order (porder) = 1:}
partition_{1~0} &\leftarrow \texttt{[2, 6, -2, 0]} \\
\text{partition sum}_{1~0} &\leftarrow 2 + 6 + 2 + 0 = 10 \\
Rice_{1~0} &\leftarrow 1~~(4 \times 2 ^ 0 < 10~,~4 \times 2 ^ 1 < 10~,~4 \times 2 ^ 2 > 10) \\
\intertext{which is encoded to $encoded~residuals_{1~0}$:
\newline
\includegraphics{figures/flac/residual-example2.pdf}}
partition_{1~1} &\leftarrow \texttt{[-1, -2, 3, -1, -3]} \\
\text{partition sum}_{1~1} &\leftarrow 1 + 2 + 3 + 1 + 3 = 10 \\
Rice_{1~1} &\leftarrow 1~~(4 \times 2 ^ 0 < 10~,~4 \times 2 ^ 1 < 10~,~4 \times 2 ^ 2 > 10) \\
\intertext{which is encoded to $encoded~residuals_{1~1}$:
\newline
\includegraphics{figures/flac/residual-example3.pdf}}
\end{align*}
\par
\noindent
Since partition order 0's 33 bits, + 4 bits for one partition header,
is smaller than partition order 1's 17 bits + 16 bits + 8 bits
for two partition headers, the ideal partition order for these residuals is 0.

\clearpage

The 33 bit partition is packaged into a complete residual block
in which:
\newline
\begin{tabular}{rl}
$partition_{0~0}$ & $\leftarrow$ \texttt{2, 6, -2, 0, -1, -2, 3, -1, -3} \\
$Rice_{0~0}$ & $\leftarrow 1$ \\
partition order & $\leftarrow 0$ \\
coding method & $\leftarrow 0$ \\
\end{tabular}
\begin{figure}[h]
\includegraphics{figures/flac/residual-example4.pdf}
\end{figure}
\par
Finally, we package these residuals into a FIXED subframe in which:
\newline
\begin{tabular}{rl}
predictor order & $\leftarrow 1$ \\
$\text{warm-up sample}_0$ & $\leftarrow $ 18 \\
\end{tabular}
\begin{figure}[h]
\includegraphics{figures/flac/residual-example5.pdf}
\end{figure}
\par
\noindent
Reducing our 10, 16-bit samples from a total of 160 bits
down to only 67 bits - or about 40\% of their original size.

\clearpage

\subsection{Computing LPC Coefficients}

{\relsize{-1}
\ALGORITHM{signed subframe samples}{LPC subframe order, QLP coefficients, QLP precision and QLP shift needed}
\SetKwFunction{WINDOW}{window}
\SetKwFunction{AUTOCORRELATE}{autocorrelate}
\SetKwFunction{LEN}{len}
\SetKw{AND}{and}
\SetKw{NOT}{not}
\SetKwData{WINDOWED}{windowed samples}
\SetKwData{AUTOCORRELATIONS}{autocorrelation values}
\SetKwData{LPCOEFFS}{LP coefficients}
\SetKwData{ERRORS}{error}
\SetKwData{ORDER}{order}
\SetKwData{QLPCOEFFS}{QLP coefficients}
\SetKwData{QLPPRECISION}{QLP precision}
\SetKwData{QLPSHIFT}{QLP shift needed}
\SetKwData{BESTLPCPARAMS}{best LPC parameters}
\SetKwData{LPCDATA}{LPC subframe data}
\tcc{windowed sample count equals subframe sample count}
\WINDOWED $\leftarrow$ \WINDOW(subframe samples)\;
\BlankLine
\tcc{autocorrelation value count equals the maximum LPC order + 1}
\AUTOCORRELATIONS $\leftarrow$ \AUTOCORRELATE(\WINDOWED)\;
\BlankLine
\eIf{$\LEN(\AUTOCORRELATIONS) > 1$ \AND \AUTOCORRELATIONS aren't all 0.0}{
  (\LPCOEFFS, \ERRORS) $\leftarrow$ compute LP coefficients from \AUTOCORRELATIONS\;
  \BlankLine
  \eIf(\tcc*[f]{from encoding parameters}){\NOT exhaustive model search}{
    \tcc{estimate which set of LP coefficients is the smallest
      and return those}
    \ORDER $\leftarrow$ estimate best order from \ERRORS, sample count and bits per sample\;
    (\QLPCOEFFS, \QLPPRECISION, \QLPSHIFT) $\leftarrow$ quantize \LPCOEFFS at estimated best \ORDER\;
    \Return (\ORDER, \QLPCOEFFS, \QLPPRECISION, \QLPSHIFT)\;
  }{
    \tcc{build a complete LPC subframe from each set of LP coefficients
    and return the parameters of the one which is smallest}
    best LPC subframe size $\leftarrow$ maximum integer\;
    \For{\ORDER = 1 \emph{\KwTo}maximum LPC order + 1}{
      (\QLPCOEFFS, \QLPPRECISION, \QLPSHIFT) $\leftarrow$ quantize \LPCOEFFS at \ORDER\;
      \LPCDATA $\leftarrow$ build LPC subframe from \ORDER, \QLPCOEFFS, \QLPPRECISION, \QLPSHIFT\;
      \If{$\LEN(\LPCDATA) < \text{best LPC subframe size}$}{
        \BESTLPCPARAMS $\leftarrow$ (\ORDER, \QLPCOEFFS, \QLPPRECISION, \QLPSHIFT)\;
        best LPC subframe size $\leftarrow \LEN(\LPCDATA)$\;
      }
      \Return \BESTLPCPARAMS\;
    }
  }
}{
  \tcc{all samples are 0, so return very basic coefficients}
  \Return (1, [0], 2, 0)\;
}
\EALGORITHM
}

\clearpage


\subsubsection{Windowing the Input Samples}
\ALGORITHM{a list of signed input sample integers}{a list of signed windowed samples as floats}
\SetKwFunction{TUKEY}{tukey}
\For{i = 0 \emph{\KwTo}sample count}{
  $windowed_i = sample_i \times \TUKEY(i)$\;
}
\Return windowed samples\;
\EALGORITHM
\par
\noindent
The \VAR{Tukey} function is defined as:
\begin{equation*}
tukey(n) =
\begin{cases}
\frac{1}{2} \times \left[1 + cos\left(\pi \times \left(\frac{2 \times n}{\alpha \times (N - 1)} - 1 \right)\right)\right] & \text{ if } 0 \leq n \leq \frac{\alpha \times (N - 1)}{2} \\
1 & \text{ if } \frac{\alpha \times (N - 1)}{2} \leq n \leq (N - 1) \times (1 - \frac{\alpha}{2}) \\
\frac{1}{2} \times \left[1 + cos\left(\pi \times \left(\frac{2 \times n}{\alpha \times (N - 1)} - \frac{2}{\alpha} + 1 \right)\right)\right] & \text{ if } (N - 1) \times (1 - \frac{\alpha}{2}) \leq n \leq (N - 1) \\
\end{cases}
\end{equation*}
\par
\noindent
Where $N$ is the total number of samples and $\alpha$ is $\nicefrac{1}{2}$.
\par
\noindent
\begin{wrapfigure}[5]{r}{3in}
\includegraphics{figures/flac/tukey.pdf}
\end{wrapfigure}
\begin{table}[h]
\begin{tabular}{r|r|r|r}
$i$ & $sample_i$ & $tukey(i)$ & $windowed_i$ \\
\hline
0 & 18 & 0 & 0.0 \\
1 & 20 & .41 & 8.2 \\
2 & 26 & .97 & 25.2 \\
3 & 24 & 1 & 24.0 \\
4 & 24 & 1 & 24.0 \\
5 & 23 & 1 & 23.0 \\
6 & 21 & 1 & 21.0 \\
7 & 24 & .97 & 23.3 \\
8 & 23 & .41 & 9.4 \\
9 & 20 & 0 & 0.0 \\
\end{tabular}
\end{table}

\clearpage

\subsubsection{Performing Autocorrelation}

\ALGORITHM{a list of signed windowed samples, the maximum LPC order}{a list of signed autocorrelation values}
\For{lag = 0 \emph{\KwTo}maximum LPC order + 1}{
  $\text{autocorrelation}_{\text{lag}} = \overset{\text{total samples} - \text{lag} - 1}{\underset{i = 0}{\sum}}\text{sample}_i \times \text{sample}_{i + \text{lag}}$\;
}
\EALGORITHM
For example, given the windowed samples:
\texttt{0.0, 8.2, 25.2, 24.0, 24.0, 23.0, 21.0, 23.3, 9.4, 0.0}
and a maximum LPC order of 3:
\begin{figure}[h]
\subfloat{
  {\relsize{-2}
    \begin{tabular}{rrrrr}
      \texttt{0.0} & $\times$ & \texttt{0.0} & $=$ & \texttt{0.00} \\
      \texttt{8.2} & $\times$ & \texttt{8.2} & $=$ & \texttt{67.24} \\
      \texttt{25.2} & $\times$ & \texttt{25.2} & $=$ & \texttt{635.04} \\
      \texttt{24.0} & $\times$ & \texttt{24.0} & $=$ & \texttt{576.00} \\
      \texttt{24.0} & $\times$ & \texttt{24.0} & $=$ & \texttt{576.00} \\
      \texttt{23.0} & $\times$ & \texttt{23.0} & $=$ & \texttt{529.00} \\
      \texttt{21.0} & $\times$ & \texttt{21.0} & $=$ & \texttt{441.00} \\
      \texttt{23.3} & $\times$ & \texttt{23.3} & $=$ & \texttt{542.89} \\
      \texttt{9.4} & $\times$ & \texttt{9.4} & $=$ & \texttt{88.36} \\
      \texttt{0.0} & $\times$ & \texttt{0.0} & $=$ & \texttt{0.00} \\
      \hline
      \multicolumn{3}{r}{$\text{autocorrelation}_0$} & $=$ & \texttt{3455.53} \\
    \end{tabular}
  }
}
\includegraphics{figures/flac/lag0.pdf}

\subfloat{
  {\relsize{-2}
    \begin{tabular}{rrrrr}
      \texttt{0.0} & $\times$ & \texttt{8.2} & $=$ & \texttt{0.00} \\
      \texttt{8.2} & $\times$ & \texttt{25.2} & $=$ & \texttt{206.64} \\
      \texttt{25.2} & $\times$ & \texttt{24.0} & $=$ & \texttt{604.80} \\
      \texttt{24.0} & $\times$ & \texttt{24.0} & $=$ & \texttt{576.00} \\
      \texttt{24.0} & $\times$ & \texttt{23.0} & $=$ & \texttt{552.00} \\
      \texttt{23.0} & $\times$ & \texttt{21.0} & $=$ & \texttt{483.00} \\
      \texttt{21.0} & $\times$ & \texttt{23.3} & $=$ & \texttt{489.30} \\
      \texttt{23.3} & $\times$ & \texttt{9.4} & $=$ & \texttt{219.02} \\
      \texttt{9.4} & $\times$ & \texttt{0.0} & $=$ & \texttt{0.00} \\
      \hline
      \multicolumn{3}{r}{$\text{autocorrelation}_1$} & $=$ & \texttt{3130.76} \\
    \end{tabular}
  }
}
\includegraphics{figures/flac/lag1.pdf}

\subfloat{
  {\relsize{-2}
    \begin{tabular}{rrrrr}
      \texttt{0.0} & $\times$ & \texttt{25.2} & $=$ & \texttt{0.00} \\
      \texttt{8.2} & $\times$ & \texttt{24.0} & $=$ & \texttt{196.80} \\
      \texttt{25.2} & $\times$ & \texttt{24.0} & $=$ & \texttt{604.80} \\
      \texttt{24.0} & $\times$ & \texttt{23.0} & $=$ & \texttt{552.00} \\
      \texttt{24.0} & $\times$ & \texttt{21.0} & $=$ & \texttt{504.00} \\
      \texttt{23.0} & $\times$ & \texttt{23.3} & $=$ & \texttt{535.90} \\
      \texttt{21.0} & $\times$ & \texttt{9.4} & $=$ & \texttt{197.40} \\
      \texttt{23.3} & $\times$ & \texttt{0.0} & $=$ & \texttt{0.00} \\
      \hline
      \multicolumn{3}{r}{$\text{autocorrelation}_2$} & $=$ & \texttt{2590.90} \\
    \end{tabular}
  }
}
\includegraphics{figures/flac/lag2.pdf}

\subfloat{
  {\relsize{-2}
    \begin{tabular}{rrrrr}
      \texttt{0.0} & $\times$ & \texttt{24.0} & $=$ & \texttt{0.00} \\
      \texttt{8.2} & $\times$ & \texttt{24.0} & $=$ & \texttt{196.80} \\
      \texttt{25.2} & $\times$ & \texttt{23.0} & $=$ & \texttt{579.60} \\
      \texttt{24.0} & $\times$ & \texttt{21.0} & $=$ & \texttt{504.00} \\
      \texttt{24.0} & $\times$ & \texttt{23.3} & $=$ & \texttt{559.20} \\
      \texttt{23.0} & $\times$ & \texttt{9.4} & $=$ & \texttt{216.20} \\
      \texttt{21.0} & $\times$ & \texttt{0.0} & $=$ & \texttt{0.00} \\
      \hline
      \multicolumn{3}{r}{$\text{autocorrelation}_3$} & $=$ & \texttt{2055.80} \\
    \end{tabular}
  }
}
\includegraphics{figures/flac/lag3.pdf}
\end{figure}
\par
\noindent
Note that the total number of autocorrelation values equals
the maximum LPC order + 1.

\clearpage

\subsubsection{LP Coefficient Calculation}
{\relsize{-1}
\ALGORITHM{a list of autocorrelation floats, the maximum LPC order}{a list of LP coefficient lists, a list of error values}
\SetKwData{LPCOEFF}{LP coefficient}
\SetKwData{ERROR}{error}
\SetKwData{AUTOCORRELATION}{autocorrelation}
\begin{tabular}{rcl}
$\kappa_0$ &$\leftarrow$ & $ \AUTOCORRELATION_1 \div \AUTOCORRELATION_0$ \\
$\LPCOEFF_{0~0}$ &$\leftarrow$ & $ \kappa_0$ \\
$\ERROR_0$ &$\leftarrow$ & $ \AUTOCORRELATION_0 \times (1 - {\kappa_0} ^ 2)$ \\
\end{tabular}\;
\For{i = 1 \emph{\KwTo}maximum LPC order + 1}{
  $q_i \leftarrow \AUTOCORRELATION_{i + 1}$\;
  \tcc{"zip" all of the previous row's LP coefficients
    \newline
    and the reversed autocorrelation values from 1 to i + 1
    \newline
    into ($c$,$a$) pairs}
  \For{j = 0 \emph{\KwTo}i}{
    \tcc{$q_i$ is $\AUTOCORRELATION_{i + 1}$ minus the sum of those mutiplied ($c$,$a$) pairs}
    $q_i \leftarrow q_i - (\LPCOEFF_{(i - 1)~j} \times \AUTOCORRELATION_{i - j})$\;
  }
  $\kappa_i = q_i \div \ERROR_{i - 1}$\;
  \BlankLine
  \tcc{"zip" all of the previous row's LP coefficients
    \newline
    and the previous row's LP coefficients reversed
    \newline
    into ($c$,$r$) pairs}
  \For{j = 0 \emph{\KwTo}i}{
    \tcc{then build a new coefficient list of $c - (\kappa_i * r)$ for each ($c$,$r$) pair}
    $\LPCOEFF_{i~j} \leftarrow \LPCOEFF_{(i - 1)~j} - (\kappa_i \times \LPCOEFF_{(i - 1)~(i - j - 1)})$\;
  }
  $\text{LP coefficient}_{i~i} \leftarrow \kappa_i$\tcc*[r]{and append $\kappa_i$ as the final coefficient in that list}
  \BlankLine
  $\ERROR_i \leftarrow \ERROR_{i - 1} \times (1 - {\kappa_i}^2)$\;
}
\Return list of LP coefficient lists, error value list\;
\EALGORITHM
}
\par
\noindent
Given a a maximum LPC order of 3 and 4 autocorrelation values:
{\relsize{-1}
\begin{align*}
\kappa_0 &\leftarrow \text{autocorrelation}_1 \div \text{autocorrelation}_0 \\
\text{LP coefficient}_{0~0} &\leftarrow \kappa_0 \\
\text{error}_0 &\leftarrow \text{autocorrelation}_0 \times (1 - {\kappa_0} ^ 2) \\
i &= 1 \\
q_1 &\leftarrow \text{autocorrelation}_2 - (\text{LP coefficient}_{0~0} \times \text{autocorrelation}_{1}) \\
\kappa_1 &\leftarrow q_1 \div error_0 \\
\text{LP coefficient}_{1~0} &\leftarrow \text{LP coefficient}_{0~0} - (\kappa_1 \times \text{LP coefficient}_{0~0}) \\
\text{LP coefficient}_{1~1} &\leftarrow \kappa_1 \\
\text{error}_1 &\leftarrow \text{error}_0 \times (1 - {\kappa_1} ^ 2) \\
i &= 2 \\
q_2 &\leftarrow \text{autocorrelation}_3 - (\text{LP coefficient}_{1~0} \times \text{autocorrelation}_{2} + \text{LP coefficient}_{1~1} \times \text{autocorrelation}_{1}) \\
\kappa_2 &\leftarrow q_2 \div \text{error}_1 \\
\text{LP coefficient}_{2~0} &\leftarrow \text{LP coefficient}_{1~0} - (\kappa_2 \times \text{LP coefficient}_{1~1}) \\
\text{LP coefficient}_{2~1} &\leftarrow \text{LP coefficient}_{1~1} - (\kappa_2 \times \text{LP coefficient}_{1~0}) \\
\text{LP coefficient}_{2~2} &\leftarrow \kappa_2 \\
\text{error}_2 &\leftarrow \text{error}_1 \times (1 - {\kappa_2} ^ 2) \\
\end{align*}
}

Performing this calculation with the autocorrelation values
\texttt{3455.53, 3130.76, 2590.90, 2055.80}:
{\relsize{-1}
\begin{align*}
\kappa_0 &\leftarrow 3130.76 \div 3455.53 = 0.906 \\
\text{LP coefficient}_{0~0} &\leftarrow 0.906 \\
\text{error}_0 &\leftarrow 3455.53 \times (1 - {0.906} ^ 2) = 619.107 \\
i &= 1 \\
q_1 &\leftarrow 2590.90 - (0.906 \times 3130.76) = -245.569 \\
\kappa_1 &\leftarrow -245.569 \div 619.107 = -0.397 \\
\text{LP coefficient}_{1~0} &\leftarrow 0.906 - (-0.397 \times 0.906) = 1.266 \\
\text{LP coefficient}_{1~1} &\leftarrow -0.397 \\
\text{error}_1 &\leftarrow 619.107 \times (1 - {-0.397} ^ 2) = 521.530 \\
i &= 2 \\
q_2 &\leftarrow 2055.80 - (1.266 \times 2590.90 + -0.397 \times 3130.76) = 18.632 \\
\kappa_2 &\leftarrow 18.632 \div 521.53 = 0.036 \\
\text{LP coefficient}_{2~0} &\leftarrow 1.266 - (0.036 \times -0.397) = 1.28 \\
\text{LP coefficient}_{2~1} &\leftarrow -0.397 - (0.036 \times 1.266) = -0.443 \\
\text{LP coefficient}_{2~2} &\leftarrow 0.036 \\
\text{error}_2 &\leftarrow 521.53 \times (1 - {0.036} ^ 2) = 520.854 \\
\end{align*}
}
\par
\noindent
With the final result of:
\begin{table}[h]
\begin{tabular}{r|rrr}
order & \multicolumn{3}{c}{LP coefficients} \\
\hline
1 & \texttt{0.906} \\
2 & \texttt{1.266} & \texttt{-0.397} \\
3 & \texttt{1.280} & \texttt{-0.443} & \texttt{0.036} \\
\end{tabular}
\end{table}
\par
\noindent
and error values: \texttt{619.107, 521.530, 520.854}

\clearpage

\subsubsection{Estimating Best Order}
{\relsize{-1}
\ALGORITHM{floating point error values, maximum LPC order, block size, sample count, bits per sample}{the best stimated order value to use}
\SetKwData{BORDER}{best order}
\SetKwData{BSIZE}{best subframe bits}
\SetKwData{ERROR}{error}
\SetKwData{ERRORSCALE}{error scale}
\SetKwFunction{MAX}{max}
\tcc{block size and sample count are typically identical until the last frame
\newline
at which point sample count may be smaller than the requested block size}
\uIf{block size $\leq 192$}{
  QLP precision $\leftarrow 7$\;
}
\uElseIf{block size $\leq 384$}{
  QLP precision $\leftarrow 8$\;
}
\uElseIf{block size $\leq 576$}{
  QLP precision $\leftarrow 9$\;
}
\uElseIf{block size $\leq 1152$}{
  QLP precision $\leftarrow 10$\;
}
\uElseIf{block size $\leq 2304$}{
  QLP precision $\leftarrow 11$\;
}
\uElseIf{block size $\leq 4608$}{
  QLP precision $\leftarrow 12$\;
}
\Else{
  QLP precision $\leftarrow 13$\;
}
\BlankLine
\ERRORSCALE $\leftarrow \frac{\log_e(2) ^ 2}{\text{sample count} \times 2}$\;
\BSIZE $\leftarrow$ maximum floating point\;
\For{i = 0 \emph{\KwTo}maximum LPC order}{
  $order \leftarrow i + 1$\;
  \uIf{$\text{error}_i > 0.0$}{
    $header~bits \leftarrow order \times (\text{bits per sample} + \text{QLP precision})$\;
    $bits~per~residual \leftarrow \MAX\left(\frac{\log_e(error_i \times \text{\ERRORSCALE})}{\log_e(2) \times 2} , 0.0\right)$\;
    $estimated~subframe~bits \leftarrow header~bits + bits~per~residual \times \text{sample count}$\;
    \BlankLine
    \If{$estimated~subframe~bits < \text{\BSIZE}$}{
      \BORDER $\leftarrow order$\;
      \BSIZE $\leftarrow estimated~subframe~bits$\;
    }
  }
  \ElseIf{$\text{error}_i = 0.0$}{
    \Return $order$\;
  }
}
\Return \BORDER\;
\EALGORITHM
}

\clearpage

\subsubsection{Quantizing LP Coefficients}
{\relsize{-1}
\ALGORITHM{LP coefficients, a positive order value, the encoding parameters block size}{QLP coefficients, QLP precision, QLP shift needed}
\SetKwFunction{MIN}{min}
\SetKwFunction{MAX}{max}
\SetKwFunction{ROUND}{round}
\uIf{block size $\leq 192$}{
  QLP precision $\leftarrow 7$\;
}
\uElseIf{block size $\leq 384$}{
  QLP precision $\leftarrow 8$\;
}
\uElseIf{block size $\leq 576$}{
  QLP precision $\leftarrow 9$\;
}
\uElseIf{block size $\leq 1152$}{
  QLP precision $\leftarrow 10$\;
}
\uElseIf{block size $\leq 2304$}{
  QLP precision $\leftarrow 11$\;
}
\uElseIf{block size $\leq 4608$}{
  QLP precision $\leftarrow 12$\;
}
\Else{
  QLP precision $\leftarrow 13$\;
}
\BlankLine
$l \leftarrow $ maximum $|c|$ for $c$ in $\text{LP Coefficient}_{order~0}$ to $\text{LP Coefficient}_{order~order}$\;
QLP shift needed $\leftarrow \text{QLP precision} - \lfloor \log_2(l) \rfloor - 2$\;
\uIf(\tcc*[f]{ensure shift fits into signed 5 bit field}){QLP shift needed $> 2 ^ 4 - 1$}{
  QLP shift needed $\leftarrow 2 ^ 4 - 1$\;
}
\ElseIf{QLP shift needed $< -(2 ^ 4)$}{
  \Return error\tcc*[r]{too much shift required for coefficients}
}

\BlankLine
\tcc{QLP min and max are the smallest and largest QLP coefficients that fit in a signed field that's "QLP precision" bits wide}
QLP max $\leftarrow 2 ^ \text{QLP precision - 1} - 1$\;
QLP min $\leftarrow -(2 ^ \text{QLP precision - 1})$\;
$error \leftarrow 0.0$\;
\eIf{$\text{QLP shift needed} \geq 0$}{
  \For{i = 0 \emph{\KwTo}order + 1}{
    $error \leftarrow error + \text{LP Coefficients}_{order~i} \times 2 ^ \text{QLP shift needed}$\;
    $\text{QLP coefficient}_i \leftarrow \MIN(\MAX(\ROUND(error), \text{QLP min}), \text{QLP max})$\;
    $error \leftarrow error - \text{QLP coefficient}_i$\;
  }
  \Return (QLP coefficients, QLP precision, QLP shift needed)\;
}(\tcc*[f]{negative shifts are not allowed, so shrink coefficients}){
  \For{i = 0 \emph{\KwTo}order + 1}{
    $error \leftarrow error + \text{LP Coefficients}_{order~i} \div 2 ^ \text{QLP shift needed}$\;
    $\text{QLP coefficient}_i \leftarrow \MIN(\MAX(\ROUND(error), \text{QLP min}), \text{QLP max})$\;
    $error \leftarrow error - \text{QLP coefficient}_i$\;
  }
  \Return (QLP coefficients, QLP precision, 0)\;
}
\EALGORITHM
}
\clearpage

For example, given the $\texttt{LP coefficients}_3$ \texttt{1.280, -0.443, 0.036},
an order of \texttt{3} and a block size of \texttt{4096}:
\begin{align*}
\text{QLP precision} &\leftarrow 12 \\
l &\leftarrow 1.280 \\
\text{QLP shift needed} &\leftarrow 12 - \lfloor \log_2(1.280) \rfloor - 2 = 10 \\
\text{QLP max} &\leftarrow 2047 \\
\text{QLP min} &\leftarrow -2048 \\
error &\leftarrow 0.0 \\
i &= 0 \\
error &\leftarrow 0.0 + 1.280 \times 2 ^ {10} = 1310.72 \\
\text{QLP coefficient}_0 &\leftarrow 1311 \\
error &\leftarrow 1310.72 - 1311 = -0.28 \\
i &= 1 \\
error &\leftarrow -0.28 + -0.443 \times 2 ^ {10} = -453.912 \\
\text{QLP coefficient}_1 &\leftarrow -454 \\
error &\leftarrow -453.912 - -454 = 0.088 \\
i &= 2 \\
error &\leftarrow 0.088 + 0.036 \times 2 ^ {10} = 36.952 \\
\text{QLP coefficient}_2 &\leftarrow 37 \\
error &\leftarrow 36.952 - 37 = -0.048 \\
\end{align*}
\par
\noindent
Resulting in the QLP coefficients \texttt{1311, -454, 37},
a QLP precision of \texttt{12} and a QLP shift needed of \texttt{10}.
These values are inserted directly into a desired QLP subframe header
and are used to calculate its residuals.

\clearpage

\subsection{Encoding an LPC Subframe}
\ALGORITHM{signed subframe samples, subframe order, QLP coefficients, QLP precision, QLP shift needed, subframe's bits per sample, wasted BPS}{an LPC subframe}
\WRITE 0 in 1 unsigned bit\tcc*[r]{pad}
\WRITE 1 in 1 unsigned bit\tcc*[r]{subframe type}
\WRITE (subframe order - 1) in 5 unsigned bits\;
\eIf{$wasted~BPS > 0$}{
  \WRITE 1 in 1 unsigned bit\;
  \WUNARY (wasted BPS - 1) with stop bit 1\;
}{
  \WRITE 0 in 1 unsigned bit\;
}
\For(\tcc*[f]{warm-up samples}){i = 0 \emph{\KwTo}subframe order}{
  \WRITE $sample_i$ in (bits per sample) signed bits\;
}
\WRITE (QLP precision - 1) in 4 unsigned bits\;
\WRITE QLP shift needed in 5 signed bits\;
\For(\tcc*[f]{QLP coefficients}){i = 0 \emph{\KwTo}subframe order}{
  \WRITE $\text{QLP coefficient}_i$ in (QLP precision) signed bits\;
}
\BlankLine
\For(\tcc*[f]{calculate signed residuals}){i = 0 \emph{\KwTo}sample count - order}{
  $residual_i \leftarrow sample_{i + order} - \left \lfloor \frac{\overset{order - 1}{\underset{j = 0}{\sum}} \text{QLP coefficient}_j \times sample_{i + order - j - 1} }{2 ^ \text{QLP shift needed}} \right \rfloor$
}
write encoded residual block based on signed residual values\;
\Return an LPC subframe\;
\EALGORITHM
\begin{figure}[h]
\includegraphics{figures/flac/lpc2.pdf}
\end{figure}

\clearpage

\subsubsection{LPC Subframe Residuals Calculcation Example}
\begin{tabular}{rl}
samples : & \texttt{18, 20, 26, 24, 24, 23, 21, 24, 23, 20} \\
order : & \texttt{3} \\
QLP precision : &\texttt{12} \\
QLP shift needed : & \texttt{10} \\
QLP coefficients : & \texttt{1311, -454, 37} \\
\end{tabular}
\newline
\begin{align*}
residual_0 &\leftarrow 24 - \left\lfloor\frac{1311 \times 26 + -454 \times 20 + 37 \times 18}{2 ^ {10}}\right\rfloor = 24 - \left\lfloor\frac{25672}{1024}\right\rfloor = 24 - 25 = -1 \\
residual_1 &\leftarrow 24 - \left\lfloor\frac{1311 \times 24 + -454 \times 26 + 37 \times 20}{2 ^ {10}}\right\rfloor = 24 - \left\lfloor\frac{20400}{1024}\right\rfloor = 24 - 19 = 5 \\
residual_2 &\leftarrow 23 - \left\lfloor\frac{1311 \times 24 + -454 \times 24 + 37 \times 26}{2 ^ {10}}\right\rfloor = 23 - \left\lfloor\frac{21530}{1024}\right\rfloor = 23 - 21 = 2 \\
residual_3 &\leftarrow 21 - \left\lfloor\frac{1311 \times 23 + -454 \times 24 + 37 \times 24}{2 ^ {10}}\right\rfloor = 21 - \left\lfloor\frac{20145}{1024}\right\rfloor = 21 - 19 = 2 \\
residual_4 &\leftarrow 24 - \left\lfloor\frac{1311 \times 21 + -454 \times 23 + 37 \times 24}{2 ^ {10}}\right\rfloor = 24 - \left\lfloor\frac{17977}{1024}\right\rfloor = 24 - 17 = 7 \\
residual_5 &\leftarrow 23 - \left\lfloor\frac{1311 \times 24 + -454 \times 21 + 37 \times 23}{2 ^ {10}}\right\rfloor = 23 - \left\lfloor\frac{22781}{1024}\right\rfloor = 23 - 22 = 1 \\
residual_6 &\leftarrow 20 - \left\lfloor\frac{1311 \times 23 + -454 \times 24 + 37 \times 21}{2 ^ {10}}\right\rfloor = 20 - \left\lfloor\frac{20034}{1024}\right\rfloor = 20 - 19 = 1
\end{align*}
Leading to a final set of 7 residual values: \texttt{-1, 5, 2, 2, 7, 1, 1}.
Encoding them to a residual block, our final LPC subframe is:
\begin{figure}[h]
\includegraphics{figures/flac/lpc-parse2.pdf}
\end{figure}

\subsection{Calculating Frame CRC-16}
CRC-16 is used to checksum the entire FLAC frame, including the header
and any padding bits after the final subframe.
Given a byte of input and the previous CRC-16 checksum,
or 0 as an initial value, the current checksum can be calculated as follows:
\begin{equation}
\text{checksum}_i = \text{CRC16}(byte\xor(\text{checksum}_{i - 1} \gg 8 ))\xor(\text{checksum}_{i - 1} \ll 8)
\end{equation}
\par
\noindent
and the checksum is always truncated to 16-bits.
\begin{table}[h]
{\relsize{-3}\ttfamily
\begin{tabular}{|r||r|r|r|r|r|r|r|r|r|r|r|r|r|r|r|r|}
\hline
 & 0x?0 & 0x?1 & 0x?2 & 0x?3 & 0x?4 & 0x?5 & 0x?6 & 0x?7 & 0x?8 & 0x?9 & 0x?A & 0x?B & 0x?C & 0x?D & 0x?E & 0x?F \\
\hline
0x0? & 0000 & 8005 & 800f & 000a & 801b & 001e & 0014 & 8011 & 8033 & 0036 & 003c & 8039 & 0028 & 802d & 8027 & 0022 \\
0x1? & 8063 & 0066 & 006c & 8069 & 0078 & 807d & 8077 & 0072 & 0050 & 8055 & 805f & 005a & 804b & 004e & 0044 & 8041 \\
0x2? & 80c3 & 00c6 & 00cc & 80c9 & 00d8 & 80dd & 80d7 & 00d2 & 00f0 & 80f5 & 80ff & 00fa & 80eb & 00ee & 00e4 & 80e1 \\
0x3? & 00a0 & 80a5 & 80af & 00aa & 80bb & 00be & 00b4 & 80b1 & 8093 & 0096 & 009c & 8099 & 0088 & 808d & 8087 & 0082 \\
0x4? & 8183 & 0186 & 018c & 8189 & 0198 & 819d & 8197 & 0192 & 01b0 & 81b5 & 81bf & 01ba & 81ab & 01ae & 01a4 & 81a1 \\
0x5? & 01e0 & 81e5 & 81ef & 01ea & 81fb & 01fe & 01f4 & 81f1 & 81d3 & 01d6 & 01dc & 81d9 & 01c8 & 81cd & 81c7 & 01c2 \\
0x6? & 0140 & 8145 & 814f & 014a & 815b & 015e & 0154 & 8151 & 8173 & 0176 & 017c & 8179 & 0168 & 816d & 8167 & 0162 \\
0x7? & 8123 & 0126 & 012c & 8129 & 0138 & 813d & 8137 & 0132 & 0110 & 8115 & 811f & 011a & 810b & 010e & 0104 & 8101 \\
0x8? & 8303 & 0306 & 030c & 8309 & 0318 & 831d & 8317 & 0312 & 0330 & 8335 & 833f & 033a & 832b & 032e & 0324 & 8321 \\
0x9? & 0360 & 8365 & 836f & 036a & 837b & 037e & 0374 & 8371 & 8353 & 0356 & 035c & 8359 & 0348 & 834d & 8347 & 0342 \\
0xA? & 03c0 & 83c5 & 83cf & 03ca & 83db & 03de & 03d4 & 83d1 & 83f3 & 03f6 & 03fc & 83f9 & 03e8 & 83ed & 83e7 & 03e2 \\
0xB? & 83a3 & 03a6 & 03ac & 83a9 & 03b8 & 83bd & 83b7 & 03b2 & 0390 & 8395 & 839f & 039a & 838b & 038e & 0384 & 8381 \\
0xC? & 0280 & 8285 & 828f & 028a & 829b & 029e & 0294 & 8291 & 82b3 & 02b6 & 02bc & 82b9 & 02a8 & 82ad & 82a7 & 02a2 \\
0xD? & 82e3 & 02e6 & 02ec & 82e9 & 02f8 & 82fd & 82f7 & 02f2 & 02d0 & 82d5 & 82df & 02da & 82cb & 02ce & 02c4 & 82c1 \\
0xE? & 8243 & 0246 & 024c & 8249 & 0258 & 825d & 8257 & 0252 & 0270 & 8275 & 827f & 027a & 826b & 026e & 0264 & 8261 \\
0xF? & 0220 & 8225 & 822f & 022a & 823b & 023e & 0234 & 8231 & 8213 & 0216 & 021c & 8219 & 0208 & 820d & 8207 & 0202 \\
\hline
\end{tabular}
}
\end{table}
\par
\noindent
For example, given the frame bytes:
\texttt{FF F8 CC 1C 00 C0 EB 00 00 00 00 00 00 00 00},
the frame's CRC-16 can be calculated:
{\relsize{-2}
\begin{align*}
\CRCSIXTEEN{0}{0xFF}{0x0000}{0xFF}{0x0000}{0x0202} \\
\CRCSIXTEEN{1}{0xF8}{0x0202}{0xFA}{0x0200}{0x001C} \\
\CRCSIXTEEN{2}{0xCC}{0x001C}{0xCC}{0x1C00}{0x1EA8} \\
\CRCSIXTEEN{3}{0x1C}{0x1EA8}{0x02}{0xA800}{0x280F} \\
\CRCSIXTEEN{4}{0x00}{0x280F}{0x28}{0x0F00}{0x0FF0} \\
\CRCSIXTEEN{5}{0xC0}{0x0FF0}{0xCF}{0xF000}{0xF2A2} \\
\CRCSIXTEEN{6}{0xEB}{0xF2A2}{0x19}{0xA200}{0x2255} \\
\CRCSIXTEEN{7}{0x00}{0x2255}{0x22}{0x5500}{0x55CC} \\
\CRCSIXTEEN{8}{0x00}{0x55CC}{0x55}{0xCC00}{0xCDFE} \\
\CRCSIXTEEN{9}{0x00}{0xCDFE}{0xCD}{0xFE00}{0x7CAD} \\
\CRCSIXTEEN{10}{0x00}{0x7CAD}{0x7C}{0xAD00}{0x2C0B} \\
\CRCSIXTEEN{11}{0x00}{0x2C0B}{0x2C}{0x0B00}{0x8BEB} \\
\CRCSIXTEEN{12}{0x00}{0x8BEB}{0x8B}{0xEB00}{0xE83A} \\
\CRCSIXTEEN{13}{0x00}{0xE83A}{0xE8}{0x3A00}{0x3870} \\
\CRCSIXTEEN{14}{0x00}{0x3870}{0x38}{0x7000}{0xF093} \\
\intertext{Thus, the next two bytes after the final subframe should be
\texttt{0xF0} and \texttt{0x93}.
Again, when the checksum bytes are run through the checksumming procedure:}
\CRCSIXTEEN{15}{0xF0}{0xF093}{0x00}{0x9300}{0x9300} \\
\CRCSIXTEEN{16}{0x93}{0x9300}{0x00}{0x0000}{0x0000}
\end{align*}
the result will also always be 0, just as in the CRC-8.
}
