\chapter{Free Lossless Audio Codec}
FLAC compresses PCM audio data losslessly using predictors and a
residual.
FLACs contain checksumming to verify their integrity, contain comment tags for
metadata and are streamable.

Except for the contents of the VORBIS\_COMMENT metadata block, everything in FLAC is big-endian.

\section{the FLAC File Stream}
\begin{figure}[h]
\includegraphics{figures/flac_stream.pdf}
\end{figure}
\par
\noindent
``Last'' is 0 when there are no additional metadata blocks and 1 when
it is the final block before the the audio frames.
``Block Length'' is the size of the metadata block data to follow,
not including the header.
\begin{figure}[h]
\begin{tabular}{| r | l |}
\hline
Block Type & Block \\
\hline
\texttt{0} & STREAMINFO \\
\texttt{1} & PADDING \\
\texttt{2} & APPLICATION \\
\texttt{3} & SEEKTABLE \\
\texttt{4} & VORBIS\_COMMENT \\
\texttt{5} & CUESHEET \\
\texttt{6} & PICTURE \\
\texttt{7-126} & reserved \\
\texttt{127} & invalid \\
\hline
\end{tabular}
\end{figure}

\pagebreak

\section{FLAC Metadata Blocks}

\subsection{STREAMINFO}
\begin{figure}[h]
\includegraphics{figures/flac_streaminfo.pdf}
\end{figure}

\subsection{PADDING}

PADDING is simply a block full of NULL (\texttt{0x00}) bytes.
Its purpose is to provide extra metadata space within the FLAC file.
By having a padding block, other metadata blocks can be grown or
shrunk without having to rewrite the entire FLAC file by removing or
adding space to the padding.


\subsection{APPLICATION}
\begin{figure}[h]
\includegraphics{figures/flac_application.pdf}
\end{figure}
\noindent
APPLICATION is a general-purpose metadata block used by a variety of
different programs.
Its contents are defined by the ASCII Application ID value.

\subsection{SEEKTABLE}
\begin{figure}[h]
\includegraphics{figures/flac_seektable.pdf}
\end{figure}

\subsection{VORBIS\_COMMENT}

\subsection{PICTURE}
