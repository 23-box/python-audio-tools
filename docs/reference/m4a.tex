\chapter{M4A}

M4A is typically AAC audio in a QuickTime container stream, though
it may also contain other formats such as MPEG-1 audio.

\section{the QuickTime file stream}
\begin{figure}[h]
\includegraphics{figures/m4a_quicktime.pdf}
\end{figure}
\par
\noindent
Unlike other chunked formats such as RIFF WAVE, QuickTime's atom chunks
may be containers for other atoms.  All of its fields are big-endian.
\subsection{a QuickTime atom}
\begin{figure}[h]
\includegraphics{figures/m4a_atom.pdf}
\end{figure}
Atom Type is an ASCII string.
Atom Length is the length of the entire atom, including the header.
If Atom Length is 0, the atom continues until the end of the file.
If Atom Length is 1, the atom has an extended size.  This means
there is a 64-bit length field immediately after the header which is
the atom's actual size.
\begin{figure}[h]
\includegraphics{figures/m4a_atom2.pdf}
\end{figure}
\subsection{Container atoms}
There is no flag or field to tell a QuickTime parser which
of its atoms are containers and which ones are not.
If an atom is known to be a container, one can treat its Atom Data
as a QuickTime stream and parse it in a recursive fashion.
\section{M4A atoms}
A typical M4A begins with an `ftyp' atom indicating its file type,
followed by a `moov' atom containing a copious amount of file metadata,
an optional `free' atom with nothing but empty space
(so that metadata can be resized, if necessary) and an `mdat' atom
containing the song data itself.
\begin{figure}[h]
\parpic[r]{
\includegraphics{figures/m4a_atoms.pdf}
}
\subsection{the ftyp atom}
\includegraphics{figures/m4a_ftyp.pdf}
\par
The `Major Brand' and `Compatible Brand' fields are ASCII strings.
`Major Brand Version' is an integer.

\subsection{the mvhd atom}
\includegraphics{figures/m4a_mvhd.pdf}
\par
If `Version' is 0, `Created Mac UTC Date', `Modified Mac UTC Date' and
`Duration' are 32-bit fields.  If it is 1, they are 64-bit fields.
\end{figure}

\pagebreak

\subsection{the tkhd atom}
\par
\begin{figure}[h]
\includegraphics{figures/m4a_tkhd.pdf}
\end{figure}
\par
\noindent
As with `mvhd', if `Version' is 0, `Created Mac UTC Date',
`Modified Mac UTC Date' and `Duration' are 32-bit fields.
If it is 1, they are 64-bit fields.

\subsection{the mdhd atom}

