%This work is licensed under the
%Creative Commons Attribution-Share Alike 3.0 United States License.
%To view a copy of this license, visit
%http://creativecommons.org/licenses/by-sa/3.0/us/ or send a letter to
%Creative Commons,
%171 Second Street, Suite 300,
%San Francisco, California, 94105, USA.

\chapter{M4A}
\label{m4a}
M4A is typically AAC audio in a QuickTime container stream, though
it may also contain other formats such as MPEG-1 audio.

\section{the QuickTime file stream}
\begin{figure}[h]
\includegraphics{figures/m4a_quicktime.pdf}
\end{figure}
\par
\noindent
Unlike other chunked formats such as RIFF WAVE, QuickTime's atom chunks
may be containers for other atoms.  All of its fields are big-endian.
\subsection{a QuickTime atom}
\begin{figure}[h]
\includegraphics{figures/m4a_atom.pdf}
\end{figure}
Atom Type is an ASCII string.
Atom Length is the length of the entire atom, including the header.
If Atom Length is 0, the atom continues until the end of the file.
If Atom Length is 1, the atom has an extended size.  This means
there is a 64-bit length field immediately after the header which is
the atom's actual size.
\begin{figure}[h]
\includegraphics{figures/m4a_atom2.pdf}
\end{figure}
\subsection{Container atoms}
There is no flag or field to tell a QuickTime parser which
of its atoms are containers and which ones are not.
If an atom is known to be a container, one can treat its Atom Data
as a QuickTime stream and parse it in a recursive fashion.
\section{M4A atoms}
A typical M4A begins with an `ftyp' atom indicating its file type,
followed by a `moov' atom containing a copious amount of file metadata,
an optional `free' atom with nothing but empty space
(so that metadata can be resized, if necessary) and an `mdat' atom
containing the song data itself.
\begin{figure}[h]
\parpic[r]{
\includegraphics{figures/m4a_atoms.pdf}
}
\subsection{the ftyp atom}
\includegraphics{figures/m4a_ftyp.pdf}
\par
The `Major Brand' and `Compatible Brand' fields are ASCII strings.
`Major Brand Version' is an integer.

\subsection{the mvhd atom}
\includegraphics{figures/m4a_mvhd.pdf}
\par
If `Version' is 0, `Created Mac UTC Date', `Modified Mac UTC Date' and
`Duration' are 32-bit fields.  If it is 1, they are 64-bit fields.
\end{figure}

\pagebreak

\subsection{the tkhd atom}
\par
\begin{figure}[h]
\includegraphics{figures/m4a_tkhd.pdf}
\end{figure}
\par
\noindent
As with `mvhd', if `Version' is 0, `Created Mac UTC Date',
`Modified Mac UTC Date' and `Duration' are 32-bit fields.
If it is 1, they are 64-bit fields.

\subsection{the mdhd atom}

The \texttt{mdhd} atom contains track information such as samples-per-second,
track length and creation/modification times.

\begin{figure}[h]
\includegraphics{figures/m4a_mdhd.pdf}
\end{figure}
\par
\noindent
As with `mvhd', if `Version' is 0, `Created Mac UTC Date',
`Modified Mac UTC Date' and `Track Length' are 32-bit fields.
If it is 1, they are 64-bit fields.

\pagebreak

\subsection{the hdlr atom}

\begin{figure}[h]
\includegraphics{figures/m4a_hdlr.pdf}
\end{figure}
\par
\noindent
`QuickTime flags', `QuickTime flags mask' and `Component Name Length'
are integers.  The rest are ASCII strings.

\subsection{the smhd atom}

\begin{figure}[h]
\includegraphics{figures/m4a_smhd.pdf}
\end{figure}

\subsection{the dref atom}

\begin{figure}[h]
\includegraphics{figures/m4a_dref.pdf}
\end{figure}

\pagebreak

\subsection{the stsd atom}

\begin{figure}[h]
\includegraphics{figures/m4a_stsd.pdf}
\end{figure}

\subsection{the mp4a atom}

The \texttt{mp4a} atom contains information such as the number of channels
and bits-per-sample.  It can be found in the \texttt{stsd} atom.

\begin{figure}[h]
\includegraphics{figures/m4a_mp4a.pdf}
\end{figure}

\pagebreak

\subsection{the stts atom}

\begin{figure}[h]
\includegraphics{figures/m4a_stts.pdf}
\end{figure}

\subsection{the stsc atom}

\begin{figure}[h]
\includegraphics{figures/m4a_stsc.pdf}
\end{figure}

\subsection{the stsz atom}

\begin{figure}[h]
\includegraphics{figures/m4a_stsz.pdf}
\end{figure}

\pagebreak

\subsection{the stco atom}

\begin{figure}[h]
\includegraphics{figures/m4a_stco.pdf}
\end{figure}
\par
\noindent
Offsets point to an absolute position in the M4A file of AAC data in
the \texttt{mdat} atom.  Therefore, if the \texttt{moov} atom size changes
(which can happen by writing new metadata in its \texttt{meta} child atom)
the \texttt{mdat} atom may move and these obsolute offsets will change.
In that instance, they \textbf{must}
be re-adjusted in the \texttt{stco} atom or the file may become unplayable.

\subsection{the meta atom}

\begin{figure}[h]
\parpic[r]{
\includegraphics{figures/m4a_meta_atoms.pdf}
}
\includegraphics{figures/m4a_meta.pdf}
The atoms within the \texttt{ilst} container are all containers themselves,
each with a \texttt{data} atom of its own.
Notice that many of \texttt{ilst}'s sub-atoms begin with the
non-ASCII 0xA9 byte.

\includegraphics{figures/m4a_data.pdf}
\par
\noindent
Text data atoms have a type of 1.
Binary data atoms typically have a type of 0.
\end{figure}
\begin{table}[h]
{\relsize{-1}
\begin{tabular}{|r|l||r|l||r|l|}
\hline
Atom & Description & Atom & Description & Atom & Description \\
\hline
\texttt{alb} & Album Nam &
\texttt{ART} & Track Artist &
\texttt{cmt} & Comments \\
\texttt{covr} & Cover Image &
\texttt{cpil} & Compilation &
\texttt{cprt} & Copyright \\
\texttt{day} & Year &
\texttt{disk} & Disc Number &
\texttt{gnre} & Genre \\
\texttt{grp} & Grouping &
\texttt{----} & iTunes-specific &
\texttt{nam} & Track Name \\
\texttt{rtng} & Rating &
\texttt{tmpo} & BMP &
\texttt{too} & Encoder \\
\texttt{trkn} & Track Number &
\texttt{wrt} & Composer &
& \\
\hline
\end{tabular}
\caption{Known \texttt{ilst} sub-atoms}
}
\end{table}

\pagebreak

\subsubsection{the trkn sub-atom}
\texttt{trkn} is a binary sub-atom of \texttt{meta} which contains
the track number.
\begin{figure}[h]
\includegraphics{figures/m4a_trkn.pdf}
\end{figure}

\subsubsection{the disk sub-atom}
\texttt{disk} is a binary sub-atom of \texttt{meta} which contains
the disc number.
For example, if the track belongs to the first disc in a set of
two discs, the sub-atom will contain that information.
\begin{figure}[h]
\includegraphics{figures/m4a_disk.pdf}
\end{figure}
