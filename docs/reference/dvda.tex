\chapter{DVD-Audio}
DVD-Audio is a format for delivering hi-fidelity, multichannel
audio on DVD media.
A DVD-Audio's \texttt{AUDIO\_TS} directory contains the
relevent data needed for decoding, spread into a lot of files
whose names are more than little cryptic at first glance.

Unlike CD audio, which is little more than a set of
1 to 99 audio tracks, a DVD-Audio disc contains
one or more titlesets.
Each titleset contains one or more titles, and each
title contains one or more tracks.
\begin{figure}[h]
\includegraphics{figures/dvda_layout.pdf}
\end{figure}
\par
\noindent
Typically, a DVD-Audio disc will contain two titlesets,
one for audio and the other for video - which we can ignore.
The first titleset will often contain two titles,
one for 2 channel audio and the other for 5.1 channel audio.
Each title will usually contain a consistent number of tracks
as MLP or PCM encoded audio.
\par
With this in mind, we can now make some sense of
the \texttt{AUDIO\_TS} directory's contents:
\vskip .25in
$\texttt{\huge{AUDIO\_TS.IFO}}$
\hfill
information about the disc, including the number of titlesets
\vskip .25in
$\texttt{\huge{ATS\_}}\underbrace{\texttt{\huge{01}}}_{Titleset}\texttt{\huge{\_0.IFO}}$
\hfill
information about all the titles in a given titleset
\vskip .25in
$\texttt{\huge{ATS\_}}\underbrace{\texttt{\huge{01}}}_{Titleset}\texttt{\huge{\_}}\underbrace{\texttt{\huge{1}}}_{AOB~\#}\texttt{\huge{.AOB}}$
\hfill
audio data for one or more tracks in a given titleset
\vskip .25in
\par
All are binary files containing one or more, 2048 byte sectors.

\clearpage

\section{AUDIO\_TS.IFO}
Known as the ``Audio Manager'' or ``AMG'',
this is primarily a container of pointers to
other files on disc.
However, for our purposes, we're only interested
in the \VAR{Audio Titleset Count} value.
\begin{figure}[h]
\includegraphics{figures/dvda_audio_ts_ifo.pdf}
\end{figure}
\section{ATS\_XX\_0.IFO}
Known as the ``Audio Titleset Information'' or ``ATSI'',
there is one of these files per \VAR{Audio Titleset Count}.
Each is typically 4096 bytes long (2 sectors).
\begin{figure}[h]
\includegraphics{figures/dvda_ats_xx_0.pdf}
\end{figure}
\par
\noindent
\VAR{Title Table Offset} is the address of the title's
information table, from the start of the second sector.
For example, given the \VAR{Title Table Offset} value of
\texttt{0x18}, we seek to address \texttt{0x818}
($\texttt{0x800} + \texttt{0x18} = \texttt{0x818}$)
to read the title's information.

\clearpage

\subsection{Title Table}

For each title in the titleset, there is a title table.

\begin{figure}[h]
\includegraphics{figures/dvda_ats_title.pdf}
\end{figure}
\par
\noindent
\VAR{Track Count} is the total number of tracks in the title.
\VAR{Title PTS Length} is the total length of the title in
PTS ticks.
There are 90000 PTS ticks per second.
\VAR{Sector Pointers Offset} is the offset value of the
sector pointers table, relative to the start of the title table.
For example, if our \VAR{Title Table Offset} value is
\texttt{0x18} and the \VAR{Sector Pointers Offset} value is
\texttt{0x1F0}, we seek to address \texttt{0xA08}
to reach the sector pointers table
($\texttt{0x818} + \texttt{0x1F0} = \texttt{0xA08}$).
\par
There is one timestamp value per track.
\VAR{Initial PTS Index} and \VAR{Track PTS Length} values
are the offset and length of the track in PTS ticks, respectively.

\subsection{Sector Pointers Table}

For each title in the titleset, there is a sector pointers table.

\begin{figure}[h]
\includegraphics{figures/dvda_ats_sectors.pdf}
\end{figure}
\par
\noindent
There is one sector pointers value per track,
as indicated in the title table.
