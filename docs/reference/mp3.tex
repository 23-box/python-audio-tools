\chapter{MP3}
MP3 is the de-facto standard for lossy audio.
It is little more than a series of MPEG frames with an
optional ID3v2 metadata header and optional ID3v1 metadata
footer.

MP3 decoders are assumed to be very tolerant of anything in
the stream that doesn't look like an MPEG frame, ignoring such
junk until the next frame is found.
Since MP3 files have no standard container format in which
non-MPEG data can be placed, metadata such as ID3 tags are often
made `sync-safe' by formatting them in a way that decoders won't
confuse tags for MPEG frames.
\section{the MP3 File Stream}
\begin{figure}[h]
\includegraphics{figures/mp3_stream.pdf}
\end{figure}
\begin{table}[h]
\begin{tabular}{|c||l||l||r|r|r||l|}
\hline
& & & \multicolumn{3}{c||}{Sample Rate} & \\
bits & MPEG ID & Description & MPEG-1 & MPEG-2 & MPEG-2.5 & Channels \\
\hline
\texttt{00} & MPEG-2.5 & reserved & 44100 & 22050 & 11025 & Stereo \\
\texttt{01} & reserved & Layer III & 48000 & 24000 & 12000 & Joint stereo \\
\texttt{10} & MPEG-2 & Layer II & 32000 & 16000 & 8000 & Dual channel stereo\\
\texttt{11} & MPEG-1 & Layer I & reserved & reserved & reserved & Mono \\
\hline
\end{tabular}
\end{table}
\par
\noindent
Layer I frames always contain 384 samples.
Layer II and Layer III frames always contain 1152 samples.
If the `Protection' bit is set, the frame header is followed by a
16 bit CRC.

\pagebreak

\begin{table}[h]
{\relsize{-2}
\begin{tabular}{|c||r|r|r|r|r|}
\hline
& MPEG-1 & MPEG-1 & MPEG-1 & MPEG-2 & MPEG-2 \\
bits & Layer-1 & Layer-2 & Layer-3 & Layer-1 & Layer-2/3 \\
\hline
\texttt{0000} & free & free & free & free & free \\
\texttt{0001} & 32 & 32 & 32 & 32 & 8 \\
\texttt{0010} & 64 & 48 & 40 & 48 & 16 \\
\texttt{0011} & 96 & 56 & 48 & 56 & 24 \\
\texttt{0100} & 128 & 64 & 56 & 64 & 32 \\
\texttt{0101} & 160 & 80 & 64 & 80 & 40 \\
\texttt{0110} & 192 & 96 & 80 & 96 & 48 \\
\texttt{0111} & 224 & 112 & 96 & 112 & 56 \\
\texttt{1000} & 256 & 128 & 112 & 128 & 64 \\
\texttt{1001} & 288 & 160 & 128 & 144 & 80 \\
\texttt{1010} & 320 & 192 & 160 & 160 & 96 \\
\texttt{1011} & 352 & 224 & 192 & 176 & 112 \\
\texttt{1100} & 384 & 256 & 224 & 192 & 128 \\
\texttt{1101} & 416 & 320 & 256 & 224 & 144 \\
\texttt{1110} & 448 & 384 & 320 & 256 & 160 \\
\texttt{1111} & bad & bad & bad & bad & bad \\
\hline
\end{tabular}
}
\caption{Bitrate in 1000 bits per second}
\end{table}
To find the total size of an MPEG frame, use one of the following
formulas:
\begin{align}
\intertext{Layer I:}
\text{Byte Length} &= \left ( \frac{12 \times \text{Bitrate}}{\text{Sample Rate}} + \text{Pad} \right ) \times 4 \\
\intertext{Layer II/III:}
\text{Byte Length} &= \frac{144 \times \text{Bitrate}}{\text{Sample Rate}} + Pad
\end{align}
For example, an MPEG-1 Layer III frame with a sampling rate of 44100,
a bitrate of 128kbps and a set pad bit is 418 bytes long, including the header.
\begin{equation}
\frac{144 \times 128000}{44100} + 1 = 418
\end{equation}

\subsection{the Xing header}

An MP3 frame header contains the track's sampling rate,
bits-per-sample and number of channels.
However, because MP3 files are little more than
concatenated MPEG frames, there is no obvious place to
store the track's total length.
Since the length of each frame is a constant number of samples,
one can calculate the track length by counting the number of frames.
This method is the most accurate but is also quite slow.

For MP3 files in which all frames have the same bitrate
- also known as constant bitrate, or CBR files -
one can divide the total size of file (minus any ID3 headers/footers),
by the bitrate to determine its length.
If an MP3 file has no Xing header in its first frame,
one can assume it is CBR.

An MP3 file that does contain a Xing header in its first frame
can be assumed to be variable bitrate, or VBR.
In that case, the rate of the first frame cannot be used as a
basis to calculate the length of the entire file.
Instead, one must use the information from the Xing header
which contains that length.

All of the fields within a Xing header are big-endian.
\begin{figure}[h]
\includegraphics{figures/mp3_xing.pdf}
\end{figure}

\section{ID3v1 tags}
ID3v1 tags are very simple metadata tags appended to an MP3 file.
All of the fields are fixed length and the text encoding is
undefined.
There are two versions of ID3v1 tags.
ID3v1.1 has a track number field as a 1 byte value
at the end of the comment field.
If the byte just before the end is not null (0x00),
assume we're dealing with a classic ID3v1 tag without a
track number.

\subsection{ID3v1}

\begin{figure}[h]
\includegraphics{figures/mp3_id3v1.pdf}
\end{figure}

\subsection{ID3v1.1}

\begin{figure}[h]
\includegraphics{figures/mp3_id3v11.pdf}
\end{figure}

\section{ID3v2 tags}

The ID3v2 tag was invented to address the deficiencies in the original
ID3v1 tag.
ID3v2 comes in three similar but not entirely compatible variants:
ID3v2.2, ID3v2.3 and ID3v2.4.
All of its fields are big-endian.

\begin{figure}[h]
\includegraphics{figures/mp3_id3v2_stream.pdf}
\end{figure}

\subsection{ID3v2.2}

\subsubsection{ID3v2.2 header}

\begin{figure}[h]
\includegraphics{figures/mp3_id3v22_header.pdf}
\end{figure}
\par
\noindent
The single Size field is split by NULL (0x00) bytes in order to make
it `sync-safe'.  That is, no possible size value will result in a false
MP3 frame sync (11 bits set in a row).

\subsubsection{ID3v2.2 frame}

\begin{figure}[h]
\includegraphics{figures/mp3_id3v22_frame.pdf}
\end{figure}
\par
\noindent
Frame ID's that begin with the letter `T' (0x54) are text frames.
These have an additional text encoding byte before the actual
text data.
All text strings may be terminated by a null character
(0x00 or 0x0000, depending on the encoding).

\pagebreak

\begin{figure}[h]
\includegraphics{figures/mp3_id3v22_textframe.pdf}
\begin{tabular}{r|l}
Encoding Byte & Text Encoding \\
\hline
\texttt{0x00} & ISO-8859-1 \\
\texttt{0x01} & UCS-16 \\
\end{tabular}
\end{figure}

\subsubsection{ID3v2.2 frame IDs}
\begin{table}[h]
{\relsize{-2}
\begin{tabular}{|c|l||c|l|}
\hline
ID & Description & ID & Description \\
\hline
\texttt{BUF} & Recommended buffer size & \texttt{CNT} & Play counter \\
\texttt{COM} & Comments & \texttt{CRA} & Audio encryption \\
\texttt{CRM} & Encrypted meta frame & \texttt{ETC} & Event timing codes \\
\texttt{EQU} & Equalization & \texttt{GEO} & General encapsulated object \\
\texttt{IPL} & Involved people list & \texttt{LNK} & Linked information \\
\texttt{MCI} & Music CD Identifier & \texttt{MLL} & MPEG location lookup table \\
\texttt{PIC} & Attached picture & \texttt{POP} & Popularimeter \\
\texttt{REV} & Reverb & \texttt{RVA} & Relative volume adjustment \\
\texttt{SLT} & Synchronized lyric/text & \texttt{STC} & Synced tempo codes \\
\texttt{TAL} & Album/Movie/Show title & \texttt{TBP} & BPM (Beats Per Minute) \\
\texttt{TCM} & Composer & \texttt{TCO} & Content type \\
\texttt{TCR} & Copyright message & \texttt{TDA} & Date \\
\texttt{TDY} & Playlist delay & \texttt{TEN} & Encoded by \\
\texttt{TFT} & File type & \texttt{TIM} & Time \\
\texttt{TKE} & Initial key & \texttt{TLA} & Language(s) \\
\texttt{TLE} & Length & \texttt{TMT} & Media type \\
\texttt{TOA} & Original artist(s)/performer(s) & \texttt{TOF} & Original filename \\
\texttt{TOL} & Original Lyricist(s)/text writer(s) & \texttt{TOR} & Original release year \\
\texttt{TOT} & Original album/Movie/Show title & \texttt{TP1} & Lead artist(s)/performer(s)/Soloist(s)/Performing group \\
\texttt{TP2} & Band/Orchestra/Accompaniment & \texttt{TP3} & Conductor/Performer refinement \\
\texttt{TP4} & Interpreted, remixed, or otherwise modified by & \texttt{TPA} & Part of a set \\
\texttt{TPB} & Publisher & \texttt{TRC} & ISRC (International Standard Recording Code) \\
\texttt{TRD} & Recording dates & \texttt{TRK} & Track number/Position in set \\
\texttt{TSI} & Size & \texttt{TSS} & Software/hardware and settings used for encoding \\
\texttt{TT1} & Content group description & \texttt{TT2} & Title/Songname/Content description \\
\texttt{TT3} & Subtitle/Description refinement & \texttt{TXT} & Lyricist/text writer \\
\texttt{TXX} & User defined text information frame & \texttt{TYE} & Year \\
\texttt{UFI} & Unique file identifier & \texttt{ULT} & Unsychronized lyric/text transcription \\
\texttt{WAF} & Official audio file webpage & \texttt{WAR} & Official artist/performer webpage \\
\texttt{WAS} & Official audio source webpage & \texttt{WCM} & Commercial information \\
\texttt{WCP} & Copyright/Legal information & \texttt{WPB} & Publishers official webpage \\
\texttt{WXX} & User defined URL link frame & & \\
\hline
\end{tabular}
}
\end{table}

\pagebreak

\subsubsection{ID3v2.2 PIC frame}

`PIC' frames are attached pictures.
This allows an ID3v2.2 tag to contain a JPEG or PNG image,
typically of album artwork which can be displayed to the user
when the track is played.

\begin{figure}[h]
\includegraphics{figures/mp3_id3v22_pic.pdf}
\end{figure}
\par
\noindent
Text Encoding is the encoding of the Description field.
Its value is either ISO-8859-1 or UCS-16 - the same as in
text frames.
Image Format is a 3 byte string indicating the format of the image,
typically `JPG' for JPEG images or 'PNG' for PNG images.
Description is a NULL-terminated C-string which contains
a text description of the image.

\begin{table}[h]
{\relsize{-1}
\begin{tabular}{|r|l||r|l|}
\hline
value & type & value & type \\
\hline
0 & Other & 1 & 32x32 pixels `file icon' (PNG only) \\
2 & Other file icon & 3 & Cover (front) \\
4 & Cover (back) & 5 & Leaflet page \\
6 & Media (e.g. label side of CD) & 7 & Lead artist / Lead performer / Soloist \\
8 & Artist / Performer & 9 & Conductor \\
10 & Band / Orchestra & 11 & Composer \\
12 & Lyricist / Text writer & 13 & Recording location \\
14 & During recording & 15 & During performance \\
16 & Movie / Video screen capture & 17 & A bright coloured fish \\
18 & Illustration & 19 & Band / Artist logotype \\
20 & Publisher / Studio logotype & &  \\
\hline
\end{tabular}
\caption{PIC image types}
}
\end{table}

\pagebreak

\subsection{ID3v2.3}
\subsubsection{ID3v2.3 header}
\begin{figure}[h]
\includegraphics{figures/mp3_id3v23_header.pdf}
\end{figure}
\par
\noindent
The single Size field is split by NULL (0x00) bytes in order to make
it `sync-safe'.

\subsubsection{ID3v2.3 frame}
\begin{figure}[h]
\includegraphics{figures/mp3_id3v23_frame.pdf}
\end{figure}
\par
\noindent
Frame ID's that begin with the letter `T' (0x54) are text frames.
These have an additional text encoding byte before the actual
text data.
All text strings may be terminated by a null character
(0x00 or 0x0000, depending on the encoding).
\begin{figure}[h]
\includegraphics{figures/mp3_id3v23_textframe.pdf}
\begin{tabular}{r|l}
Encoding Byte & Text Encoding \\
\hline
\texttt{0x00} & ISO-8859-1 \\
\texttt{0x01} & UCS-16 \\
\end{tabular}
\end{figure}

\pagebreak

\subsubsection{ID3v2.3 frame IDs}

\begin{table}[h]
{\relsize{-2}
\begin{tabular}{|c|l||c|l|}
\hline
ID & Description & ID & Description \\
\hline
\texttt{AENC} & Audio encryption & \texttt{APIC} & Attached picture \\
\texttt{COMM} & Comments & \texttt{COMR} & Commercial frame \\
\texttt{ENCR} & Encryption method registration & \texttt{EQUA} & Equalization \\
\texttt{ETCO} & Event timing codes & \texttt{GEOB} & General encapsulated object \\
\texttt{GRID} & Group identification registration & \texttt{IPLS} & Involved people list \\
\texttt{LINK} & Linked information & \texttt{MCDI} & Music CD identifier \\
\texttt{MLLT} & MPEG location lookup table & \texttt{OWNE} & Ownership frame \\
\texttt{PRIV} & Private frame & \texttt{PCNT} & Play counter \\
\texttt{POPM} & Popularimeter & \texttt{POSS} & Position synchronisation frame \\
\texttt{RBUF} & Recommended buffer size & \texttt{RVAD} & Relative volume adjustment \\
\texttt{RVRB} & Reverb & \texttt{SYLT} & Synchronized lyric/text \\
\texttt{SYTC} & Synchronized tempo codes & \texttt{TALB} & Album/Movie/Show title \\
\texttt{TBPM} & BPM (beats per minute) & \texttt{TCOM} & Composer \\
\texttt{TCON} & Content type & \texttt{TCOP} & Copyright message \\
\texttt{TDAT} & Date & \texttt{TDLY} & Playlist delay \\
\texttt{TENC} & Encoded by & \texttt{TEXT} & Lyricist/Text writer \\
\texttt{TFLT} & File type & \texttt{TIME} & Time \\
\texttt{TIT1} & Content group description & \texttt{TIT2} & Title/songname/content description \\
\texttt{TIT3} & Subtitle/Description refinement & \texttt{TKEY} & Initial key \\
\texttt{TLAN} & Language(s) & \texttt{TLEN} & Length \\
\texttt{TMED} & Media type & \texttt{TOAL} & Original album/movie/show title \\
\texttt{TOFN} & Original filename & \texttt{TOLY} & Original lyricist(s)/text writer(s) \\
\texttt{TOPE} & Original artist(s)/performer(s) & \texttt{TORY} & Original release year \\
\texttt{TOWN} & File owner/licensee & \texttt{TPE1} & Lead performer(s)/Soloist(s) \\
\texttt{TPE2} & Band/orchestra/accompaniment & \texttt{TPE3} & Conductor/performer refinement \\
\texttt{TPE4} & Interpreted, remixed, or otherwise modified by & \texttt{TPOS} & Part of a set \\
\texttt{TPUB} & Publisher & \texttt{TRCK} & Track number/Position in set \\
\texttt{TRDA} & Recording dates & \texttt{TRSN} & Internet radio station name \\
\texttt{TRSO} & Internet radio station owner & \texttt{TSIZ} & Size \\
\texttt{TSRC} & ISRC (international standard recording code) & \texttt{TSSE} & Software/Hardware and encoding settings \\
\texttt{TYER} & Year & \texttt{TXXX} & User defined text information frame \\
\texttt{UFID} & Unique file identifier & \texttt{USER} & Terms of use \\
\texttt{USLT} & Unsychronized lyric/text transcription & \texttt{WCOM} & Commercial information \\
\texttt{WCOP} & Copyright/Legal information & \texttt{WOAF} & Official audio file webpage \\
\texttt{WOAR} & Official artist/performer webpage & \texttt{WOAS} & Official audio source webpage \\
\texttt{WORS} & Official internet radio station homepage & \texttt{WPAY} & Payment \\
\texttt{WPUB} & Publishers official webpage & \texttt{WXXX} & User defined URL link frame \\
\hline
\end{tabular}
}
\end{table}

\pagebreak

\subsubsection{ID3v2.3 APIC frame}
\begin{figure}[h]
\includegraphics{figures/mp3_id3v23_apic.pdf}
\end{figure}
Text Encoding is the encoding of the Description field.
Its value is either ISO-8859-1 or UCS-16 - the same as in
text frames.
MIME Type is a NULL-terminated, ASCII C-string which contains the
image's MIME type, such as `image/jpeg' or `image/png'.
Description is a NULL-terminated C-string which contains
a text description of the image.
\begin{table}[h]
{\relsize{-1}
\begin{tabular}{|r|l||r|l|}
\hline
value & type & value & type \\
\hline
0 & Other & 1 & 32x32 pixels `file icon' (PNG only) \\
2 & Other file icon & 3 & Cover (front) \\
4 & Cover (back) & 5 & Leaflet page \\
6 & Media (e.g. label side of CD) & 7 & Lead artist / Lead performer / Soloist \\
8 & Artist / Performer & 9 & Conductor \\
10 & Band / Orchestra & 11 & Composer \\
12 & Lyricist / Text writer & 13 & Recording location \\
14 & During recording & 15 & During performance \\
16 & Movie / Video screen capture & 17 & A bright coloured fish \\
18 & Illustration & 19 & Band / Artist logotype \\
20 & Publisher / Studio logotype & &  \\
\hline
\end{tabular}
\caption{APIC image types}
}
\end{table}
