\chapter{MP3}
MP3 is the de-facto standard for lossy audio.
It is little more than a series of MPEG frames with an
optional ID3v2 metadata header and optional ID3v1 metadata
footer.

MP3 decoders are assumed to be very tolerant of anything in
the stream that doesn't look like an MPEG frame, ignoring such
junk until the next frame is found.
Since MP3 files have no standard container format in which
non-MPEG data can be placed, metadata such as ID3 tags are often
made `sync-safe' by formatting them in a way that decoders won't
confuse tags for MPEG frames.
\section{the MP3 File Stream}
\begin{figure}[h]
\includegraphics{figures/mp3_stream.pdf}
\end{figure}
\begin{table}[h]
\begin{tabular}{|c||l||l||r|r|r||l|}
\hline
& & & \multicolumn{3}{c||}{Sample Rate} & \\
bits & MPEG ID & Description & MPEG-1 & MPEG-2 & MPEG-2.5 & Channels \\
\hline
\texttt{00} & MPEG-2.5 & reserved & 44100 & 22050 & 11025 & Stereo \\
\texttt{01} & reserved & Layer III & 48000 & 24000 & 12000 & Joint stereo \\
\texttt{10} & MPEG-2 & Layer II & 32000 & 16000 & 8000 & Dual channel stereo\\
\texttt{11} & MPEG-1 & Layer I & reserved & reserved & reserved & Mono \\
\hline
\end{tabular}
\end{table}
\par
\noindent
Layer I frames always contain 384 samples.
Layer II and Layer III frames always contain 1152 samples.
If the `Protection' bit is set, the frame header is followed by a
16 bit CRC.

\pagebreak

\begin{table}[h]
{\relsize{-2}
\begin{tabular}{|c||r|r|r|r|r|}
\hline
& MPEG-1 & MPEG-1 & MPEG-1 & MPEG-2 & MPEG-2 \\
bits & Layer-1 & Layer-2 & Layer-3 & Layer-1 & Layer-2/3 \\
\hline
\texttt{0000} & free & free & free & free & free \\
\texttt{0001} & 32 & 32 & 32 & 32 & 8 \\
\texttt{0010} & 64 & 48 & 40 & 48 & 16 \\
\texttt{0011} & 96 & 56 & 48 & 56 & 24 \\
\texttt{0100} & 128 & 64 & 56 & 64 & 32 \\
\texttt{0101} & 160 & 80 & 64 & 80 & 40 \\
\texttt{0110} & 192 & 96 & 80 & 96 & 48 \\
\texttt{0111} & 224 & 112 & 96 & 112 & 56 \\
\texttt{1000} & 256 & 128 & 112 & 128 & 64 \\
\texttt{1001} & 288 & 160 & 128 & 144 & 80 \\
\texttt{1010} & 320 & 192 & 160 & 160 & 96 \\
\texttt{1011} & 352 & 224 & 192 & 176 & 112 \\
\texttt{1100} & 384 & 256 & 224 & 192 & 128 \\
\texttt{1101} & 416 & 320 & 256 & 224 & 144 \\
\texttt{1110} & 448 & 384 & 320 & 256 & 160 \\
\texttt{1111} & bad & bad & bad & bad & bad \\
\hline
\end{tabular}
}
\caption{Bitrate in 1000 bits per second}
\end{table}
To find the total size of an MPEG frame, use one of the following
formulas:
\begin{align}
\intertext{Layer I:}
\text{Byte Length} &= \left ( \frac{12 \times \text{Bitrate}}{\text{Sample Rate}} + \text{Pad} \right ) \times 4 \\
\intertext{Layer II/III:}
\text{Byte Length} &= \frac{144 \times \text{Bitrate}}{\text{Sample Rate}} + Pad
\end{align}
For example, an MPEG-1 Layer III frame with a sampling rate of 44100,
a bitrate of 128kbps and a set pad bit is 418 bytes long, including the header.
\begin{equation}
\frac{144 \times 128000}{44100} + 1 = 418
\end{equation}

\subsection{the Xing header}

An MP3 frame header contains the track's sampling rate,
bits-per-sample and number of channels.
However, because MP3 files are little more than
concatenated MPEG frames, there is no obvious place to
store the track's total length.
Since the length of each frame is a constant number of samples,
one can calculate the track length by counting the number of frames.
This method is the most accurate but is also quite slow.

For MP3 files in which all frames have the same bitrate
- also known as constant bitrate, or CBR files -
one can divide the total size of file (minus any ID3 headers/footers),
by the bitrate to determine its length.
If an MP3 file has no Xing header in its first frame,
one can assume it is CBR.

An MP3 file that does contain a Xing header in its first frame
can be assumed to be variable bitrate, or VBR.
In that case, the rate of the first frame cannot be used as a
basis to calculate the length of the entire file.
Instead, one must use the information from the Xing header
which contains that length.

All of the fields within a Xing header are big-endian.
\begin{figure}[h]
\includegraphics{figures/mp3_xing.pdf}
\end{figure}

\subsection{ID3v1 tags}
ID3v1 tags are very simple metadata tags appended to an MP3 file.
All of the fields are fixed length and the text encoding is
undefined.
There are two versions of ID3v1 tags.
ID3v1.1 has a track number field as a 1 byte value
at the end of the comment field.
If the byte just before the end is not null (0x00),
assume we're dealing with a classic ID3v1 tag without a
track number.

\subsubsection{ID3v1}

\begin{figure}[h]
\includegraphics{figures/mp3_id3v1.pdf}
\end{figure}

\subsubsection{ID3v1.1}

\begin{figure}[h]
\includegraphics{figures/mp3_id3v11.pdf}
\end{figure}
