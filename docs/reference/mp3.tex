%This work is licensed under the
%Creative Commons Attribution-Share Alike 3.0 United States License.
%To view a copy of this license, visit
%http://creativecommons.org/licenses/by-sa/3.0/us/ or send a letter to
%Creative Commons,
%171 Second Street, Suite 300,
%San Francisco, California, 94105, USA.

\chapter{MP3}
MP3 is the de-facto standard for lossy audio.
It is little more than a series of MPEG frames with an
optional ID3v2 metadata header and optional ID3v1 metadata
footer.

MP3 decoders are assumed to be very tolerant of anything in
the stream that doesn't look like an MPEG frame, ignoring such
junk until the next frame is found.
Since MP3 files have no standard container format in which
non-MPEG data can be placed, metadata such as ID3 tags are often
made `sync-safe' by formatting them in a way that decoders won't
confuse tags for MPEG frames.
\section{the MP3 File Stream}
\begin{figure}[h]
\includegraphics{figures/mp3_stream.pdf}
\end{figure}
\begin{table}[h]
\begin{tabular}{|c||l||l||r|r|r||l|}
\hline
& & & \multicolumn{3}{c||}{Sample Rate} & \\
bits & MPEG ID & Description & MPEG-1 & MPEG-2 & MPEG-2.5 & Channels \\
\hline
\texttt{00} & MPEG-2.5 & reserved & 44100 & 22050 & 11025 & Stereo \\
\texttt{01} & reserved & Layer III & 48000 & 24000 & 12000 & Joint stereo \\
\texttt{10} & MPEG-2 & Layer II & 32000 & 16000 & 8000 & Dual channel stereo\\
\texttt{11} & MPEG-1 & Layer I & reserved & reserved & reserved & Mono \\
\hline
\end{tabular}
\end{table}
\par
\noindent
Layer I frames always contain 384 samples.
Layer II and Layer III frames always contain 1152 samples.
If the \VAR{Protection} bit is 0, the frame header is followed by a
16 bit CRC.

\pagebreak

\begin{table}[h]
{\relsize{-2}
\begin{tabular}{|c||r|r|r|r|r|}
\hline
& MPEG-1 & MPEG-1 & MPEG-1 & MPEG-2 & MPEG-2 \\
bits & Layer-1 & Layer-2 & Layer-3 & Layer-1 & Layer-2/3 \\
\hline
\texttt{0000} & free & free & free & free & free \\
\texttt{0001} & 32 & 32 & 32 & 32 & 8 \\
\texttt{0010} & 64 & 48 & 40 & 48 & 16 \\
\texttt{0011} & 96 & 56 & 48 & 56 & 24 \\
\texttt{0100} & 128 & 64 & 56 & 64 & 32 \\
\texttt{0101} & 160 & 80 & 64 & 80 & 40 \\
\texttt{0110} & 192 & 96 & 80 & 96 & 48 \\
\texttt{0111} & 224 & 112 & 96 & 112 & 56 \\
\texttt{1000} & 256 & 128 & 112 & 128 & 64 \\
\texttt{1001} & 288 & 160 & 128 & 144 & 80 \\
\texttt{1010} & 320 & 192 & 160 & 160 & 96 \\
\texttt{1011} & 352 & 224 & 192 & 176 & 112 \\
\texttt{1100} & 384 & 256 & 224 & 192 & 128 \\
\texttt{1101} & 416 & 320 & 256 & 224 & 144 \\
\texttt{1110} & 448 & 384 & 320 & 256 & 160 \\
\texttt{1111} & bad & bad & bad & bad & bad \\
\hline
\end{tabular}
}
\caption{Bitrate in 1000 bits per second}
\end{table}
To find the total size of an MPEG frame, use one of the following
formulas:
\begin{align}
\intertext{Layer I:}
\text{Byte Length} &= \left ( \frac{12 \times \text{Bitrate}}{\text{Sample Rate}} + \text{Pad} \right ) \times 4 \\
\intertext{Layer II/III:}
\text{Byte Length} &= \frac{144 \times \text{Bitrate}}{\text{Sample Rate}} + \text{Pad}
\end{align}
For example, an MPEG-1 Layer III frame with a sampling rate of 44100,
a bitrate of 128kbps and a set pad bit is 418 bytes long, including the header.
\begin{equation}
\frac{144 \times 128000}{44100} + 1 = 418
\end{equation}

\subsection{the Xing Header}

An MP3 frame header contains the track's sampling rate,
bits-per-sample and number of channels.
However, because MP3 files are little more than
concatenated MPEG frames, there is no obvious place to
store the track's total length.
Since the length of each frame is a constant number of samples,
one can calculate the track length by counting the number of frames.
This method is the most accurate but is also quite slow.

For MP3 files in which all frames have the same bitrate
- also known as constant bitrate, or CBR files -
one can divide the total size of file (minus any ID3 headers/footers),
by the bitrate to determine its length.
If an MP3 file has no Xing header in its first frame,
one can assume it is CBR.

An MP3 file that does contain a Xing header in its first frame
can be assumed to be variable bitrate, or VBR.
In that case, the rate of the first frame cannot be used as a
basis to calculate the length of the entire file.
Instead, one must use the information from the Xing header
which contains that length.

All of the fields within a Xing header are big-endian.
\begin{figure}[h]
\includegraphics{figures/mp3_xing.pdf}
\end{figure}

\section{ID3v1 Tags}
ID3v1 tags are very simple metadata tags appended to an MP3 file.
All of the fields are fixed length and the text encoding is
undefined.
There are two versions of ID3v1 tags.
ID3v1.1 has a track number field as a 1 byte value
at the end of the comment field.
If the byte just before the end is not null (0x00),
assume we're dealing with a classic ID3v1 tag without a
track number.

\subsection{ID3v1}

\begin{figure}[h]
\includegraphics{figures/mp3_id3v1.pdf}
\end{figure}

\subsection{ID3v1.1}

\begin{figure}[h]
\includegraphics{figures/mp3_id3v11.pdf}
\end{figure}

\section{ID3v2 Tags}

The ID3v2 tag was invented to address the deficiencies in the original
ID3v1 tag.
ID3v2 comes in three similar but not entirely compatible variants:
ID3v2.2, ID3v2.3 and ID3v2.4.
All of its fields are big-endian.

\begin{figure}[h]
\includegraphics{figures/mp3_id3v2_stream.pdf}
\end{figure}

\subsection{ID3v2.2}

\subsubsection{ID3v2.2 Header}

\begin{figure}[h]
\includegraphics{figures/mp3_id3v22_header.pdf}
\end{figure}
\par
\noindent
The single Size field is split by NULL (0x00) bytes in order to make
it `sync-safe'.
That is, no possible size value will result in a false
MP3 frame sync (11 bits set in a row).
This size field is the length of the entire tag, not including the header.

\subsubsection{CNT Frame}
\begin{figure}[h]
\includegraphics{figures/id3v22/cnt.pdf}
\end{figure}
This frame is a 32-bit counter of the number of times the track
has been played.
According to the spec, additional bytes are added should the counter
overflow.
In practice, audio players typically don't update tags during playback
and one is unlikely to exceed 4 billion plays of a single track.

\clearpage

\subsubsection{COM Frame}
\begin{figure}[h]
\includegraphics{figures/id3v22/com.pdf}
\end{figure}
This frame is a lengthy string of comment text.
\VAR{Encoding} is 0 for Latin-1, 1 for UCS-2.
\VAR{Language} is a 3 byte ASCII string.
\VAR{Short Content Description} is a NULL-terminated string
containing a short description of the comment.
Note that for UCS-2 comments, the NULL terminator is 2 bytes.
The remainder of the frame is the comment text itself.

\subsubsection{CRA Frame}
\begin{figure}[h]
\includegraphics{figures/id3v22/cra.pdf}
\end{figure}
This frame indicates audio encryption information.
\VAR{Owner Identifier} is a NULL-terminated string
containing a URL, email address or link.
\VAR{Preview Start} and \VAR{Preview Length}
indicate how many of the file's frames are unencrypted.

\subsubsection{CRM Frame}
\begin{figure}[h]
\includegraphics{figures/id3v22/crm.pdf}
\end{figure}
This frame contains encrypted side data such as pictures or text.
\VAR{Owner Identifier} and \VAR{Content / Explanation} are
NULL-terminated strings.

\clearpage

\subsubsection{GEO Frame}
\begin{figure}[h]
\includegraphics{figures/id3v22/geo.pdf}
\end{figure}
This frame contains an embedded file.
\VAR{MIME Type} and \VAR{Filename} are encoded as NULL-terminated
Latin-1 strings.
\VAR{Description} is encoded as a Latin-1 or UCS-2 NULL-terminated
string, depending on if \VAR{Encoding} is 0 or 1, respectively.
The remainder of the frame is the file's binary data.

\subsubsection{IPL Frame}
\begin{figure}[h]
\includegraphics{figures/id3v22/ipl.pdf}
\end{figure}
This frame contains a list of people involved in the file's creation.
Each string is encoded as a Latin-1 or UCS-2 NULL-terminated
string, depending on if \VAR{Encoding} is 0 or 1, respectively.

\subsubsection{MCI Frame}
\begin{figure}[h]
\includegraphics{figures/id3v22/mci.pdf}
\end{figure}
This frame contains the binary data from a CD's Table of Contents.
That TOC consists of a 4 byte header followed by 8 bytes per CD track.
Valid \VAR{TRK} frames are required if this frame is present.

\clearpage

\subsubsection{PIC Frame}
\begin{figure}[h]
\includegraphics{figures/id3v22/pic.pdf}
\end{figure}
This frame contains an embedded image file.
\VAR{Image Format} is a 3 byte string indicating the format of the image,
typically `JPG' for JPEG images or 'PNG' for PNG images.
\VAR{Description} is a Latin-1 or UCS-2 NULL-terminated string,
depending on if \VAR{Encoding} is 0 or 1, respectively.

\begin{table}[h]
{\relsize{-1}
\begin{tabular}{|r|l||r|l|}
\hline
value & type & value & type \\
\hline
0 & Other & 1 & 32x32 pixels `file icon' (PNG only) \\
2 & Other file icon & 3 & Cover (front) \\
4 & Cover (back) & 5 & Leaflet page \\
6 & Media (e.g. label side of CD) & 7 & Lead artist / Lead performer / Soloist \\
8 & Artist / Performer & 9 & Conductor \\
10 & Band / Orchestra & 11 & Composer \\
12 & Lyricist / Text writer & 13 & Recording location \\
14 & During recording & 15 & During performance \\
16 & Movie / Video screen capture & 17 & A bright colored fish \\
18 & Illustration & 19 & Band / Artist logotype \\
20 & Publisher / Studio logotype & &  \\
\hline
\end{tabular}
\caption{PIC image types}
}
\end{table}

\subsubsection{POP Frame}
\begin{figure}[h]
\includegraphics{figures/id3v22/pop.pdf}
\end{figure}
This frame contains a ``popularimeter''.
\VAR{User Email} is a NULL-terminated string of the user's email address.
\VAR{Rating} is a value from 1 (worst) to 255 (best), where 0 is unknown.
\VAR{Play Counter} is the number of times the track has been played.
This frame may occur multiple times, but each must have a different email
address.

\clearpage

\subsubsection{REV Frame}
\begin{figure}[h]
\includegraphics{figures/id3v22/rev.pdf}
\end{figure}
This frame contains user-defined reverb values for echo effects.

\subsubsection{RVA Frame}
\begin{figure}[h]
\includegraphics{figures/id3v22/rva.pdf}
\end{figure}
This frame contains a user-defined relative volume adjustment
to apply to each channel.
If \VAR{Left Channel} is set, the volume change is used to increment
the left channel's values.
If not set, the volume change is used to decrement the left channel's
values.

\subsubsection{SLT Frame}
\begin{figure}[h]
\includegraphics{figures/id3v22/slt.pdf}
\end{figure}
This is a frame of synced lyrics for karaoke.
\VAR{Timestamp Format} is 1 for MPEG frames or 2 for milliseconds.
The NULL-terminated text strings are designed to be displayed
as individual syllables at specific timestamps.

\subsubsection{Text Frames}
\begin{figure}[h]
\includegraphics{figures/id3v22/t__.pdf}
\end{figure}
These frames are for textual information.
\VAR{Encoding} is 0 for Latin-1 encoding, 1 for UCS-2 encoding.
\begin{table}[h]
{\relsize{-2}
\begin{tabular}{|r|l|}
\hline
Frame ID & Meaning \\
\hline
\texttt{TAL} & album / movie / show title \\
\texttt{TBP} & BPM (beats per minute) \\
\texttt{TCM} & composer \\
\texttt{TCO} & content type \\
\texttt{TCR} & copyright message \\
\texttt{TDA} & date \\
\texttt{TDY} & playlist delay \\
\texttt{TEN} & encoded by \\
\texttt{TFT} & file type \\
\texttt{TIM} & time \\
\texttt{TKE} & initial key \\
\texttt{TLA} & language(s) \\
\texttt{TLE} & length \\
\texttt{TMT} & media type \\
\texttt{TOA} & original artist(s) / performer(s) \\
\texttt{TOF} & original filename \\
\texttt{TOL} & original Lyricist(s) / text writer(s) \\
\texttt{TOR} & original release year \\
\texttt{TOT} & original album / movie / show title \\
\texttt{TP1} & lead artist(s) / performer(s) / soloist(s) / performing group \\
\texttt{TP2} & band / orchestra / accompaniment \\
\texttt{TP3} & conductor / performer refinement \\
\texttt{TP4} & interpreted, remixed, or otherwise modified by \\
\texttt{TPA} & album number / part of a set \\
\texttt{TPB} & publisher \\
\texttt{TRC} & ISRC (International Standard Recording Code) \\
\texttt{TRD} & recording dates \\
\texttt{TRK} & track number / position in set \\
\texttt{TSI} & size \\
\texttt{TSS} & software / hardware and settings used for encoding \\
\texttt{TT1} & content group description \\
\texttt{TT2} & title / songname / content description \\
\texttt{TT3} & subtitle / description refinement \\
\texttt{TXT} & lyricist / text writer \\
\texttt{TYE} & year \\
\hline
\end{tabular}
}
\end{table}
\par
The \texttt{TRK} and \texttt{TPA} numeric fields may be extended
with the ``/'' character, indicating the field's total number.
For example, a \texttt{TRK} of ``3/5'' means the track is number
3 out of a total of 5.

\begin{figure}[h]
\includegraphics{figures/id3v22/txx.pdf}
\end{figure}
This frame is for user-defined text information.
\VAR{Description} is a NULL-terminated string indicating
what the frame is for.
\VAR{Encoding} is 0 for Latin-1 encoding, 1 for UCS-2 encoding.

\subsubsection{URL Frames}

\begin{figure}[h]
\includegraphics{figures/id3v22/w__.pdf}
\end{figure}
These frames are for URL links to external resources.
\par
\begin{table}[h]
\begin{tabular}{|r|l|}
\hline
Frame ID & Meaning \\
\hline
\texttt{WAF} & official audio file webpage \\
\texttt{WAR} & official artist / performer webpage \\
\texttt{WAS} & official audio source webpage \\
\texttt{WCM} & commercial information \\
\texttt{WCP} & copyright / legal information \\
\texttt{WPB} & publishers official webpage \\
\hline
\end{tabular}
\end{table}
\begin{figure}[h]
\includegraphics{figures/id3v22/wxx.pdf}
\end{figure}

\clearpage

\subsection{ID3v2.3}
\subsubsection{ID3v2.3 Header}
\begin{figure}[h]
\includegraphics{figures/mp3_id3v23_header.pdf}
\end{figure}
\par
\noindent
The single Size field is split by NULL (0x00) bytes in order to make
it `sync-safe'.
This size field is the length of the entire tag, not including the header.

\subsubsection{ID3v2.3 Frame}
\begin{figure}[h]
\includegraphics{figures/mp3_id3v23_frame.pdf}
\end{figure}
\par
\noindent
Frame IDs that begin with the letter `T' (0x54) are text frames.
These have an additional text encoding byte before the actual
text data.
All text strings may be terminated by a null character
(0x00 or 0x0000, depending on the encoding).
\begin{figure}[h]
\includegraphics{figures/mp3_id3v23_textframe.pdf}
\begin{tabular}{r|l}
Encoding Byte & Text Encoding \\
\hline
\texttt{0x00} & ISO-8859-1 \\
\texttt{0x01} & UCS-16 \\
\end{tabular}
\end{figure}

\pagebreak

\subsubsection{ID3v2.3 APIC Frame}
\begin{figure}[h]
\includegraphics{figures/mp3_id3v23_apic.pdf}
\end{figure}
Text Encoding is the encoding of the Description field.
Its value is either ISO-8859-1 or UCS-16 - the same as in
text frames.
MIME Type is a NULL-terminated, ASCII C-string which contains the
image's MIME type, such as `image/jpeg' or `image/png'.
Description is a NULL-terminated C-string which contains
a text description of the image.
\begin{table}[h]
{\relsize{-1}
\begin{tabular}{|r|l||r|l|}
\hline
value & type & value & type \\
\hline
0 & Other & 1 & 32x32 pixels `file icon' (PNG only) \\
2 & Other file icon & 3 & Cover (front) \\
4 & Cover (back) & 5 & Leaflet page \\
6 & Media (e.g. label side of CD) & 7 & Lead artist / Lead performer / Soloist \\
8 & Artist / Performer & 9 & Conductor \\
10 & Band / Orchestra & 11 & Composer \\
12 & Lyricist / Text writer & 13 & Recording location \\
14 & During recording & 15 & During performance \\
16 & Movie / Video screen capture & 17 & A bright colored fish \\
18 & Illustration & 19 & Band / Artist logotype \\
20 & Publisher / Studio logotype & &  \\
\hline
\end{tabular}
\caption{APIC image types}
}
\end{table}

\pagebreak

\subsubsection{ID3v2.3 Frame IDs}

\begin{multicols}{3}
{\relsize{-2}
\begin{description}
\item[\texttt{AENC}] Audio encryption
\item[\texttt{APIC}] Attached picture
\item[\texttt{COMM}] Comments
\item[\texttt{COMR}] Commercial frame
\item[\texttt{ENCR}] Encryption method registration
\item[\texttt{EQUA}] Equalization
\item[\texttt{ETCO}] Event timing codes
\item[\texttt{GEOB}] General encapsulated object
\item[\texttt{GRID}] Group identification registration
\item[\texttt{IPLS}] Involved people list
\item[\texttt{LINK}] Linked information
\item[\texttt{MCDI}] Music CD identifier
\item[\texttt{MLLT}] MPEG location lookup table
\item[\texttt{OWNE}] Ownership frame
\item[\texttt{PRIV}] Private frame
\item[\texttt{PCNT}] Play counter
\item[\texttt{POPM}] Popularimeter
\item[\texttt{POSS}] Position synchronisation frame
\item[\texttt{RBUF}] Recommended buffer size
\item[\texttt{RVAD}] Relative volume adjustment
\item[\texttt{RVRB}] Reverb
\item[\texttt{SYLT}] Synchronized lyric / text
\item[\texttt{SYTC}] Synchronized tempo codes
\item[\texttt{TALB}] Album /Movie / Show title
\item[\texttt{TBPM}] BPM (beats per minute)
\item[\texttt{TCOM}] Composer
\item[\texttt{TCON}] Content type
\item[\texttt{TCOP}] Copyright message
\item[\texttt{TDAT}] Date
\item[\texttt{TDLY}] Playlist delay
\item[\texttt{TENC}] Encoded by
\item[\texttt{TEXT}] Lyricist / Text writer
\item[\texttt{TFLT}] File type
\item[\texttt{TIME}] Time
\item[\texttt{TIT1}] Content group description
\item[\texttt{TIT2}] Title / songname / content description
\item[\texttt{TIT3}] Subtitle / Description refinement
\item[\texttt{TKEY}] Initial key
\item[\texttt{TLAN}] Language(s)
\item[\texttt{TLEN}] Length
\item[\texttt{TMED}] Media type
\item[\texttt{TOAL}] Original album/movie/show title
\item[\texttt{TOFN}] Original filename
\item[\texttt{TOLY}] Original lyricist(s) / text writer(s)
\item[\texttt{TOPE}] Original artist(s) / performer(s)
\item[\texttt{TORY}] Original release year
\item[\texttt{TOWN}] File owner / licensee
\item[\texttt{TPE1}] Lead performer(s) / Soloist(s)
\item[\texttt{TPE2}] Band / orchestra / accompaniment
\item[\texttt{TPE3}] Conductor / performer refinement
\item[\texttt{TPE4}] Interpreted, remixed, or otherwise modified by
\item[\texttt{TPOS}] Part of a set
\item[\texttt{TPUB}] Publisher
\item[\texttt{TRCK}] Track number / Position in set
\item[\texttt{TRDA}] Recording dates
\item[\texttt{TRSN}] Internet radio station name
\item[\texttt{TRSO}] Internet radio station owner
\item[\texttt{TSIZ}] Size
\item[\texttt{TSRC}] ISRC (international standard recording code)
\item[\texttt{TSSE}] Software/Hardware and encoding settings
\item[\texttt{TYER}] Year
\item[\texttt{TXXX}] User defined text information frame
\item[\texttt{UFID}] Unique file identifier
\item[\texttt{USER}] Terms of use
\item[\texttt{USLT}] Unsychronized lyric / text transcription
\item[\texttt{WCOM}] Commercial information
\item[\texttt{WCOP}] Copyright / Legal information
\item[\texttt{WOAF}] Official audio file webpage
\item[\texttt{WOAR}] Official artist/performer webpage
\item[\texttt{WOAS}] Official audio source webpage
\item[\texttt{WORS}] Official internet radio station homepage
\item[\texttt{WPAY}] Payment
\item[\texttt{WPUB}] Publishers official webpage
\item[\texttt{WXXX}] User defined URL link frame
\end{description}
}
\end{multicols}

\subsection{ID3v2.4}
\subsubsection{ID3v2.4 Header}
\begin{figure}[h]
\includegraphics{figures/mp3_id3v24_header.pdf}
\end{figure}

\subsubsection{ID3v2.4 Frame}
\begin{figure}[h]
\includegraphics{figures/mp3_id3v24_frame.pdf}
\end{figure}
\par
\noindent
Frame IDs that begin with the letter `T' (0x54) are text frames.
These have an additional text encoding byte before the actual
text data.
All text strings may be terminated by a null character
(0x00 or 0x0000, depending on the encoding).
\begin{figure}[h]
\includegraphics{figures/mp3_id3v24_textframe.pdf}
\begin{tabular}{r|l}
Encoding Byte & Text Encoding \\
\hline
\texttt{0x00} & ISO-8859-1 \\
\texttt{0x01} & UTF-16 \\
\texttt{0x02} & UTF-16BE \\
\texttt{0x03} & UTF-8
\end{tabular}
\end{figure}

\pagebreak

\subsubsection{ID3v2.4 APIC Frame}
\begin{figure}[h]
\includegraphics{figures/mp3_id3v24_apic.pdf}
\end{figure}
\par
\noindent
Text Encoding is the encoding of the Description field.
Its value is either ISO-8859-1, UTF-16 or UTF-8 - the same as in
text frames.
MIME Type is a NULL-terminated, ASCII C-string which contains the
image's MIME type, such as `image/jpeg' or `image/png'.
Description is a NULL-terminated C-string which contains
a text description of the image.
\begin{table}[h]
{\relsize{-1}
\begin{tabular}{|r|l||r|l|}
\hline
value & type & value & type \\
\hline
0 & Other & 1 & 32x32 pixels `file icon' (PNG only) \\
2 & Other file icon & 3 & Cover (front) \\
4 & Cover (back) & 5 & Leaflet page \\
6 & Media (e.g. label side of CD) & 7 & Lead artist / Lead performer / Soloist \\
8 & Artist / Performer & 9 & Conductor \\
10 & Band / Orchestra & 11 & Composer \\
12 & Lyricist / Text writer & 13 & Recording location \\
14 & During recording & 15 & During performance \\
16 & Movie / Video screen capture & 17 & A bright colored fish \\
18 & Illustration & 19 & Band / Artist logotype \\
20 & Publisher / Studio logotype & &  \\
\hline
\end{tabular}
\caption{APIC image types}
}
\end{table}

\pagebreak

\subsubsection{ID3v2.4 Frame IDs}

\begin{multicols}{3}
{\relsize{-2}
\begin{description}
\item[\texttt{AENC}] Audio encryption
\item[\texttt{APIC}] Attached picture
\item[\texttt{ASPI}] Audio seek point index
\item[\texttt{COMM}] Comments
\item[\texttt{COMR}] Commercial frame
\item[\texttt{ENCR}] Encryption method registration
\item[\texttt{EQU2}] Equalisation (2)
\item[\texttt{ETCO}] Event timing codes
\item[\texttt{GEOB}] General encapsulated object
\item[\texttt{GRID}] Group identification registration
\item[\texttt{LINK}] Linked information
\item[\texttt{MCDI}] Music CD identifier
\item[\texttt{MLLT}] MPEG location lookup table
\item[\texttt{OWNE}] Ownership frame
\item[\texttt{PRIV}] Private frame
\item[\texttt{PCNT}] Play counter
\item[\texttt{POPM}] Popularimeter
\item[\texttt{POSS}] Position synchronisation frame
\item[\texttt{RBUF}] Recommended buffer size
\item[\texttt{RVA2}] Relative volume adjustment (2)
\item[\texttt{RVRB}] Reverb
\item[\texttt{SEEK}] Seek frame
\item[\texttt{SIGN}] Signature frame
\item[\texttt{SYLT}] Synchronised lyric/text
\item[\texttt{SYTC}] Synchronised tempo codes
\item[\texttt{TALB}] Album/Movie/Show title
\item[\texttt{TBPM}] BPM (beats per minute)
\item[\texttt{TCOM}] Composer
\item[\texttt{TCON}] Content type
\item[\texttt{TCOP}] Copyright message
\item[\texttt{TDEN}] Encoding time
\item[\texttt{TDLY}] Playlist delay
\item[\texttt{TDOR}] Original release time
\item[\texttt{TDRC}] Recording time
\item[\texttt{TDRL}] Release time
\item[\texttt{TDTG}] Tagging time
\item[\texttt{TENC}] Encoded by
\item[\texttt{TEXT}] Lyricist/Text writer
\item[\texttt{TFLT}] File type
\item[\texttt{TIPL}] Involved people list
\item[\texttt{TIT1}] Content group description
\item[\texttt{TIT2}] Title/songname/content description
\item[\texttt{TIT3}] Subtitle/Description refinement
\item[\texttt{TKEY}] Initial key
\item[\texttt{TLAN}] Language(s)
\item[\texttt{TLEN}] Length
\item[\texttt{TMCL}] Musician credits list
\item[\texttt{TMED}] Media type
\item[\texttt{TMOO}] Mood
\item[\texttt{TOAL}] Original album/movie/show title
\item[\texttt{TOFN}] Original filename
\item[\texttt{TOLY}] Original lyricist(s)/text writer(s)
\item[\texttt{TOPE}] Original artist(s)/performer(s)
\item[\texttt{TOWN}] File owner/licensee
\item[\texttt{TPE1}] Lead performer(s)/Soloist(s)
\item[\texttt{TPE2}] Band/orchestra/accompaniment
\item[\texttt{TPE3}] Conductor/performer refinement
\item[\texttt{TPE4}] Interpreted, remixed, or otherwise modified by
\item[\texttt{TPOS}] Part of a set
\item[\texttt{TPRO}] Produced notice
\item[\texttt{TPUB}] Publisher
\item[\texttt{TRCK}] Track number/Position in set
\item[\texttt{TRSN}] Internet radio station name
\item[\texttt{TRSO}] Internet radio station owner
\item[\texttt{TSOA}] Album sort order
\item[\texttt{TSOP}] Performer sort order
\item[\texttt{TSOT}] Title sort order
\item[\texttt{TSRC}] ISRC (international standard recording code)
\item[\texttt{TSSE}] Software/Hardware and settings used for encoding
\item[\texttt{TSST}] Set subtitle
\item[\texttt{TXXX}] User defined text information frame
\item[\texttt{UFID}] Unique file identifier
\item[\texttt{USER}] Terms of use
\item[\texttt{USLT}] Unsynchronised lyric/text transcription
\item[\texttt{WCOM}] Commercial information
\item[\texttt{WCOP}] Copyright/Legal information
\item[\texttt{WOAF}] Official audio file webpage
\item[\texttt{WOAR}] Official artist/performer webpage
\item[\texttt{WOAS}] Official audio source webpage
\item[\texttt{WORS}] Official Internet radio station homepage
\item[\texttt{WPAY}] Payment
\item[\texttt{WPUB}] Publishers official webpage
\item[\texttt{WXXX}] User defined URL link frame
\end{description}
}
\end{multicols}


