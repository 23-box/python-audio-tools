\chapter{Introduction}
This book is intended as a reference for anyone who's ever looked
at their collection of audio files and wondered how they worked.
Though still a work-in-progress, my goal is to create documentation
on the full decoding/encoding process of as many audio formats as
possible.

Though to be honest, the audience for this is myself.
I enjoy figuring out the little details of how these formats operate.
And as I figure them out and implement them in Python Audio Tools,
I then add some documentation here on what I've just discovered.
That way, when I have to come back to something six months from now,
I can return to some written documentation instead of having to go
directly to my source code.

Therefore, I try to make my documentation as clear and concise
as possible.
Otherwise, what's the advantage over simply diving back into the source?
Yet this process often turns into a challenge of its own;
I'll discover that a topic I thought I'd understood wasn't so
easy to grasp once I had to simplify and explain it to some
hypothetical future self.
Thus, I'll have to learn it better in order to explain it better.

That said, there's still much work left to do.
Because it's a repository of my knowledge, it also illustrates
the limits of my knowledge.
Many formats are little more than ``stubs'', containing
just enough information to extract such metadata as
sample rate or bits-per-sample.
These are formats in which my Python Audio Tools passes the
encoding/decoding task to a binary ``black-box'' executable
since I haven't yet taken the time to learn how to perform that
work myself.
But my hope is that as I learn more, this work will become
more fleshed-out and widely useful.

In the meantime, by including it with Python Audio Tools,
my hope is that someone else with some passing interest might also
get some use out of what I've learned.
And though I strive for accuracy (for my own sake, at least)
I cannot guarantee it.
When in doubt, consult the references on page \pageref{references}
for links to external sources which may have additional information.

