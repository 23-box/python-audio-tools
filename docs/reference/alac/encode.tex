%This work is licensed under the
%Creative Commons Attribution-Share Alike 3.0 United States License.
%To view a copy of this license, visit
%http://creativecommons.org/licenses/by-sa/3.0/us/ or send a letter to
%Creative Commons,
%171 Second Street, Suite 300,
%San Francisco, California, 94105, USA.

\section{ALAC Encoding}

To encode an ALAC file, we need a stream of PCM sample integers
along with that stream's sample rate, bits-per-sample and number of
channels.
We'll start by encoding all of the non-audio ALAC atoms,
most of which are contained within the \ATOM{moov} atom.
There's over twenty atoms in a typical ALAC file,
most of which are packed with seemingly redundant or
nonessential data,
so it will take awhile before we can move on to the actual
audio encoding process.

Remember, all of an ALAC's fields are big-endian.

%This work is licensed under the
%Creative Commons Attribution-Share Alike 3.0 United States License.
%To view a copy of this license, visit
%http://creativecommons.org/licenses/by-sa/3.0/us/ or send a letter to
%Creative Commons,
%171 Second Street, Suite 300,
%San Francisco, California, 94105, USA.

\subsection{ALAC Atoms}
\begin{wrapfigure}[6]{r}{1.5in}
\includegraphics{alac/figures/atoms.pdf}
\end{wrapfigure}
We'll encode our ALAC file in iTunes order, which means
it contains the \ATOM{ftyp}, \ATOM{moov}, \ATOM{free} and
\ATOM{mdat} atoms, in that order.

\subsubsection{the ftyp Atom}

\begin{table}[h]
\begin{tabular}{|l|r|l|}
\hline
Field & Size & Value \\
\hline
atom length & 32 & 32 \\
atom type & 32 & `ftyp' (\texttt{0x66747970}) \\
\hline
major brand & 32 & `M4A ' (\texttt{0x4d344120}) \\
major brand version & 32 & \texttt{0} \\
compatible brand & 32 & `M4A ' (\texttt{0x4d344120}) \\
compatible brand & 32 & `mp42' (\texttt{0x6d703432}) \\
compatible brand & 32 & `isom' (\texttt{0x69736f6d}) \\
compatible brand & 32 & \texttt{0x00000000} \\
\hline
\end{tabular}
\end{table}

\subsubsection{the moov Atom}

\begin{table}[h]
\begin{tabular}{|l|r|l|}
\hline
Field & Size & Value \\
\hline
atom length & 32 & \ATOM{mvhd} size + \ATOM{trak} size + \ATOM{udta} size + 8 \\
atom type & 32 & `moov' (\texttt{0x6d6f6f76}) \\
\hline
\ATOM{mvhd} atom & \ATOM{mvhd} size & \ATOM{mvhd} data \\
\ATOM{trak} atom & \ATOM{trak} size & \ATOM{trak} data \\
\ATOM{udta} atom & \ATOM{udta} size & \ATOM{udta} data \\
\hline
\end{tabular}
\end{table}

\clearpage

\subsubsection{the mvhd Atom}

\begin{table}[h]
\begin{tabular}{|l|r|l|}
\hline
Field & Size & Value \\
\hline
atom length & 32 & 108/120 \\
atom type & 32 & `mvhd' (\texttt{0x6d766864}) \\
\hline
version & 8 & \texttt{0x00} \\
flags & 24 & \texttt{0x000000} \\
created date & 32/64 & creation date as Mac UTC \\
modified date & 32/64 & modification date as Mac UTC \\
time scale & 32 & sample rate \\
duration & 32/64 & total PCM frames \\
playback speed & 32 & \texttt{0x10000} \\
user volume & 16 & \texttt{0x100} \\
padding & 80 & \texttt{0x00000000000000000000} \\
window geometry matrix a & 32 & \texttt{0x10000} \\
window geometry matrix b & 32 & \texttt{0} \\
window geometry matrix u & 32 & \texttt{0} \\
window geometry matrix c & 32 & \texttt{0} \\
window geometry matrix d & 32 & \texttt{0x10000} \\
window geometry matrix v & 32 & \texttt{0} \\
window geometry matrix x & 32 & \texttt{0} \\
window geometry matrix y & 32 & \texttt{0} \\
window geometry matrix w & 32 & \texttt{0x40000000} \\
QuickTime preview & 64 & \texttt{0} \\
QuickTime still poster & 32 & \texttt{0} \\
QuickTime selection time & 64 & \texttt{0} \\
QuickTime current time & 32 & \texttt{0} \\
next track ID & 32 & \texttt{2} \\
\hline
\end{tabular}
\end{table}

If \VAR{version} is 0, \VAR{created date}, \VAR{modified date} and
\VAR{duration} are 32 bit fields.
Otherwise, they are 64 bit fields.
The \VAR{created date} and \VAR{modified date} are seconds
since the Macintosh Epoch, which is 00:00:00, January 1st, 1904.\footnote{Why 1904?  It's the first leap year of the 20th century.}
To convert a Unix Epoch timestamp (seconds since January 1st, 1970) to
a Macintosh Epoch, one needs to add 24,107 days -
or \texttt{2082844800} seconds.

\clearpage

\subsubsection{the trak Atom}
\begin{tabular}{|l|r|l|}
\hline
Field & Size & Value \\
\hline
atom length & 32 & \ATOM{tkhd} size + \ATOM{mdia} size + 8 \\
atom type & 32 & `trak' (\texttt{0x7472616b}) \\
\hline
\ATOM{tkhd} atom & \ATOM{tkhd} size & \ATOM{tkhd} data \\
\ATOM{mdia} atom & \ATOM{mdia} size & \ATOM{mdia} data \\
\hline
\end{tabular}

\subsubsection{the tkhd Atom}

\begin{table}[h]
\begin{tabular}{|l|r|l|}
\hline
Field & Size & Value \\
\hline
atom length & 32 & 92/104 \\
atom type & 32 & `tkhd' (\texttt{0x746b6864}) \\
\hline
version & 8 & \texttt{0x00} \\
padding & 20 & \texttt{0x000000} \\
track in poster & 1 & \texttt{0} \\
track in preview & 1 & \texttt{1} \\
track in movie & 1 & \texttt{1} \\
track enabled & 1 & \texttt{1} \\
created date & 32/64 & creation date as Mac UTC \\
modified date & 32/64 & modification date as Mac UTC \\
track ID & 32 & \texttt{1} \\
padding & 32 & \texttt{0x00000000} \\
duration & 32/64 & total PCM frames \\
padding & 64 & \texttt{0x0000000000000000} \\
video layer & 16 & \texttt{0} \\
QuickTime alternate & 16 & \texttt{0} \\
volume & 16 & \texttt{0x1000} \\
padding & 16 & \texttt{0x0000} \\
video geometry matrix a & 32 & \texttt{0x10000} \\
video geometry matrix b & 32 & \texttt{0} \\
video geometry matrix u & 32 & \texttt{0} \\
video geometry matrix c & 32 & \texttt{0} \\
video geometry matrix d & 32 & \texttt{0x10000} \\
video geometry matrix v & 32 & \texttt{0} \\
video geometry matrix x & 32 & \texttt{0} \\
video geometry matrix y & 32 & \texttt{0} \\
video geometry matrix w & 32 & \texttt{0x40000000} \\
video width & 32 & \texttt{0} \\
video height & 32 & \texttt{0} \\
\hline
\end{tabular}
\end{table}

\clearpage

\subsubsection{the mdia Atom}

\begin{table}[h]
\begin{tabular}{|l|r|l|}
\hline
Field & Size & Value \\
\hline
atom length & 32 & \ATOM{mdhd} size + \ATOM{hdlr} size + \ATOM{minf} size + 8 \\
atom type & 32 & `mdia' (\texttt{0x6d646961}) \\
\hline
\ATOM{mdhd} atom & \ATOM{mdhd} size & \ATOM{mdhd} data \\
\ATOM{hdlr} atom & \ATOM{hdlr} size & \ATOM{hdlr} data \\
\ATOM{minf} atom & \ATOM{minf} size & \ATOM{minf} data \\
\hline
\end{tabular}
\end{table}

\subsubsection{the mdhd Atom}

\begin{table}[h]
\begin{tabular}{|l|r|l|}
\hline
Field & Size & Value \\
\hline
atom length & 32 & 32/44 \\
atom type & 32 & `mdhd' (\texttt{0x6d646864}) \\
\hline
version & 8 & \texttt{0x00} \\
flags & 24 & \texttt{0x000000} \\
created date & 32/64 & creation date as Mac UTC \\
modified date & 32/64 & modification date as Mac UTC \\
time scale & 32 & sample rate \\
duration & 32/64 & total PCM frames \\
padding & 1 & \texttt{0} \\
language & 5 & \\
language & 5 & language value as ISO 639-2 \\
language & 5 & \\
QuickTime quality & 16 & \texttt{0} \\
\hline
\end{tabular}
\end{table}
Note the three, 5-bit \VAR{language} fields.
By adding 0x60 to each value and converting the result to ASCII characters,
the result is an \href{http://www.loc.gov/standards/iso639-2/}{ISO 639-2}
string of the file's language representation.
For example, given the values \texttt{0x15}, \texttt{0x0E} and \texttt{0x04}:
\begin{align*}
\text{language}_0 &= \texttt{0x15} + \texttt{0x60} = \texttt{0x75} = \texttt{u} \\
\text{language}_1 &= \texttt{0x0E} + \texttt{0x60} = \texttt{0x6E} = \texttt{n} \\
\text{language}_2 &= \texttt{0x04} + \texttt{0x60} = \texttt{0x64} = \texttt{d}
\end{align*}
Which is the code `\texttt{und}', meaning `undetermined' - which is typical.

\clearpage

\subsubsection{the hdlr Atom}
\label{alac_hdlr}
\begin{tabular}{|l|r|l|}
\hline
Field & Size & Value \\
\hline
atom length & 32 & 33 + component \\
atom type & 32 & `hdlr' (\texttt{0x68646c72}) \\
\hline
version & 8 & \texttt{0x00} \\
flags & 24 & \texttt{0x000000} \\
QuickTime type & 32 & \texttt{0x00000000} \\
QuickTime subtype & 32 & `soun' (\texttt{0x736f756e}) \\
QuickTime manufacturer & 32 & \texttt{0x00000000} \\
QuickTime component reserved flags & 32 & \texttt{0x00000000} \\
QuickTime component reserved flags mask & 32 & \texttt{0x00000000} \\
component name length & 8 & \texttt{0x00} \\
component name & component name length $\times$ 8 & \\
\hline
\end{tabular}


\subsubsection{the minf Atom}
\begin{tabular}{|l|r|l|}
\hline
Field & Size & Value \\
\hline
atom length & 32 & \ATOM{smhd} size + \ATOM{dinf} size + \ATOM{stbl} size + 8 \\
atom type & 32 & `minf' (\texttt{0x6d696e66}) \\
\hline
\ATOM{smhd} atom & \ATOM{smhd} size & \ATOM{smhd} data \\
\ATOM{dinf} atom & \ATOM{dinf} size & \ATOM{dinf} data \\
\ATOM{stbl} atom & \ATOM{stbl} size & \ATOM{stbl} data \\
\hline
\end{tabular}

\subsubsection{the smhd Atom}
\begin{tabular}{|l|r|l|}
\hline
Field & Size & Value \\
\hline
atom length & 32 & 16 \\
atom type & 32 & `smhd' (\texttt{0x736d6864}) \\
\hline
version & 8 & \texttt{0x00} \\
flags & 24 & \texttt{0x000000} \\
audio balance & 16 & \texttt{0x0000} \\
padding & 16 & \texttt{0x0000} \\
\hline
\end{tabular}

\subsubsection{the dinf Atom}
\begin{tabular}{|l|r|l|}
\hline
Field & Size & Value \\
\hline
atom length & 32 & \ATOM{dref} size + 8 \\
atom type & 32 & `dinf' (\texttt{0x64696e66}) \\
\hline
\ATOM{dref} atom & \ATOM{dref} size & \ATOM{dref} data \\
\hline
\end{tabular}

\clearpage

\subsubsection{the dref Atom}

\begin{table}[h]
\begin{tabular}{|l|r|l|}
\hline
Field & Size & Value \\
\hline
atom length & 32 & 28 \\
atom type & 32 & `dref' (\texttt{0x64726566}) \\
\hline
version & 8 & \texttt{0x00} \\
flags & 24 & \texttt{0x000000} \\
number of references & 32 & \texttt{1} \\
\hline
\hline
reference atom size & 32 & \texttt{12} \\
reference atom type & 32 & `url ' (\texttt{0x75726c20}) \\
reference atom data & 32 & \texttt{0x00000001} \\
\hline
\end{tabular}
\end{table}

\subsubsection{the stbl Atom}

\begin{table}[h]
\begin{tabular}{|l|r|l|}
\hline
Field & Size & Value \\
\hline
atom length & 32 & \ATOM{stsd} size + \ATOM{stts} size + \ATOM{stsc} size + \\
& & \ATOM{stsz} size + \ATOM{stco} size + 8 \\
atom type & 32 & `stbl' (\texttt{0x7374626c}) \\
\hline
\ATOM{stsd} atom & \ATOM{stsd} size & \ATOM{stsd} data \\
\ATOM{stts} atom & \ATOM{stts} size & \ATOM{stts} data \\
\ATOM{stsc} atom & \ATOM{stsc} size & \ATOM{stsc} data \\
\ATOM{stsz} atom & \ATOM{stsz} size & \ATOM{stsz} data \\
\ATOM{stco} atom & \ATOM{stco} size & \ATOM{stco} data \\
\hline
\end{tabular}
\end{table}

\subsubsection{the stsd Atom}

\begin{table}[h]
\begin{tabular}{|l|r|l|}
\hline
Field & Size & Value \\
\hline
atom length & 32 & \ATOM{alac} size + 16 \\
atom type & 32 & `stsd' (\texttt{0x73747364}) \\
\hline
version & 8 & \texttt{0x00} \\
flags & 24 & \texttt{0x000000} \\
number of descriptions & 32 & \texttt{1} \\
\hline
\ATOM{alac} atom & \ATOM{alac} size & \ATOM{alac} data \\
\hline
\end{tabular}
\end{table}

\clearpage

\subsubsection{the alac Atom}

\begin{table}[h]
\begin{tabular}{|l|r|l|}
\hline
Field & Size & Value \\
\hline
atom length & 32 & 72 \\
atom type & 32 & `alac' (\texttt{0x616c6163}) \\
\hline
reserved & 48 & \texttt{0x000000000000} \\
reference index & 16 & \texttt{1} \\
version & 16 & \texttt{0} \\
revision level & 16 & \texttt{0} \\
vendor & 32 & \texttt{0x00000000} \\
channels & 16 & channel count \\
bits per sample & 16 & bits per sample \\
compression ID & 16 & \texttt{0} \\
audio packet size & 16 & \texttt{0} \\
sample rate & 32 & \texttt{0xAC440000} \\
\hline
\hline
atom length & 32 & 36 \\
atom type & 32 & `alac' (\texttt{0x616c6163}) \\
\hline
padding & 32 & \texttt{0x00000000} \\
max samples per frame & 32 & largest number of PCM frames per ALAC frame \\
padding & 8 & \texttt{0x00} \\
sample size & 8 & bits per sample \\
history multiplier & 8 & \texttt{40} \\
initial history & 8 & \texttt{10} \\
maximum K & 8 & \texttt{14} \\
channels & 8 & channel count \\
unknown & 16 & \texttt{0x00FF} \\
max coded frame size & 32 & largest ALAC frame size, in bytes \\
bitrate & 32 & $((\text{\ATOM{mdat} size} \times 8 ) \div (\text{total PCM frames} \div \text{sample rate}))$ \\
sample rate & 32 & sample rate \\
\hline
\end{tabular}
\end{table}
The \VAR{history multiplier}, \VAR{initial history} and \VAR{maximum K}
values are encode-time options, typically set to 40, 10 and 14,
respectively.

Note that the \VAR{bitrate} field can't be known in advance;
we must fill that value with 0 for now and then
return to this atom once encoding is completed
and its size has been determined.

\clearpage

\subsubsection{the stts Atom}

\begin{table}[h]
\begin{tabular}{|l|r|l|}
\hline
Field & Size & Value \\
\hline
atom length & 32 & number of times $\times$ 8 + 16\\
atom type & 32 & `stts' (\texttt{0x73747473}) \\
\hline
version & 8 & \texttt{0x00} \\
flags & 24 & \texttt{0x000000} \\
number of times & 32 & \\
\hline
frame count 1 & 32 & number of occurrences \\
frame duration 1 & 32 & PCM frame count \\
\hline
\multicolumn{3}{|c|}{...} \\
\hline
\end{tabular}
\end{table}
This atom keeps track of how many different sizes of ALAC frames
occur in the ALAC file, in PCM frames.
It will typically have only two ``times'', the block size we're
using for most of our samples and the final block size for
any remaining samples.

For example, let's imagine encoding a 1 minute audio file
at 44100Hz with a block size of 4096 frames.
This file has a total of 2,646,000 PCM frames ($60 \times 44100 = 2646000$).
2,646,000 PCM frames divided by a 4096 block size means
we have 645 ALAC frames of size 4096, and 1 ALAC frame of size 4080.

Therefore:
\begin{align*}
\text{number of times} &= 2 \\
\text{frame count}_1 &= 645 \\
\text{frame duration}_1 &= 4096 \\
\text{frame count}_2 &= 1 \\
\text{frame duration}_2 &= 4080
\end{align*}

\subsubsection{the stsc Atom}

\begin{table}[h]
\begin{tabular}{|l|r|l|}
\hline
Field & Size & Value \\
\hline
atom length & 32 & entries $\times$ 12 + 16 \\
atom type & 32 & `stsc' (\texttt{0x73747363}) \\
\hline
version & 8 & \texttt{0x00} \\
flags & 24 & \texttt{0x000000} \\
number of entries & 32 & \\
\hline
first chunk & 32 & \\
ALAC frames per chunk & 32 & \\
description index & 32 & \texttt{1} \\
\hline
\multicolumn{3}{|c|}{...} \\
\hline
\end{tabular}
\end{table}

This atom stores how many ALAC frames are in a given ``chunk''.
In this instance a ``chunk'' represents an entry in
the \ATOM{stco} atom table, used for seeking backwards and forwards
through the file.
\VAR{First chunk} is the starting offset of its frames-per-chunk
value, beginning at 1.

As an example, let's take a one minute, 44100Hz audio file
that's been broken into 130 chunks
(each with an entry in the \ATOM{stco} atom).
Its \ATOM{stsc} entries would typically be:
\begin{align*}
\text{first chunk}_1 &= 1 \\
\text{frames per chunk}_1 &= 5 \\
\text{first chunk}_2 &= 130 \\
\text{frames per chunk}_2 &= 1
\end{align*}
What this means is that chunks 1 through 129 have 5 ALAC frames each
while chunk 130 has 1 ALAC frame.
This is a total of 646 ALAC frames, which matches the contents of
the \ATOM{stts} atom.

\subsubsection{the stsz Atom}

\begin{tabular}{|l|r|l|}
\hline
Field & Size & Value \\
\hline
atom length & 32 & sizes $\times$ 4 + 20 \\
atom type & 32 & `stsz' (\texttt{0x7374737a}) \\
\hline
version & 8 & \texttt{0x00} \\
flags & 24 & \texttt{0x000000} \\
block byte size & 32 & \texttt{0x00000000} \\
number of sizes & 32 & \\
\hline
frame size & 32 & \\
\hline
\multicolumn{3}{|c|}{...} \\
\hline
\end{tabular}

This atom is a list of ALAC frame sizes, each in bytes.
For example, our 646 frame file would have 646 corresponding
\ATOM{stsz} entries.

\subsubsection{the stco Atom}

\begin{tabular}{|l|r|l|}
\hline
Field & Size & Value \\
\hline
atom length & 32 & offset $\times$ 4 + 16 \\
atom type & 32 & `stco' (\texttt{0x7374636f}) \\
\hline
version & 8 & \texttt{0x00} \\
flags & 24 & \texttt{0x000000} \\
number of offsets & 32 & \\
\hline
frame offset & 32 & \\
\hline
\multicolumn{3}{|c|}{...} \\
\hline
\end{tabular}

This atom is a list of absolute file offsets for each chunk, where
each chunk is typically 5 ALAC frames large.

\clearpage

\subsubsection{the udta Atom}

\begin{tabular}{|l|r|l|}
\hline
Field & Size & Value \\
\hline
atom length & 32 & \ATOM{meta} size + 8 \\
atom type & 32 & `udta' (\texttt{0x75647461}) \\
\hline
\ATOM{meta} atom & \ATOM{meta} size & \ATOM{meta} data \\
\hline
\end{tabular}

\subsubsection{the meta Atom}

\begin{tabular}{|l|r|l|}
\hline
Field & Size & Value \\
\hline
atom length & 32 & \ATOM{hdlr} size + \ATOM{ilst} size + \ATOM{free} size + 12 \\
atom type & 32 & `meta' (\texttt{0x6d657461}) \\
\hline
version & 8 & \texttt{0x00} \\
flags & 24 & \texttt{0x000000} \\
\hline
\ATOM{hdlr} atom & \ATOM{hdlr} size & \ATOM{hdlr} data \\
\ATOM{ilst} atom & \ATOM{ilst} size & \ATOM{ilst} data \\
\ATOM{free} atom & \ATOM{free} size & \ATOM{free} data \\
\hline
\end{tabular}

\subsubsection{the hdlr atom (revisited)}

\begin{tabular}{|l|r|l|}
\hline
Field & Size & Value \\
\hline
atom length & 32 & 34 \\
atom type & 32 & `hdlr' (\texttt{0x68646c72}) \\
\hline
version & 8 & \texttt{0x00} \\
flags & 24 & \texttt{0x000000} \\
QuickTime type & 32 & \texttt{0x00000000} \\
QuickTime subtype & 32 & `mdir' (\texttt{0x6d646972}) \\
QuickTime manufacturer & 32 & `appl' (\texttt{0x6170706c}) \\
QuickTime component reserved flags & 32 & \texttt{0x00000000} \\
QuickTime component reserved flags mask & 32 & \texttt{0x00000000} \\
component name length & 8 & \texttt{0x00} \\
component name & 0 & \\
\hline
\end{tabular}

This atom is laid out identically to the ALAC file's primary
\ATOM{hdlr} atom (described on page \pageref{alac_hdlr}).
The only difference is the contents of its fields.

\subsubsection{the ilst Atom}

This atom is a collection of \ATOM{data} sub-atoms
and is described on page \pageref{m4a_meta}.

\subsubsection{the free Atom}

These atoms are simple collection of NULL bytes which can easily be
resized to make room for other atoms without rewriting the entire file.


\clearpage

\subsection{Encoding mdat Atom}
\ALGORITHM{PCM frames, various encoding parameters:
\newline
\begin{tabular}{rl}
parameter & typical value \\
\hline
block size & 4096 \\
initial history & 40 \\
history multiplier & 10 \\
maximum K & 14 \\
interlacing shift & 2 \\
minimum interlacing leftweight & 0 \\
maximum interlacing leftweight & 4 \\
\end{tabular}
}{an encoded \texttt{mdat} atom}
\SetKwData{BLOCKSIZE}{block size}
$0 \rightarrow$ \WRITE 32 unsigned bits\tcc*[r]{placeholder length}
$\texttt{"mdat"} \rightarrow$ \WRITE 4 bytes\;
\While{PCM frames remain}{
  take \BLOCKSIZE PCM frames from the input\;
  \hyperref[alac:encode_frameset]{write PCM frames to frameset}\;
}
return to start of \texttt{mdat} atom and write actual length\;
\EALGORITHM
\begin{figure}[h]
\includegraphics{alac/figures/stream.pdf}
\end{figure}

\clearpage

\subsection{Encoding Frameset}
\label{alac:encode_frameset}
{\relsize{-2}
  \input{alac/algorithms/encode_frameset}
}

%% \subsubsection{Channel Assignment}
%% \begin{tabular}{r|l}
%% channels & assignment \\
%% \hline
%% 1 & mono \\
%% 2 & left, right \\
%% 3 & center, left, right \\
%% 4 & center, left, right, center surround \\
%% 5 & center, left, right, left surround, right surround \\
%% 6 & center, left, right, left surround, right surround, LFE \\
%% 7 & center, left, right, left surround, right surround, center surround, LFE \\
%% 8 & center, left center, right center, left, right, left surround, right surround, LFE \\
%% \end{tabular}

\clearpage

\subsection{Encoding Frame}
\label{alac:encode_frame}
{\relsize{-1}
  \input{alac/algorithms/encode_frame}
}

\clearpage

\subsection{Encoding Uncompressed Frame}
\label{alac:write_uncompressed_frame}
{\relsize{-1}
  \input{alac/algorithms/encode_uncompressed_frame}
}

\begin{figure}[h]
  \includegraphics{alac/figures/uncompressed_frame.pdf}
\end{figure}

\clearpage

\subsection{Encoding Compressed Frame}
\label{alac:write_compressed_frame}
{\relsize{-1}
  \input{alac/algorithms/encode_compressed_frame}
}

\clearpage

\subsection{Encoding Non-Interlaced Frame}
\label{alac:write_non_interlaced_frame}
{\relsize{-1}
  \input{alac/algorithms/encode_noninterlaced_frame}
}
\begin{figure}[h]
  \includegraphics{alac/figures/noninterlaced_frame.pdf}
\end{figure}

\clearpage

\subsection{Writing Subframe Header}
\label{alac:write_subframe_header}
\input{alac/algorithms/write_subframe_header}
\begin{figure}[h]
\includegraphics{alac/figures/subframe_header.pdf}
\end{figure}
\par
\noindent
For example, given the QLP coefficients
\texttt{1170, -1088, 565, -161},
the subframe header is written as:
\begin{figure}[h]
\includegraphics{alac/figures/subframe-build.pdf}
\end{figure}

\clearpage

\subsection{Encoding Interlaced Frame}
\label{alac:write_interlaced_frame}
{\relsize{-1}
  \input{alac/algorithms/encode_interlaced_frame}
}

\clearpage

\begin{figure}[h]
  \includegraphics{alac/figures/interlaced_frame.pdf}
\end{figure}

\subsubsection{Correlating Channels}
\label{alac:correlate_channels}
{\relsize{-1}
  \input{alac/algorithms/correlate_channels}
\par
\noindent
For example, given an \VAR{interlacing shift} value of 2 and an
\VAR{interlacing leftweight} value of 3:
\par
\noindent
{\relsize{-1}
\begin{tabular}{r||r|r||>{$}r<{$}|>{$}r<{$}|}
$i$ & $\textsf{channel}_{0~i}$ & $\textsf{channel}_{1~i}$ & \textsf{correlated}_{0~i} & \textsf{correlated}_{1~i} \\
\hline
0 & 18 & 2 & 2 + \lfloor((18 - 2) \times 3) \div 2 ^ 2\rfloor = 14 & 18 - 2 = 16 \\
1 & 20 & 3 & 3 + \lfloor((20 - 3) \times 3) \div 2 ^ 2\rfloor = 15 & 20 - 3 = 17 \\
2 & 26 & 0 & 0 + \lfloor((26 - 0) \times 3) \div 2 ^ 2\rfloor = 19 & 26 - 0 = 26 \\
3 & 24 & -1 & -1 + \lfloor((24 + 1) \times 3) \div 2 ^ 2\rfloor = 17 & 24 + 1 = 25 \\
4 & 24 & 0 & 0 + \lfloor((24 - 0) \times 3) \div 2 ^ 2\rfloor = 18 & 24 - 0 = 24 \\
\end{tabular}
}
}

\clearpage

%This work is licensed under the
%Creative Commons Attribution-Share Alike 3.0 United States License.
%To view a copy of this license, visit
%http://creativecommons.org/licenses/by-sa/3.0/us/ or send a letter to
%Creative Commons,
%171 Second Street, Suite 300,
%San Francisco, California, 94105, USA.

\subsection{Encoding LPC Subframe}
\label{flac:encode_lpc_subframe}
{\relsize{-1}
  \input{flac/algorithms/encode_lpc_subframe}
}

\clearpage

\subsubsection{Calculating QLP Precision}
\label{flac:calculate_qlp_precision}
{\relsize{-2}
  \input{flac/algorithms/calculate_qlp_precision}
}

\subsubsection{Windowing the Input Samples}
\label{flac:window}
{\relsize{-1}
  \input{flac/algorithms/encode_window_samples}
}

\par
For example, given the input samples: \texttt{[18, 20, 26, 24, 24, 23, 21, 24, 23, 20]}
\begin{wrapfigure}[5]{r}{3in}
\includegraphics{flac/figures/tukey.pdf}
\end{wrapfigure}
\begin{table}[h]
\begin{tabular}{r|r|r|r}
$i$ & $\textsf{sample}_i$ & $\textsf{Tukey}_i$ & $\textsf{windowed}_i$ \\
\hline
0 & 18 & 0.00 & 0.0 \\
1 & 20 & 0.41 & 8.2 \\
2 & 26 & 0.97 & 25.2 \\
3 & 24 & 1.00 & 24.0 \\
4 & 24 & 1.00 & 24.0 \\
5 & 23 & 1.00 & 23.0 \\
6 & 21 & 1.00 & 21.0 \\
7 & 24 & 0.97 & 23.3 \\
8 & 23 & 0.41 & 9.4 \\
9 & 20 & 0.00 & 0.0 \\
\end{tabular}
\end{table}

\clearpage

\subsubsection{Performing Autocorrelation}
\label{flac:autocorrelate}
{\relsize{-1}
  \input{flac/algorithms/encode_autocorrelate}
}
For example, given the windowed samples:
\texttt{[0.0, 8.2, 25.2, 24.0, 24.0, 23.0, 21.0, 23.3, 9.4, 0.0]}
and a maximum LPC order of 3:
\begin{figure}[h]
\subfloat{
  {\relsize{-2}
    \begin{tabular}{rrrrr}
      \texttt{0.0} & $\times$ & \texttt{0.0} & $=$ & \texttt{0.00} \\
      \texttt{8.2} & $\times$ & \texttt{8.2} & $=$ & \texttt{67.24} \\
      \texttt{25.2} & $\times$ & \texttt{25.2} & $=$ & \texttt{635.04} \\
      \texttt{24.0} & $\times$ & \texttt{24.0} & $=$ & \texttt{576.00} \\
      \texttt{24.0} & $\times$ & \texttt{24.0} & $=$ & \texttt{576.00} \\
      \texttt{23.0} & $\times$ & \texttt{23.0} & $=$ & \texttt{529.00} \\
      \texttt{21.0} & $\times$ & \texttt{21.0} & $=$ & \texttt{441.00} \\
      \texttt{23.3} & $\times$ & \texttt{23.3} & $=$ & \texttt{542.89} \\
      \texttt{9.4} & $\times$ & \texttt{9.4} & $=$ & \texttt{88.36} \\
      \texttt{0.0} & $\times$ & \texttt{0.0} & $=$ & \texttt{0.00} \\
      \hline
      \multicolumn{3}{r}{$\textsf{autocorrelated}_0$} & $=$ & \texttt{3455.53} \\
    \end{tabular}
  }
}
\includegraphics{flac/figures/lag0.pdf}

\subfloat{
  {\relsize{-2}
    \begin{tabular}{rrrrr}
      \texttt{0.0} & $\times$ & \texttt{8.2} & $=$ & \texttt{0.00} \\
      \texttt{8.2} & $\times$ & \texttt{25.2} & $=$ & \texttt{206.64} \\
      \texttt{25.2} & $\times$ & \texttt{24.0} & $=$ & \texttt{604.80} \\
      \texttt{24.0} & $\times$ & \texttt{24.0} & $=$ & \texttt{576.00} \\
      \texttt{24.0} & $\times$ & \texttt{23.0} & $=$ & \texttt{552.00} \\
      \texttt{23.0} & $\times$ & \texttt{21.0} & $=$ & \texttt{483.00} \\
      \texttt{21.0} & $\times$ & \texttt{23.3} & $=$ & \texttt{489.30} \\
      \texttt{23.3} & $\times$ & \texttt{9.4} & $=$ & \texttt{219.02} \\
      \texttt{9.4} & $\times$ & \texttt{0.0} & $=$ & \texttt{0.00} \\
      \hline
      \multicolumn{3}{r}{$\textsf{autocorrelated}_1$} & $=$ & \texttt{3130.76} \\
    \end{tabular}
  }
}
\includegraphics{flac/figures/lag1.pdf}

\subfloat{
  {\relsize{-2}
    \begin{tabular}{rrrrr}
      \texttt{0.0} & $\times$ & \texttt{25.2} & $=$ & \texttt{0.00} \\
      \texttt{8.2} & $\times$ & \texttt{24.0} & $=$ & \texttt{196.80} \\
      \texttt{25.2} & $\times$ & \texttt{24.0} & $=$ & \texttt{604.80} \\
      \texttt{24.0} & $\times$ & \texttt{23.0} & $=$ & \texttt{552.00} \\
      \texttt{24.0} & $\times$ & \texttt{21.0} & $=$ & \texttt{504.00} \\
      \texttt{23.0} & $\times$ & \texttt{23.3} & $=$ & \texttt{535.90} \\
      \texttt{21.0} & $\times$ & \texttt{9.4} & $=$ & \texttt{197.40} \\
      \texttt{23.3} & $\times$ & \texttt{0.0} & $=$ & \texttt{0.00} \\
      \hline
      \multicolumn{3}{r}{$\textsf{autocorrelated}_2$} & $=$ & \texttt{2590.90} \\
    \end{tabular}
  }
}
\includegraphics{flac/figures/lag2.pdf}

\subfloat{
  {\relsize{-2}
    \begin{tabular}{rrrrr}
      \texttt{0.0} & $\times$ & \texttt{24.0} & $=$ & \texttt{0.00} \\
      \texttt{8.2} & $\times$ & \texttt{24.0} & $=$ & \texttt{196.80} \\
      \texttt{25.2} & $\times$ & \texttt{23.0} & $=$ & \texttt{579.60} \\
      \texttt{24.0} & $\times$ & \texttt{21.0} & $=$ & \texttt{504.00} \\
      \texttt{24.0} & $\times$ & \texttt{23.3} & $=$ & \texttt{559.20} \\
      \texttt{23.0} & $\times$ & \texttt{9.4} & $=$ & \texttt{216.20} \\
      \texttt{21.0} & $\times$ & \texttt{0.0} & $=$ & \texttt{0.00} \\
      \hline
      \multicolumn{3}{r}{$\textsf{autocorrelated}_3$} & $=$ & \texttt{2055.80} \\
    \end{tabular}
  }
}
\includegraphics{flac/figures/lag3.pdf}
\end{figure}
\par
\noindent
Note that the total number of autocorrelation values equals
the maximum LPC order + 1.

\clearpage

\subsubsection{LP Coefficient Calculation}
\label{flac:compute_lp_coeffs}
\input{flac/algorithms/encode_lp_coeffs}

\clearpage

\subsubsection{LP Coefficient Calculation Example}
Given a maximum LPC order of 3 and 4 autocorrelation values:
{\relsize{-1}
  \begin{align*}
    \kappa_0 &\leftarrow \textsf{autocorrelation}_1 \div \textsf{autocorrelation}_0 \\
    \textsf{LP coefficient}_{0~0} &\leftarrow \kappa_0 \\
    \textsf{error}_0 &\leftarrow \textsf{autocorrelation}_0 \times (1 - {\kappa_0} ^ 2) \\
    i &= 1 \\
    q_1 &\leftarrow \textsf{autocorrelation}_2 - (\textsf{LP coefficient}_{0~0} \times \textsf{autocorrelation}_{1}) \\
    \kappa_1 &\leftarrow q_1 \div error_0 \\
    \textsf{LP coefficient}_{1~0} &\leftarrow \textsf{LP coefficient}_{0~0} - (\kappa_1 \times \textsf{LP coefficient}_{0~0}) \\
    \textsf{LP coefficient}_{1~1} &\leftarrow \kappa_1 \\
    \textsf{error}_1 &\leftarrow \textsf{error}_0 \times (1 - {\kappa_1} ^ 2) \\
    i &= 2 \\
    q_2 &\leftarrow \textsf{autocorrelation}_3 - (\textsf{LP coefficient}_{1~0} \times \textsf{autocorrelation}_{2} + \textsf{LP coefficient}_{1~1} \times \textsf{autocorrelation}_{1}) \\
    \kappa_2 &\leftarrow q_2 \div \textsf{error}_1 \\
    \textsf{LP coefficient}_{2~0} &\leftarrow \textsf{LP coefficient}_{1~0} - (\kappa_2 \times \textsf{LP coefficient}_{1~1}) \\
    \textsf{LP coefficient}_{2~1} &\leftarrow \textsf{LP coefficient}_{1~1} - (\kappa_2 \times \textsf{LP coefficient}_{1~0}) \\
    \textsf{LP coefficient}_{2~2} &\leftarrow \kappa_2 \\
    \textsf{error}_2 &\leftarrow \textsf{error}_1 \times (1 - {\kappa_2} ^ 2) \\
\end{align*}
}
\par
\noindent
With \textsf{autocorrelation} values: \texttt{[3455.53, 3130.76, 2590.90, 2055.80]}
{\relsize{-1}
  \begin{align*}
    \kappa_0 &\leftarrow 3130.76 \div 3455.53 = 0.906 \\
    \textsf{LP coefficient}_{0~0} &\leftarrow \textbf{0.906} \\
    \textsf{error}_0 &\leftarrow 3455.53 \times (1 - {0.906} ^ 2) = \textbf{619.107} \\
    i &= 1 \\
    q_1 &\leftarrow 2590.90 - (0.906 \times 3130.76) = -245.569 \\
    \kappa_1 &\leftarrow -245.569 \div 619.107 = -0.397 \\
    \textsf{LP coefficient}_{1~0} &\leftarrow 0.906 - (-0.397 \times 0.906) = \textbf{1.266} \\
    \textsf{LP coefficient}_{1~1} &\leftarrow \textbf{-0.397} \\
    \textsf{error}_1 &\leftarrow 619.107 \times (1 - {-0.397} ^ 2) = \textbf{521.530} \\
    i &= 2 \\
    q_2 &\leftarrow 2055.80 - (1.266 \times 2590.90 + -0.397 \times 3130.76) = 18.632 \\
    \kappa_2 &\leftarrow 18.632 \div 521.53 = 0.036 \\
    \textsf{LP coefficient}_{2~0} &\leftarrow 1.266 - (0.036 \times -0.397) = \textbf{1.28} \\
    \textsf{LP coefficient}_{2~1} &\leftarrow -0.397 - (0.036 \times 1.266) = \textbf{-0.443} \\
    \textsf{LP coefficient}_{2~2} &\leftarrow \textbf{0.036} \\
    \textsf{error}_2 &\leftarrow 521.53 \times (1 - {0.036} ^ 2) = \textbf{520.854} \\
  \end{align*}
}

\clearpage

\subsubsection{Estimating Best Order}
\label{flac:estimate_best_order}
\input{flac/algorithms/encode_best_order}

\clearpage

\subsubsection{Estimating Best Order Example}

Given the error values \texttt{[619.107, 521.530, 520.854]},
a block size of 10, 16 bits per sample, a QLP precision of 12 and maximum LPC order of 3:
\begin{align*}
  \textsf{error scale} &\leftarrow \frac{({\log_e 2}) ^ 2}{10 \times 2} = 0.024 \\
  i &\leftarrow 0 \\
  o &\leftarrow 0 + 1 = 1 \\
  \textsf{header bits}_1 &\leftarrow 1 \times (16 + 12) = 28 \\
  \textsf{bits per residual}_1 &\leftarrow \frac{\log_e(619.107 \times 0.024)}{({\log_e 2}) \times 2} = 1.947 \\
  \textsf{subframe bits}_1 &\leftarrow 28 + 1.947 \times (10 - 1) = \textbf{45.523} \\
  i &\leftarrow 1 \\
  o &\leftarrow 1 + 1 = 2 \\
  \textsf{header bits}_2 &\leftarrow 2 \times (16 + 12) = 56 \\
  \textsf{bits per residual}_2 &\leftarrow \frac{\log_e(521.530 \times 0.024)}{({\log_e 2}) \times 2} = 1.823 \\
  \textsf{subframe bits}_2 &\leftarrow 56 + 1.823 \times (10 - 2) = \textbf{70.584} \\
  i &\leftarrow 2 \\
  o &\leftarrow 2 + 1 = 3 \\
  \textsf{header bits}_3 &\leftarrow 3 \times (16 + 12) = 84 \\
  \textsf{bits per residual}_3 &\leftarrow \frac{\log_e(520.854 \times 0.024)}{({\log_e 2}) \times 2} = 1.822 \\
  \textsf{subframe bits}_3 &\leftarrow 84 + 1.822 \times (10 - 3) = \textbf{96.754} \\
\end{align*}
\par
\noindent
Since the $\textsf{subframe bits}_1$ value of 45.523 is the smallest,
the best LPC order to use is 1.

\clearpage

\subsubsection{Quantizing LP Coefficients}
\label{flac:quantize_lp_coeffs}
{\relsize{-1}
  \input{flac/algorithms/encode_quantize_coeffs}
}

\clearpage

\subsubsection{Quantizing LP Coefficients Example}

Given the $\textsf{LP coefficient}_3$ \texttt{[1.280, -0.443, 0.036]},
an \textsf{order} of \texttt{3} and a \textsf{QLP precision} \texttt{12}:
\begin{align*}
l &\leftarrow 1.280 \\
\textsf{QLP shift} &\leftarrow 12 - \lfloor \log_2(1.280) \rfloor - 2 = 10 \\
\textsf{QLP max} &\leftarrow 2 ^ {12 - 1} - 1 = 2047 \\
\textsf{QLP min} &\leftarrow -(2 ^ {12 - 1}) = -2048 \\
\textsf{error} &\leftarrow 0.0 \\
i &= 0 \\
\textsf{error} &\leftarrow 0.0 + 1.280 \times 2 ^ {10} = 1310.72 \\
\textsf{QLP coefficient}_0 &\leftarrow 1311 \\
\textsf{error} &\leftarrow 1310.72 - 1311 = -0.28 \\
i &= 1 \\
\textsf{error} &\leftarrow -0.28 + -0.443 \times 2 ^ {10} = -453.912 \\
\textsf{QLP coefficient}_1 &\leftarrow -454 \\
\textsf{error} &\leftarrow -453.912 - -454 = 0.088 \\
i &= 2 \\
\textsf{error} &\leftarrow 0.088 + 0.036 \times 2 ^ {10} = 36.952 \\
\textsf{QLP coefficient}_2 &\leftarrow 37 \\
\textsf{error} &\leftarrow 36.952 - 37 = -0.048 \\
\end{align*}
\par
\noindent
Resulting in the QLP coefficients \texttt{1311, -454, 37}
and a QLP shift of \texttt{10}.
These values, in addition to QLP precision,
are inserted directly into a desired QLP subframe header
and are also used to calculate its residuals.

\clearpage

\subsection{Writing an LPC Subframe}
\label{flac:write_lpc_subframe}
{\relsize{-1}
  \input{flac/algorithms/write_lpc_subframe}
}
\begin{figure}[h]
\includegraphics{flac/figures/lpc.pdf}
\end{figure}

\clearpage

\subsubsection{LPC Subframe Residuals Calculation Example}
{\relsize{-1}
  \begin{tabular}{rl}
    \textsf{samples} : & \texttt{[18, 20, 26, 24, 24, 23, 21, 24, 23, 20]} \\
    \textsf{block size} : & 10 \\
    \textsf{subframe's bits per sample} : & 16 \\
    \textsf{wasted BPS} : & 0 \\
    \textsf{LPC order} : & \texttt{3} \\
    \textsf{QLP precision} : &\texttt{12} \\
    \textsf{QLP shift needed} : & \texttt{10} \\
    \textsf{QLP coefficients} : & \texttt{[1311, -454, 37]} \\
  \end{tabular}
  \newline
  \begin{align*}
    \textsf{residual}_0 &\leftarrow 24 - \left\lfloor\frac{(1311 \times 26) + (-454 \times 20) + (37 \times 18)}{2 ^ {10}}\right\rfloor = -1 \\
    \textsf{residual}_1 &\leftarrow 24 - \left\lfloor\frac{(1311 \times 24) + (-454 \times 26) + (37 \times 20)}{2 ^ {10}}\right\rfloor = 5 \\
    \textsf{residual}_2 &\leftarrow 23 - \left\lfloor\frac{(1311 \times 24) + (-454 \times 24) + (37 \times 26)}{2 ^ {10}}\right\rfloor = 2 \\
    \textsf{residual}_3 &\leftarrow 21 - \left\lfloor\frac{(1311 \times 23) + (-454 \times 24) + (37 \times 24)}{2 ^ {10}}\right\rfloor = 2 \\
    \textsf{residual}_4 &\leftarrow 24 - \left\lfloor\frac{(1311 \times 21) + (-454 \times 23) + (37 \times 24)}{2 ^ {10}}\right\rfloor = 7 \\
    \textsf{residual}_5 &\leftarrow 23 - \left\lfloor\frac{(1311 \times 24) + (-454 \times 21) + (37 \times 23)}{2 ^ {10}}\right\rfloor = 1 \\
    \textsf{residual}_6 &\leftarrow 20 - \left\lfloor\frac{(1311 \times 23) + (-454 \times 24) + (37 \times 21)}{2 ^ {10}}\right\rfloor = 1
  \end{align*}
  Leading to a final set of 7 residual values: \texttt{[-1, 5, 2, 2, 7, 1, 1]}.
  Encoding them to a residual block, our final LPC subframe is:
}
\begin{figure}[h]
\includegraphics{flac/figures/lpc-parse2.pdf}
\end{figure}


\clearpage

%This work is licensed under the
%Creative Commons Attribution-Share Alike 3.0 United States License.
%To view a copy of this license, visit
%http://creativecommons.org/licenses/by-sa/3.0/us/ or send a letter to
%Creative Commons,
%171 Second Street, Suite 300,
%San Francisco, California, 94105, USA.

\subsection{Residual Encoding}
\label{flac:write_residual_block}
{\relsize{-1}
  \input{flac/algorithms/encode_residual}
}
\begin{figure}[h]
\includegraphics{flac/figures/residual.pdf}
\end{figure}

\clearpage

\subsubsection{Calculating Residual Partitions}
\label{flac:calculate_residual_partitions}
{\relsize{-1}
  \input{flac/algorithms/calculate_residual_partitions}
}

\clearpage

\subsubsection{Residual Encoding Example}
Given a set of residuals \texttt{[2, 6, -2, 0, -1, -2, 3, -1, -3]},
block size of 10 and predictor order of 1:
{\relsize{-1}
  \begin{align*}
  \intertext{$\text{partition order}~o = 0$:}
  \textsf{partition start}_{0~0} &\leftarrow 0 \\
  \textsf{partition samples}_{0~0} &\leftarrow 10 \div 2 ^ 0 - 1 = 9 \\
  \textsf{partition}_{0~0} &\leftarrow \texttt{[2, 6, -2, 0, -1, -2, 3, -1, -3]} \\
  \textsf{partition sum}_{0~0} &\leftarrow 2 + 6 + 2 + 0 + 1 + 2 + 3 + 1 + 3 = 20 \\
  \textsf{Rice}_{0~0} &\leftarrow \lfloor\log_2 (20 \div 9) \rfloor = 1 \\
  \textsf{partition size}_{0~0} &\leftarrow 4 + ((1 + 1) \times 9) + \left\lfloor\frac{20}{2 ^ {1 - 1}}\right\rfloor - \left\lfloor\frac{9}{2}\right\rfloor = \textbf{38} \\
  \intertext{$\text{partition order}~o = 1$:}
  \textsf{partition start}_{1~0} &\leftarrow 0 \\
  \textsf{partition samples}_{1~0} &\leftarrow 10 \div 2 ^ 1 - 1 = 4 \\
  \textsf{partition}_{1~0} &\leftarrow \texttt{[2, 6, -2, 0]} \\
  \textsf{partition sum}_{1~0} &\leftarrow 2 + 6 + 2 + 0 = 10 \\
  \textsf{Rice}_{1~0} &\leftarrow \lfloor\log_2 (10 \div 4) \rfloor = 1 \\
  \textsf{partition size}_{1~0} &\leftarrow 4 + ((1 + 1) \times 4) + \left\lfloor\frac{10}{2 ^ {1 - 1}}\right\rfloor - \left\lfloor\frac{4}{2}\right\rfloor = \textbf{20} \\
  \textsf{partition start}_{1~1} &\leftarrow (1 \times 10 \div 2 ^ {1}) - 1 = 4 \\
  \textsf{partition samples}_{1~1} &\leftarrow 10 \div 2 ^ 1 = 5 \\
  \textsf{partition}_{1~1} &\leftarrow \texttt{[-1, -2, 3, -1, -3]} \\
  \textsf{partition sum}_{1~1} &\leftarrow 1 + 2 + 3 + 1 + 3 = 10 \\
  \textsf{Rice}_{1~1} &\leftarrow \lfloor\log_2 (10 \div 5) \rfloor = 1 \\
  \textsf{partition size}_{1~1} &\leftarrow 4 + ((1 + 1) \times 5) + \left\lfloor\frac{10}{2 ^ {1 - 1}}\right\rfloor - \left\lfloor\frac{5}{2}\right\rfloor = \textbf{22} \\
\end{align*}
\par
\noindent
Since block size of $10 \bmod 2 ^ 2 \neq 0$, we stop at partition order 1
because the list of residuals can't be divided equally into more partitions.
And because $\textsf{total partitions size}_0$ of 38 is smaller than
$\textsf{total partitions size}_1$ of 42 (20 + 22), we use partition order 0
to encode our residuals into a single partition with 9 residuals.
}
\begin{figure}[h]
  \includegraphics{flac/figures/residuals-enc-example.pdf}
\end{figure}


