%This work is licensed under the
%Creative Commons Attribution-Share Alike 3.0 United States License.
%To view a copy of this license, visit
%http://creativecommons.org/licenses/by-sa/3.0/us/ or send a letter to
%Creative Commons,
%171 Second Street, Suite 300,
%San Francisco, California, 94105, USA.

\subsection{ALAC Atoms}
\begin{wrapfigure}[6]{r}{1.5in}
\includegraphics{alac/figures/atoms.pdf}
\end{wrapfigure}
We'll encode our ALAC file in iTunes order, which means
it contains the \ATOM{ftyp}, \ATOM{moov}, \ATOM{free} and
\ATOM{mdat} atoms, in that order.

\subsubsection{the ftyp Atom}

\begin{table}[h]
\begin{tabular}{|l|r|l|}
\hline
Field & Size & Value \\
\hline
atom length & 32 & 32 \\
atom type & 32 & `ftyp' (\texttt{0x66747970}) \\
\hline
major brand & 32 & `M4A ' (\texttt{0x4d344120}) \\
major brand version & 32 & \texttt{0} \\
compatible brand & 32 & `M4A ' (\texttt{0x4d344120}) \\
compatible brand & 32 & `mp42' (\texttt{0x6d703432}) \\
compatible brand & 32 & `isom' (\texttt{0x69736f6d}) \\
compatible brand & 32 & \texttt{0x00000000} \\
\hline
\end{tabular}
\end{table}

\subsubsection{the moov Atom}

\begin{table}[h]
\begin{tabular}{|l|r|l|}
\hline
Field & Size & Value \\
\hline
atom length & 32 & \ATOM{mvhd} size + \ATOM{trak} size + \ATOM{udta} size + 8 \\
atom type & 32 & `moov' (\texttt{0x6d6f6f76}) \\
\hline
\ATOM{mvhd} atom & \ATOM{mvhd} size & \ATOM{mvhd} data \\
\ATOM{trak} atom & \ATOM{trak} size & \ATOM{trak} data \\
\ATOM{udta} atom & \ATOM{udta} size & \ATOM{udta} data \\
\hline
\end{tabular}
\end{table}

\clearpage

\subsubsection{the mvhd Atom}

\begin{table}[h]
\begin{tabular}{|l|r|l|}
\hline
Field & Size & Value \\
\hline
atom length & 32 & 108/120 \\
atom type & 32 & `mvhd' (\texttt{0x6d766864}) \\
\hline
version & 8 & \texttt{0x00} \\
flags & 24 & \texttt{0x000000} \\
created date & 32/64 & creation date as Mac UTC \\
modified date & 32/64 & modification date as Mac UTC \\
time scale & 32 & sample rate \\
duration & 32/64 & total PCM frames \\
playback speed & 32 & \texttt{0x10000} \\
user volume & 16 & \texttt{0x100} \\
padding & 80 & \texttt{0x00000000000000000000} \\
window geometry matrix a & 32 & \texttt{0x10000} \\
window geometry matrix b & 32 & \texttt{0} \\
window geometry matrix u & 32 & \texttt{0} \\
window geometry matrix c & 32 & \texttt{0} \\
window geometry matrix d & 32 & \texttt{0x10000} \\
window geometry matrix v & 32 & \texttt{0} \\
window geometry matrix x & 32 & \texttt{0} \\
window geometry matrix y & 32 & \texttt{0} \\
window geometry matrix w & 32 & \texttt{0x40000000} \\
QuickTime preview & 64 & \texttt{0} \\
QuickTime still poster & 32 & \texttt{0} \\
QuickTime selection time & 64 & \texttt{0} \\
QuickTime current time & 32 & \texttt{0} \\
next track ID & 32 & \texttt{2} \\
\hline
\end{tabular}
\end{table}

If \VAR{version} is 0, \VAR{created date}, \VAR{modified date} and
\VAR{duration} are 32 bit fields.
Otherwise, they are 64 bit fields.
The \VAR{created date} and \VAR{modified date} are seconds
since the Macintosh Epoch, which is 00:00:00, January 1st, 1904.\footnote{Why 1904?  It's the first leap year of the 20th century.}
To convert a Unix Epoch timestamp (seconds since January 1st, 1970) to
a Macintosh Epoch, one needs to add 24,107 days -
or \texttt{2082844800} seconds.

\clearpage

\subsubsection{the trak Atom}
\begin{tabular}{|l|r|l|}
\hline
Field & Size & Value \\
\hline
atom length & 32 & \ATOM{tkhd} size + \ATOM{mdia} size + 8 \\
atom type & 32 & `trak' (\texttt{0x7472616b}) \\
\hline
\ATOM{tkhd} atom & \ATOM{tkhd} size & \ATOM{tkhd} data \\
\ATOM{mdia} atom & \ATOM{mdia} size & \ATOM{mdia} data \\
\hline
\end{tabular}

\subsubsection{the tkhd Atom}

\begin{table}[h]
\begin{tabular}{|l|r|l|}
\hline
Field & Size & Value \\
\hline
atom length & 32 & 92/104 \\
atom type & 32 & `tkhd' (\texttt{0x746b6864}) \\
\hline
version & 8 & \texttt{0x00} \\
padding & 20 & \texttt{0x000000} \\
track in poster & 1 & \texttt{0} \\
track in preview & 1 & \texttt{1} \\
track in movie & 1 & \texttt{1} \\
track enabled & 1 & \texttt{1} \\
created date & 32/64 & creation date as Mac UTC \\
modified date & 32/64 & modification date as Mac UTC \\
track ID & 32 & \texttt{1} \\
padding & 32 & \texttt{0x00000000} \\
duration & 32/64 & total PCM frames \\
padding & 64 & \texttt{0x0000000000000000} \\
video layer & 16 & \texttt{0} \\
QuickTime alternate & 16 & \texttt{0} \\
volume & 16 & \texttt{0x1000} \\
padding & 16 & \texttt{0x0000} \\
video geometry matrix a & 32 & \texttt{0x10000} \\
video geometry matrix b & 32 & \texttt{0} \\
video geometry matrix u & 32 & \texttt{0} \\
video geometry matrix c & 32 & \texttt{0} \\
video geometry matrix d & 32 & \texttt{0x10000} \\
video geometry matrix v & 32 & \texttt{0} \\
video geometry matrix x & 32 & \texttt{0} \\
video geometry matrix y & 32 & \texttt{0} \\
video geometry matrix w & 32 & \texttt{0x40000000} \\
video width & 32 & \texttt{0} \\
video height & 32 & \texttt{0} \\
\hline
\end{tabular}
\end{table}

\clearpage

\subsubsection{the mdia Atom}

\begin{table}[h]
\begin{tabular}{|l|r|l|}
\hline
Field & Size & Value \\
\hline
atom length & 32 & \ATOM{mdhd} size + \ATOM{hdlr} size + \ATOM{minf} size + 8 \\
atom type & 32 & `mdia' (\texttt{0x6d646961}) \\
\hline
\ATOM{mdhd} atom & \ATOM{mdhd} size & \ATOM{mdhd} data \\
\ATOM{hdlr} atom & \ATOM{hdlr} size & \ATOM{hdlr} data \\
\ATOM{minf} atom & \ATOM{minf} size & \ATOM{minf} data \\
\hline
\end{tabular}
\end{table}

\subsubsection{the mdhd Atom}

\begin{table}[h]
\begin{tabular}{|l|r|l|}
\hline
Field & Size & Value \\
\hline
atom length & 32 & 32/44 \\
atom type & 32 & `mdhd' (\texttt{0x6d646864}) \\
\hline
version & 8 & \texttt{0x00} \\
flags & 24 & \texttt{0x000000} \\
created date & 32/64 & creation date as Mac UTC \\
modified date & 32/64 & modification date as Mac UTC \\
time scale & 32 & sample rate \\
duration & 32/64 & total PCM frames \\
padding & 1 & \texttt{0} \\
language & 5 & \\
language & 5 & language value as ISO 639-2 \\
language & 5 & \\
QuickTime quality & 16 & \texttt{0} \\
\hline
\end{tabular}
\end{table}
Note the three, 5-bit \VAR{language} fields.
By adding 0x60 to each value and converting the result to ASCII characters,
the result is an \href{http://www.loc.gov/standards/iso639-2/}{ISO 639-2}
string of the file's language representation.
For example, given the values \texttt{0x15}, \texttt{0x0E} and \texttt{0x04}:
\begin{align*}
\text{language}_0 &= \texttt{0x15} + \texttt{0x60} = \texttt{0x75} = \texttt{u} \\
\text{language}_1 &= \texttt{0x0E} + \texttt{0x60} = \texttt{0x6E} = \texttt{n} \\
\text{language}_2 &= \texttt{0x04} + \texttt{0x60} = \texttt{0x64} = \texttt{d}
\end{align*}
Which is the code `\texttt{und}', meaning `undetermined' - which is typical.

\clearpage

\subsubsection{the hdlr Atom}
\label{alac_hdlr}
\begin{tabular}{|l|r|l|}
\hline
Field & Size & Value \\
\hline
atom length & 32 & 33 + component \\
atom type & 32 & `hdlr' (\texttt{0x68646c72}) \\
\hline
version & 8 & \texttt{0x00} \\
flags & 24 & \texttt{0x000000} \\
QuickTime type & 32 & \texttt{0x00000000} \\
QuickTime subtype & 32 & `soun' (\texttt{0x736f756e}) \\
QuickTime manufacturer & 32 & \texttt{0x00000000} \\
QuickTime component reserved flags & 32 & \texttt{0x00000000} \\
QuickTime component reserved flags mask & 32 & \texttt{0x00000000} \\
component name length & 8 & \texttt{0x00} \\
component name & component name length $\times$ 8 & \\
\hline
\end{tabular}


\subsubsection{the minf Atom}
\begin{tabular}{|l|r|l|}
\hline
Field & Size & Value \\
\hline
atom length & 32 & \ATOM{smhd} size + \ATOM{dinf} size + \ATOM{stbl} size + 8 \\
atom type & 32 & `minf' (\texttt{0x6d696e66}) \\
\hline
\ATOM{smhd} atom & \ATOM{smhd} size & \ATOM{smhd} data \\
\ATOM{dinf} atom & \ATOM{dinf} size & \ATOM{dinf} data \\
\ATOM{stbl} atom & \ATOM{stbl} size & \ATOM{stbl} data \\
\hline
\end{tabular}

\subsubsection{the smhd Atom}
\begin{tabular}{|l|r|l|}
\hline
Field & Size & Value \\
\hline
atom length & 32 & 16 \\
atom type & 32 & `smhd' (\texttt{0x736d6864}) \\
\hline
version & 8 & \texttt{0x00} \\
flags & 24 & \texttt{0x000000} \\
audio balance & 16 & \texttt{0x0000} \\
padding & 16 & \texttt{0x0000} \\
\hline
\end{tabular}

\subsubsection{the dinf Atom}
\begin{tabular}{|l|r|l|}
\hline
Field & Size & Value \\
\hline
atom length & 32 & \ATOM{dref} size + 8 \\
atom type & 32 & `dinf' (\texttt{0x64696e66}) \\
\hline
\ATOM{dref} atom & \ATOM{dref} size & \ATOM{dref} data \\
\hline
\end{tabular}

\clearpage

\subsubsection{the dref Atom}

\begin{table}[h]
\begin{tabular}{|l|r|l|}
\hline
Field & Size & Value \\
\hline
atom length & 32 & 28 \\
atom type & 32 & `dref' (\texttt{0x64726566}) \\
\hline
version & 8 & \texttt{0x00} \\
flags & 24 & \texttt{0x000000} \\
number of references & 32 & \texttt{1} \\
\hline
\hline
reference atom size & 32 & \texttt{12} \\
reference atom type & 32 & `url ' (\texttt{0x75726c20}) \\
reference atom data & 32 & \texttt{0x00000001} \\
\hline
\end{tabular}
\end{table}

\subsubsection{the stbl Atom}

\begin{table}[h]
\begin{tabular}{|l|r|l|}
\hline
Field & Size & Value \\
\hline
atom length & 32 & \ATOM{stsd} size + \ATOM{stts} size + \ATOM{stsc} size + \\
& & \ATOM{stsz} size + \ATOM{stco} size + 8 \\
atom type & 32 & `stbl' (\texttt{0x7374626c}) \\
\hline
\ATOM{stsd} atom & \ATOM{stsd} size & \ATOM{stsd} data \\
\ATOM{stts} atom & \ATOM{stts} size & \ATOM{stts} data \\
\ATOM{stsc} atom & \ATOM{stsc} size & \ATOM{stsc} data \\
\ATOM{stsz} atom & \ATOM{stsz} size & \ATOM{stsz} data \\
\ATOM{stco} atom & \ATOM{stco} size & \ATOM{stco} data \\
\hline
\end{tabular}
\end{table}

\subsubsection{the stsd Atom}

\begin{table}[h]
\begin{tabular}{|l|r|l|}
\hline
Field & Size & Value \\
\hline
atom length & 32 & \ATOM{alac} size + 16 \\
atom type & 32 & `stsd' (\texttt{0x73747364}) \\
\hline
version & 8 & \texttt{0x00} \\
flags & 24 & \texttt{0x000000} \\
number of descriptions & 32 & \texttt{1} \\
\hline
\ATOM{alac} atom & \ATOM{alac} size & \ATOM{alac} data \\
\hline
\end{tabular}
\end{table}

\clearpage

\subsubsection{the alac Atom}

\begin{table}[h]
\begin{tabular}{|l|r|l|}
\hline
Field & Size & Value \\
\hline
atom length & 32 & 72 \\
atom type & 32 & `alac' (\texttt{0x616c6163}) \\
\hline
reserved & 48 & \texttt{0x000000000000} \\
reference index & 16 & \texttt{1} \\
version & 16 & \texttt{0} \\
revision level & 16 & \texttt{0} \\
vendor & 32 & \texttt{0x00000000} \\
channels & 16 & channel count \\
bits per sample & 16 & bits per sample \\
compression ID & 16 & \texttt{0} \\
audio packet size & 16 & \texttt{0} \\
sample rate & 32 & \texttt{0xAC440000} \\
\hline
\hline
atom length & 32 & 36 \\
atom type & 32 & `alac' (\texttt{0x616c6163}) \\
\hline
padding & 32 & \texttt{0x00000000} \\
max samples per frame & 32 & largest number of PCM frames per ALAC frame \\
padding & 8 & \texttt{0x00} \\
sample size & 8 & bits per sample \\
history multiplier & 8 & \texttt{40} \\
initial history & 8 & \texttt{10} \\
maximum K & 8 & \texttt{14} \\
channels & 8 & channel count \\
unknown & 16 & \texttt{0x00FF} \\
max coded frame size & 32 & largest ALAC frame size, in bytes \\
bitrate & 32 & $((\text{\ATOM{mdat} size} \times 8 ) \div (\text{total PCM frames} \div \text{sample rate}))$ \\
sample rate & 32 & sample rate \\
\hline
\end{tabular}
\end{table}
The \VAR{history multiplier}, \VAR{initial history} and \VAR{maximum K}
values are encode-time options, typically set to 40, 10 and 14,
respectively.

Note that the \VAR{bitrate} field can't be known in advance;
we must fill that value with 0 for now and then
return to this atom once encoding is completed
and its size has been determined.

\clearpage

\subsubsection{the stts Atom}

\begin{table}[h]
\begin{tabular}{|l|r|l|}
\hline
Field & Size & Value \\
\hline
atom length & 32 & number of times $\times$ 8 + 16\\
atom type & 32 & `stts' (\texttt{0x73747473}) \\
\hline
version & 8 & \texttt{0x00} \\
flags & 24 & \texttt{0x000000} \\
number of times & 32 & \\
\hline
frame count 1 & 32 & number of occurrences \\
frame duration 1 & 32 & PCM frame count \\
\hline
\multicolumn{3}{|c|}{...} \\
\hline
\end{tabular}
\end{table}
This atom keeps track of how many different sizes of ALAC frames
occur in the ALAC file, in PCM frames.
It will typically have only two ``times'', the block size we're
using for most of our samples and the final block size for
any remaining samples.

For example, let's imagine encoding a 1 minute audio file
at 44100Hz with a block size of 4096 frames.
This file has a total of 2,646,000 PCM frames ($60 \times 44100 = 2646000$).
2,646,000 PCM frames divided by a 4096 block size means
we have 645 ALAC frames of size 4096, and 1 ALAC frame of size 4080.

Therefore:
\begin{align*}
\text{number of times} &= 2 \\
\text{frame count}_1 &= 645 \\
\text{frame duration}_1 &= 4096 \\
\text{frame count}_2 &= 1 \\
\text{frame duration}_2 &= 4080
\end{align*}

\subsubsection{the stsc Atom}

\begin{table}[h]
\begin{tabular}{|l|r|l|}
\hline
Field & Size & Value \\
\hline
atom length & 32 & entries $\times$ 12 + 16 \\
atom type & 32 & `stsc' (\texttt{0x73747363}) \\
\hline
version & 8 & \texttt{0x00} \\
flags & 24 & \texttt{0x000000} \\
number of entries & 32 & \\
\hline
first chunk & 32 & \\
ALAC frames per chunk & 32 & \\
description index & 32 & \texttt{1} \\
\hline
\multicolumn{3}{|c|}{...} \\
\hline
\end{tabular}
\end{table}

This atom stores how many ALAC frames are in a given ``chunk''.
In this instance a ``chunk'' represents an entry in
the \ATOM{stco} atom table, used for seeking backwards and forwards
through the file.
\VAR{First chunk} is the starting offset of its frames-per-chunk
value, beginning at 1.

As an example, let's take a one minute, 44100Hz audio file
that's been broken into 130 chunks
(each with an entry in the \ATOM{stco} atom).
Its \ATOM{stsc} entries would typically be:
\begin{align*}
\text{first chunk}_1 &= 1 \\
\text{frames per chunk}_1 &= 5 \\
\text{first chunk}_2 &= 130 \\
\text{frames per chunk}_2 &= 1
\end{align*}
What this means is that chunks 1 through 129 have 5 ALAC frames each
while chunk 130 has 1 ALAC frame.
This is a total of 646 ALAC frames, which matches the contents of
the \ATOM{stts} atom.

\subsubsection{the stsz Atom}

\begin{tabular}{|l|r|l|}
\hline
Field & Size & Value \\
\hline
atom length & 32 & sizes $\times$ 4 + 20 \\
atom type & 32 & `stsz' (\texttt{0x7374737a}) \\
\hline
version & 8 & \texttt{0x00} \\
flags & 24 & \texttt{0x000000} \\
block byte size & 32 & \texttt{0x00000000} \\
number of sizes & 32 & \\
\hline
frame size & 32 & \\
\hline
\multicolumn{3}{|c|}{...} \\
\hline
\end{tabular}

This atom is a list of ALAC frame sizes, each in bytes.
For example, our 646 frame file would have 646 corresponding
\ATOM{stsz} entries.

\subsubsection{the stco Atom}

\begin{tabular}{|l|r|l|}
\hline
Field & Size & Value \\
\hline
atom length & 32 & offset $\times$ 4 + 16 \\
atom type & 32 & `stco' (\texttt{0x7374636f}) \\
\hline
version & 8 & \texttt{0x00} \\
flags & 24 & \texttt{0x000000} \\
number of offsets & 32 & \\
\hline
frame offset & 32 & \\
\hline
\multicolumn{3}{|c|}{...} \\
\hline
\end{tabular}

This atom is a list of absolute file offsets for each chunk, where
each chunk is typically 5 ALAC frames large.

\clearpage

\subsubsection{the udta Atom}

\begin{tabular}{|l|r|l|}
\hline
Field & Size & Value \\
\hline
atom length & 32 & \ATOM{meta} size + 8 \\
atom type & 32 & `udta' (\texttt{0x75647461}) \\
\hline
\ATOM{meta} atom & \ATOM{meta} size & \ATOM{meta} data \\
\hline
\end{tabular}

\subsubsection{the meta Atom}

\begin{tabular}{|l|r|l|}
\hline
Field & Size & Value \\
\hline
atom length & 32 & \ATOM{hdlr} size + \ATOM{ilst} size + \ATOM{free} size + 12 \\
atom type & 32 & `meta' (\texttt{0x6d657461}) \\
\hline
version & 8 & \texttt{0x00} \\
flags & 24 & \texttt{0x000000} \\
\hline
\ATOM{hdlr} atom & \ATOM{hdlr} size & \ATOM{hdlr} data \\
\ATOM{ilst} atom & \ATOM{ilst} size & \ATOM{ilst} data \\
\ATOM{free} atom & \ATOM{free} size & \ATOM{free} data \\
\hline
\end{tabular}

\subsubsection{the hdlr atom (revisited)}

\begin{tabular}{|l|r|l|}
\hline
Field & Size & Value \\
\hline
atom length & 32 & 34 \\
atom type & 32 & `hdlr' (\texttt{0x68646c72}) \\
\hline
version & 8 & \texttt{0x00} \\
flags & 24 & \texttt{0x000000} \\
QuickTime type & 32 & \texttt{0x00000000} \\
QuickTime subtype & 32 & `mdir' (\texttt{0x6d646972}) \\
QuickTime manufacturer & 32 & `appl' (\texttt{0x6170706c}) \\
QuickTime component reserved flags & 32 & \texttt{0x00000000} \\
QuickTime component reserved flags mask & 32 & \texttt{0x00000000} \\
component name length & 8 & \texttt{0x00} \\
component name & 0 & \\
\hline
\end{tabular}

This atom is laid out identically to the ALAC file's primary
\ATOM{hdlr} atom (described on page \pageref{alac_hdlr}).
The only difference is the contents of its fields.

\subsubsection{the ilst Atom}

This atom is a collection of \ATOM{data} sub-atoms
and is described on page \pageref{m4a_meta}.

\subsubsection{the free Atom}

These atoms are simple collection of NULL bytes which can easily be
resized to make room for other atoms without rewriting the entire file.
