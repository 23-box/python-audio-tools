\chapter{Ogg Vorbis}
\label{vorbis}
Ogg Vorbis is Vorbis audio in an Ogg container.
Ogg containers are a series of Ogg pages, each containing
one or more segments of data.
All of the fields within Ogg Vorbis are little-endian.
\section{Ogg file stream}
\begin{figure}[h]
\includegraphics{figures/ogg_stream.pdf}
\end{figure}
\parpic[r]{
\begin{tabular}{|c|l|}
\hline
bits & Header Type \\
\hline
\texttt{001} & Continuation \\
\texttt{010} & Beginning of Stream \\
\texttt{100} & End of Stream \\
\hline
\end{tabular}
}
`Granule position' is a time marker.
In the case of Ogg Vorbis, it is the sample count.

`Bitstream Serial Number' is an identifier for the given
bitstream which is unique within the Ogg file.
For instance, an Ogg file might contain both video and
audio pages, interleaved.
The Ogg pages for the audio will have a different
serial number from those of the video so that the decoder
knows where to send the data of each.

`Page Sequence Number' is an integer counter which starts from 0
and increments 1 for each Ogg page.
Multiple bitstreams will have separate sequence numbers.

`Checksum' is a 32-bit checksum of the entire Ogg page.

The `Page Segments' value indicates how many segments are in
this Ogg page.
Each segment will have an 8-bit length.
If that length is 255, it indicates the next segment is
part of the current one and should be concatenated with it when
creating packets from the segments.
In this way, packets larger than 255 bytes can be stored in
an Ogg page.
If the final segment in the Ogg page has a length of 255 bytes,
the packet it is a part of continues into the next Ogg page.

\subsection{Ogg packets}
\begin{figure}[h]
\includegraphics{figures/ogg_packets.pdf}
\end{figure}
\par
\noindent
This is an example Ogg stream to illustrate a few key points about
the format.
Note that Ogg pages may have one or more segments,
and packets are composed of one of more segments,
yet the boundaries between packets are segments
that are less than 255 bytes long.
Which segment belongs to which Ogg page is not important
for building packets.

\section{the Identification packet}
The first packet within a Vorbis stream is the Identification packet.
This contains the sample rate and number of channels.
Vorbis does not have a bits-per-sample field, as samples
are stored internally as floating point values and are
converted into a certain number of bits in the decoding process.
To find the total samples, use the `Granule Position' value
in the stream's final Ogg page.
\begin{figure}[h]
\includegraphics{figures/vorbis_identification.pdf}
\end{figure}

\clearpage

\section{the Comment packet}
\label{vorbiscomment}
The second packet within a Vorbis stream is the Comment packet.

\begin{figure}[h]
\includegraphics{figures/vorbis_comment.pdf}
\end{figure}

The length fields are all little-endian.
The Vendor String and Comment Strings are all UTF-8 encoded.
Keys are not case-sensitive and may occur multiple times,
indicating multiple values for the same field.
For instance, a track with multiple artists may have
more than one \texttt{ARTIST}.

\begin{multicols}{2}
{\relsize{-2}
\begin{description}
\item[ALBUM] album name
\item[ARTIST] artist name, band name, composer, author, etc.
\item[CATALOGNUMBER*] CD spine number
\item[COMPOSER*] the work's author
\item[CONDUCTOR*] performing ensemble's leader
\item[COPYRIGHT] copyright attribution
\item[DATE] recording date
\item[DESCRIPTION] a short description
\item[DISCNUMBER*] disc number for multi-volume work
\item[ENGINEER*] the recording masterer
\item[ENSEMBLE*] performing group
\item[GENRE] a short music genre label
\item[GUEST ARTIST*] collaborating artist
\item[ISRC] ISRC number for the track
\item[LICENSE] license information
\item[LOCATION] recording location
\item[OPUS*] number of the work
\item[ORGANIZATION] record label
\item[PART*] track's movement title
\item[PERFORMER] performer name, orchestra, actor, etc.
\item[PRODUCER*] person responsible for the project
\item[PRODUCTNUMBER*] UPC, EAN, or JAN code
\item[PUBLISHER*] album's publisher
\item[RELEASE DATE*] date the album was published
\item[REMIXER*] person who created the remix
\item[SOURCE ARTIST*] artist of the work being performed
\item[SOURCE MEDIUM*] CD, radio, cassette, vinyl LP, etc.
\item[SOURCE WORK*] a soundtrack's original work
\item[SPARS*] DDD, ADD, AAD, etc.
\item[SUBTITLE*] for multiple track names in a single file
\item[TITLE] track name
\item[TRACKNUMBER] track number
\item[VERSION] track version
\end{description}
}
\end{multicols}
\par
\noindent
Fields marked with * are proposed extension fields and not part of the official Vorbis comment specification.
