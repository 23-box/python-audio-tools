%This work is licensed under the
%Creative Commons Attribution-Share Alike 3.0 United States License.
%To view a copy of this license, visit
%http://creativecommons.org/licenses/by-sa/3.0/us/ or send a letter to
%Creative Commons,
%171 Second Street, Suite 300,
%San Francisco, California, 94105, USA.

\chapter{Ogg Vorbis}
\label{vorbis}
Ogg Vorbis is Vorbis audio in an Ogg container.
Ogg containers are a series of Ogg pages, each containing
one or more segments of data.
All of the fields within Ogg Vorbis are little-endian.
\section{Ogg File Stream}
\begin{figure}[h]
\includegraphics{figures/ogg_stream.pdf}
\end{figure}
\parpic[r]{
\begin{tabular}{|c|l|}
\hline
bits & Header Type \\
\hline
\texttt{001} & Continuation \\
\texttt{010} & Beginning of Stream \\
\texttt{100} & End of Stream \\
\hline
\end{tabular}
}
\VAR{Granule position} is a time marker.
In the case of Ogg Vorbis, it is the sample count.

\VAR{Bitstream Serial Number} is an identifier for the given
bitstream which is unique within the Ogg file.
For instance, an Ogg file might contain both video and
audio pages, interleaved.
The Ogg pages for the audio will have a different
serial number from those of the video so that the decoder
knows where to send the data of each.

\VAR{Page Sequence Number} is an integer counter which starts from 0
and increments 1 for each Ogg page.
Multiple bitstreams will have separate sequence numbers.

\VAR{Checksum} is a 32-bit checksum of the entire Ogg page.

The \VAR{Page Segments} value indicates how many segments are in
this Ogg page.
Each segment will have an 8-bit length.
If that length is 255, it indicates the next segment is
part of the current one and should be concatenated with it when
creating packets from the segments.
In this way, packets larger than 255 bytes can be stored in
an Ogg page.
If the final segment in the Ogg page has a length of 255 bytes,
the packet it is a part of continues into the next Ogg page.

\subsection{Ogg Packets}
\begin{figure}[h]
\includegraphics{figures/ogg_packets.pdf}
\end{figure}
\par
\noindent
This is an example Ogg stream to illustrate a few key points about
the format.
Note that Ogg pages may have one or more segments,
and packets are composed of one of more segments,
yet the boundaries between packets are segments
that are less than 255 bytes long.
Which segment belongs to which Ogg page is not important
for building packets.

\section{Vorbis Headers}

The first three packets in a valid Ogg Vorbis file must be
\VAR{Identification}, \VAR{Comment} and \VAR{Setup}, in that order.
Each header packet is prefixed by a 7 byte common header.

\clearpage

\subsection{the Identification Packet}
The first packet within a Vorbis stream is the Identification packet.
This contains the sample rate and number of channels.
Vorbis does not have a bits-per-sample field, as samples
are stored internally as floating point values and are
converted into a certain number of bits in the decoding process.
To find the total samples, use the \VAR{Granule Position} value
in the stream's final Ogg page.
\begin{figure}[h]
\includegraphics{figures/vorbis_identification.pdf}
\end{figure}
\par
\noindent
\VAR{Channels} and \VAR{Sample Rate} must be greater than 0.
The two, 4-bit \VAR{Blocksize} fields are stored as a power of 2.
For example:
\begin{align*}
\text{Blocksize}_0 &= 2 ^ {\text{Field}_0} \\
\text{Blocksize}_1 &= 2 ^ {\text{Field}_1}
\end{align*}
where $\text{Blocksize}_i$ must be 64, 128, 256, 512, 1024, 2048, 4096 or 8192
and
\linebreak
$\text{Blocksize}_0 \leq \text{Blocksize}_1$

\clearpage

\subsection{the Comment Packet}
\label{vorbiscomment}
The second packet within a Vorbis stream is the Comment packet.

\begin{figure}[h]
\includegraphics{figures/vorbis_comment.pdf}
\end{figure}

The \VAR{Vendor String} and \VAR{Comment Strings} are all UTF-8 encoded.
Keys are not case-sensitive and may occur multiple times,
indicating multiple values for the same field.
For instance, a track with multiple artists may have
more than one \texttt{ARTIST}.

\begin{multicols}{2}
{\relsize{-2}
\begin{description}
\item[ALBUM] album name
\item[ARTIST] artist name, band name, composer, author, etc.
\item[CATALOGNUMBER*] CD spine number
\item[COMPOSER*] the work's author
\item[CONDUCTOR*] performing ensemble's leader
\item[COPYRIGHT] copyright attribution
\item[DATE] recording date
\item[DESCRIPTION] a short description
\item[DISCNUMBER*] disc number for multi-volume work
\item[ENGINEER*] the recording masterer
\item[ENSEMBLE*] performing group
\item[GENRE] a short music genre label
\item[GUEST ARTIST*] collaborating artist
\item[ISRC] ISRC number for the track
\item[LICENSE] license information
\item[LOCATION] recording location
\item[OPUS*] number of the work
\item[ORGANIZATION] record label
\item[PART*] track's movement title
\item[PERFORMER] performer name, orchestra, actor, etc.
\item[PRODUCER*] person responsible for the project
\item[PRODUCTNUMBER*] UPC, EAN, or JAN code
\item[PUBLISHER*] album's publisher
\item[RELEASE DATE*] date the album was published
\item[REMIXER*] person who created the remix
\item[SOURCE ARTIST*] artist of the work being performed
\item[SOURCE MEDIUM*] CD, radio, cassette, vinyl LP, etc.
\item[SOURCE WORK*] a soundtrack's original work
\item[SPARS*] DDD, ADD, AAD, etc.
\item[SUBTITLE*] for multiple track names in a single file
\item[TITLE] track name
\item[TRACKNUMBER] track number
\item[VERSION] track version
\end{description}
}
\end{multicols}
\par
\noindent
Fields marked with * are proposed extension fields and not part of the official Vorbis comment specification.

\subsection{the Setup Packet}

The third packet in the Vorbis stream is the Setup packet.

\begin{figure}[h]
\includegraphics{figures/vorbis_setup_packet.pdf}
\end{figure}

It contains six sections of data required for decoding.

\clearpage

\subsubsection{Codebooks}

The \VAR{Codebooks} section of the setup packet stores
the Huffman lookup trees.

\begin{figure}[h]
\includegraphics{figures/vorbis_codebooks.pdf}
\end{figure}
\par
\noindent
This section contains two optional sets of data,
a list of Huffman table entry lengths
and the lookup table values each entry length may resolve to.
\VAR{Total Entries} indicates the total number of entry lengths present.
These lengths may be stored in one of three ways:
unordered without sparse entries, unordered with sparse entries
and ordered.

Unordered without sparse entries is the simplest method;
each entry length is stored as a 5 bit value, plus one.
Unordered with sparse entries is almost as simple;
each 5 bit length is prefixed by a single bit indicating
whether it is present or not.

Ordered entries are more complicated.
The initial length is stored as a 5 bit value, plus one.
The number of entries with that length are stored as a series of
\VAR{Length Count} values in the bitstream, whose sizes
are determined by the number of remaining entries.
\begin{align*}
\text{Length Count}_i \text{ Size} &= \lfloor\log_2 (\text{Remaining Entries}_i)\rfloor + 1
\intertext{For example, given a \VAR{Total Entries} value of 8 and an
\VAR{Initial Length} value of 2:}
\text{Length Count}_0 \text{ Size} &= \lfloor\log_2 8\rfloor + 1 = 4 \text{ bits}
\end{align*}
which means we read a 4 bit value to determine how many
\VAR{Entry Length} values are 2 bits long.
Therefore, if $\text{Length Count}_0$ is read from the bitstream as 2,
our \VAR{Entry Length} values are:
\begin{align*}
\text{Entry Length}_0 &= 2 \\
\text{Entry Length}_1 &= 2
\end{align*}
and the next set of lengths are 3 bits long.
Since we still have remaining entries, we read another \VAR{Length Count}
value of the length:
\begin{equation*}
\text{Length Count}_1 \text{ Size} = \lfloor\log_2 6\rfloor + 1 = 3 \text{ bits}
\end{equation*}
Thus, if $\text{Length Count}_1$ is also a value of 2 from the bitstream,
our \VAR{Entry Length} values are:
\begin{align*}
\text{Entry Length}_2 &= 3 \\
\text{Entry Length}_3 &= 3
\end{align*}
and the next set of lengths are 4 bits long.
We then read one more \VAR{Length Count} value:
\begin{equation*}
\text{Length Count}_2 \text{ Size} = \lfloor\log_2 4\rfloor + 1 = 3 \text{ bits}
\end{equation*}
Finally, if $\text{Length Count}_2$ is 4 from the bitstream,
our \VAR{Entry Length} values are:
\begin{align*}
\text{Entry Length}_4 &= 4 \\
\text{Entry Length}_5 &= 4 \\
\text{Entry Length}_6 &= 4 \\
\text{Entry Length}_7 &= 4
\end{align*}
At this point, we've assigned lengths to all the values
indicated by \VAR{Total Entries}, so the process is complete.

\clearpage

\subsubsection{Transforming Entry Lengths to Huffman Tree}

Once a set of entry length values is parsed from the stream,
transforming them into a Huffman decision tree
is performed by taking our entry lengths from
$\text{Entry Length}_0$ to $\text{Entry Length}_{total - 1}$
and placing them in the tree recursively such that the 0 bit
branches are populated first.
For example, given the parsed entry length values:
\begin{align*}
\text{Entry Length}_0 &= 2 & \text{Entry Length}_1 &= 4 & \text{Entry Length}_2 &= 4 & \text{Entry Length}_3 &= 4 \\
\text{Entry Length}_4 &= 4 & \text{Entry Length}_5 &= 2 & \text{Entry Length}_6 &= 3 & \text{Entry Length}_7 &= 3
\end{align*}
\par
\noindent
We first place $\text{Entry Length}_0$ into the Huffman tree as a 2 bit value.
Since the zero bits are filled first when adding a node 2 bits deep,
it initially looks like:

\begin{figure}[h]
\includegraphics{figures/vorbis_huffman_example1.pdf}
\caption{$\text{Entry Length}_0$ placed with 2 bits}
\end{figure}
\par
\noindent
We then place $\text{Entry Length}_1$ into the Huffman tree as a 4 bit value.
Since the \texttt{0 0} branch is already populated by $\text{Entry Length}_0$,
we switch to the empty \texttt{0 1} branch as follows:
\begin{figure}[h]
\includegraphics{figures/vorbis_huffman_example2.pdf}
\caption{$\text{Entry Length}_1$ placed with 4 bits}
\end{figure}
\par
\noindent
The 4 bit $\text{Entry Length}_2$, $\text{Entry Length}_3$ and $\text{Entry Length}_4$
values are placed similarly along the \texttt{0 1} branch:
\begin{figure}[h]
\includegraphics{figures/vorbis_huffman_example3.pdf}
\caption{$\text{Entry Length}_2$, $\text{Entry Length}_3$ and $\text{Entry Length}_4$ placed with 4 bits}
\end{figure}
\par
\noindent
Finally, the remaining three entries are populated along the \texttt{1} branch:
\begin{figure}[h]
\includegraphics{figures/vorbis_huffman_example4.pdf}
\caption{$\text{Entry Length}_5$, $\text{Entry Length}_6$ and $\text{Entry Length}_7$ are placed}
\end{figure}

\subsubsection{The Lookup Table}

The lookup table is only present if \VAR{Lookup Type} is 1 or 2.
A \VAR{Lookup Type} of 0 indicates no lookup table, while anything
greater than 2 is an error.

\VAR{Minimum Value} and \VAR{Delta Value} are 32-bit floating point values
which can be parsed in the following way:
\begin{figure}[h]
\includegraphics{figures/vorbis_float32.pdf}
\end{figure}
\begin{equation*}
\text{Float} =
\begin{cases}
\text{Mantissa} \times 2 ^ {\text{Exponent} - 788} & \text{ if Sign = 0} \\
-\text{Mantissa} \times 2 ^ {\text{Exponent} - 788} & \text{ if Sign = 1}
\end{cases}
\end{equation*}
\par
\VAR{Value Bits} indicates the size of each value in bits, plus one.
\VAR{Sequence P} is a 1 bit flag.
The total number of multiplicand values depends on whether \VAR{Lookup Type}
is 1 or 2.
\begin{align*}
\intertext{if \VAR{Lookup Type} = 1:}
\text{Multiplicand Count} &= \text{max}(\text{Int}) ^ \text{Dimensions} \text{ where } \text{Int} ^ \text{Dimensions} \leq \text{Total Entries}
\intertext{if \VAR{Lookup Type} = 2:}
\text{Multiplicand Count} &= \text{Dimensions} \times \text{Total Entries}
\end{align*}
\par
The \VAR{Multiplicand} values themselves are a list of unsigned integers,
each \VAR{Value Bits} (+ 1) bits large.


\clearpage

\section{Channel Assignment}
\begin{table}[h]
{\relsize{-1}
\begin{tabular}{|c|r|r|r|r|r|r|r|r|}
\hline
channel & & & & & & & & \\
count & channel 1 & channel 2 & channel 3 & channel 4 & channel 5 & channel 6 & channel 7 & channel 8 \\
\hline
\multirow{2}{1em}{1} & front & & & & & & & \\
                     & center & & & & & & & \\
\hline
\multirow{2}{1em}{2} & front & front & & & & & & \\
                     & left  & right & & & & & & \\
\hline
\multirow{2}{1em}{3} & front & front & front & & & & & \\
                     & left  & center & right & & & & & \\
\hline
\multirow{2}{1em}{4} & front & front & back & back & & & & \\
                     & left  & right & left & right & & & & \\
\hline
\multirow{2}{1em}{5} & front & front  & front & back & back & & & \\
                     & left  & center & right & left & right & & & \\
\hline
\multirow{2}{1em}{6} & front & front  & front & back & back & & & \\
                     & left  & center & right & left & right & LFE & & \\
\hline
\multirow{2}{1em}{7} & front & front  & front & side & side  & back   & & \\
                     & left  & center & right & left & right & center & LFE& \\
\hline
\multirow{2}{1em}{8} & front & front  & front & side & side  & back   & back & \\
                     & left  & center & right & left & right & left & right & LFE \\
\hline
8+ & \multicolumn{8}{c|}{defined by application} \\
\hline
\end{tabular}
}
\end{table}
