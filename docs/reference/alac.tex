%This work is licensed under the
%Creative Commons Attribution-Share Alike 3.0 United States License.
%To view a copy of this license, visit
%http://creativecommons.org/licenses/by-sa/3.0/us/ or send a letter to
%Creative Commons,
%171 Second Street, Suite 300,
%San Francisco, California, 94105, USA.

\chapter{Apple Lossless}

Apple's lossless audio codec, informally referred to as ``ALAC'',
is lossless audio inside a QuickTime container - similar to M4A.
Its stream is the same collection of atoms as covered on page
\pageref{m4a}.
The key difference is the contents of its \texttt{mdat} atom.

\section{the ALAC file stream}
\begin{wrapfigure}[6]{r}{1.5in}
\includegraphics{figures/alac_atoms.pdf}
\end{wrapfigure}

This is the typical arrangements of ALAC atoms as encoded by iTunes.
As you can see, it is almost identical to the layout of AAC audio.
One of the key differences is that ALAC's \texttt{stsd} atom
contains an \texttt{alac} description sub-atom rather than an
\texttt{mp4a} description atom.

Its layout is as follows:

\begin{figure}[h]
\includegraphics{figures/alac_alac_atom.pdf}
\end{figure}

\pagebreak

\section{ALAC decoding}

An ALAC stream is made up of individual frames within the \texttt{mdat}
atom, as follows:

\begin{figure}[h]
\includegraphics{figures/alac_stream.pdf}
\end{figure}
\par
\noindent
ALAC frames come in two varieties: compressed and uncompressed,
depending on the \VAR{Is Not Compressed} bit in the frame header.
An uncompressed frame is laid out as follows:
\label{alac_uncompressed}
\begin{figure}[h]
\includegraphics{figures/alac_uncompressed_frame.pdf}
\end{figure}
\par
\noindent
The number of PCM frames in an ALAC frame depends on the \VAR{Has Sample Size}
bit.
If set, the number of PCM frames equals the 32-bit \VAR{Sample Size}
value.
If not set, the total number of PCM frames equals the \VAR{Max Coded Frame Size}
value in the \texttt{alac} atom and the \VAR{Sample Size} value is omitted.
ALAC streams typically use the same number of samples per frame
until the end of the stream, at which point the leftover samples
are placed in a different-sized frame.

Uncompressed frames interleave samples between channels during decoding.
For example, a 2 channel frame places $\text{Sample}_1$ on $\text{Channel}_1$,
$\text{Sample}_2$ on $\text{Channel}_2$,
$\text{Sample}_3$ on $\text{Channel}_1$,
$\text{Sample}_4$ on $\text{Channel}_2$ and so on.

Finally, note that all ALAC frames have a footer of the bits `\texttt{111}'
and padding as needed such that each frame begins on an aligned byte boundary.

\pagebreak

A compressed frame is laid out as follows:

\begin{figure}[h]
\includegraphics{figures/alac_compressed_frame.pdf}
\end{figure}
\par
\noindent
\VAR{Interlacing Shift} and \VAR{Interlacing Leftweight} are used
for channel decorrelation after the subframes have been decoded.

There is one subframe header and one residual block per
channel.
If \VAR{Wasted Bits} is greater than 0, there is also
a block of wasted bits samples after the subframe headers but before
the residuals.

\label{alac_wasted_bits}
\VAR{Wasted Bits} is an attempt to store the least significant bits
of each sample more efficiently at high bits-per-sample, where
that data will often be indistinguishable from random noise.
In effect, it's a block of interlaced uncompressed samples
(similar to an uncompressed ALAC frame) each $8 \times \text{\VAR{Wasted Bits}}$
bits large.
Those wasted bits are then prepended to each sample after channel
decorrelation.
I'll explain this process in more detail later.

Each subframe header is laid out as follows:

\begin{figure}[h]
\includegraphics{figures/alac_subframe_header.pdf}
\end{figure}
\par
\noindent
There are \VAR{Coefficient Count} number of coefficients in each
subframe header, each a 16-bit signed value.

\pagebreak

\subsection{Residual decoding}

There are \VAR{Sample Size} number of residuals per residual block.
Decoding a residual block requires knowing the \VAR{Initial History},
\VAR{History Multiplier}, \VAR{Maximum K}, \VAR{Channels}, \VAR{Wasted Bits} and
\VAR{Bits per Sample} values, from the \texttt{alac} atom and frame header.
Fortunately, most of these values are rarely used;
we'll mostly be concerned with the \VAR{History} and \VAR{History Multiplier}.

For each residual block, \VAR{History} starts with the value of
\VAR{Initial History} and will change during residual decoding.
We use it to calculate $\kappa$ using the following formula:
\begin{equation}
\kappa = \left\lfloor\log_2 \left( \frac{\text{history}}{2 ^ 9} + 3 \right) \right\rfloor
\end{equation}
Note that if $\kappa$ exceeds the \VAR{Maximum K} value from the
\texttt{alac} atom, \VAR{Maximum K} is used instead.

We then need to know the \VAR{Bits per Sample} value of the residual
block, which is equal to:
\begin{equation}
\text{Bits per Sample}_{Residual} = \text{Bits per Sample}_{ALAC} - (\text{Wasted Bits} \times 8) + \text{Channels} - 1
\end{equation}
For mono streams, this is typically equal to the stream's
\VAR{Bits per Sample},
while for stereo streams it's often the \VAR{Bits per Sample} plus 1.

\begin{wrapfigure}[14]{r}{2.5in}
\includegraphics{figures/alac_read_residual.pdf}
\end{wrapfigure}
The $\kappa$ and \VAR{Bits per Sample} values are used to read a
single unsigned residual value in the following way:

The initial bit reading portion of the process
involves reading a unary value with a stop bit of `0'
and a maximum value of 8.
If the maximum value is exceeded, we read \VAR{Bits per Sample}
number of bits as our final value
(the only place \VAR{Bits per Sample} is used throughout the
residual decoding process).

Otherwise, we read $\kappa$ number of \VAR{extra} bits if $\kappa > 1$.
If \VAR{extra} is greater than 1, we return
$(\text{\VAR{unary}} \times (2^k - 1)) + (\text{\VAR{extra}} - 1)$.
If not, we push a single \VAR{extra} bit back on the stream and return our
$\text{\VAR{unary}} \times (2^k - 1)$ value.

\clearpage

We perform the following to convert unsigned residuals to signed residuals
which are returned to the subframe decoder:
\begin{equation}
\text{signed} =
\begin{cases}
\lfloor (\text{unsigned} + 1) \div 2 \rfloor & \text{if unsigned is even} \\
-\lfloor (\text{unsigned} + 1) \div 2 \rfloor & \text{if unsigned is odd}
\end{cases}
\end{equation}

Finally, we use our unsigned value to update \VAR{history}
before reading the next residual:
{\relsize{-1}
\begin{equation*}
\text{history} =
\begin{cases}
\text{history} + (\text{unsigned} \times \text{history multiplier}) - \left\lfloor\frac{\text{history} \times \text{history multiplier}}{2^9}\right\rfloor & \text{if unsigned} \leq 65535 \\
65535 & \text{if unsigned} > 65535
\end{cases}
\end{equation*}
}

Thus far, residual decoding isn't overly complex.
We simply calculate $\kappa$ from \VAR{history}, read a residual,
update \VAR{history} from its unsigned value and repeat the process
until we've read an entire subframe's worth of residuals.
The bulk of an ALAC file's residuals will be read in this way.
However, ALAC also features an ``escape code'' for large chunks
of 0 residuals (which may happen during a long stretch of
digital silence, for example).

If \VAR{history} ever falls below 128, this special case is triggered.
First, we read a special \VAR{block size} residual value
with a \VAR{Bits per Sample} value of 16 and a $\kappa$\footnote{Again, $\kappa$ cannot exceed \VAR{Maximum K} as encoded in the \texttt{alac} atom.} value of:
\begin{equation}
\kappa_{\text{blocksize}} = 7 - \log_2(\text{history}) + \frac{\text{history} + 16}{64}
\end{equation}
This \VAR{block size} is how many 0 residuals to send outright -
which may be 0, indicating no 0 residuals to send.
Either way, if $\text{\VAR{block size}} \leq 65535$,
we add 1 to the unsigned value of the next residual in the block (if any).
Finally, \VAR{history} is automatically set to 0.

\clearpage

\subsubsection{Residual decoding example}

In this example, we'll decode a group of residuals in which our
\VAR{Initial History} is 1130 and our \VAR{History Multiplier} is 40.
As a spoiler, $\kappa$ (the amount of non-unary bits to read
after each unary value) will remain 2 for this batch of residuals,
but why that is so will be explained below.

\begin{figure}[h]
\includegraphics{figures/alac_residual.pdf}
\end{figure}
\begin{itemize}
\setlength{\itemsep}{0in}
\setlength{\parskip}{0in}
\item Residual 1
\begin{itemize}
\item $\kappa = \lfloor\log_2((1130 \div 2^9) + 3)\rfloor = \lfloor\log_2(5)\rfloor = 2$
\item $\text{unsigned}_1 = (\textbf{0} \times (2^2 - 1)) + (\textbf{3} - 1) = \textbf{2}$
\item $\text{residual}_1 = \lfloor (2 + 1) \div 2 \rfloor = 1$
\item $\text{history} = 1130 + (\textbf{2} \times 40) - \lfloor(1130 \times 40) \div 2^9\rfloor = 1130 + 80 - 88 = 1122$
\end{itemize}
\item Residual 2
\begin{itemize}
\item $\kappa = \lfloor\log_2((1122 \div 2^9) + 3)\rfloor = \lfloor\log_2(5)\rfloor = 2$
\item $\text{unsigned}_2 = (\textbf{2} \times (2^2 - 1)) + (\textbf{3} - 1) = \textbf{8}$
\item $\text{residual}_2 = \lfloor (8 + 1) \div 2 \rfloor = 4$
\item $\text{history} = 1122 + (\textbf{8} \times 40) - \lfloor(1122 \times 40) \div 2^9\rfloor = 1122 + 320 - 87 = 1355$
\end{itemize}
\item Residual 3
\begin{itemize}
\item $\kappa = \lfloor\log_2((1355 \div 2^9) + 3)\rfloor = \lfloor\log_2(5)\rfloor = 2$
\item $\text{unsigned}_3 = (\textbf{1} \times (2^2 - 1)) = \textbf{3}$\hfill (note that we ``unread'' one 0 bit here)
\item $\text{residual}_3 = -\lfloor (3 + 1) \div 2 \rfloor = -2$
\item $\text{history} = 1355 + (\textbf{3} \times 40) - \lfloor(1355 \times 40) \div 2^9\rfloor = 1355 + 120 - 105 = 1370$
\end{itemize}
\item Residual 4
\begin{itemize}
\item $\kappa = \lfloor\log_2((1370 \div 2^9) + 3)\rfloor = \lfloor\log_2(5)\rfloor = 2$
\item $\text{unsigned}_4 = (\textbf{0} \times (2^2 - 1)) + (\textbf{3} - 1) = \textbf{2}$
\item $\text{residual}_4 = \lfloor (2 + 1) \div 2 \rfloor = 1$
\item $\text{history} = 1370 + (\textbf{2} \times 40) - \lfloor(1370 \times 40) \div 2^9\rfloor = 1370 + 80 - 107 = 1343$
\end{itemize}
\end{itemize}
Thus, our batch of signed residual values are 1, 4, -2 and 1.

\clearpage

\subsection{Subframe calculation}

% example:
% decode residuals [16, 1, 9, -1, -1, 1, -4, 7, -7, -1]
% with starting coeffs [1122, -766, 107, 122]
% and prediction_quantitization 9
% to samples [16, 17, 26, 25, 24, 23, 20, 24, 23, 20]

Given our list of decoded residual values;
along with a list of coefficients,
a \VAR{Coefficient Count} and a \VAR{Prediction Quantitization} value
(all from the subframe header),
we can now generate a list of signed subframe samples for a
given channel.

The first residual is always the first output sample:
\begin{align*}
\text{Sample}_0 &= \text{Residual}_0 \\
\intertext{Then, for the next \VAR{Coefficient Count} number of residuals:}
\text{Sample}_i &= \text{Residual}_i + \text{Sample}_{i - 1}
\end{align*}

For example, given that we have a \VAR{Coefficient Count} of 4
and our first five residuals are 16, 1, 9, -1 and -1;
our first five sample values are:
\begin{align*}
\text{Sample}_0 &= \textbf{16} \\
\text{Sample}_1 = 16 + 1 &= \textbf{17} \\
\text{Sample}_2 = 17 + 9 &= \textbf{26} \\
\text{Sample}_3 = 26 - 1 &= \textbf{25} \\
\text{Sample}_4 = 25 - 1 &= \textbf{24}
\end{align*}
These are our ``starting point'' samples upon which the remainder
of the subframe will be built.

Subsequent samples are calculated in the following way:

\begin{align*}
\text{LPC Sum}_i &= \overset{coeffs - 1}{\underset{j = 0}{\sum}}
\text{Coefficient}_j \times (\text{Sample}_{i - j - 1} - \text{Sample}_{i - coeffs - 1}) \\
\text{Sample}_i &= \left\lfloor \frac{\text{LPC Sum}_i + 2^{\text{Predictor Quantitization - 1}}} {2 ^ \text{Predictor Quantitization}}\right\rfloor + Residual_i + \text{Sample}_{i - coeffs - 1} \\
\intertext{For example, given $\text{Residual}_5$ = 1,
\VAR{Predictor Quantitization} = 9
and the coefficients 1122, -766, 107 and 122:}
\text{LPC Sum}_5 &=
(1122 \times (24 - 16)) + (-766 \times (25 - 16)) +
 (107 \times (26 - 16)) + (122 \times (17 - 16)) \\
&= (1122 \times 8) + (-766 \times 9) + (107 \times 10) + (122 \times 1) \\
&= 8976 + -6894 + 1070 + 122 = \textbf{3274} \\
\text{Sample}_5 &= \left\lfloor\frac{\textbf{3274} + 2^8}{2^9}\right\rfloor + 1 + 16
= \left\lfloor\frac{3530}{512}\right\rfloor + 1 + 16 = \textbf{23}
\end{align*}
But before calculating $\text{Sample}_6$,
we need to adjust our coefficient list.

\clearpage
\label{alac_coefficient_adjustment}
Updating the coefficient list first requires the \VAR{sign} function:
{\relsize{-2}
\begin{equation}
\text{sign}(x) =
\begin{cases}
1 & \text{if } x > 0 \\
0 & \text{if } x = 0 \\
-1 & \text{if } x < 0
\end{cases}
\end{equation}
}
We then take our \VAR{samples}, index $i$, \VAR{coefficients},
coefficient \VAR{count}, \VAR{residual}, \VAR{predictor quantitization}
values and go through the following routine if $\text{residual} \neq 0$:
\par
{\relsize{-1}
\begin{tabular}{|l>{$}r<{$}c>{$}l<{$}|}
\hline
Step 1. & \text{original sign} &=& \text{sign}(residual) \\
Step 2. & j &=& 0 \\
Step 3. & v &=& \text{Sample}_{i - \text{count} - 1} - \text{Sample}_{i - \text{count} + j} \\
Step 4. & sign &=& \begin{cases}\text{sign}(v) & \text{if original sign} > 0 \\
-\text{sign}(v) & \text{if original sign} \leq 0\end{cases} \\
Step 5. & \text{Coefficient}_{\text{count} - j - 1} &=& \text{Coefficient}_{\text{count} - j - 1} - sign \\
Step 6. & residual &=& residual - \left( \left\lfloor\frac{v \times sign}{2 ^ \text{Predictor Quantitization}}\right\rfloor \times (j + 1) \right) \\
Step 7. & j &=& j + 1 \\
Step 8. & \multicolumn{3}{l|}{if ($j < count$) and ($\text{sign}(residual) = \text{original sign}$), goto step 3} \\
\hline
\end{tabular}
}
\par
\noindent
To continue our example:
{\relsize{-2}
\begin{align*}
\text{original sign} &= \text{sign}(1) = \textbf{1} \\
j &= 0 \\
v &= \text{Sample}_0 - \text{Sample}_{1} = 16 - 17 = \textbf{-1} \\
sign &= \text{sign}(-1) = \textbf{-1} \\
\text{Coefficient}_3 &= \text{Coefficient}_3 - -1 = 122 + 1 = \textbf{123} \\
residual &= residual - \left( \left\lfloor\frac{-1 \times -1}{2 ^ 9}\right\rfloor \times (0 + 1)\right) = 1 - (0 \times 1) = \textbf{1} \\
j &= 1 \\
v &= \text{Sample}_0 - \text{Sample}_{2} = 16 - 26 = \textbf{-10} \\
sign &= \text{sign}(-10) = \textbf{-1} \\
\text{Coefficient}_2 &= \text{Coefficient}_2 - -1 = 107 + 1 = \textbf{108} \\
residual &= residual - \left( \left\lfloor\frac{-10 \times -1}{2 ^ 9}\right\rfloor \times (1 + 1)\right) = 1 - (0 \times 2) = \textbf{1} \\
j &= 2 \\
v &= \text{Sample}_0 - \text{Sample}_{3} = 16 - 25 = \textbf{-9} \\
sign &= \text{sign}(-9) = \textbf{-1} \\
\text{Coefficient}_1 &= \text{Coefficient}_1 - -1 = -766 + 1 = \textbf{-765} \\
residual &= residual - \left( \left\lfloor\frac{-9 \times -1}{2 ^ 9}\right\rfloor \times (2 + 1)\right) = 1 - (0 \times 3) = \textbf{1} \\
j &= 3 \\
v &= \text{Sample}_0 - \text{Sample}_{4} = 16 - 25 = \textbf{-8} \\
sign &= \text{sign}(-8) = \textbf{-1} \\
\text{Coefficient}_0 &= \text{Coefficient}_0 - -1 = 1122 + 1 = \textbf{1123} \\
residual &= residual - \left( \left\lfloor\frac{-8 \times -1}{2 ^ 9}\right\rfloor \times (3 + 1)\right) = 1 - (0 \times 4) = \textbf{1} \\
j &= 4\text{ and stop, since 4 equals our coefficient count} \\
\end{align*}
}

\clearpage

Given that $\text{Residual}_6$ is -4, the calculation of $\text{Sample}_6$
is as follows:
\begin{align*}
\text{LPC Sum}_6 &= (1123 \times (23 - 17)) + (-765 \times (24 - 17)) + (108 \times (25 - 17)) + (123 \times (26 - 17)) \\
&= (1123 \times 6) + (-765 \times 7) + (108 \times 8) + (123 \times 9) \\
&= 6738 + -5355 + 864 + 1107 = \textbf{3354} \\
\text{Sample}_6 &= \left\lfloor\frac{\textbf{3354} + 2^8}{2^9}\right\rfloor + -4 + 17 = \left\lfloor\frac{3610}{512}\right\rfloor + -4 + 17 = \textbf{20} \\
%% \end{align*}
%% \begin{align*}
\text{original sign} &= \text{sign}(-4) = -1 \\
j &= 0 \\
v &= \text{Sample}_1 - \text{Sample}_2 = 17 - 26 = \textbf{-9} \\
sign &= -\text{sign}(-9) = \textbf{1} \\
\text{Coefficient}_3 &= \text{Coefficient}_3 - 1 = 123 - 1 = \textbf{122} \\
residual &= residual - \left( \left\lfloor\frac{-9 \times 1}{2 ^ 9}\right\rfloor \times (0 + 1)\right) = -4 - (-1 \times 1) = \textbf{-3} \\
j &= 1 \\
v &= \text{Sample}_1 - \text{Sample}_3 = 17 - 25 = \textbf{-8} \\
sign &= -\text{sign}(-8) = \textbf{1} \\
\text{Coefficient}_2 &= \text{Coefficient}_2 - 1 = 108 - 1 = \textbf{107} \\
residual &= residual - \left( \left\lfloor\frac{-8 \times 1}{2 ^ 9}\right\rfloor \times (1 + 1)\right) = -3 - (-1 \times 2) = \textbf{-1} \\
j &= 2 \\
v &= \text{Sample}_1 - \text{Sample}_4 = 17 - 24 = \textbf{-7} \\
sign &= -\text{sign}(-7) = \textbf{1} \\
\text{Coefficient}_1 &= \text{Coefficient}_1 - 1 = -765 - 1 = \textbf{-766} \\
residual &= residual - \left( \left\lfloor\frac{-7 \times 1}{2 ^ 9}\right\rfloor \times (2 + 1)\right) = -1 - (-1 \times 3) = \textbf{2} \\
&\text{and stop, since sign}(2) \neq \text{sign}(-4) \text{ (our \VAR{original sign} value)}
\end{align*}

So, the coefficients for $\text{Sample}_7$ are 1123, -766, 107 and 122.

\clearpage

\subsection{Channel decorrelation}
\label{alac_decorrelation}
If we have more than one channel of output, the next step is
performing channel decorrelation.
If our \VAR{Interlacing Leftweight} value from the frame header is 0,
our channels are stored independently.
In that case, the $\text{Channel}_1$ is our left samples and
$\text{Channel}_2$ is our right samples.

If \VAR{Interlacing Leftweight} is greater than zero,
we calculate samples as follows:
\begin{align*}
\text{Right}_i &= \text{Channel}_1 - \left\lfloor\frac{\text{Channel}_2 \times \text{Interlacing Leftweight}}{2 ^ \text{Interlacing Shift}}\right\rfloor \\
\text{Left}_i &= \text{Channel}_2 + \text{Right}_i
\end{align*}

For example, given the $\text{Channel}_1$ samples of 14, 15, 19, 17, 18;
the $\text{Channel}_2$ samples of 16, 17, 26, 25, 24,
an \VAR{Interlacing Shift} value of 2 and an \VAR{Interlacing Leftweight}
values of 3, we calculate output samples as follows:
\begin{table}[h]
\begin{tabular}{|c||>{$}r<{$}|>{$}r<{$}||>{$}r<{$}|>{$}r<{$}|}
\hline
Sample & \text{Channel}_1 & \text{Channel}_2 & \text{Right}_i & \text{Left}_i \\
\hline
0 & 14 & 16 & 14 - \lfloor(16 \times 3) \div 2^2\rfloor = \textbf{2} & 16 + \textbf{2} = \textbf{18} \\
1 & 15 & 17 & 15 - \lfloor(17 \times 3) \div 2^2\rfloor = \textbf{3} & 17 + \textbf{3} = \textbf{20} \\
2 & 19 & 26 & 19 - \lfloor(26 \times 3) \div 2^2\rfloor = \textbf{0} & 26 + \textbf{0} = \textbf{26} \\
3 & 17 & 25 & 17 - \lfloor(25 \times 3) \div 2^2\rfloor = \textbf{-1} & 25 + \textbf{-1} = \textbf{24} \\
4 & 18 & 24 & 18 - \lfloor(24 \times 3) \div 2^2\rfloor = \textbf{0} & 24 + \textbf{0} = \textbf{24} \\
\hline
\end{tabular}
\end{table}

\subsection{Wasted bits}

As explained on page \pageref{alac_wasted_bits}, a compressed ALAC frame
with \VAR{Wasted Bits} stores them interleaved between channels.
Then, after channel decorrelation, these verbatim values are
prepended to each sample.
For example, given a 2 channel stream with 24 bits-per-sample and
a \VAR{Wasted Bits} value of 1 (meaning our ``wasted'' samples
are 8 bits large), our final output is as follows:
\begin{figure}[h]
\includegraphics{figures/alac_wasted.pdf}
\end{figure}

\clearpage

\section{ALAC encoding}

To encode an ALAC file, we need a stream of PCM sample integers
along with that stream's sample rate, bits-per-sample and number of
channels.
We'll start by encoding all of the non-audio ALAC atoms,
most of which are contained within the \ATOM{moov} atom.
There's over twenty atoms in a typical ALAC file,
most of which are packed with seemingly redundant or
nonessential data,
so it will take awhile before we can move on to the actual
audio encoding process.

Remember, all of an ALAC's fields are big-endian.

\subsection{ALAC atoms}

We'll encode our ALAC file in iTunes order, which means
it contains the \ATOM{ftyp}, \ATOM{moov}, \ATOM{free} and
\ATOM{mdat} atoms, in that order.

\subsubsection{the ftyp atom}

\begin{table}[h]
\begin{tabular}{|l|r|l|}
\hline
Field & Size & Value \\
\hline
atom length & 32 & 32 \\
atom type & 32 & `ftyp' (\texttt{0x66747970}) \\
\hline
major brand & 32 & `M4A ' (\texttt{0x4d344120}) \\
major brand version & 32 & \texttt{0} \\
compatible brand & 32 & `M4A ' (\texttt{0x4d344120}) \\
compatible brand & 32 & `mp42' (\texttt{0x6d703432}) \\
compatible brand & 32 & `isom' (\texttt{0x69736f6d}) \\
compatible brand & 32 & \texttt{0x00000000} \\
\hline
\end{tabular}
\end{table}

\subsubsection{the moov atom}

\begin{table}[h]
\begin{tabular}{|l|r|l|}
\hline
Field & Size & Value \\
\hline
atom length & 32 & \ATOM{mvhd} size + \ATOM{trak} size + \ATOM{udta} size + 8 \\
atom type & 32 & `moov' (\texttt{0x6d6f6f76}) \\
\hline
\ATOM{mvhd} atom & \ATOM{mvhd} size & \ATOM{mvhd} data \\
\ATOM{trak} atom & \ATOM{trak} size & \ATOM{trak} data \\
\ATOM{udta} atom & \ATOM{udta} size & \ATOM{udta} data \\
\hline
\end{tabular}
\end{table}

\clearpage

\subsubsection{the mvhd atom}

\begin{table}[h]
\begin{tabular}{|l|r|l|}
\hline
Field & Size & Value \\
\hline
atom length & 32 & 108/120 \\
atom type & 32 & `mvhd' (\texttt{0x6d766864}) \\
\hline
version & 8 & \texttt{0x00} \\
flags & 24 & \texttt{0x000000} \\
created date & 32/64 & creation date as Mac UTC \\
modified date & 32/64 & modification date as Mac UTC \\
time scale & 32 & sample rate \\
duration & 32/64 & total PCM frames \\
playback speed & 32 & \texttt{0x10000} \\
user volume & 16 & \texttt{0x100} \\
padding & 80 & \texttt{0x00000000000000000000} \\
window geometry matrix a & 32 & \texttt{0x10000} \\
window geometry matrix b & 32 & \texttt{0} \\
window geometry matrix u & 32 & \texttt{0} \\
window geometry matrix c & 32 & \texttt{0} \\
window geometry matrix d & 32 & \texttt{0x10000} \\
window geometry matrix v & 32 & \texttt{0} \\
window geometry matrix x & 32 & \texttt{0} \\
window geometry matrix y & 32 & \texttt{0} \\
window geometry matrix w & 32 & \texttt{0x40000000} \\
QuickTime preview & 64 & \texttt{0} \\
QuickTime still poster & 32 & \texttt{0} \\
QuickTime selection time & 64 & \texttt{0} \\
QuickTime current time & 32 & \texttt{0} \\
next track ID & 32 & \texttt{2} \\
\hline
\end{tabular}
\end{table}

If \VAR{version} is 0, \VAR{created date}, \VAR{modified date} and
\VAR{duration} are 32 bit fields.
Otherwise, they are 64 bit fields.
The \VAR{created date} and \VAR{modified date} are seconds
since the Macintosh Epoch, which is 00:00:00, January 1st, 1904.\footnote{Why 1904?  It's the first leap year of the 20th century.}
To convert a Unix Epoch timestamp (seconds since January 1st, 1970) to
a Macintosh Epoch, one needs to add 24,107 days -
or \texttt{2082844800} seconds.

\clearpage

\subsubsection{the trak atom}
\begin{tabular}{|l|r|l|}
\hline
Field & Size & Value \\
\hline
atom length & 32 & \ATOM{tkhd} size + \ATOM{mdia} size + 8 \\
atom type & 32 & `trak' (\texttt{0x7472616b}) \\
\hline
\ATOM{tkhd} atom & \ATOM{tkhd} size & \ATOM{tkhd} data \\
\ATOM{mdia} atom & \ATOM{mdia} size & \ATOM{mdia} data \\
\hline
\end{tabular}

\subsubsection{the tkhd atom}

\begin{table}[h]
\begin{tabular}{|l|r|l|}
\hline
Field & Size & Value \\
\hline
atom length & 32 & 92/104 \\
atom type & 32 & `tkhd' (\texttt{0x746b6864}) \\
\hline
version & 8 & \texttt{0x00} \\
padding & 20 & \texttt{0x000000} \\
track in poster & 1 & \texttt{0} \\
track in preview & 1 & \texttt{1} \\
track in movie & 1 & \texttt{1} \\
track enabled & 1 & \texttt{1} \\
created date & 32/64 & creation date as Mac UTC \\
modified date & 32/64 & modification date as Mac UTC \\
track ID & 32 & \texttt{1} \\
padding & 32 & \texttt{0x00000000} \\
duration & 32/64 & total PCM frames \\
padding & 64 & \texttt{0x0000000000000000} \\
video layer & 16 & \texttt{0} \\
QuickTime alternate & 16 & \texttt{0} \\
volume & 16 & \texttt{0x1000} \\
padding & 16 & \texttt{0x0000} \\
video geometry matrix a & 32 & \texttt{0x10000} \\
video geometry matrix b & 32 & \texttt{0} \\
video geometry matrix u & 32 & \texttt{0} \\
video geometry matrix c & 32 & \texttt{0} \\
video geometry matrix d & 32 & \texttt{0x10000} \\
video geometry matrix v & 32 & \texttt{0} \\
video geometry matrix x & 32 & \texttt{0} \\
video geometry matrix y & 32 & \texttt{0} \\
video geometry matrix w & 32 & \texttt{0x40000000} \\
video width & 32 & \texttt{0} \\
video height & 32 & \texttt{0} \\
\hline
\end{tabular}
\end{table}

\clearpage

\subsubsection{the mdia atom}

\begin{table}[h]
\begin{tabular}{|l|r|l|}
\hline
Field & Size & Value \\
\hline
atom length & 32 & \ATOM{mdhd} size + \ATOM{hdlr} size + \ATOM{minf} size + 8 \\
atom type & 32 & `mdia' (\texttt{0x6d646961}) \\
\hline
\ATOM{mdhd} atom & \ATOM{mdhd} size & \ATOM{mdhd} data \\
\ATOM{hdlr} atom & \ATOM{hdlr} size & \ATOM{hdlr} data \\
\ATOM{minf} atom & \ATOM{minf} size & \ATOM{minf} data \\
\hline
\end{tabular}
\end{table}

\subsubsection{the mdhd atom}

\begin{table}[h]
\begin{tabular}{|l|r|l|}
\hline
Field & Size & Value \\
\hline
atom length & 32 & 32/44 \\
atom type & 32 & `mdhd' (\texttt{0x6d646864}) \\
\hline
version & 8 & \texttt{0x00} \\
flags & 24 & \texttt{0x000000} \\
created date & 32/64 & creation date as Mac UTC \\
modified date & 32/64 & modification date as Mac UTC \\
time scale & 32 & sample rate \\
duration & 32/64 & total PCM frames \\
padding & 1 & \texttt{0} \\
language & 5 & \\
language & 5 & language value as ISO 639-2 \\
language & 5 & \\
QuickTime quality & 16 & \texttt{0} \\
\hline
\end{tabular}
\end{table}
Note the three, 5-bit \VAR{language} fields.
By adding 0x60 to each value and converting the result to ASCII characters,
the result is an \href{http://www.loc.gov/standards/iso639-2/}{ISO 639-2}
string of the file's language representation.
For example, given the values \texttt{0x15}, \texttt{0x0E} and \texttt{0x04}:
\begin{align*}
\text{language}_0 &= \texttt{0x15} + \texttt{0x60} = \texttt{0x75} = \texttt{u} \\
\text{language}_1 &= \texttt{0x0E} + \texttt{0x60} = \texttt{0x6E} = \texttt{n} \\
\text{language}_2 &= \texttt{0x04} + \texttt{0x60} = \texttt{0x64} = \texttt{d}
\end{align*}
Which is the code `\texttt{und}', meaning `undetermined' - which is typical.

\clearpage

\subsubsection{the hdlr atom}
\label{alac_hdlr}
\begin{tabular}{|l|r|l|}
\hline
Field & Size & Value \\
\hline
atom length & 32 & 33 + component \\
atom type & 32 & `hdlr' (\texttt{0x68646c72}) \\
\hline
version & 8 & \texttt{0x00} \\
flags & 24 & \texttt{0x000000} \\
QuickTime type & 32 & \texttt{0x00000000} \\
QuickTime subtype & 32 & `soun' (\texttt{0x736f756e}) \\
QuickTime manufacturer & 32 & \texttt{0x00000000} \\
QuickTime component reserved flags & 32 & \texttt{0x00000000} \\
QuickTime component reserved flags mask & 32 & \texttt{0x00000000} \\
component name length & 8 & \texttt{0x00} \\
component name & component name length $\times$ 8 & \\
\hline
\end{tabular}


\subsubsection{the minf atom}
\begin{tabular}{|l|r|l|}
\hline
Field & Size & Value \\
\hline
atom length & 32 & \ATOM{smhd} size + \ATOM{dinf} size + \ATOM{stbl} size + 8 \\
atom type & 32 & `minf' (\texttt{0x6d696e66}) \\
\hline
\ATOM{smhd} atom & \ATOM{smhd} size & \ATOM{smhd} data \\
\ATOM{dinf} atom & \ATOM{dinf} size & \ATOM{dinf} data \\
\ATOM{stbl} atom & \ATOM{stbl} size & \ATOM{stbl} data \\
\hline
\end{tabular}

\subsubsection{the smhd atom}
\begin{tabular}{|l|r|l|}
\hline
Field & Size & Value \\
\hline
atom length & 32 & 16 \\
atom type & 32 & `smhd' (\texttt{0x736d6864}) \\
\hline
version & 8 & \texttt{0x00} \\
flags & 24 & \texttt{0x000000} \\
audio balance & 16 & \texttt{0x0000} \\
padding & 16 & \texttt{0x0000} \\
\hline
\end{tabular}

\subsubsection{the dinf atom}
\begin{tabular}{|l|r|l|}
\hline
Field & Size & Value \\
\hline
atom length & 32 & \ATOM{dref} size + 8 \\
atom type & 32 & `dinf' (\texttt{0x64696e66}) \\
\hline
\ATOM{dref} atom & \ATOM{dref} size & \ATOM{dref} data \\
\hline
\end{tabular}

\clearpage

\subsubsection{the dref atom}

\begin{table}[h]
\begin{tabular}{|l|r|l|}
\hline
Field & Size & Value \\
\hline
atom length & 32 & 28 \\
atom type & 32 & `dref' (\texttt{0x64726566}) \\
\hline
version & 8 & \texttt{0x00} \\
flags & 24 & \texttt{0x000000} \\
number of references & 32 & \texttt{1} \\
\hline
\hline
reference atom size & 32 & \texttt{12} \\
reference atom type & 32 & `url ' (\texttt{0x75726c20}) \\
reference atom data & 32 & \texttt{0x00000001} \\
\hline
\end{tabular}
\end{table}

\subsubsection{the stbl atom}

\begin{table}[h]
\begin{tabular}{|l|r|l|}
\hline
Field & Size & Value \\
\hline
atom length & 32 & \ATOM{stsd} size + \ATOM{stts} size + \ATOM{stsc} size + \\
& & \ATOM{stsz} size + \ATOM{stco} size + 8 \\
atom type & 32 & `stbl' (\texttt{0x7374626c}) \\
\hline
\ATOM{stsd} atom & \ATOM{stsd} size & \ATOM{stsd} data \\
\ATOM{stts} atom & \ATOM{stts} size & \ATOM{stts} data \\
\ATOM{stsc} atom & \ATOM{stsc} size & \ATOM{stsc} data \\
\ATOM{stsz} atom & \ATOM{stsz} size & \ATOM{stsz} data \\
\ATOM{stco} atom & \ATOM{stco} size & \ATOM{stco} data \\
\hline
\end{tabular}
\end{table}

\subsubsection{the stsd atom}

\begin{table}[h]
\begin{tabular}{|l|r|l|}
\hline
Field & Size & Value \\
\hline
atom length & 32 & \ATOM{alac} size + 16 \\
atom type & 32 & `stsd' (\texttt{0x73747364}) \\
\hline
version & 8 & \texttt{0x00} \\
flags & 24 & \texttt{0x000000} \\
number of descriptions & 32 & \texttt{1} \\
\hline
\ATOM{alac} atom & \ATOM{alac} size & \ATOM{alac} data \\
\hline
\end{tabular}
\end{table}

\clearpage

\subsubsection{the alac atom}

\begin{table}[h]
\begin{tabular}{|l|r|l|}
\hline
Field & Size & Value \\
\hline
atom length & 32 & 72 \\
atom type & 32 & `alac' (\texttt{0x616c6163}) \\
\hline
reserved & 48 & \texttt{0x000000000000} \\
reference index & 16 & \texttt{1} \\
version & 16 & \texttt{0} \\
revision level & 16 & \texttt{0} \\
vendor & 32 & \texttt{0x00000000} \\
channels & 16 & channel count \\
bits per sample & 16 & bits per sample \\
compression ID & 16 & \texttt{0} \\
audio packet size & 16 & \texttt{0} \\
%FIXME
%not entirely sure on this one.  it might be a float
sample rate & 16 & sample rate \\
padding & 16 & \texttt{0x0000} \\
\hline
\hline
atom length & 32 & 36 \\
atom type & 32 & `alac' (\texttt{0x616c6163}) \\
\hline
padding & 32 & \texttt{0x00000000} \\
max samples per frame & 32 & largest number of PCM frames per ALAC frame \\
padding & 8 & \texttt{0x00} \\
sample size & 8 & bits per sample \\
history multiplier & 8 & \texttt{40} \\
initial history & 8 & \texttt{10} \\
maximum K & 8 & \texttt{14} \\
channels & 8 & channel count \\
unknown & 16 & \texttt{0x00FF} \\
max coded frame size & 32 & largest ALAC frame size, in bytes \\
bitrate & 32 & $((\text{\ATOM{mdat} size} \times 8 ) \div (\text{total PCM frames} \div \text{sample rate}))$ \\
sample rate & 32 & sample rate \\
\hline
\end{tabular}
\end{table}
The \VAR{history multiplier}, \VAR{initial history} and \VAR{maximum K}
values are encode-time options, typically set to 40, 10 and 14,
respectively.

Note that the \VAR{bitrate} field can't be known in advance;
we must fill that value with 0 for now and then
return to this atom once encoding is completed
and its size has been determined.

\clearpage

\subsubsection{the stts atom}

\begin{table}[h]
\begin{tabular}{|l|r|l|}
\hline
Field & Size & Value \\
\hline
atom length & 32 & number of times $\times$ 8 + 16\\
atom type & 32 & `stts' (\texttt{0x73747473}) \\
\hline
version & 8 & \texttt{0x00} \\
flags & 24 & \texttt{0x000000} \\
number of times & 32 & \\
\hline
frame count 1 & 32 & number of occurrances \\
frame duration 1 & 32 & PCM frame count \\
\hline
\multicolumn{3}{|c|}{...} \\
\hline
\end{tabular}
\end{table}
This atom keeps track of how many different sizes of ALAC frames
occur in the ALAC file, in PCM frames.
It will typically have only two ``times'', the block size we're
using for most of our samples and the final block size for
any remaining samples.

For example, let's imagine encoding a 1 minute audio file
at 44100Hz with a block size of 4096 frames.
This file has a total of 2,646,000 PCM frames ($60 \times 44100 = 2646000$).
2,646,000 PCM frames divided by a 4096 block size means
we have 645 ALAC frames of size 4096, and 1 ALAC frame of size 4080.

Therefore:
\begin{align*}
\text{number of times} &= 2 \\
\text{frame count}_1 &= 645 \\
\text{frame duration}_1 &= 4096 \\
\text{frame count}_2 &= 1 \\
\text{frame duration}_2 &= 4080
\end{align*}

\subsubsection{the stsc atom}

\begin{table}[h]
\begin{tabular}{|l|r|l|}
\hline
Field & Size & Value \\
\hline
atom length & 32 & entries $\times$ 12 + 16 \\
atom type & 32 & `stsc' (\texttt{0x73747363}) \\
\hline
version & 8 & \texttt{0x00} \\
flags & 24 & \texttt{0x000000} \\
number of entries & 32 & \\
\hline
first chunk & 32 & \\
ALAC frames per chunk & 32 & \\
description index & 32 & \texttt{1} \\
\hline
\multicolumn{3}{|c|}{...} \\
\hline
\end{tabular}
\end{table}

This atom stores how many ALAC frames are in a given ``chunk''.
In this instance a ``chunk'' represents an entry in
the \ATOM{stco} atom table, used for seeking backwards and forwards
through the file.
\VAR{First chunk} is the starting offset of its frames-per-chunk
value, beginning at 1.

As an example, let's take a one minute, 44100Hz audio file
that's been broken into 130 chunks
(each with an entry in the \ATOM{stco} atom).
Its \ATOM{stsc} entries would typically be:
\begin{align*}
\text{first chunk}_1 &= 1 \\
\text{frames per chunk}_1 &= 5 \\
\text{first chunk}_2 &= 130 \\
\text{frames per chunk}_2 &= 1
\end{align*}
What this means is that chunks 1 through 129 have 5 ALAC frames each
while chunk 130 has 1 ALAC frame.
This is a total of 646 ALAC frames, which matches the contents of
the \ATOM{stts} atom.

\subsubsection{the stsz atom}

\begin{tabular}{|l|r|l|}
\hline
Field & Size & Value \\
\hline
atom length & 32 & sizes $\times$ 4 + 20 \\
atom type & 32 & `stsz' (\texttt{0x7374737a}) \\
\hline
version & 8 & \texttt{0x00} \\
flags & 24 & \texttt{0x000000} \\
block byte size & 32 & \texttt{0x00000000} \\
number of sizes & 32 & \\
\hline
frame size & 32 & \\
\hline
\multicolumn{3}{|c|}{...} \\
\hline
\end{tabular}

This atom is a list of ALAC frame sizes, each in bytes.
For example, our 646 frame file would have 646 corresponding
\ATOM{stsz} entries.

\subsubsection{the stco atom}

\begin{tabular}{|l|r|l|}
\hline
Field & Size & Value \\
\hline
atom length & 32 & offset $\times$ 4 + 16 \\
atom type & 32 & `stco' (\texttt{0x7374636f}) \\
\hline
version & 8 & \texttt{0x00} \\
flags & 24 & \texttt{0x000000} \\
number of offsets & 32 & \\
\hline
frame offset & 32 & \\
\hline
\multicolumn{3}{|c|}{...} \\
\hline
\end{tabular}

This atom is a list of absolute file offsets for each chunk, where
each chunk is typically 5 ALAC frames large.

\clearpage

\subsubsection{the udta atom}

\begin{tabular}{|l|r|l|}
\hline
Field & Size & Value \\
\hline
atom length & 32 & \ATOM{meta} size + 8 \\
atom type & 32 & `udta' (\texttt{0x75647461}) \\
\hline
\ATOM{meta} atom & \ATOM{meta} size & \ATOM{meta} data \\
\hline
\end{tabular}

\subsubsection{the meta atom}

\begin{tabular}{|l|r|l|}
\hline
Field & Size & Value \\
\hline
atom length & 32 & \ATOM{hdlr} size + \ATOM{ilst} size + \ATOM{free} size + 12 \\
atom type & 32 & `meta' (\texttt{0x6d657461}) \\
\hline
version & 8 & \texttt{0x00} \\
flags & 24 & \texttt{0x000000} \\
\hline
\ATOM{hdlr} atom & \ATOM{hdlr} size & \ATOM{hdlr} data \\
\ATOM{ilst} atom & \ATOM{ilst} size & \ATOM{ilst} data \\
\ATOM{free} atom & \ATOM{free} size & \ATOM{free} data \\
\hline
\end{tabular}

\subsubsection{the hdlr atom (revisited)}

\begin{tabular}{|l|r|l|}
\hline
Field & Size & Value \\
\hline
atom length & 32 & 34 \\
atom type & 32 & `hdlr' (\texttt{0x68646c72}) \\
\hline
version & 8 & \texttt{0x00} \\
flags & 24 & \texttt{0x000000} \\
QuickTime type & 32 & \texttt{0x00000000} \\
QuickTime subtype & 32 & `mdir' (\texttt{0x6d646972}) \\
QuickTime manufacturer & 32 & `appl' (\texttt{0x6170706c}) \\
QuickTime component reserved flags & 32 & \texttt{0x00000000} \\
QuickTime component reserved flags mask & 32 & \texttt{0x00000000} \\
component name length & 8 & \texttt{0x00} \\
component name & 0 & \\
\hline
\end{tabular}

This atom is laid out identically to the ALAC file's primary
\ATOM{hdlr} atom (described on page \pageref{alac_hdlr}).
The only difference is the contents of its fields.

\subsubsection{the ilst atom}

This atom is a collection of \ATOM{data} sub-atoms
and is described on page \pageref{m4a_meta}.

\subsubsection{the free atom}

These atoms are simple collection of NULL bytes which can easily be
resized to make room for other atoms without rewriting the entire file.

\clearpage

\subsection{Compressed and uncompressed frames}

Now that we've built a set of ALAC atoms, the next step is to break
our audio into a set of PCM frames of a certain block size -
typically 4096.
We then have to decide whether to turn them into a compressed
or uncompressed frame.
This is done by first attempting a compressed frame while keeping
track of its size.
If that compressed frame's size is greater than what an
uncompressed frame would be
($(\text{block size} \times \text{channels} \times \text{bits per sample} \div 8) + 7$), use an uncompressed frame instead.\footnote{See page \pageref{alac_uncompressed}}
For any audio data that isn't random noise, compressed frames
will be the better choice.

As you'll recall, a compressed frame is laid out as follows:

\begin{figure}[h]
\includegraphics{figures/alac_compressed_frame.pdf}
\end{figure}
\par
\noindent
\VAR{Channel Count} will be consistent throughout the stream:
0 for mono streams, 1 for stereo streams.
\VAR{Has Sample Size} will be 0 so long as we have enough
remaining samples to fill our consistent block size.
But, at the end of the stream, this value will be 1 to cover any
remaining samples whose size will be stored in the \VAR{Sample Size} field.
Next, we use a \VAR{Wasted Bits} value of 1 if our stream's bits-per-sample
is greater than 16.

In the event we have wasted bits, we simply chop off the bottom 8 bits
of each sample and store them as a block of wasted bits samples
(whose values are typically random noise, which does not compress well)
and then treat the stream as having 16 bits per sample
throughout the remainder of compression.

\clearpage

\subsection{Channel correlation}

For stereo streams, we must determine good \VAR{Interlacing Shift}
and \VAR{Interlacing Leftweight} values to exploit similarities
between the left and right channels.

In this case, we'll use an \VAR{Interlacing Shift} value of 2
and try all all \VAR{Interlacing Leftweight} values between 0 and 4
(inclusive), using the one that generates the smallest subframes.

Correlating our left and right channel samples using
\VAR{Interlacing Shift} and \VAR{Interlacing Leftweight} is done
using the inverse of decorrelation:
\begin{align*}
\text{Channel}_1 &= \text{Right}_i + \left\lfloor\frac{(\text{Left}_i - \text{Right}_i) \times \text{Interlacing Leftweight}}{2 ^ \text{Interlacing Shift}}\right\rfloor \\
\text{Channel}_2 &= \text{Left}_i - \text{Right}_i
\end{align*}
\par
For example, given the left channels 18, 20, 26, 24, 24;
the right channels of 2, 3, 0, -1. 0; an \VAR{Interlacing Shift} value
of 2 and an \VAR{Interlacing Leftweight} value of 3, we
calculate ALAC channels as follows:
\begin{table}[h]
\begin{tabular}{|c||>{$}r<{$}|>{$}r<{$}||>{$}r<{$}|>{$}r<{$}|}
\hline
Sample & \text{Left}_i & \text{Right}_i & \text{Channel}_1 & \text{Channel}_2 \\
\hline
0 & 18 & 2 & 2 + \left\lfloor\frac{(18 - 2) \times 3}{2 ^ 2}\right\rfloor = 2 + 12 = \textbf{14} & 18 - 2 = \textbf{16} \\
1 & 20 & 3 & 3 + \left\lfloor\frac{(20 - 3) \times 3}{2 ^ 2}\right\rfloor = 3 + 12 = \textbf{15} & 20 - 3 = \textbf{17} \\
2 & 26 & 0 & 0 + \left\lfloor\frac{(26 - 0) \times 3}{2 ^ 2}\right\rfloor = 0 + 19 = \textbf{19} & 26 - 0 = \textbf{26} \\
3 & 24 & -1 & -1 + \left\lfloor\frac{(24 + 1) \times 3}{2 ^ 2}\right\rfloor = -1 + 18 = \textbf{17} & 24 - -1 = \textbf{25} \\
4 & 24 & 0 & 0 + \left\lfloor\frac{(24 - 0) \times 3}{2 ^ 2}\right\rfloor = 0 + 18 = \textbf{18} & 24 - 0 = \textbf{24} \\
\hline
\end{tabular}
\end{table}

\subsection{Coefficient calculation}
%example samples: [14, 15, 19, 17, 18, 17, 16, 18, 17, 15,
%                  13, 13, 12, 12, 15, 17, 16, 17, 16, 13]

Given a list of correlated samples, we need to generate
one subframe header per channel by performing coefficient calculation
to generate \VAR{Prediction Quantitization},
\VAR{Coefficient Count} and a set of \VAR{Coefficient} values.
We'll set \VAR{Prediction Type} to 0 and \VAR{Rice Modifier} to 4.
\begin{figure}[h]
\includegraphics{figures/alac_subframe_header.pdf}
\end{figure}
\par
\noindent
We'll adapt FLAC's LPC subframe calculation routine\footnote{See page \pageref{flac_lpc_encoding}} for this purpose.

\clearpage

\subsubsection{Windowing}
%windowed to: [0.000000,2.196699,9.500000,14.510408,18.000000,
%              17.000000,16.000000,18.000000,17.000000,15.000000,
%              13.000000,13.000000,12.000000,12.000000,15.000000,
%              17.000000,13.656854,8.500000,2.343146,0.000000]

The first step in ALAC subframe encoding is `windowing' the input signal.
Put simply, this is a process of multiplying each input sample by
an equivalent value from the window, which are floats from 0.0 to 1.0.
In this case, the default is a Tukey window with a ratio of 0.5.
A Tukey window is a combination of the Hann and Rectangular windows.
The ratio of 0.5 means there's 0.5 samples in the Hann window per
sample in the Rectangular window.

\begin{align}
\text{hann}(n) &= \frac{1}{2} \left ( 1 - \cos \left ( \frac{2 \pi n}{\text{sample count} - 1} \right ) \right ) \\
\text{rectangle}(n) &= 1.0
\end{align}

The Tukey window is defined by taking a Hann window, splitting it at
the halfway point, and inserting a Rectangular window between the
two.

\begin{figure}[h]
\subfloat{
\includegraphics{figures/flac_hann.pdf}
}
\subfloat{
\includegraphics{figures/flac_rectangular.pdf}
}
\subfloat{
\includegraphics{figures/flac_tukey.pdf}
}
\end{figure}

Let's run through a short example with 20 samples:

{\relsize{-1}
\begin{tabular}{|r|r >{$}c<{$} r >{$}c<{$} r|}
\hline
& input & & Tukey & & windowed \\
index & sample & & window & & signal \\
\hline
0 & 14 & \times & 0.0000 & = & 0.00 \\
1 & 15 & \times & 0.1464 & = & 2.20 \\
2 & 19 & \times & 0.5000 & = & 9.50 \\
3 & 17 & \times & 0.8536 & = & 14.51 \\
4 & 18 & \times & 1.0000 & = & 18.00 \\
5 & 17 & \times & 1.0000 & = & 17.00 \\
6 & 16 & \times & 1.0000 & = & 16.00 \\
7 & 18 & \times & 1.0000 & = & 18.00 \\
8 & 17 & \times & 1.0000 & = & 17.00 \\
9 & 15 & \times & 1.0000 & = & 15.00 \\
10 & 13 & \times & 1.0000 & = & 13.00 \\
11 & 13 & \times & 1.0000 & = & 13.00 \\
12 & 12 & \times & 1.0000 & = & 12.00 \\
13 & 12 & \times & 1.0000 & = & 12.00 \\
14 & 15 & \times & 1.0000 & = & 15.00 \\
15 & 17 & \times & 1.0000 & = & 17.00 \\
16 & 16 & \times & 0.8536 & = & 13.66 \\
17 & 17 & \times & 0.5000 & = & 8.50 \\
18 & 16 & \times & 0.1464 & = & 2.34 \\
19 & 13 & \times & 0.0000 & = & 0.00 \\
\hline
\end{tabular}
}

\subsubsection{Computing autocorrelation}
%autocorrelated to: [3416.877417,3314.071374,3078.904744,2807.422836,
%                    2554.560660,2325.480970,2107.853553,1903.292893,
%                    1701.213203]

Once our input samples have been converted to a windowed signal,
we then compute the autocorrelation values from that signal.
Each autocorrelation value is determined by multiplying the signal's
samples by the samples of a lagged version of that same signal,
and then taking the sum.
The lagged signal is simply the original signal with \VAR{lag} number of
samples removed from the beginning.

\begin{figure}[h]
\includegraphics{figures/alac_lag.pdf}
\end{figure}

In this example are autocorrelation values are:
3416.9513, 3314.1450, 3078.9564, 2807.4600, 2554.6000,
2325.5300, 2107.9100, 1903.3500 and 1701.2700.

\clearpage

\subsubsection{LP coefficient calculation}

Calculating the LP coefficients uses the Levinson-Durbin recursive method.\footnote{This algorithm is taken from \LINK{http://www.engineer.tamuk.edu/SPark/chap7.pdf} }
Our inputs are $M$, the maximum coefficient count, and $r$ autocorrelation
values, from $r(0)$ to $r(M - 1)$.
Our outputs are $a$, a list of LP coefficient lists from
$a_{11}$ to $a_{(M - 1)(M - 1)}$, and $E$, a list
of error values from $E_0$ to $E_{(M - 1)}$.
$q_m$ and $\kappa_m$ are temporary values.
\par
\noindent
Initial values:
\begin{align*}
&& E_0 &= r(0) \\
&& a_{11} &= \kappa_1 = \frac{r(1)}{E_0} \\
&& E_1 &= E_0 {(1 - {{\kappa_1} ^ 2})}
\intertext{With $m \geq 2$, the following recurive algorithm is performed:}
\text{Step 1. } && q_m &= r(m) - \overset{m - 1}{\underset{i = 1}{\sum}} a_{i(m - 1)}r(m - i) \\
\text{Step 2. } && \kappa_m &= \frac{q_m}{E_{(m - 1)}} \\
\text{Step 3. } && a_{mm} &= \kappa_m \\
\text{Step 4. } && a_{im} &= a_{i(m - 1)} - \kappa_m a_{(m - i)(m - 1)} \text{ for $i = 1$, $i = 2$,...,$i = m - 1$} \\
\text{Step 5. } && E_m &= E_{m - 1}(1 - {\kappa_m} ^ 2) \\
\text{Step 6. } &&& \text{If $m < M$ then $m \leftarrow m + 1$ and goto step 1.  If $m = M$ then stop.}
\end{align*}

Let's run through an example in which $M = 9$,
$r(0) = 3417$, $r(1) = 3314$, $r(2) = 3079$, $r(3) = 2807$,
$r(4) = 2555$, $r(5) = 2326$, $r(6) = 2108$, $r(7) = 1903$
and $r(8) = 1701$:
\clearpage

{\relsize{-2}
\begin{align*}
E_0 &= r(0) = 3417 \\
a_{11} &= \kappa_1 = \frac{r(1)}{E_0} = \frac{3314}{3417} = 0.97 \\
E_1 &= E_o(1 - {\kappa_1} ^ 2) = 3417(1 - .97 ^ 2) = 201.95 \\
q_2 &= r(2) - \overset{1}{\underset{i = 1}{\sum}} a_{i1}r(2 - i) = 3079 - (0.97)(3314) = -135.58 \\
\kappa_2 &= \frac{q_2}{E_1} = \frac{-135.58}{201.95} = -0.671 \\
a_{22} &= \kappa_2 = -0.671 \\
a_{12} &= a_{11} - \kappa_2 a_{11} = 0.97 - (-0.671)(0.97) = 1.621 \\
E_2 &= E_1 (1 - {\kappa_2} ^ 2 ) = 201.95(1 - {-0.671 ^ 2}) = 111.024 \\
q_3 &= r(3) - \overset{2}{\underset{i = 1}{\sum}} a_{i2} r(3 - i) = 2807 - ((1.621)(3079) + (-0.671)(3314)) = 39.635 \\
\kappa_3 &= \frac{q_3}{E_2} = \frac{39.635}{111.024} = 0.357 \\
a_{33} &= \kappa_3 = 0.357 \\
a_{13} &= a_{12} - \kappa_3 a_{22} = 1.621 - (0.357)(-0.671) = 1.861 \\
a_{23} &= a_{22} - \kappa_3 a_{12} = -0.671 - (0.357)(1.621) = -1.25 \\
E_3 &= E_2 (1 - {\kappa_3} ^ 2 ) = 111.024(1 - {0.357 ^ 2}) = 96.874 \\
q_4 &= r(4) - \overset{3}{\underset{i = 1}{\sum}} a_{i3} r(4 - i) = 2555 - ((1.861)(2807) + (-1.25)(3079) + (0.357)(3314)) = -3.175 \\
\kappa_4 &= \frac{q_4}{E_3} = \frac{-3.175}{96.874} = -0.033 \\
a_{44} &= \kappa_4 = -0.033 \\
a_{14} &= a_{13} - \kappa_4 a_{33} = 1.861 - (-0.033)(0.357) = 1.873 \\
a_{24} &= a_{23} - \kappa_4 a_{23} = -1.25 - (-0.033)(-1.25) = -1.291 \\
a_{34} &= a_{33} - \kappa_4 a_{33} = 0.357 - (-0.033)(1.861) = 0.418 \\
E_4 &= E_3 (1 - {\kappa_4} ^ 2) = 96.874(1 - {-0.033 ^ 2}) = 96.769
\end{align*}
}
Calculating $E_5$ through $E_8$ and $a_{15}$ though $a_{88}$ will be
left as an exercise for the reader.
Our final values are:
{\relsize{-2}
\begin{align*}
a_{11} &= 0.970 &&&&&&&&&&&&&& \\
a_{12} &= 1.621 & a_{22} &= -0.67 &&&&&&&&&&&& \\
a_{13} &= 1.861 & a_{23} &= -1.25 & a_{33} &= 0.357 &&&&&&&&&& \\
a_{14} &= 1.873 & a_{24} &= -1.29 & a_{34} &= 0.418 & a_{44} &= -0.03 &&&&&&&& \\
a_{15} &= 1.868 & a_{25} &= -1.28 & a_{35} &= 0.340 & a_{45} &= 0.122 & a_{55} &= -0.14 &&&&&& \\
a_{16} &= 1.874 & a_{26} &= -1.28 & a_{36} &= 0.341 & a_{46} &= 0.119 & a_{56} &= -0.09 & a_{66} &= 0.00 &&&& \\
a_{17} &= 1.875 & a_{27} &= -1.27 & a_{37} &= 0.332 & a_{47} &= 0.091 & a_{57} &= 0.011 & a_{67} &= -0.15 & a_{77} &= 0.078 & & \\
a_{18} &= 1.892 & a_{28} &= -1.30 & a_{38} &= 0.338 & a_{48} &= 0.112 & a_{58} &= 0.082 & a_{68} &= -0.42 & a_{78} &= 0.479 & a_{88} &= -0.214 \\
E_1 &= 202.0 & E_2 &= 111.0 & E_3 &= 96.9 & E_4 &= 96.8 & E_5 &= 96.1 & E_6 &= 96.1 & E_7 &= 95.6 & E_8 &= 91.2
\end{align*}
}
\par
\noindent
These values have been rounded to the nearest significant digit
and will not be an exact match to those generated by a computer.

\clearpage

\subsubsection{Best order estimation}

At this point, we have an array of prospective LP coefficient lists,
a list of error values and must decide which LPC order to use.
Making an estimation requires the total number of samples
in the subframe, the number of overhead bits per order (by default,
this is the number of bits per sample in the subframe, plus 5),
and an error scale constant in addition to the LPC error values:
\begin{equation}
\text{Error Scale} = \frac{\ln(2) ^ 2}{2 \times \text{Total Samples}}
\end{equation}
\noindent
Once the error scale has been calculated, one can generate a
\VAR{Bits per Residual} estimation function which, given an \VAR{LPC Error}
value, returns what its name implies:
\begin{equation}
\text{Bits per Residual}(\text{LPC Error}) = \frac{\ln(\text{Error Scale} \times \text{LPC Error})}{2 \times \ln(2)}
\end{equation}
With this function, we can estimate how many bits the entire ALAC subframe
will take for each \VAR{LPC Error} value and its associated \VAR{Order}:
{\relsize{-2}
\begin{equation*}
\text{Total Bits}(\text{LPC Error},\text{Order}) = \text{Bits per Residual}(\text{LPC Error}) \times (\text{Total Samples} - \text{Order}) + (\text{Order} \times \text{Overhead bits})
\end{equation*}
}
Continuing with our example, we have 20 samples which gives us an
error scale of: $\frac{{\ln(2) ^ 2}}{2 \times 20} = \frac{.6931 ^ 2}{40} = .01201$

Now, we'll estimate the bits used by order 4 and order 8,
which use LPC Error values 96.8 and 91.2, respectively:

\begin{align*}
\intertext{At LPC order 4, our bits per residual are:}
\frac{\ln(.01201 \times 96.8)}{2 \times \ln (2)} = \frac{\ln(1.163)}{1.386} &= 0.1089 \\
\intertext{And our total bits are:}
(0.1089 \times (20 - 1)) + (1 \times (16 + 5)) = 2.069 + 21 &= 23.069 \\
\intertext{At LPC order 8, our bits per residual are:}
\frac{\ln(.01201 \times 91.2)}{2 \times \ln(2)} = \frac{\ln(1.095)}{1.386} &= 0.065 \\
\intertext{And our total bits are:}
(0.065 \times (20 - 2)) + (2 \times (16 + 5)) = 1.17 + 42 &= 43.17 \\
\end{align*}
\par
\noindent
Therefore, since the total bits for order 4 are the smallest,
the best order for this group of samples is 4.

\subsubsection{Quantizing coefficients}

Quantizing coefficients is a process of taking a list of LP Coefficients
along with a QLP Coefficients Precision value and
returning a list of Coefficients and a Prediction Quantitization value.
The first step is determining the upper and lower limits of the
Coefficients:
\begin{align}
\text{Coefficient Maximum} &= 2 ^ {16 - 1} - 1 = 32767\\
\text{Coefficient Mininum} &= - 2 ^ {16 - 1} = -32768
\end{align}
\par
\noindent
Now we determine the initial Prediction Quantitization value:
\begin{equation}
\text{quantitization} = 16 - \left \lceil \frac{\log(\max(|\text{LP Coefficients}|))}{\log(2) } \right \rceil - 1
\end{equation}
\par
\noindent
where `quantitization' is adjusted if necessary such that:
$0 \leq \text{quantitization} \leq \texttt{0xF}$
, since it must fit into a 4-bit unsigned field.

Continuing our ongoing example, let's assume we're quantizing the LP
coefficients 1.873, -1.29, 0.418 and -0.03.
So our prediction quantitization should be:
\begin{equation*}
\text{shift} = 16 - \left \lceil \frac{\log(1.873)}{\log(2)} \right \rceil - 1
= 16 - \left \lceil \frac{0.6275}{0.6931} \right \rceil - 1 = 16 - 1 - 1 = 14
\end{equation*}
\par
\noindent
Finally, we determine the Coefficient values themselves via a small
recursive routine:
\begin{align}
X(i) &= E(i - 1) + (\text{LP Coefficient}_i \times 2 ^ {quantitization}) \\
\text{Coefficient}_i &= \text{round}(X(i)) \\
E(i) &= X(i) - \text{Coefficient}_i
\end{align}
\par
\noindent
where $E(0) = 0$ and each Coefficient is adjusted
prior to calculating the next $E(i)$ value such that:
$\text{Coefficient Minimum} \leq \text{Coefficient}_i \leq \text{Coefficient Maximum}$

So to finish our example:
{\relsize{-2}
\begin{align*}
X(1) &= E(0) + (1.873 \times 2 ^ {14} ) = 0 + 30687.232 = \textbf{30687.232} \\
\text{QLP Coefficient}_1 &= \text{round}(30687.232) = \textbf{30687} \\
E(1) &= X(1) - \text{QLP Coefficient}_1 = 30687.232 - 30687= \textbf{0.232} \\
X(2) &= E(1) + (-1.29 \times 2 ^ {14} ) = 0.232 + -21135.36 = \textbf{-21135.128} \\
\text{QLP Coefficient}_2 &= \text{round}(-21135.128) = \textbf{-21135} \\
E(2) &= X(2) - \text{QLP Coefficient}_2 = -21135.128 - -21135 = \textbf{-0.128} \\
X(3) &= E(2) + (0.418 \times 2 ^ {14} ) = -0.128 + 6848.512 = \textbf{6848.384} \\
\text{QLP Coefficient}_3 &= \text{round}(6848.384) = \textbf{6848} \\
E(3) &= X(3) - \text{QLP Coefficient}_3 = 6848.384 - 6848 = \textbf{0.384} \\
X(4) &= E(3) + (-0.03 \times 2 ^ {14} ) = 0.384 - 491.52 = \textbf{-491.136} \\
\text{QLP Coefficient}_4 &= \text{round}(-491.136) = \textbf{-491} \\
\end{align*}
}
\par
\noindent
Therefore, the \VAR{Coefficient Count} is 4,
the \VAR{Coefficient} values are 30687, -21135, 6848 and -491,
and the \VAR{Prediction Qauntitization} value is 14.


\clearpage

\subsection{Subframe calculation}

Given our list of correlated samples; along with a list of
coefficients, a \VAR{Coefficient Count} and a \VAR{Prediction Quantitization}
value, we can now generate a list of signed residuals values
for a given channel.

The first residual is always the first input sample:
\begin{align*}
\text{Residual}_0 &= \text{Sample}_0 \\
\intertext{Then, for the next \VAR{Coefficient Count} number of samples:}
\text{Residual}_i &= \text{Sample}_i - \text{Sample}_{i - 1}
\end{align*}

For example, given that we have a \VAR{Coefficient Count} of 4
and our first five samples are 16, 17, 26, 25 and 24;
our first five residual values are:
\begin{align*}
\text{Residual}_0 &= \textbf{16} \\
\text{Residual}_1 = 17 - 16 &= \textbf{1} \\
\text{Residual}_2 = 26 - 17 &= \textbf{9} \\
\text{Residual}_3 = 25 - 26 &= \textbf{-1} \\
\text{Residual}_4 = 24 - 25 &= \textbf{-1}
\end{align*}

These are our ``starting point'' residuals upon which the
remainder of the residuals will be calculated from.

Subsequent residuals are calculated in the following way:
\begin{align*}
\text{LPC Sum}_i &= \overset{coeffs - 1}{\underset{j = 0}{\sum}}
\text{Coefficient}_j \times (\text{Sample}_{i - j - 1} - \text{Sample}_{i - coeffs - 1}) \\
\text{Residual}_i &= \text{Sample}_i - \left( \left\lfloor \frac{\text{LPC Sum}_i + 2^{\text{Predictor Quantitization - 1}}} {2 ^ \text{Predictor Quantitization}}\right\rfloor + \text{Sample}_{i - coeffs - 1} \right) \\
\intertext{For example, given $\text{Sample}_5$ = 23,
\VAR{Predictor Quantitization} = 9 and the coefficients
1122, -766, 107 and 122:}
\text{LPC Sum}_5 &=
(1122 \times (24 - 16)) + (-766 \times (25 - 16)) +
 (107 \times (26 - 16)) + (122 \times (17 - 16)) \\
&= (1122 \times 8) + (-766 \times 9) + (107 \times 10) + (122 \times 1) \\
&= 8976 + -6894 + 1070 + 122 = \textbf{3274} \\
\text{Residual}_5 &= 23 - \left( \left\lfloor \frac{\textbf{3274} + 2^8}{2 ^ 9}\right\rfloor + 16 \right) =  23 - \left( \left\lfloor \frac{3530}{512} \right\rfloor + 16 \right) = 23 - (6 + 16) = \textbf{1}
\end{align*}
But before calculation $\text{Residual}_6$, we need to adjust
our coefficient list in the same way as residual decoding.

\clearpage

As described on page \pageref{alac_coefficient_adjustment},
we run through the following process:
\par
{\relsize{-1}
\begin{tabular}{|l>{$}r<{$}c>{$}l<{$}|}
\hline
Step 1. & \text{original sign} &=& \text{sign}(residual) \\
Step 2. & j &=& 0 \\
Step 3. & v &=& \text{Sample}_{i - \text{count} - 1} - \text{Sample}_{i - \text{count} + j} \\
Step 4. & sign &=& \begin{cases}\text{sign}(v) & \text{if original sign} > 0 \\
-\text{sign}(v) & \text{if original sign} \leq 0\end{cases} \\
Step 5. & \text{Coefficient}_{\text{count} - j - 1} &=& \text{Coefficient}_{\text{count} - j - 1} - sign \\
Step 6. & residual &=& residual - \left( \left\lfloor\frac{v \times sign}{2 ^ \text{Predictor Quantitization}}\right\rfloor \times (j + 1) \right) \\
Step 7. & j &=& j + 1 \\
Step 8. & \multicolumn{3}{l|}{if ($j < count$) and ($\text{sign}(residual) = \text{original sign}$), goto step 3} \\
\hline
\end{tabular}
}
\par
\noindent
To continue our example:
{\relsize{-1}
\begin{align*}
\text{original sign} &= \text{sign}(1) = \textbf{1} \\
j &= 0 \\
v &= \text{Sample}_0 - \text{Sample}_{1} = 16 - 17 = \textbf{-1} \\
sign &= \text{sign}(-1) = \textbf{-1} \\
\text{Coefficient}_3 &= \text{Coefficient}_3 - -1 = 122 + 1 = \textbf{123} \\
residual &= residual - \left( \left\lfloor\frac{-1 \times -1}{2 ^ 9}\right\rfloor \times (0 + 1)\right) = 1 - (0 \times 1) = \textbf{1} \\
j &= 1 \\
v &= \text{Sample}_0 - \text{Sample}_{2} = 16 - 26 = \textbf{-10} \\
sign &= \text{sign}(-10) = \textbf{-1} \\
\text{Coefficient}_2 &= \text{Coefficient}_2 - -1 = 107 + 1 = \textbf{108} \\
residual &= residual - \left( \left\lfloor\frac{-10 \times -1}{2 ^ 9}\right\rfloor \times (1 + 1)\right) = 1 - (0 \times 2) = \textbf{1} \\
j &= 2 \\
v &= \text{Sample}_0 - \text{Sample}_{3} = 16 - 25 = \textbf{-9} \\
sign &= \text{sign}(-9) = \textbf{-1} \\
\text{Coefficient}_1 &= \text{Coefficient}_1 - -1 = -766 + 1 = \textbf{-765} \\
residual &= residual - \left( \left\lfloor\frac{-9 \times -1}{2 ^ 9}\right\rfloor \times (2 + 1)\right) = 1 - (0 \times 3) = \textbf{1} \\
j &= 3 \\
v &= \text{Sample}_0 - \text{Sample}_{4} = 16 - 25 = \textbf{-8} \\
sign &= \text{sign}(-8) = \textbf{-1} \\
\text{Coefficient}_0 &= \text{Coefficient}_0 - -1 = 1122 + 1 = \textbf{1123} \\
residual &= residual - \left( \left\lfloor\frac{-8 \times -1}{2 ^ 9}\right\rfloor \times (3 + 1)\right) = 1 - (0 \times 4) = \textbf{1} \\
j &= 4\text{ and stop, since 4 equals our coefficient count} \\
\end{align*}
}
\par
Which, you'll notice, is identical to the procedure used during
residual decoding.

\clearpage

Given that $\text{Sample}_6$ is 20, the calculation of $\text{Residual}_6$
is as follows:
\begin{align*}
\text{LPC Sum}_6 &= (1123 \times (23 - 17)) + (-765 \times (24 - 17)) + (108 \times (25 - 17)) + (123 \times (26 - 17)) \\
&= (1123 \times 6) + (-765 \times 7) + (108 \times 8) + (123 \times 9) \\
&= 6738 + -5355 + 864 + 1107 = \textbf{3354} \\
\text{Residual}_6 &= 20 - \left( \left\lfloor \frac{\textbf{3354} + 2^8}{2 ^ 9}\right\rfloor + 17 \right) =  20 - \left( \left\lfloor \frac{3610}{512} \right\rfloor + 17 \right) = 20 - (7 + 17) = \textbf{-4} \\
%% \end{align*}
%% \begin{align*}
\text{original sign} &= \text{sign}(-4) = -1 \\
j &= 0 \\
v &= \text{Sample}_1 - \text{Sample}_2 = 17 - 26 = \textbf{-9} \\
sign &= -\text{sign}(-9) = \textbf{1} \\
\text{Coefficient}_3 &= \text{Coefficient}_3 - 1 = 123 - 1 = \textbf{122} \\
residual &= residual - \left( \left\lfloor\frac{-9 \times 1}{2 ^ 9}\right\rfloor \times (0 + 1)\right) = -4 - (-1 \times 1) = \textbf{-3} \\
j &= 1 \\
v &= \text{Sample}_1 - \text{Sample}_3 = 17 - 25 = \textbf{-8} \\
sign &= -\text{sign}(-8) = \textbf{1} \\
\text{Coefficient}_2 &= \text{Coefficient}_2 - 1 = 108 - 1 = \textbf{107} \\
residual &= residual - \left( \left\lfloor\frac{-8 \times 1}{2 ^ 9}\right\rfloor \times (1 + 1)\right) = -3 - (-1 \times 2) = \textbf{-1} \\
j &= 2 \\
v &= \text{Sample}_1 - \text{Sample}_4 = 17 - 24 = \textbf{-7} \\
sign &= -\text{sign}(-7) = \textbf{1} \\
\text{Coefficient}_1 &= \text{Coefficient}_1 - 1 = -765 - 1 = \textbf{-766} \\
residual &= residual - \left( \left\lfloor\frac{-7 \times 1}{2 ^ 9}\right\rfloor \times (2 + 1)\right) = -1 - (-1 \times 3) = \textbf{2} \\
&\text{and stop, since sign}(2) \neq \text{sign}(-4) \text{ (our \VAR{original sign} value)}
\end{align*}

So, the coefficients for $\text{Residual}_7$ are 1123, -766, 107 and 122.

\clearpage

\subsection{Residual encoding}

The final step of frame encoding is writing our list of residual values.
As with decoding, this also requires knowing several additional
values whose defaults are as follows:
\begin{table}[h]
\begin{tabular}{|r|l|}
\hline
value & default \\
\hline
Initial History & 10 \\
History Multiplier & 40 \\
Maximum K & 14 \\
Bits per Sample & $\text{Stream's BPS} - (\text{Wasted Bits} \times 8) + \text{Channels} - 1$ \\
\hline
\end{tabular}
\end{table}
\par
For each residual block, \VAR{History} starts with the value of
\VAR{Initial History} and will change during residual encoding.
We use it to calculate $\kappa$ using the following formula:
\begin{equation}
\kappa = \left\lfloor\log_2 \left( \frac{\text{history}}{2 ^ 9} + 3 \right) \right\rfloor
\end{equation}
Note that if $\kappa$ exceeds the \VAR{Maximum K} value from the
\texttt{alac} atom, \VAR{Maximum K} is used instead.

\begin{wrapfigure}[20]{r}{2.17in}
\includegraphics{figures/alac_write_residual.pdf}
\end{wrapfigure}

The $\kappa$ and \VAR{Bits per Sample} values are used to write
a single unsigned residual value $r$ in the following way:

First, we convert our signed residual to an unsigned value:
\begin{equation*}
\text{unsigned} =
\begin{cases}
\text{signed} \times 2 & \text{if signed} \geq 0 \\
(-\text{signed} \times 2) - 1 & \text{if signed} < 0
\end{cases}
\end{equation*}

We then divide our unsigned residual into quotient and remainder values:
\begin{align*}
q &= \lfloor r \div (2^k - 1) \rfloor \\
e &= r - (q \times (2^k - 1))
\end{align*}

If $q$ is greater than 8, we write a 9 bit value \texttt{0x1FF}
(i.e. the bits `\texttt{1 1 1 1 1 1 1 1 1 1}')
before writing the value $r$ verbatim using \VAR{Bits per Sample}
number of bits.

If $q$ is greater than 0, we write its value in unary.
Otherwise, we write a single \texttt{0} bit.

Finally, if $e$ is greater than 0, we write the value
$e + 1$ using $\kappa$ of bits.
Otherwise, we write the value 0 using $\kappa - 1$ number of bits.
Assuming $\kappa$ is greater than 1, of course.

\clearpage

As with decoding, we then use our unsigned value to update \VAR{history}
before encoding the next residual:
{\relsize{-1}
\begin{equation*}
\text{history} =
\begin{cases}
\text{history} + (\text{unsigned} \times \text{history multiplier}) - \left\lfloor\frac{\text{history} \times \text{history multiplier}}{2^9}\right\rfloor & \text{if unsigned} \leq 65535 \\
65535 & \text{if unsigned} > 65535
\end{cases}
\end{equation*}
}
Should history ever fall below 128, we generate a special
residual value.
That residual's \VAR{Bits per Sample} value is 16, its
$\kappa$ value is:
\begin{equation}
\kappa_{\text{blocksize}} = 7 - \log_2(\text{history}) + \frac{\text{history} + 16}{64}
\end{equation}
and its value is how many consecutive `0' residuals follow -
which may be 0 if the next residual is not `0'.
Either way, if $\text{\VAR{block size}} \leq 65535$, we
subtract 1 to the next unsigned value of the next residual in the block
(if any).  Finally, \VAR{history} is automatically set to 0.

\subsubsection{Residual encoding example}

In this example, we'll encode a group of residuals in which our
\VAR{Initial History} is 1290, our \VAR{History Multiplier} is 40
and our signed residual values are 1, 9 and -1:
\begin{itemize}
\setlength{\itemsep}{0in}
\setlength{\parskip}{0in}
\item Residual 1
\begin{itemize}
\item $\kappa = \lfloor\log_2((1290 \div 2^9) + 3)\rfloor = \lfloor\log_2(5)\rfloor = 2$
\item $\text{unsigned}_1 = 2 \times 1 = \textbf{2}$
\item $\text{history} = 1290 + (2 \times 40) - \lfloor (1290 \times 40) \div 2^9 \rfloor = 1370 - 100 = 1270$
\end{itemize}
\item Residual 2
\begin{itemize}
\item $\kappa = \lfloor\log_2((1270 \div 2^9) + 3)\rfloor = \lfloor\log_2(5)\rfloor = 2$
\item $\text{unsigned}_2 = 2 \times 9 = \textbf{18}$
\item $\text{history} = 1270 + (18 \times 40) - \lfloor (1270 \times 40) \div 2^9 \rfloor = 1990 - 99 = 1891$
\end{itemize}
\item Residual 3
\begin{itemize}
\item $\kappa = \lfloor\log_2((1891 \div 2^9) + 3)\rfloor = \lfloor\log_2(6)\rfloor = 2$
\item $\text{unsigned}_3 = (1 \times 2) - 1 = \textbf{1}$
\item $\text{history} = 1891 + (1 \times 40) - \lfloor (1891 \times 40) \div 2^9 \rfloor = 1931 - 147 = 1784$
\end{itemize}
\end{itemize}
\begin{figure}[h]
\includegraphics{figures/alac_residual_w.pdf}
\end{figure}
