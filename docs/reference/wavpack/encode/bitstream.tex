%This work is licensed under the
%Creative Commons Attribution-Share Alike 3.0 United States License.
%To view a copy of this license, visit
%http://creativecommons.org/licenses/by-sa/3.0/us/ or send a letter to
%Creative Commons,
%171 Second Street, Suite 300,
%San Francisco, California, 94105, USA.

\subsection{Writing Bitstream}
\label{wavpack:write_bitstream}
\input{wavpack/algorithms/write_bitstream}

\clearpage

\subsection{Encoding Bitstream}
\label{wavpack:encode_bitstream}
{\relsize{-1}
  \input{wavpack/algorithms/encode_bitstream}
}

\clearpage

\subsubsection{Writing Modified Elias Gamma Code}
\label{wavpack:write_egc}
\input{wavpack/algorithms/write_egc}

\begin{figure}[h]
  \includegraphics{wavpack/figures/residuals.pdf}
\end{figure}

\clearpage

\subsubsection{\texttt{encode\_residual} Function}
\label{wavpack:encode_residual}
{\relsize{-1}
  \input{wavpack/algorithms/encode_residual}
}

\clearpage

\subsubsection{\texttt{flush} Function}
\label{wavpack:flush_residual}
{\relsize{-1}
  \input{wavpack/algorithms/flush_residual}
}

\clearpage

\subsubsection{\texttt{encode\_zeroes} Function}
\label{wavpack:encode_zeroes}
{\relsize{-1}
  \input{wavpack/algorithms/encode_zeroes}
}

\clearpage


\subsubsection{Residual Encoding Example}
{\relsize{-1}
  Given the residuals:
  \newline
  \begin{tabular}{rr}
    channel 0 residuals : & \texttt{[-61,~-33,~-18,~~1,~20,~35,~50,~62,~68,~71]}\\
    channel 1 residuals : & \texttt{[~31,~~32,~~36,~37,~35,~31,~25,~18,~10,~~0]}\\
  \end{tabular}
  \newline
  And entropies:
  \newline
  \begin{tabular}{rr}
    channel 0 entropy : & \texttt{[118, 194, 322]} \\
    channel 1 entropy : & \texttt{[118, 176, 212]} \\
  \end{tabular}
}

\begin{table}[h]
{\relsize{-2}
\begin{tabular}{|>{$}r<{$}||>{$}r<{$}|>{$}r<{$}|>{$}r<{$}|>{$}r<{$}|>{$}r<{$}|>{$}r<{$}|}
i & r_i & \text{unsigned}_i &\text{sign}_i &
%% \text{median}_{c~0} & \text{median}_{c~1} & \text{median}_{c~2} &
m_i & \text{offset}_i & \text{add}_i \\
\hline
0 & -61 &
60 & 1 &
%% \left\lfloor\frac{118}{2 ^ 4}\right\rfloor + 1 = 8 & \left\lfloor\frac{194}{2 ^ 4}\right\rfloor + 1 = 13 & \left\lfloor\frac{322}{2 ^ 4}\right\rfloor + 1 = 21 &
3 & 60 - (8 + 13 + ((3 - 2) \times 21)) = 18 & 21 - 1 = 20
\\
1 & 31 &
31 & 0 &
%% \left\lfloor\frac{118}{2 ^ 4}\right\rfloor + 1 = 8 & \left\lfloor\frac{176}{2 ^ 4}\right\rfloor + 1 = 12 & \left\lfloor\frac{212}{2 ^ 4}\right\rfloor + 1 = 14 &
2 & 31 - (8 + 12) = 11 & 14 - 1 = 13
\\
\hline
2 & -33 &
32 & 1 &
%% \left\lfloor\frac{123}{2 ^ 4}\right\rfloor + 1 = 8 & \left\lfloor\frac{214}{2 ^ 4}\right\rfloor + 1 = 14 & \left\lfloor\frac{377}{2 ^ 4}\right\rfloor + 1 = 24 &
2 & 32 - (8 + 14) = 10 & 24 - 1 = 23
\\
3 & 32 &
32 & 0 &
%% \left\lfloor\frac{123}{2 ^ 4}\right\rfloor + 1 = 8 & \left\lfloor\frac{191}{2 ^ 4}\right\rfloor + 1 = 12 & \left\lfloor\frac{198}{2 ^ 4}\right\rfloor + 1 = 13 &
2 & 32 - (8 + 12) = 12 & 13 - 1 = 12
\\
\hline
4 & -18 &
17 & 1 &
%% \left\lfloor\frac{128}{2 ^ 4}\right\rfloor + 1 = 9 & \left\lfloor\frac{234}{2 ^ 4}\right\rfloor + 1 = 15 & \left\lfloor\frac{353}{2 ^ 4}\right\rfloor + 1 = 23 &
1 & 17 - 9 = 8 & 15 - 1 = 14
\\
5 & 36 &
36 & 0 &
%% \left\lfloor\frac{128}{2 ^ 4}\right\rfloor + 1 = 9 & \left\lfloor\frac{206}{2 ^ 4}\right\rfloor + 1 = 13 & \left\lfloor\frac{184}{2 ^ 4}\right\rfloor + 1 = 12 &
3 & 36 - (9 + 13 + ((3 - 2) \times 12)) = 2 & 12 - 1 = 11
\\
\hline
6 & 1 &
1 & 0 &
%% \left\lfloor\frac{138}{2 ^ 4}\right\rfloor + 1 = 9 & \left\lfloor\frac{226}{2 ^ 4}\right\rfloor + 1 = 15 & \left\lfloor\frac{353}{2 ^ 4}\right\rfloor + 1 = 23 &
0 & 1 & 9 - 1 = 8
\\
7 & 37 &
37 & 0 &
%% \left\lfloor\frac{138}{2 ^ 4}\right\rfloor + 1 = 9 & \left\lfloor\frac{226}{2 ^ 4}\right\rfloor + 1 = 15 & \left\lfloor\frac{214}{2 ^ 4}\right\rfloor + 1 = 14 &
2 & 37 - (9 + 15) = 13 & 14 - 1 = 13
\\
\hline
8 & 20 &
20 & 0 &
%% \left\lfloor\frac{134}{2 ^ 4}\right\rfloor + 1 = 9 & \left\lfloor\frac{226}{2 ^ 4}\right\rfloor + 1 = 15 & \left\lfloor\frac{353}{2 ^ 4}\right\rfloor + 1 = 23 &
1 & 20 - 9 = 11 & 15 - 1 = 14
\\
9 & 35 &
35 & 0 &
%% \left\lfloor\frac{148}{2 ^ 4}\right\rfloor + 1 = 10 & \left\lfloor\frac{246}{2 ^ 4}\right\rfloor + 1 = 16 & \left\lfloor\frac{200}{2 ^ 4}\right\rfloor + 1 = 13 &
2 & 35 - (10 + 16) = 9 & 13 - 1 = 12
\\
\hline
10 & 35 &
35 & 0 &
%% \left\lfloor\frac{144}{2 ^ 4}\right\rfloor + 1 = 10 & \left\lfloor\frac{218}{2 ^ 4}\right\rfloor + 1 = 14 & \left\lfloor\frac{353}{2 ^ 4}\right\rfloor + 1 = 23 &
2 & 35 - (10 + 14) = 11 & 23 - 1 = 22
\\
11 & 31 &
31 & 0 &
%% \left\lfloor\frac{158}{2 ^ 4}\right\rfloor + 1 = 10 & \left\lfloor\frac{266}{2 ^ 4}\right\rfloor + 1 = 17 & \left\lfloor\frac{186}{2 ^ 4}\right\rfloor + 1 = 12 &
2 & 31 - (10 + 17) = 4 & 12 - 1 = 11
\\
\hline
12 & 50 &
50 & 0 &
%% \left\lfloor\frac{154}{2 ^ 4}\right\rfloor + 1 = 10 & \left\lfloor\frac{238}{2 ^ 4}\right\rfloor + 1 = 15 & \left\lfloor\frac{331}{2 ^ 4}\right\rfloor + 1 = 21 &
3 & 50 - (10 + 15 + ((3 - 2) \times 21)) = 4 & 21 - 1 = 20
\\
13 & 25 &
25 & 0 &
%% \left\lfloor\frac{168}{2 ^ 4}\right\rfloor + 1 = 11 & \left\lfloor\frac{291}{2 ^ 4}\right\rfloor + 1 = 19 & \left\lfloor\frac{174}{2 ^ 4}\right\rfloor + 1 = 11 &
1 & 25 - 11 = 14 & 19 - 1 = 18
\\
\hline
14 & 62 &
62 & 0 &
%% \left\lfloor\frac{164}{2 ^ 4}\right\rfloor + 1 = 11 & \left\lfloor\frac{258}{2 ^ 4}\right\rfloor + 1 = 17 & \left\lfloor\frac{386}{2 ^ 4}\right\rfloor + 1 = 25 &
3 & 62 - (11 + 17 + ((3 - 2) \times 25)) = 9 & 25 - 1 = 24
\\
15 & 18 &
18 & 0 &
%% \left\lfloor\frac{178}{2 ^ 4}\right\rfloor + 1 = 12 & \left\lfloor\frac{281}{2 ^ 4}\right\rfloor + 1 = 18 & \left\lfloor\frac{174}{2 ^ 4}\right\rfloor + 1 = 11 &
1 & 18 - 12 = 6 & 18 - 1 = 17
\\
\hline
16 & 68 &
68 & 0 &
%% \left\lfloor\frac{174}{2 ^ 4}\right\rfloor + 1 = 11 & \left\lfloor\frac{283}{2 ^ 4}\right\rfloor + 1 = 18 & \left\lfloor\frac{451}{2 ^ 4}\right\rfloor + 1 = 29 &
3 & 68 - (11 + 18 + ((3 - 2) \times 29)) = 10 & 29 - 1 = 28
\\
17 & 10 &
10 & 0 &
%% \left\lfloor\frac{188}{2 ^ 4}\right\rfloor + 1 = 12 & \left\lfloor\frac{271}{2 ^ 4}\right\rfloor + 1 = 17 & \left\lfloor\frac{174}{2 ^ 4}\right\rfloor + 1 = 11 &
0 & 10 & 12 - 1 = 11
\\
\hline
18 & 71 &
71 & 0 &
%% \left\lfloor\frac{184}{2 ^ 4}\right\rfloor + 1 = 12 & \left\lfloor\frac{308}{2 ^ 4}\right\rfloor + 1 = 20 & \left\lfloor\frac{526}{2 ^ 4}\right\rfloor + 1 = 33 &
3 & 71 - (12 + 20 + ((3 - 2) \times 33)) = 6 & 33 - 1 = 32
\\
19 & 0 &
0 & 0 &
%% \left\lfloor\frac{184}{2 ^ 4}\right\rfloor + 1 = 12 & \left\lfloor\frac{271}{2 ^ 4}\right\rfloor + 1 = 17 & \left\lfloor\frac{174}{2 ^ 4}\right\rfloor + 1 = 11 &
0 & 0 & 12 - 1 = 11
\\
\hline
\end{tabular}
\vskip .25in
\begin{tabular}{|>{$}r<{$}||>{$}r<{$}|>{$}r<{$}|>{$}r<{$}||>{$}r<{$}|>{$}r<{$}|>{$}r<{$}|>{$}r<{$}>{$}r<{$}|}
i & m_i & \text{offset}_i & \text{add}_i & u_i & p_i & e_i & r_i & b_i \\
\hline
0 & 3 &
18 & 20 &
(3 \times 2) + 1 = 7 &
\lfloor\log_2(20)\rfloor = 4 &
2 ^ {(4 + 1)} - 20 - 1 = 11 &
(18 + 11) \div 2 = 14 &
1 \\
1 & 2 &
11 & 13 &
(2 \times 2) - 1 = 3 &
\lfloor\log_2(13)\rfloor = 3 &
2 ^ {(3 + 1)} - 13 - 1 = 2 &
(11 + 2) \div 2 = 6 &
1 \\
\hline
2 & 2 &
10 & 23 &
(2 \times 2) - 1 = 3 &
\lfloor\log_2(23)\rfloor = 4 &
2 ^ {(4 + 1)} - 23 - 1 = 8 &
(10 + 8) \div 2 = 9 &
0 \\
3 & 2 &
12 & 12 &
(2 \times 2) - 1 = 3 &
\lfloor\log_2(12)\rfloor = 3 &
2 ^ {(3 + 1)} - 12 - 1 = 3 &
(12 + 3) \div 2 = 7 &
1 \\
\hline
4 & 1 &
8 & 14 &
(1 \times 2) - 1 = 1 &
\lfloor\log_2(14)\rfloor = 3 &
2 ^ {(3 + 1)} - 14 - 1 = 1 &
(8 + 1) \div 2 = 4 &
1 \\
5 & 3 &
2 & 11 &
(3 - 1) \times 2 = 4 &
\lfloor\log_2(11)\rfloor = 3 &
2 ^ {(3 + 1)} - 11 - 1 = 4 &
2 & \\
\hline
6 & 0 &
1 & 8 &
\textit{undefined} &
\lfloor\log_2(8)\rfloor = 3 &
2 ^ {(3 + 1)} - 8 - 1 = 7 &
1 & \\
7 & 2 &
13 & 13 &
(2 \times 2) + 1 = 5 &
\lfloor\log_2(13)\rfloor = 3 &
2 ^ {(3 + 1)} - 13 - 1 = 2 &
(13 + 2) \div 2 = 7 &
1 \\
\hline
8 & 1 &
11 & 14 &
(1 \times 2) - 1 = 1 &
\lfloor\log_2(14)\rfloor = 3 &
2 ^ {(3 + 1)} - 14 - 1 = 1 &
(11 + 1) \div 2 = 6 &
0 \\
9 & 2 &
9 & 12 &
(2 \times 2) - 1 = 3 &
\lfloor\log_2(12)\rfloor = 3 &
2 ^ {(3 + 1)} - 12 - 1 = 3 &
(9 + 3) \div 2 = 6 &
0 \\
\hline
10 & 2 &
11 & 22 &
(2 \times 2) - 1 = 3 &
\lfloor\log_2(22)\rfloor = 4 &
2 ^ {(4 + 1)} - 22 - 1 = 9 &
(11 + 9) \div 2 = 10 &
0 \\
11 & 2 &
4 & 11 &
(2 \times 2) - 1 = 3 &
\lfloor\log_2(11)\rfloor = 3 &
2 ^ {(3 + 1)} - 11 - 1 = 4 &
(4 + 4) \div 2 = 4 &
0 \\
\hline
12 & 3 &
4 & 20 &
(3 \times 2) - 1 = 5 &
\lfloor\log_2(20)\rfloor = 4 &
2 ^ {(4 + 1)} - 20 - 1 = 11 &
4 & \\
13 & 1 &
14 & 18 &
(1 \times 2) - 1 = 1 &
\lfloor\log_2(18)\rfloor = 4 &
2 ^ {(4 + 1)} - 18 - 1 = 13 &
(14 + 13) \div 2 = 13 &
1 \\
\hline
14 & 3 &
9 & 24 &
(3 \times 2) - 1 = 5 &
\lfloor\log_2(24)\rfloor = 4 &
2 ^ {(4 + 1)} - 24 - 1 = 7 &
(9 + 7) \div 2 = 8 &
0 \\
15 & 1 &
6 & 17 &
(1 \times 2) - 1 = 1 &
\lfloor\log_2(17)\rfloor = 4 &
2 ^ {(4 + 1)} - 17 - 1 = 14 &
6 & \\
\hline
16 & 3 &
10 & 28 &
(3 - 1) \times 2 = 4 &
\lfloor\log_2(28)\rfloor = 4 &
2 ^ {(4 + 1)} - 28 - 1 = 3 &
(9 + 3) \div 2 = 6 &
0 \\
17 & 0 &
10 & 11 &
\textit{undefined} &
\lfloor\log_2(11)\rfloor = 3 &
2 ^ {(3 + 1)} - 11 - 1 = 4 &
(10 + 4) \div 2 = 7 &
0 \\
\hline
18 & 3 &
6 & 32 &
3 \times 2 = 6 &
\lfloor\log_2(32)\rfloor = 5 &
2 ^ {(5 + 1)} - 32 - 1 = 31 &
6 & \\
19 & 0 &
0 & 11 &
\textit{undefined} &
\lfloor\log_2(11)\rfloor = 3 &
2 ^ {(3 + 1)} - 11 - 1 = 4 &
0 & \\
\hline
\end{tabular}
}
\end{table}

\clearpage

\begin{figure}[h]
  \includegraphics[width=6in,keepaspectratio]{wavpack/figures/residuals_parse.pdf}
\end{figure}

\clearpage

\subsubsection{2nd Residual Encoding Example}
This example is more simplified to demonstrate how the \VAR{zeroes}
value propagates in two instances.
\vskip .25in
{\relsize{-2}
\renewcommand{\arraystretch}{1.75}
\begin{tabular}{|r|r|>{$}r<{$}>{$}r<{$}>{$}r<{$}||>{$}r<{$}|>{$}r<{$}>{$}r<{$}>{$}r<{$}>{$}r<{$}>{$}r<{$}|l}
  $i$ & $r_i$ & \text{entropy}_{0~0} & \text{entropy}_{0~1} & \text{entropy}_{0~2}  & u_i & \text{zeroes}_i & m_i & \text{offset}_i & \text{add}_i & \text{sign}_i \\
\cline{0-10}
-2 & & & & & \textit{und.} & & & & & \\
-1 & & {\color{red}0} & 0 & 0 & {\color{red}\textit{und.}} & \textit{und.} & \textit{und.} & \textit{und.} & \textit{und.} & \textit{und.} \\
0 & 1 & 0 & 0 & 0 & 3 & {\color{blue}0} & 1 & 0 & 0 & 0 \\
1 & 2 & 5 & 0 & 0 & 3 & \textit{und.} & 2 & 0 & 0 & 0 \\
2 & 3 & 10 & 5 & 0 & 5 & \textit{und.} & 3 & 0 & 0 & 0 \\
3 & 2 & 15 & 10 & 5 & 2 & \textit{und.} & 2 & 0 & 0 & 0 \\
4 & 1 & 20 & 15 & 3 & \textit{und.} & \textit{und.} & 0 & 1 & 1 & 0 \\
5 & 0 & 18 & 15 & 3 & 0 & \textit{und.} & 0 & 0 & 1 & 0 \\
6 & 0 & 16 & 15 & 3 & \textit{und.} & \textit{und.} & 0 & 0 & 1 & 0 \\
7 & 0 & 14 & 15 & 3 & 0 & \textit{und.} & 0 & 0 & 0 & 0 \\
8 & 0 & 12 & 15 & 3 & \textit{und.} & \textit{und.} & 0 & 0 & 0 & 0 \\
9 & 0 & 10 & 15 & 3 & 0 & \textit{und.} & 0 & 0 & 0 & 0 \\
10 & 0 & 8 & 15 & 3 & \textit{und.} & \textit{und.} & 0 & 0 & 0 & 0 \\
11 & 0 & 6 & 15 & 3 & 0 & \textit{und.} & 0 & 0 & 0 & 0 \\
12 & 0 & 4 & 15 & 3 & \textit{und.} & \textit{und.} & 0 & 0 & 0 & 0 \\
13 & 0 & 2 & 15 & 3 & 0 & \textit{und.} & 0 & 0 & 0 & 0 \\
14 & 0 & {\color{red}0} & 15 & 3 & {\color{red}\textit{und.}} & \textit{und.} & 0 & 0 & 0 & 0 \\
\cline{0-10}
15 & 0 & 0 & 15 & 3 & \textit{und.} & 1 & \textit{und.} & \textit{und.} & \textit{und.} & \textit{und.} & \multirow{10}{1em}{\begin{sideways}block of 10 zero residuals\end{sideways}} \\
16 & 0 & 0 & 0 & 0 & \textit{und.} & 2 & \textit{und.} & \textit{und.} & \textit{und.} & \textit{und.} \\
17 & 0 & 0 & 0 & 0 & \textit{und.} & 3 & \textit{und.} & \textit{und.} & \textit{und.} & \textit{und.} \\
18 & 0 & 0 & 0 & 0 & \textit{und.} & 4 & \textit{und.} & \textit{und.} & \textit{und.} & \textit{und.} \\
19 & 0 & 0 & 0 & 0 & \textit{und.} & 5 & \textit{und.} & \textit{und.} & \textit{und.} & \textit{und.} \\
20 & 0 & 0 & 0 & 0 & \textit{und.} & 6 & \textit{und.} & \textit{und.} & \textit{und.} & \textit{und.} \\
21 & 0 & 0 & 0 & 0 & \textit{und.} & 7 & \textit{und.} & \textit{und.} & \textit{und.} & \textit{und.} \\
22 & 0 & 0 & 0 & 0 & \textit{und.} & 8 & \textit{und.} & \textit{und.} & \textit{und.} & \textit{und.} \\
23 & 0 & 0 & 0 & 0 & \textit{und.} & 9 & \textit{und.} & \textit{und.} & \textit{und.} & \textit{und.} \\
24 & 0 & 0 & 0 & 0 & \textit{und.} & 10 & \textit{und.} & \textit{und.} & \textit{und.} & \textit{und.} \\
\cline{0-10}
25 & -1 & 0 & 0 & 0 & 1 & {\color{blue}10} & 0 & 0 & 0 & 1 \\
26 & -2 & 0 & 0 & 0 & 1 & \textit{und.} & 1 & 0 & 0 & 1 \\
27 & -3 & 5 & 0 & 0 & 3 & \textit{und.} & 2 & 0 & 0 & 1 \\
28 & -2 & 10 & 5 & 0 & 0 & \textit{und.} & 1 & 0 & 0 & 1 \\
29 & -1 & 15 & 3 & 0 & \textit{und.} & \textit{und.} & 0 & 0 & 0 & 1 \\
\cline{0-10}
\end{tabular}
}

\clearpage

For $r_0$, because $\text{entropy}_{0~0} = 0$ and
$u_{(-1)} = \textit{undefined}$\footnote{as determined by the \texttt{unary\_undefined} function},
we must handle a block of zeroes in some way.
But because $r_0 \neq 0$, we prepend a ``false alarm'' block of zeroes
and encode the residual normally.

For $r_{15}$, because $\text{entropy}_{0~0} = 0$,
$u_{14} = \textit{undefined}$ and $r_{15} = 0$,
we flush $\text{residual}_{14}$'s values and begin a block of zeroes
which continues until $r_{25} \neq 0$.
This block of zeroes is prepended to $\text{residual}_{25}$'s
values, which are flushed once $\text{residual}_{26}$ is encoded
and $u_{25}$ can be calculated from $u_{24}$ and $m_{26}$.

\begin{figure}[h]
  \includegraphics{wavpack/figures/residuals_parse2.pdf}
\end{figure}
