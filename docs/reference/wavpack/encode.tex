%Copyright (C) 2007-2014  Brian Langenberger
%This work is licensed under the
%Creative Commons Attribution-Share Alike 3.0 United States License.
%To view a copy of this license, visit
%http://creativecommons.org/licenses/by-sa/3.0/us/ or send a letter to
%Creative Commons,
%171 Second Street, Suite 300,
%San Francisco, California, 94105, USA.

\section{WavPack Encoding}

{\relsize{-1}
  \input{wavpack/algorithms/encode_wavpack}
}

\clearpage

\subsection{Determine Block Split}
\label{wavpack:block_split}
\ALGORITHM{input stream's channel assignment}{number of blocks per set, list of channel counts per block}
\SetKwData{BLOCKCOUNT}{block count}
\SetKwData{BLOCKCHANNELS}{block channels}
\Switch(\tcc*[f]{split channels by left/right pairs}){channel assignment}{
  \uCase{mono}{
    $\text{\BLOCKCOUNT} \leftarrow 1$\;
    $\text{\BLOCKCHANNELS} \leftarrow \texttt{[1]}$\;
  }
  \uCase{front left, front right}{
    $\text{\BLOCKCOUNT} \leftarrow 1$\;
    $\text{\BLOCKCHANNELS} \leftarrow \texttt{[2]}$\;
  }
  \uCase{front left, front right, front center}{
    $\text{\BLOCKCOUNT} \leftarrow 2$\;
    $\text{\BLOCKCHANNELS} \leftarrow \texttt{[2, 1]}$\;
  }
  \uCase{front left, front right, back left, back right}{
    $\text{\BLOCKCOUNT} \leftarrow 2$\;
    $\text{\BLOCKCHANNELS} \leftarrow \texttt{[2, 2]}$\;
  }
  \uCase{front left, front right, front center, back center}{
    $\text{\BLOCKCOUNT} \leftarrow 3$\;
    $\text{\BLOCKCHANNELS} \leftarrow \texttt{[2, 1, 1]}$\;
  }
  \uCase{front left, front right, front center, back left, back right}{
    $\text{\BLOCKCOUNT} \leftarrow 3$\;
    $\text{\BLOCKCHANNELS} \leftarrow \texttt{[2, 1, 2]}$\;
  }
  \uCase{front left, front right, front center, LFE, back left, back right}{
    $\text{\BLOCKCOUNT} \leftarrow 4$\;
    $\text{\BLOCKCHANNELS} \leftarrow \texttt{[2, 1, 1, 2]}$\;
  }
  \Other(\tcc*[f]{save them independently}){
    $\text{\BLOCKCOUNT} \leftarrow$ channel count\;
    $\text{\BLOCKCHANNELS} \leftarrow$ 1 per channel\;
  }
}
\Return \BLOCKCOUNT and \BLOCKCHANNELS
\EALGORITHM
\vskip 1ex
\par
\noindent
One could invent alternate channel splits for other obscure assignments.
WavPack's only requirement is that all channels must be in
Wave order\footnote{see page \pageref{wave_channel_assignment}}
and each block must contain 1 or 2 channels.

\begin{figure}[h]
\includegraphics{wavpack/figures/block_channels.pdf}
\end{figure}

\clearpage

\subsection{Encoding Parameters}
\label{wavpack:encoding_parameters}
{\relsize{-1}
  \input{wavpack/algorithms/encoding_parameters}
}

\clearpage

\subsection{Encoding Block}
\label{wavpack:encode_block}
{\relsize{-2}
  \input{wavpack/algorithms/encode_block}
}

\clearpage

\subsection{Calculating Maximum Magnitude}
\label{wavpack:calc_maximum_magnitude}
{\relsize{-1}
  \input{wavpack/algorithms/calculate_max_magnitude}
}

\subsection{Calculating Wasted Bits Per Sample}
\label{wavpack:calc_wasted_bps}
{\relsize{-1}
  \input{wavpack/algorithms/calculate_wasted_bps}
where the \texttt{wasted} function is defined as:
\begin{equation*}
\texttt{wasted}(x) =
\begin{cases}
\infty & \text{if } x = 0 \\
0 & \text{if } x \bmod 2 = 1 \\
1 + \texttt{wasted}(x \div 2) & \text{if } x \bmod 2 = 0 \\
\end{cases}
\end{equation*}
}

\clearpage

\subsection{Joint Stereo Conversion}
\label{wavpack:calc_joint_stereo}
\input{wavpack/algorithms/apply_joint_stereo}

\subsubsection{Joint Stereo Example}
\begin{table}[h]
{\relsize{-1}
\begin{tabular}{|r|r|r||>{$}r<{$}|>{$}r<{$}|}
$i$ & $\textsf{left}_i$ & $\textsf{right}_i$ & \textsf{mid}_i & \textsf{side}_i \\
\hline
0 & 0 & 64 & 0 - 64 = -64 & \lfloor(0 + 64) \div 2\rfloor = 32 \\
1 & 16 & 62 & 16 - 62 = -46 & \lfloor(16 + 62) \div 2\rfloor = 39 \\
2 & 31 & 56 & 31 - 56 = -25 & \lfloor(31 + 56) \div 2\rfloor = 43 \\
3 & 44 & 47 & 44 - 47 = -3 & \lfloor(44 + 47) \div 2\rfloor = 45 \\
4 & 54 & 34 & 54 - 34 = 20 & \lfloor(54 + 34) \div 2\rfloor = 44 \\
5 & 61 & 20 & 61 - 20 = 41 & \lfloor(61 + 20) \div 2\rfloor = 40 \\
6 & 64 & 4 & 64 - 4 = 60 & \lfloor(64 + 4) \div 2\rfloor = 34 \\
7 & 63 & -12 & 63 - -12 = 75 & \lfloor(63 + -12) \div 2\rfloor = 25 \\
8 & 58 & -27 & 58 - -27 = 85 & \lfloor(58 + -27) \div 2\rfloor = 15 \\
9 & 49 & -41 & 49 - -41 = 90 & \lfloor(49 + -41) \div 2\rfloor = 4 \\
\end{tabular}
}
\end{table}

\clearpage

\subsection{Writing Block Parameters}
\label{wavpack:write_block_parameters}
{\relsize{-1}
  \input{wavpack/algorithms/write_block_parameters}
}

\clearpage

\subsection{Writing Sub Block Header}
\label{wavpack:write_sub_block_header}
\input{wavpack/algorithms/write_sub_block_header}

\clearpage

%This work is licensed under the
%Creative Commons Attribution-Share Alike 3.0 United States License.
%To view a copy of this license, visit
%http://creativecommons.org/licenses/by-sa/3.0/us/ or send a letter to
%Creative Commons,
%171 Second Street, Suite 300,
%San Francisco, California, 94105, USA.

\subsection{Decode Decorrelation Terms}
\label{wavpack:decode_decorrelation_terms}
\ALGORITHM{\VAR{actual size 1 less} and \VAR{sub block size} values from sub block header, sub block data}{a list of signed decorrelation term integers, a list of unsigned decorrelation delta integers\footnote{$\text{term}_p$ and $\text{delta}_p$ indicate the term and delta values for decorrelation pass $p$}}
\SetKwData{PASSES}{passes}
\SetKwData{SUBBLOCKSIZE}{sub block size}
\SetKwData{ACTUALSIZEONELESS}{actual size 1 less}
\SetKwData{TERM}{term}
\SetKwData{DELTA}{delta}
\SetKw{OR}{or}
\SetKw{KwDownTo}{downto}
\eIf{$\text{\ACTUALSIZEONELESS} = 0$}{
  \PASSES $\leftarrow \text{\SUBBLOCKSIZE} \times 2$\;
}{
  \PASSES $\leftarrow \text{\SUBBLOCKSIZE} \times 2 - 1$\;
}
\ASSERT $\text{\PASSES} \leq 16$\;
\BlankLine
\For(\tcc*[f]{populate in reverse order}){$p \leftarrow \PASSES$ \emph{\KwDownTo}0}{
  $\text{\TERM}_p \leftarrow$ (\READ 5 unsigned bits) - 5\;
  \ASSERT $\text{\TERM}_p$ \IN \texttt{[-3, -2, -1, 1, 2, 3, 4, 5, 6, 7, 8, 17, 18]}
  \BlankLine
  $\text{\DELTA}_p \leftarrow$ \READ 3 unsigned bits\;
}
\Return $\left\lbrace\begin{tabular}{l}
decorrelation \TERM \\
decorrelation \DELTA \\
\end{tabular}\right.$\;
\EALGORITHM

\begin{figure}[h]
  \includegraphics{wavpack/figures/decorr_terms.pdf}
\end{figure}

\clearpage

\subsubsection{Reading Decorrelation Terms Example}

\begin{figure}[h]
\includegraphics{wavpack/figures/terms_parse.pdf}
\end{figure}
\begin{center}
{\renewcommand{\arraystretch}{1.25}
\begin{tabular}{>{$}r<{$}>{$}c<{$}>{$}r<{$}|>{$}r<{$}>{$}r<{$}>{$}r<{$}}
\text{decorrelation term}_4 & \leftarrow & 23 - 5 = 18 &
\text{decorrelation delta}_4 & \leftarrow & 2 \\
\text{decorrelation term}_3 & \leftarrow & 23 - 5 = 18 &
\text{decorrelation delta}_3 & \leftarrow & 2 \\
\text{decorrelation term}_2 & \leftarrow & 7 - 5 = 2 &
\text{decorrelation delta}_2 & \leftarrow & 2 \\
\text{decorrelation term}_1 & \leftarrow & 22 - 5 = 17 &
\text{decorrelation delta}_1 & \leftarrow & 2 \\
\text{decorrelation term}_0 & \leftarrow & 8 - 5 = 3 &
\text{decorrelation delta}_0 & \leftarrow & 2 \\
\end{tabular}
\renewcommand{\arraystretch}{1.0}
}
\end{center}


\clearpage

%This work is licensed under the
%Creative Commons Attribution-Share Alike 3.0 United States License.
%To view a copy of this license, visit
%http://creativecommons.org/licenses/by-sa/3.0/us/ or send a letter to
%Creative Commons,
%171 Second Street, Suite 300,
%San Francisco, California, 94105, USA.

\subsection{Writing Decorrelation Weights}
\label{wavpack:write_decorr_weights}
\input{wavpack/algorithms/write_decorrelation_weights}
\begin{figure}[h]
  \includegraphics{wavpack/figures/decorr_weights.pdf}
\end{figure}
\clearpage
For example, given the decorrelation weight values:
\begin{table}[h]
\begin{tabular}{rrrrr}
$p$ & $\textsf{weight}_{p~0}$ & $\textsf{weight}_{p~1}$ &
$\texttt{store\_weight}(\textsf{weight}_{p~0})$ &
$\texttt{store\_weight}(\textsf{weight}_{p~1})$ \\
\hline
0 & 16 & 24 & 2 & 3 \\
1 & 48 & 48 & 6 & 6 \\
2 & 32 & 32 & 4 & 4 \\
3 & 48 & 48 & 6 & 6 \\
4 & 48 & 48 & 6 & 6 \\
\end{tabular}
\end{table}
\par
\noindent
the decorrelation weights subframe is written as:
\begin{figure}[h]
\includegraphics{wavpack/figures/decorr_weights_parse.pdf}
\end{figure}


\clearpage

%Copyright (C) 2007-2015  Brian Langenberger
%This work is licensed under the
%Creative Commons Attribution-Share Alike 3.0 United States License.
%To view a copy of this license, visit
%http://creativecommons.org/licenses/by-sa/3.0/us/ or send a letter to
%Creative Commons,
%171 Second Street, Suite 300,
%San Francisco, California, 94105, USA.

\subsection{Reading Decorrelation Samples}
\label{wavpack:read_decorrelation_samples}
{\relsize{-1}
  \input{wavpack/algorithms/read_decorrelation_samples}
}
\begin{figure}[h]
  \includegraphics{wavpack/figures/decorr_samples.pdf}
\end{figure}

\clearpage

\subsubsection{Reading exp2 Values}
\label{wavpack:exp2}
{\relsize{-1}
  \input{wavpack/algorithms/decode_wv_exp2}
}
\par
\noindent
where \texttt{wexp}(\textit{x}) is defined from the following table:
\vskip .10in
\par
\noindent
{\relsize{-3}\ttfamily
\begin{tabular}{| c | c | c | c | c | c | c | c | c | c | c | c | c | c | c | c | c |}
\hline
& 0x?0 & 0x?1 & 0x?2 & 0x?3 & 0x?4 & 0x?5 & 0x?6 & 0x?7 & 0x?8 & 0x?9 & 0x?A & 0x?B & 0x?C & 0x?D & 0x?E & 0x?F \\
\hline
0x0? & 256 & 257 & 257 & 258 & 259 & 259 & 260 & 261 & 262 & 262 & 263 & 264 & 264 & 265 & 266 & 267 \\
0x1? & 267 & 268 & 269 & 270 & 270 & 271 & 272 & 272 & 273 & 274 & 275 & 275 & 276 & 277 & 278 & 278 \\
0x2? & 279 & 280 & 281 & 281 & 282 & 283 & 284 & 285 & 285 & 286 & 287 & 288 & 288 & 289 & 290 & 291 \\
0x3? & 292 & 292 & 293 & 294 & 295 & 296 & 296 & 297 & 298 & 299 & 300 & 300 & 301 & 302 & 303 & 304 \\
0x4? & 304 & 305 & 306 & 307 & 308 & 309 & 309 & 310 & 311 & 312 & 313 & 314 & 314 & 315 & 316 & 317 \\
0x5? & 318 & 319 & 320 & 321 & 321 & 322 & 323 & 324 & 325 & 326 & 327 & 328 & 328 & 329 & 330 & 331 \\
0x6? & 332 & 333 & 334 & 335 & 336 & 337 & 337 & 338 & 339 & 340 & 341 & 342 & 343 & 344 & 345 & 346 \\
0x7? & 347 & 348 & 349 & 350 & 350 & 351 & 352 & 353 & 354 & 355 & 356 & 357 & 358 & 359 & 360 & 361 \\
0x8? & 362 & 363 & 364 & 365 & 366 & 367 & 368 & 369 & 370 & 371 & 372 & 373 & 374 & 375 & 376 & 377 \\
0x9? & 378 & 379 & 380 & 381 & 382 & 383 & 384 & 385 & 386 & 387 & 388 & 389 & 391 & 392 & 393 & 394 \\
0xA? & 395 & 396 & 397 & 398 & 399 & 400 & 401 & 402 & 403 & 405 & 406 & 407 & 408 & 409 & 410 & 411 \\
0xB? & 412 & 413 & 415 & 416 & 417 & 418 & 419 & 420 & 421 & 422 & 424 & 425 & 426 & 427 & 428 & 429 \\
0xC? & 431 & 432 & 433 & 434 & 435 & 436 & 438 & 439 & 440 & 441 & 442 & 444 & 445 & 446 & 447 & 448 \\
0xD? & 450 & 451 & 452 & 453 & 454 & 456 & 457 & 458 & 459 & 461 & 462 & 463 & 464 & 466 & 467 & 468 \\
0xE? & 470 & 471 & 472 & 473 & 475 & 476 & 477 & 478 & 480 & 481 & 482 & 484 & 485 & 486 & 488 & 489 \\
0xF? & 490 & 492 & 493 & 494 & 496 & 497 & 498 & 500 & 501 & 502 & 504 & 505 & 506 & 508 & 509 & 511 \\
\hline
\end{tabular}
}

\clearpage

\subsubsection{Reading Decorrelation Samples Example}
Given a stereo block containing the sub-block:
\begin{figure}[h]
\includegraphics{wavpack/figures/decorr_samples_parse.pdf}
\end{figure}
\begin{center}
{\relsize{-2}
  \SetKwFunction{EXP}{wexp}
\begin{tabular}{r|r|r|>{$}r<{$}|>{$}r<{$}}
$p$ & $\text{term}_p$ & $s$ &
\text{sample}_{p~0~s} &
\text{sample}_{p~1~s} \\
\hline
4 & 18 & 0 &
-\lfloor \texttt{wexp}(1841 \bmod{256}) \div 2 ^ {9 - \lfloor 1841 \div 2 ^ 8 \rfloor} \rfloor = -73 &
\lfloor \EXP(1487 \bmod{256}) \div 2 ^ {9 - \lfloor 1487 \div 2 ^ 8 \rfloor} \rfloor = 28 \\
& & 1 &
-\lfloor \EXP(1865 \bmod{256}) \div 2 ^ {9 - \lfloor 1865 \div 2 ^ 8 \rfloor} \rfloor = -78 &
\lfloor \EXP(1459 \bmod{256}) \div 2 ^ {9 - \lfloor 1459 \div 2 ^ 8 \rfloor} \rfloor = 26 \\
\hline
3 & 18 & 0 & 0 & 0 \\
& & 1 & 0 & 0 \\
\hline
2 & 2 & 0 & 0 & 0 \\
& & 1 & 0 & 0 \\
\hline
1 & 17 & 0 & 0 & 0 \\
& & 1 & 0 & 0 \\
\hline
0 & 3 & 0 & 0 & 0 \\
& & 1 & 0 & 0 \\
& & 2 & 0 & 0 \\
\hline
\end{tabular}
\renewcommand{\arraystretch}{1.0}
}
\end{center}


\clearpage

%Copyright (C) 2007-2014  Brian Langenberger
%This work is licensed under the
%Creative Commons Attribution-Share Alike 3.0 United States License.
%To view a copy of this license, visit
%http://creativecommons.org/licenses/by-sa/3.0/us/ or send a letter to
%Creative Commons,
%171 Second Street, Suite 300,
%San Francisco, California, 94105, USA.

\subsection{Writing Entropy Variables}
\label{wavpack:write_entropy}
\input{wavpack/algorithms/write_entropy_variables}

\begin{figure}[h]
  \includegraphics{wavpack/figures/entropy_vars.pdf}
\end{figure}

\clearpage

\subsubsection{Writing Entropy Variables Example}

\begin{table}[h]
{\relsize{-2}
  \renewcommand{\arraystretch}{1.5}
\begin{tabular}{r|>{$}r<{$}|>{$}r<{$}|>{$}r<{$}}
  $\text{entropy}_{c~i}$ & $a$ & $c$ & \text{value}_{c~i} \\
  \hline
  118 &
  |118| + \lfloor |118| \div 2 ^ 9\rfloor = 118 &
  \lfloor\log_2(118)\rfloor + 1 = 7 &
  7 \times 2 ^ 8 + \texttt{wlog}((118 \times 2 ^ {9 - 7}) \bmod 256) = 2018 \\
  194 &
  |194| + \lfloor |194| \div 2 ^ 9\rfloor = 194 &
  \lfloor\log_2(194)\rfloor + 1 = 8 &
  8 \times 2 ^ 8 + \texttt{wlog}((194 \times 2 ^ {9 - 8}) \bmod 256) = 2202 \\
  322 &
  |322| + \lfloor |322| \div 2 ^ 9\rfloor = 322 &
  \lfloor\log_2(322)\rfloor + 1 = 9 &
  9 \times 2 ^ 8 + \LOG(\lfloor 322 \div 2 ^ {9 - 9}\rfloor \bmod 256) = 2389 \\
  \hline
  118 &
  |118| + \lfloor |118| \div 2 ^ 9\rfloor = 118 &
  \lfloor\log_2(118)\rfloor + 1 = 7 &
  7 \times 2 ^ 8 + \texttt{wlog}((118 \times 2 ^ {9 - 7}) \bmod 256) = 2018 \\
  176 &
  |176| + \lfloor |176| \div 2 ^ 9\rfloor = 176 &
  \lfloor\log_2(176)\rfloor + 1 = 8 &
  8 \times 2 ^ 8 + \texttt{wlog}((176 \times 2 ^ {9 - 8}) \bmod 256) = 2166 \\
  212 &
  |212| + \lfloor |212| \div 2 ^ 9\rfloor = 212 &
  \lfloor\log_2(212)\rfloor + 1 = 8 &
  8 \times 2 ^ 8 + \texttt{wlog}((212 \times 2 ^ {9 - 8}) \bmod 256) = 2234 \\
\end{tabular}
}
\end{table}
\begin{figure}[h]
  \includegraphics{wavpack/figures/entropy_vars_parse.pdf}
\end{figure}


\clearpage

%Copyright (C) 2007-2014  Brian Langenberger
%This work is licensed under the
%Creative Commons Attribution-Share Alike 3.0 United States License.
%To view a copy of this license, visit
%http://creativecommons.org/licenses/by-sa/3.0/us/ or send a letter to
%Creative Commons,
%171 Second Street, Suite 300,
%San Francisco, California, 94105, USA.

\subsection{Channel Correlation}
\label{wavpack:correlate_channels}
{\relsize{-1}
  \input{wavpack/algorithms/correlate_channels}
}

\clearpage

\subsection{1 Channel Correlation Pass}
\label{wavpack:correlate_1ch}
{\relsize{-1}
  \input{wavpack/algorithms/correlate_1ch}
}

\clearpage

\subsection{2 Channel Correlation Passes}
\label{wavpack:correlate_2ch}
{\relsize{-1}
  \input{wavpack/algorithms/correlate_2ch}
}

\clearpage

\subsubsection{2 Channel Correlation Pass, Term -1}
\label{wavpack:correlate_2ch_1}
{\relsize{-1}
  \input{wavpack/algorithms/correlate_2ch_1}
}

\subsubsection{2 Channel Correlation Pass, Term -2}
\label{wavpack:correlate_2ch_2}
{\relsize{-1}
  \input{wavpack/algorithms/correlate_2ch_2}
}

\clearpage

\subsubsection{2 Channel Correlation Pass, Term -3}
\label{wavpack:correlate_2ch_3}
{\relsize{-1}
  \input{wavpack/algorithms/correlate_2ch_3}
}

\clearpage

\subsection{Channel Correlation Example}
\begin{figure}[h]
{\relsize{-1}
  \subfloat{
    \begin{tabular}{|r|r|r|}
      \multicolumn{3}{c}{Correlation Terms} \\
      \hline
      $p$ & $\textsf{term}_p$ & $\textsf{delta}_p$ \\
      \hline
      0 & 3 & 2 \\
      1 & 17 & 2 \\
      2 & 2 & 2 \\
      3 & 18 & 2 \\
      4 & 18 & 2 \\
      \hline
    \end{tabular}
  }
  \subfloat{
    \begin{tabular}{|r|r|r|}
      \multicolumn{3}{c}{Correlation Weights} \\
      \hline
      $p$ & $\textsf{weight}_{p~0}$ & $\textsf{weight}_{p~1}$ \\
      \hline
      0 & 16 & 24 \\
      1 & 48 & 48 \\
      2 & 32 & 32 \\
      3 & 48 & 48 \\
      4 & 48 & 48 \\
      \hline
    \end{tabular}
  }
  \subfloat{
    \begin{tabular}{|r|r|r|}
      \multicolumn{3}{c}{Correlation Samples} \\
      \hline
      $p$ & $\textsf{sample}_{p~0~s}$ & $\textsf{sample}_{p~1~s}$ \\
      \hline
      0 & \texttt{[0, 0, 0]} & \texttt{[0, 0, 0]} \\
      1 & \texttt{[0, 0]} & \texttt{[0, 0]} \\
      2 & \texttt{[0, 0]} & \texttt{[0, 0]} \\
      3 & \texttt{[0, 0]} & \texttt{[0, 0]} \\
      4 & \texttt{[-73, -78]} & \texttt{[28, 26]} \\
      \hline
    \end{tabular}
  }
}
\end{figure}
\par
\noindent
we combine them into a single set of arguments for each correlation pass:
\begin{table}[h]
{\relsize{-1}
  \begin{tabular}{|r|r|r|r|r|r|}
    \hline
    & $\textbf{pass}_0$ & $\textbf{pass}_1$ & $\textbf{pass}_2$ &
    $\textbf{pass}_3$ & $\textbf{pass}_3$ \\
    \hline
    $\textsf{term}_p$ & 18 & 18 & 2 & 17 & 3 \\
    $\textsf{delta}_p$ & 2 & 2 & 2 & 2 & 2 \\
    $\textsf{weight}_{p~0}$ & 48 & 48 & 32 & 48 & 16 \\
    $\textsf{sample}_{p~0~s}$ & \texttt{[-73, -78]} & \texttt{[0, 0]} &
    \texttt{[0, 0]} & \texttt{[0, 0]} & \texttt{[0, 0, 0]} \\
    $\textsf{weight}_{p~1}$ & 48 & 48 & 32 & 48 & 24 \\
    $\textsf{sample}_{p~1~s}$ & \texttt{[28, 26]} & \texttt{[0, 0]} &
    \texttt{[0, 0]} & \texttt{[0, 0]} & \texttt{[0, 0, 0]} \\
    \hline
  \end{tabular}
}
\end{table}
\par
\noindent
which we apply to the residuals from the bitstream sub-block:
\par
\noindent
{\relsize{-1}
  \begin{tabular}{|r|r|r|r|r|r|}
    \hline
    $\textsf{channel}_{0~i}$ &
    after $\textbf{pass}_0$ &
    after $\textbf{pass}_1$ &
    after $\textbf{pass}_2$ &
    after $\textbf{pass}_3$ &
    after $\textbf{pass}_4$ \\
    \hline
    -64 & -61 & -61 & -61 & -61 & -61 \\
    -46 & -43 & -39 & -39 & -33 & -33 \\
    -25 & -23 & -21 & -19 & -18 & -18 \\
    -3 & -2 & -1 & 0 & 0 & 1 \\
    20 & 20 & 20 & 21 & 20 & 20 \\
    41 & 39 & 37 & 37 & 35 & 35 \\
    60 & 57 & 54 & 53 & 50 & 50 \\
    75 & 71 & 67 & 66 & 62 & 62 \\
    85 & 80 & 75 & 73 & 68 & 68 \\
    90 & 84 & 79 & 77 & 72 & 71 \\
    \hline
    \hline
    $\textsf{channel}_{1~i}$ &
    after $\textbf{pass}_0$ &
    after $\textbf{pass}_1$ &
    after $\textbf{pass}_2$ &
    after $\textbf{pass}_3$ &
    after $\textbf{pass}_4$ \\
    \hline
    32 & 31 & 31 & 31 & 31 & 31 \\
    39 & 37 & 35 & 35 & 32 & 32 \\
    43 & 41 & 39 & 38 & 36 & 36 \\
    45 & 43 & 41 & 40 & 38 & 37 \\
    44 & 41 & 39 & 38 & 36 & 35 \\
    40 & 38 & 36 & 34 & 32 & 31 \\
    34 & 32 & 30 & 28 & 26 & 25 \\
    25 & 23 & 21 & 20 & 19 & 18 \\
    15 & 14 & 13 & 12 & 11 & 10 \\
    4 & 3 & 2 & 1 & 1 & 0 \\
    \hline
  \end{tabular}
}
\par
\noindent
Resulting in final correlated samples:
\newline
\begin{tabular}{rr}
$\textsf{residual}_0$ : & \texttt{[-61,~-33,~-18,~~1,~20,~35,~50,~62,~68,~71]} \\
$\textsf{residual}_1$ : & \texttt{[~31,~~32,~~36,~37,~35,~31,~25,~18,~10,~~0]} \\
\end{tabular}

\clearpage

{\relsize{-2}
\begin{tabular}{r||r|>{$}r<{$}|>{$}r<{$}|>{$}r<{$}|>{$}r<{$}}
& $i$ & \textsf{uncorrelated}_i & \textsf{temp}_i & \textsf{correlated}_i & \textsf{weight}_{i + 1} \\
\hline
%%START
\multirow{10}{1em}{\begin{sideways}$\textbf{pass}_0$ - term 18\end{sideways}}
& 0 & -64 &
\lfloor(3 \times -73 + 78) \div 2\rfloor = -71 &
-64 - \lfloor(48 \times -71 + 2 ^ 9) \div 2 ^ {10}\rfloor = -61 &
48 + 2 = 50
\\
& 1 & -46 &
\lfloor(3 \times -64 + 73) \div 2\rfloor = -60 &
-46 - \lfloor(50 \times -60 + 2 ^ 9) \div 2 ^ {10}\rfloor = -43 &
50 + 2 = 52
\\
& 2 & -25 &
\lfloor(3 \times -46 + 64) \div 2\rfloor = -37 &
-25 - \lfloor(52 \times -37 + 2 ^ 9) \div 2 ^ {10}\rfloor = -23 &
52 + 2 = 54
\\
& 3 & -3 &
\lfloor(3 \times -25 + 46) \div 2\rfloor = -15 &
-3 - \lfloor(54 \times -15 + 2 ^ 9) \div 2 ^ {10}\rfloor = -2 &
54 + 2 = 56
\\
& 4 & 20 &
\lfloor(3 \times -3 + 25) \div 2\rfloor = 8 &
20 - \lfloor(56 \times 8 + 2 ^ 9) \div 2 ^ {10}\rfloor = 20 &
56 + 2 = 58
\\
& 5 & 41 &
\lfloor(3 \times 20 + 3) \div 2\rfloor = 31 &
41 - \lfloor(58 \times 31 + 2 ^ 9) \div 2 ^ {10}\rfloor = 39 &
58 + 2 = 60
\\
& 6 & 60 &
\lfloor(3 \times 41 - 20) \div 2\rfloor = 51 &
60 - \lfloor(60 \times 51 + 2 ^ 9) \div 2 ^ {10}\rfloor = 57 &
60 + 2 = 62
\\
& 7 & 75 &
\lfloor(3 \times 60 - 41) \div 2\rfloor = 69 &
75 - \lfloor(62 \times 69 + 2 ^ 9) \div 2 ^ {10}\rfloor = 71 &
62 + 2 = 64
\\
& 8 & 85 &
\lfloor(3 \times 75 - 60) \div 2\rfloor = 82 &
85 - \lfloor(64 \times 82 + 2 ^ 9) \div 2 ^ {10}\rfloor = 80 &
64 + 2 = 66
\\
& 9 & 90 &
\lfloor(3 \times 85 - 75) \div 2\rfloor = 90 &
90 - \lfloor(66 \times 90 + 2 ^ 9) \div 2 ^ {10}\rfloor = 84 &
66 + 2 = 68
\\
\hline
\hline
\multirow{10}{1em}{\begin{sideways}$\textbf{pass}_1$ - term 18\end{sideways}}
& 0 & -61 &
\lfloor(3 \times 0 - 0) \div 2\rfloor = 0 &
-61 - \lfloor(48 \times 0 + 2 ^ 9) \div 2 ^ {10}\rfloor = -61 &
48 + 0 = 48
\\
& 1 & -43 &
\lfloor(3 \times -61 - 0) \div 2\rfloor = -92 &
-43 - \lfloor(48 \times -92 + 2 ^ 9) \div 2 ^ {10}\rfloor = -39 &
48 + 2 = 50
\\
& 2 & -23 &
\lfloor(3 \times -43 + 61) \div 2\rfloor = -34 &
-23 - \lfloor(50 \times -34 + 2 ^ 9) \div 2 ^ {10}\rfloor = -21 &
50 + 2 = 52
\\
& 3 & -2 &
\lfloor(3 \times -23 + 43) \div 2\rfloor = -13 &
-2 - \lfloor(52 \times -13 + 2 ^ 9) \div 2 ^ {10}\rfloor = -1 &
52 + 2 = 54
\\
& 4 & 20 &
\lfloor(3 \times -2 + 23) \div 2\rfloor = 8 &
20 - \lfloor(54 \times 8 + 2 ^ 9) \div 2 ^ {10}\rfloor = 20 &
54 + 2 = 56
\\
& 5 & 39 &
\lfloor(3 \times 20 + 2) \div 2\rfloor = 31 &
39 - \lfloor(56 \times 31 + 2 ^ 9) \div 2 ^ {10}\rfloor = 37 &
56 + 2 = 58
\\
& 6 & 57 &
\lfloor(3 \times 39 - 20) \div 2\rfloor = 48 &
57 - \lfloor(58 \times 48 + 2 ^ 9) \div 2 ^ {10}\rfloor = 54 &
58 + 2 = 60
\\
& 7 & 71 &
\lfloor(3 \times 57 - 39) \div 2\rfloor = 66 &
71 - \lfloor(60 \times 66 + 2 ^ 9) \div 2 ^ {10}\rfloor = 67 &
60 + 2 = 62
\\
& 8 & 80 &
\lfloor(3 \times 71 - 57) \div 2\rfloor = 78 &
80 - \lfloor(62 \times 78 + 2 ^ 9) \div 2 ^ {10}\rfloor = 75 &
62 + 2 = 64
\\
& 9 & 84 &
\lfloor(3 \times 80 - 71) \div 2\rfloor = 84 &
84 - \lfloor(64 \times 84 + 2 ^ 9) \div 2 ^ {10}\rfloor = 79 &
64 + 2 = 66
\\
\hline
\hline
\multirow{10}{1em}{\begin{sideways}$\textbf{pass}_2$ - term 2\end{sideways}}
& 0 & -61 & &
-61 - \lfloor(32 \times 0 + 2 ^ 9) \div 2 ^ {10}\rfloor = -61 &
32 + 0 = 32
\\
& 1 & -39 & &
-39 - \lfloor(32 \times 0 + 2 ^ 9) \div 2 ^ {10}\rfloor = -39 &
32 + 0 = 32
\\
& 2 & -21 & &
-21 - \lfloor(32 \times -61 + 2 ^ 9) \div 2 ^ {10}\rfloor = -19 &
32 + 2 = 34
\\
& 3 & -1 & &
-1 - \lfloor(34 \times -39 + 2 ^ 9) \div 2 ^ {10}\rfloor = 0 &
34 + 0 = 34
\\
& 4 & 20 & &
20 - \lfloor(34 \times -21 + 2 ^ 9) \div 2 ^ {10}\rfloor = 21 &
34 - 2 = 32
\\
& 5 & 37 & &
37 - \lfloor(32 \times -1 + 2 ^ 9) \div 2 ^ {10}\rfloor = 37 &
32 - 2 = 30
\\
& 6 & 54 & &
54 - \lfloor(30 \times 20 + 2 ^ 9) \div 2 ^ {10}\rfloor = 53 &
30 + 2 = 32
\\
& 7 & 67 & &
67 - \lfloor(32 \times 37 + 2 ^ 9) \div 2 ^ {10}\rfloor = 66 &
32 + 2 = 34
\\
& 8 & 75 & &
75 - \lfloor(34 \times 54 + 2 ^ 9) \div 2 ^ {10}\rfloor = 73 &
34 + 2 = 36
\\
& 9 & 79 & &
79 - \lfloor(36 \times 67 + 2 ^ 9) \div 2 ^ {10}\rfloor = 77 &
36 + 2 = 38
\\
\hline
\hline
\multirow{10}{1em}{\begin{sideways}$\textbf{pass}_3$ - term 17\end{sideways}}
& 0 & -61 &
2 \times 0 - 0 = 0 &
-61 - \lfloor(48 \times 0 + 2 ^ 9) \div 2 ^ {10}\rfloor = -61 &
48 + 0 = 48
\\
& 1 & -39 &
2 \times -61 - 0 = -122 &
-39 - \lfloor(48 \times -122 + 2 ^ 9) \div 2 ^ {10}\rfloor = -33 &
48 + 2 = 50
\\
& 2 & -19 &
2 \times -39 + 61 = -17 &
-19 - \lfloor(50 \times -17 + 2 ^ 9) \div 2 ^ {10}\rfloor = -18 &
50 + 2 = 52
\\
& 3 & 0 &
2 \times -19 + 39 = 1 &
0 - \lfloor(52 \times 1 + 2 ^ 9) \div 2 ^ {10}\rfloor = 0 &
52 + 0 = 52
\\
& 4 & 21 &
2 \times 0 + 19 = 19 &
21 - \lfloor(52 \times 19 + 2 ^ 9) \div 2 ^ {10}\rfloor = 20 &
52 + 2 = 54
\\
& 5 & 37 &
2 \times 21 - 0 = 42 &
37 - \lfloor(54 \times 42 + 2 ^ 9) \div 2 ^ {10}\rfloor = 35 &
54 + 2 = 56
\\
& 6 & 53 &
2 \times 37 - 21 = 53 &
53 - \lfloor(56 \times 53 + 2 ^ 9) \div 2 ^ {10}\rfloor = 50 &
56 + 2 = 58
\\
& 7 & 66 &
2 \times 53 - 37 = 69 &
66 - \lfloor(58 \times 69 + 2 ^ 9) \div 2 ^ {10}\rfloor = 62 &
58 + 2 = 60
\\
& 8 & 73 &
2 \times 66 - 53 = 79 &
73 - \lfloor(60 \times 79 + 2 ^ 9) \div 2 ^ {10}\rfloor = 68 &
60 + 2 = 62
\\
& 9 & 77 &
2 \times 73 - 66 = 80 &
77 - \lfloor(62 \times 80 + 2 ^ 9) \div 2 ^ {10}\rfloor = 72 &
62 + 2 = 64
\\
\hline
\hline
\multirow{10}{1em}{\begin{sideways}$\textbf{pass}_4$ - term 3\end{sideways}}
& 0 & -61 & &
-61 - \lfloor(16 \times 0 + 2 ^ 9) \div 2 ^ {10}\rfloor = -61 &
16 + 0 = 16
\\
& 1 & -33 & &
-33 - \lfloor(16 \times 0 + 2 ^ 9) \div 2 ^ {10}\rfloor = -33 &
16 + 0 = 16
\\
& 2 & -18 & &
-18 - \lfloor(16 \times 0 + 2 ^ 9) \div 2 ^ {10}\rfloor = -18 &
16 + 0 = 16
\\
& 3 & 0 & &
0 - \lfloor(16 \times -61 + 2 ^ 9) \div 2 ^ {10}\rfloor = 1 &
16 - 2 = 14
\\
& 4 & 20 & &
20 - \lfloor(14 \times -33 + 2 ^ 9) \div 2 ^ {10}\rfloor = 20 &
14 - 2 = 12
\\
& 5 & 35 & &
35 - \lfloor(12 \times -18 + 2 ^ 9) \div 2 ^ {10}\rfloor = 35 &
12 - 2 = 10
\\
& 6 & 50 & &
50 - \lfloor(10 \times 0 + 2 ^ 9) \div 2 ^ {10}\rfloor = 50 &
10 + 0 = 10
\\
& 7 & 62 & &
62 - \lfloor(10 \times 20 + 2 ^ 9) \div 2 ^ {10}\rfloor = 62 &
10 + 2 = 12
\\
& 8 & 68 & &
68 - \lfloor(12 \times 35 + 2 ^ 9) \div 2 ^ {10}\rfloor = 68 &
12 + 2 = 14
\\
& 9 & 72 & &
72 - \lfloor(14 \times 50 + 2 ^ 9) \div 2 ^ {10}\rfloor = 71 &
14 + 2 = 16
\\
%%END
\end{tabular}
}
\begin{center}
$\text{channel}_0$ correlation passes
\end{center}


\clearpage

%Copyright (C) 2007-2014  Brian Langenberger
%This work is licensed under the
%Creative Commons Attribution-Share Alike 3.0 United States License.
%To view a copy of this license, visit
%http://creativecommons.org/licenses/by-sa/3.0/us/ or send a letter to
%Creative Commons,
%171 Second Street, Suite 300,
%San Francisco, California, 94105, USA.

\subsection{Reading Bitstream}
\label{wavpack:read_bitstream}
{\relsize{-1}
  \input{wavpack/algorithms/read_bitstream}
}


\subsubsection{Reading Modified Elias Gamma Code}
\label{wavpack:read_egc}
{\relsize{-1}
  \input{wavpack/algorithms/read_egc}
}

\clearpage

\subsubsection{Reading Residual Value}
\label{wavpack:read_residual}
{\relsize{-1}
  \input{wavpack/algorithms/read_residual}
}

\clearpage

\subsubsection{Determining Base and Add Values}
\label{wavpack:decode_base_add}
{\relsize{-1}
  \input{wavpack/algorithms/decode_base_add}
}

\clearpage

\begin{figure}[h]
  \includegraphics{wavpack/figures/residuals.pdf}
\end{figure}

\subsubsection{Residual Decoding Example}
Given a 2 channel block with $\text{entropies}_0 = \texttt{[118, 194, 322]}$
and $\text{entropies}_1 = \texttt{[118, 176, 212]}$:

\begin{figure}[h]
\includegraphics[width=6in,keepaspectratio]{wavpack/figures/residuals_parse.pdf}
\end{figure}
\par
\noindent
Calculations of $\text{entropy}_0$ and $\text{entropy}_1$
are left as an exercise for the reader.

\clearpage

\begin{table}[h]
{\relsize{-3}
\begin{tabular}{|>{$}r<{$}||>{$}r<{$}|>{$}r<{$}|>{$}r<{$}|>{$}r<{$}|}
i & u_i & m_i &
\text{base} & \text{add} \\
%% \text{entropy}_{c~0} & \text{entropy}_{c~1} & \text{entropy}_{c~2} \\
\hline
0 &
7 &
\lfloor 7 \div 2 \rfloor = 3 &
2 + \left\lfloor\frac{118}{2 ^ 4}\right\rfloor + \left\lfloor\frac{194}{2 ^ 4}\right\rfloor + \left(\left\lfloor\frac{322}{2 ^ 4}\right\rfloor \times 1\right) = 42 & \lfloor 322 \div 2 ^ 4 \rfloor = 20 \\
%% 118 + 5 = 123 & 194 + 20 = 214 & 322 + 55 = 377
1 &
3 &
\lfloor 3 \div 2 \rfloor + 1 = 2 &
2 + \left\lfloor\frac{118}{2 ^ 4}\right\rfloor + \left\lfloor\frac{176}{2 ^ 4}\right\rfloor = 20 & \lfloor 212 \div 2 ^ 4 \rfloor = 13 \\
%% 118 + 5 = 123 & 176 + 15 = 191 & 212 - 35 = 198
\hline
2 &
3 &
\lfloor 3 \div 2 \rfloor + 1 = 2 &
2 + \left\lfloor\frac{123}{2 ^ 4}\right\rfloor + \left\lfloor\frac{214}{2 ^ 4}\right\rfloor = 22 & \lfloor 377 \div 2 ^ 4 \rfloor = 23 \\
%% 123 + 5 = 128 & 214 + 20 = 234 & 377 - 60 = 353
3 &
3 &
\lfloor 3 \div 2 \rfloor + 1 = 2 &
2 + \left\lfloor\frac{123}{2 ^ 4}\right\rfloor + \left\lfloor\frac{191}{2 ^ 4}\right\rfloor = 20 & \lfloor 198 \div 2 ^ 4 \rfloor = 12 \\
%% 123 + 5 = 128 & 191 + 15 = 206 & 198 - 35 = 184
\hline
4 &
1 &
\lfloor 1 \div 2 \rfloor + 1 = 1 &
1 + \left\lfloor\frac{128}{2 ^ 4}\right\rfloor = 9 & \lfloor 234 \div 2 ^ 4 \rfloor = 14 \\
%% 128 + 10 = 138 & 234 - 8 = 226 & 353
5 &
4 &
\lfloor 4 \div 2 \rfloor + 1 = 3 &
2 + \left\lfloor\frac{128}{2 ^ 4}\right\rfloor + \left\lfloor\frac{206}{2 ^ 4}\right\rfloor + \left(\left\lfloor\frac{184}{2 ^ 4}\right\rfloor \times 1\right) = 34 & \lfloor 184 \div 2 ^ 4 \rfloor = 11 \\
%% 128 + 10 = 138 & 206 + 20 = 226 & 184 + 30 = 214
\hline
6 &
\textit{undefined} &
0 &
0 & \lfloor 138 \div 2 ^ 4 \rfloor = 8 \\
%% 138 - 4 = 134 & 226 & 353
7 &
5 &
\lfloor 5 \div 2 \rfloor = 2 &
2 + \left\lfloor\frac{138}{2 ^ 4}\right\rfloor + \left\lfloor\frac{226}{2 ^ 4}\right\rfloor = 24 & \lfloor 214 \div 2 ^ 4 \rfloor = 13 \\
%% 138 + 10 = 148 & 226 + 20 = 246 & 214 - 35 = 200
\hline
8 &
1 &
\lfloor 1 \div 2 \rfloor + 1 = 1 &
1 + \left\lfloor\frac{134}{2 ^ 4}\right\rfloor = 9 & \lfloor 226 \div 2 ^ 4 \rfloor = 14 \\
%% 134 + 10 = 144 & 226 - 8 = 218 & 353
9 &
3 &
\lfloor 3 \div 2 \rfloor + 1 = 2 &
2 + \left\lfloor\frac{148}{2 ^ 4}\right\rfloor + \left\lfloor\frac{246}{2 ^ 4}\right\rfloor = 26 & \lfloor 200 \div 2 ^ 4 \rfloor = 12 \\
%% 148 + 10 = 158 & 246 + 20 = 266 & 200 - 35 = 186
\hline
10 &
3 &
\lfloor 3 \div 2 \rfloor + 1 = 2 &
2 + \left\lfloor\frac{144}{2 ^ 4}\right\rfloor + \left\lfloor\frac{218}{2 ^ 4}\right\rfloor = 24 & \lfloor 353 \div 2 ^ 4 \rfloor = 22 \\
%% 144 + 10 = 154 & 218 + 20 = 238 & 353 - 55 = 331
11 &
3 &
\lfloor 3 \div 2 \rfloor + 1 = 2 &
2 + \left\lfloor\frac{158}{2 ^ 4}\right\rfloor + \left\lfloor\frac{266}{2 ^ 4}\right\rfloor = 27 & \lfloor 186 \div 2 ^ 4 \rfloor = 11 \\
%% 158 + 10 = 168 & 266 + 25 = 291 & 186 - 30 = 174
\hline
12 &
5 &
\lfloor 5 \div 2 \rfloor + 1 = 3 &
2 + \left\lfloor\frac{154}{2 ^ 4}\right\rfloor + \left\lfloor\frac{238}{2 ^ 4}\right\rfloor + \left(\left\lfloor\frac{331}{2 ^ 4}\right\rfloor \times 1\right) = 46 & \lfloor 331 \div 2 ^ 4 \rfloor = 20 \\
%% 154 + 10 = 164 & 238 + 20 = 258 & 331 + 55 = 386
13 &
1 &
\lfloor 1 \div 2 \rfloor + 1 = 1 &
1 + \left\lfloor\frac{168}{2 ^ 4}\right\rfloor = 11 & \lfloor 291 \div 2 ^ 4 \rfloor = 18 \\
%% 168 + 10 = 178 & 291 - 10 = 281 & 174
\hline
14 &
5 &
\lfloor 5 \div 2 \rfloor + 1 = 3 &
2 + \left\lfloor\frac{164}{2 ^ 4}\right\rfloor + \left\lfloor\frac{258}{2 ^ 4}\right\rfloor + \left(\left\lfloor\frac{386}{2 ^ 4}\right\rfloor \times 1\right) = 53 & \lfloor 386 \div 2 ^ 4 \rfloor = 24 \\
%% 164 + 10 = 174 & 258 + 25 = 283 & 386 + 65 = 451
15 &
1 &
\lfloor 1 \div 2 \rfloor + 1 = 1 &
1 + \left\lfloor\frac{178}{2 ^ 4}\right\rfloor = 12 & \lfloor 281 \div 2 ^ 4 \rfloor = 17 \\
%% 178 + 10 = 188 & 281 - 10 = 271 & 174
\hline
16 &
4 &
\lfloor 4 \div 2 \rfloor + 1 = 3 &
2 + \left\lfloor\frac{174}{2 ^ 4}\right\rfloor + \left\lfloor\frac{283}{2 ^ 4}\right\rfloor + \left(\left\lfloor\frac{451}{2 ^ 4}\right\rfloor \times 1\right) = 58 & \lfloor 451 \div 2 ^ 4 \rfloor = 28 \\
%% 174 + 10 = 184 & 283 + 25 = 308 & 451 + 75 = 526
17 &
\textit{undefined} &
0 &
0 & \lfloor 188 \div 2 ^ 4 \rfloor = 11 \\
%% 188 - 4 = 184 & 271 & 174
\hline
18 &
6 &
\lfloor 6 \div 2 \rfloor = 3 &
2 + \left\lfloor\frac{184}{2 ^ 4}\right\rfloor + \left\lfloor\frac{308}{2 ^ 4}\right\rfloor + \left(\left\lfloor\frac{526}{2 ^ 4}\right\rfloor \times 1\right) = 65 & \lfloor 526 \div 2 ^ 4 \rfloor = 32 \\
%% 184 + 10 = 194 & 308 + 25 = 333 & 526 + 85 = 611
19 &
\textit{undefined} &
0 &
0 & \lfloor 184 \div 2 ^ 4 \rfloor = 11 \\
%% 184 - 4 = 180 & 271 & 174
\hline
\end{tabular}
}
\vskip .25in
{\relsize{-3}
\renewcommand{\arraystretch}{1.5}
\begin{tabular}{|>{$}r<{$}|>{$}r<{$}|>{$}r<{$}||>{$}r<{$}|>{$}r<{$}|>{$}r<{$}|>{$}r<{$}|>{$}r<{$}|>{$}r<{$}|>{$}r<{$}|}
i & \text{base} & \text{add} & p & e & r_i & b_i & unsigned & sign_i & residual_i \\
\hline
0 &
42 & 20 &
\lfloor\log_2(20)\rfloor = 4 &
2 ^ {4 + 1} - 20 - 1 = 11 &
14 &
1 & 42 + (14 \times 2) - 11 + 1 = 60 &
1 & -60 - 1 = -61
\\
1 &
20 & 13 &
\lfloor\log_2(13)\rfloor = 3 &
2 ^ {3 + 1} - 13 - 1 = 2 &
6 &
1 & 20 + (6 \times 2) - 2 + 1 = 31 &
0 & 31
\\
\hline
2 &
22 & 23 &
\lfloor\log_2(23)\rfloor = 4 &
2 ^ {4 + 1} - 23 - 1 = 8 &
9 &
0 & 22 + (9 \times 2) - 8 + 0 = 32 &
1 & -32 - 1 = -33
\\
3 &
20 & 12 &
\lfloor\log_2(12)\rfloor = 3 &
2 ^ {3 + 1} - 12 - 1 = 3 &
7 &
1 & 20 + (7 \times 2) - 3 + 1 = 32 &
0 & 32
\\
\hline
4 &
9 & 14 &
\lfloor\log_2(14)\rfloor = 3 &
2 ^ {3 + 1} - 14 - 1 = 1 &
4 &
1 & 9 + (4 \times 2) - 1 + 1 = 17 &
1 & -17 - 1 = -18
\\
5 &
34 & 11 &
\lfloor\log_2(11)\rfloor = 3 &
2 ^ {3 + 1} - 11 - 1 = 4 &
2 &
 & 34 + 2 = 36 &
0 & 36
\\
\hline
6 &
0 & 8 &
\lfloor\log_2(8)\rfloor = 3 &
2 ^ {3 + 1} - 8 - 1 = 7 &
1 &
 & 0 + 1 = 1 &
0 & 1
\\
7 &
24 & 13 &
\lfloor\log_2(13)\rfloor = 3 &
2 ^ {3 + 1} - 13 - 1 = 2 &
7 &
1 & 24 + (7 \times 2) - 2 + 1 = 37 &
0 & 37
\\
\hline
8 &
9 & 14 &
\lfloor\log_2(14)\rfloor = 3 &
2 ^ {3 + 1} - 14 - 1 = 1 &
6 &
0 & 9 + (6 \times 2) - 1 + 0 = 20 &
0 & 20
\\
9 &
26 & 12 &
\lfloor\log_2(12)\rfloor = 3 &
2 ^ {3 + 1} - 12 - 1 = 3 &
6 &
0 & 26 + (6 \times 2) - 3 + 0 = 35 &
0 & 35
\\
\hline
10 &
24 & 22 &
\lfloor\log_2(22)\rfloor = 4 &
2 ^ {4 + 1} - 22 - 1 = 9 &
10 &
0 & 24 + (10 \times 2) - 9 + 0 = 35 &
0 & 35
\\
11 &
27 & 11 &
\lfloor\log_2(11)\rfloor = 3 &
2 ^ {3 + 1} - 11 - 1 = 4 &
4 &
0 & 27 + (4 \times 2) - 4 + 0 = 31 &
0 & 31
\\
\hline
12 &
46 & 20 &
\lfloor\log_2(20)\rfloor = 4 &
2 ^ {4 + 1} - 20 - 1 = 11 &
4 &
 & 46 + 4 = 50 &
0 & 50
\\
13 &
11 & 18 &
\lfloor\log_2(18)\rfloor = 4 &
2 ^ {4 + 1} - 18 - 1 = 13 &
13 &
1 & 11 + (13 \times 2) - 13 + 1 = 25 &
0 & 25
\\
\hline
14 &
53 & 24 &
\lfloor\log_2(24)\rfloor = 4 &
2 ^ {4 + 1} - 24 - 1 = 7 &
8 &
0 & 53 + (8 \times 2) - 7 + 0 = 62 &
0 & 62
\\
15 &
12 & 17 &
\lfloor\log_2(17)\rfloor = 4 &
2 ^ {4 + 1} - 17 - 1 = 14 &
6 &
 & 12 + 6 = 18 &
0 & 18
\\
\hline
16 &
58 & 28 &
\lfloor\log_2(28)\rfloor = 4 &
2 ^ {4 + 1} - 28 - 1 = 3 &
6 &
1 & 58 + (6 \times 2) - 3 + 1 = 68 &
0 & 68
\\
17 &
0 & 11 &
\lfloor\log_2(11)\rfloor = 3 &
2 ^ {3 + 1} - 11 - 1 = 4 &
7 &
0 & 0 + (7 \times 2) - 4 + 0 = 10 &
0 & 10
\\
\hline
18 &
65 & 32 &
\lfloor\log_2(32)\rfloor = 5 &
2 ^ {5 + 1} - 32 - 1 = 31 &
6 &
 & 65 + 6 = 71 &
0 & 71
\\
19 &
0 & 11 &
\lfloor\log_2(11)\rfloor = 3 &
2 ^ {3 + 1} - 11 - 1 = 4 &
0 &
 & 0 + 0 = 0 &
0 & 0
\\
\hline
\end{tabular}
}
{\relsize{-1}
  \vskip .2in
  Resulting in:
  \newline
  \begin{tabular}{rr}
    channel 0 residuals : & \texttt{[-61,~-33,~-18,~~1,~20,~35,~50,~62,~68,~71]}\\
    channel 1 residuals : & \texttt{[~31,~~32,~~36,~37,~35,~31,~25,~18,~10,~~0]}\\
  \end{tabular}
}
\end{table}

\clearpage

\subsubsection{2nd Residual Decoding Example}
Given a 1 channel block with $\text{entropies}_0 = \texttt{[0, 0, 0]}$:

\begin{figure}[h]
\includegraphics{wavpack/figures/residuals_parse2.pdf}
\end{figure}
\par
\noindent
This residual block demonstrates how a long run of 0 residuals is read
when $\text{entropy}_{0~0}$ falls to a low enough value.
Calculation of $\text{entropies}_{0~1}$ and $\text{entropies}_{0~2}$
are left as an exercise for the reader.

\clearpage

\begin{table}[h]
{\relsize{-3}
\renewcommand{\arraystretch}{1.5}
\begin{tabular}{|>{$}r<{$}||>{$}r<{$}|>{$}r<{$}|>{$}r<{$}|>{$}r<{$}||>{$}r<{$}|>{$}r<{$}|>{$}r<{$}|}
i & u_i & m_i &
\text{base} & \text{add} & \text{entropy}_{0~0} \\
\hline
& \multicolumn{5}{c|}{read modified Elias gamma code: $t \leftarrow 0~,~\text{zeroes} \leftarrow 0$} \\
0 &
3 & \lfloor 3 \div 2\rfloor = 1 &
1 + \left\lfloor\frac{0}{2 ^ 4}\right\rfloor = 1 &
\left\lfloor\frac{0}{2 ^ 4}\right\rfloor = 0 &
0 + 5 = 5 %% & 0 - 0 = 0 & 0
\\
1 &
3 & \lfloor 3 \div 2\rfloor + 1 = 2 &
2 + \left\lfloor\frac{5}{2 ^ 4}\right\rfloor + \left\lfloor\frac{0}{2 ^ 4}\right\rfloor = 2 &
\left\lfloor\frac{0}{2 ^ 4}\right\rfloor = 0 &
5 + 5 = 10 %% & 0 + 5 = 5 & 0 - 0 = 0
\\
2 &
5 & \lfloor 5 \div 2\rfloor + 1 = 3 &
2 + \left\lfloor\frac{10}{2 ^ 4}\right\rfloor + \left\lfloor\frac{5}{2 ^ 4}\right\rfloor + \left(\left\lfloor\frac{0}{2 ^ 4}\right\rfloor \times 1\right) = 3 &
\left\lfloor\frac{0}{2 ^ 4}\right\rfloor = 0 &
10 + 5 = 15 %% & 5 + 5 = 10 & 0 + 5 = 5
\\
3 &
2 & \lfloor 2 \div 2\rfloor + 1 = 2 &
2 + \left\lfloor\frac{15}{2 ^ 4}\right\rfloor + \left\lfloor\frac{10}{2 ^ 4}\right\rfloor = 2 &
\left\lfloor\frac{5}{2 ^ 4}\right\rfloor = 0 &
15 + 5 = 20 %% & 10 + 5 = 15 & 5 - 2 = 3
\\
4 &
\textit{undefined} & 0 &
0 & \left\lfloor\frac{20}{2 ^ 4}\right\rfloor = 1 &
20 - 2 = 18 %% & 15 & 3
\\
5 &
0 & \lfloor 0 \div 2\rfloor = 0 &
0 & \left\lfloor\frac{18}{2 ^ 4}\right\rfloor = 1 &
18 - 2 = 16 %% & 15 & 3
\\
6 &
\textit{undefined} & 0 &
0 & \left\lfloor\frac{16}{2 ^ 4}\right\rfloor = 1 &
16 - 2 = 14 %% & 15 & 3
\\
7 &
0 & \lfloor 0 \div 2\rfloor = 0 &
0 & \left\lfloor\frac{14}{2 ^ 4}\right\rfloor = 0 &
14 - 2 = 12 %% & 15 & 3
\\
8 &
\textit{undefined} & 0 &
0 & \left\lfloor\frac{12}{2 ^ 4}\right\rfloor = 0 &
12 - 2 = 10 %% & 15 & 3
\\
9 &
0 & \lfloor 0 \div 2\rfloor = 0 &
0 & \left\lfloor\frac{10}{2 ^ 4}\right\rfloor = 0 &
10 - 2 = 8 %% & 15 & 3
\\
10 &
\textit{undefined} & 0 &
0 & \left\lfloor\frac{8}{2 ^ 4}\right\rfloor = 0 &
8 - 2 = 6 %% & 15 & 3
\\
11 &
0 & \lfloor 0 \div 2\rfloor = 0 &
0 & \left\lfloor\frac{6}{2 ^ 4}\right\rfloor = 0 &
6 - 2 = 4 %% & 15 & 3
\\
12 &
\textit{undefined} & 0 &
0 & \left\lfloor\frac{4}{2 ^ 4}\right\rfloor = 0 &
4 - 2 = 2 %% & 15 & 3
\\
13 &
0 & \lfloor 0 \div 2\rfloor = 0 &
0 & \left\lfloor\frac{2}{2 ^ 4}\right\rfloor = 0 &
2 - 2 = 0 %% & 15 & 3
\\
14 &
\textit{\color{red}undefined} & 0 &
0 & \left\lfloor\frac{0}{2 ^ 4}\right\rfloor = 0 &
0 - 0 = {\color{red}0} %% & 15 & 3
\\
15-24 & \multicolumn{5}{c|}{read modified Elias gamma code: $t \leftarrow 4~,~p \leftarrow 2~,~\text{zeroes} \leftarrow 2 ^ {(4 - 1)} + 2 = 10$} \\
25 &
1 & \lfloor 1 \div 2\rfloor = 0 &
0 & \left\lfloor\frac{0}{2 ^ 4}\right\rfloor = 0 &
0 - 0 = 0 %% & 0 & 0
\\
26 &
1 & \lfloor 1 \div 2\rfloor + 1 = 1 &
1 + \left\lfloor\frac{0}{2 ^ 4}\right\rfloor = 1 &
\left\lfloor\frac{0}{2 ^ 4}\right\rfloor = 0 &
0 + 5 = 5 %% & 0 - 0 = 0 & 0
\\
27 &
3 & \lfloor 3 \div 2\rfloor + 1 = 2 &
2 + \left\lfloor\frac{5}{2 ^ 4}\right\rfloor + \left\lfloor\frac{0}{2 ^ 4}\right\rfloor = 2 &
\left\lfloor\frac{0}{2 ^ 4}\right\rfloor = 0 &
5 + 5 = 10 %% & 0 + 5 = 5 & 0 - 0 = 0
\\
28 &
0 & \lfloor 0 \div 2\rfloor + 1 = 1 &
1 + \left\lfloor\frac{10}{2 ^ 4}\right\rfloor = 1 &
\left\lfloor\frac{5}{2 ^ 4}\right\rfloor = 0 &
10 + 5 = 15 %% & 5 - 2 = 3 & 0
\\
29 &
\textit{undefined} & 0 &
0 & \left\lfloor\frac{15}{2 ^ 4}\right\rfloor = 0 &
15 - 2 = 13 %% & 3 & 0
\\
\hline
\end{tabular}
}
\vskip .25in
{\relsize{-3}
\begin{tabular}{|>{$}r<{$}|>{$}r<{$}|>{$}r<{$}||>{$}r<{$}|>{$}r<{$}|>{$}r<{$}|>{$}r<{$}|>{$}r<{$}|>{$}r<{$}|>{$}r<{$}|}
i & \text{base} & \text{add} & p & e & r_i & b_i & unsigned & sign_i & residual_i \\
\hline
0 &
1 & 0 &
& & & & 1 &
0 & 1
\\
1 &
2 & 0 &
& & & & 2 &
0 & 2
\\
2 &
3 & 0 &
& & & & 3 &
0 & 3
\\
3 &
2 & 0 &
& & & & 2 &
0 & 2
\\
4 &
0 & 1 &
\lfloor\log_2(1)\rfloor = 0 &
2 ^ {0 + 1} - 1 - 1 = 0 &
0 &
1 & 0 + (0 \times 2) - 0 + 1 = 1 &
0 & 1
\\
5 &
0 & 1 &
\lfloor\log_2(1)\rfloor = 0 &
2 ^ {0 + 1} - 1 - 1 = 0 &
0 &
0 & 0 + (0 \times 2) - 0 + 0 = 0 &
0 & 0
\\
6 &
0 & 1 &
\lfloor\log_2(1)\rfloor = 0 &
2 ^ {0 + 1} - 1 - 1 = 0 &
0 &
0 & 0 + (0 \times 2) - 0 + 0 = 0 &
0 & 0
\\
7 &
0 & 0 &
& & & & 0 &
0 & 0
\\
8 &
0 & 0 &
& & & & 0 &
0 & 0
\\
9 &
0 & 0 &
& & & & 0 &
0 & 0
\\
10 &
0 & 0 &
& & & & 0 &
0 & 0
\\
11 &
0 & 0 &
& & & & 0 &
0 & 0
\\
12 &
0 & 0 &
& & & & 0 &
0 & 0
\\
13 &
0 & 0 &
& & & & 0 &
0 & 0
\\
14 &
0 & 0 &
& & & & 0 &
0 & 0
\\
15-24 & \multicolumn{8}{c|}{long run of 10, 0 residual values} & 0 \\
25 &
0 & 0 &
& & & & 0 &
1 & -0 - 1 = -1
\\
26 &
1 & 0 &
& & & & 1 &
1 & -1 - 1 = -2
\\
27 &
2 & 0 &
& & & & 2 &
1 & -2 - 1 = -3
\\
28 &
1 & 0 &
& & & & 1 &
1 & -1 - 1 = -2
\\
29 &
0 & 0 &
& & & & 0 &
1 & -0 - 1 = -1
\\
\hline
\end{tabular}
}
{\relsize{-3}
  \vskip .2in
    Resulting in channel 0 residuals:
    \newline
    \texttt{[~~1,~~2,~~3,~~2,~~1,~~0,~~0,~~0,~~0,~~0,~~0,~~0,~~0,~~0,~~0,~~0,~~0,~~0,~~0,~~0,~~0,~~0,~~0,~~0,~~0,~-1,~-2,~-3,~-2,~-1]}
}
\end{table}


\clearpage

\subsection{Writing RIFF WAVE Header and Footer}
\label{wavpack:write_wave_header}
\begin{figure}[h]
  \includegraphics{wavpack/figures/pcm_sandwich.pdf}
\end{figure}


\subsection{Writing MD5 Sum}
\label{wavpack:write_md5}
MD5 sum is calculated as if the PCM data had been read from
a wave file's \texttt{data} chunk.
That is, the samples are converted to little-endian format
and are signed if the stream's bits-per-sample is greater than 8.
\begin{figure}[h]
  \includegraphics{wavpack/figures/md5sum.pdf}
\end{figure}

\subsection{Writing Extended Integers}
\label{wavpack:write_extended_integers}
\begin{figure}[h]
  \includegraphics{wavpack/figures/extended_integers.pdf}
\end{figure}

\clearpage

\subsection{Writing Channel Info}
\label{wavpack:write_channel_info}
\begin{figure}[h]
  \includegraphics{wavpack/figures/channel_info.pdf}
\end{figure}


\subsection{Writing Sample Rate}
\label{wavpack:write_sample_rate}
\begin{figure}[h]
  \includegraphics{wavpack/figures/sample_rate.pdf}
\end{figure}

\clearpage

\subsection{Writing Block Header}
\label{wavpack:write_block_header}
{\relsize{-1}
  \input{wavpack/algorithms/write_block_header}
}

\clearpage

\subsection{Round-Tripping Correlation Weights}
\label{wavpack:roundtrip_weights}
Because the final weight values of one block
may not be exactly representable in a correlation weights sub-block,
it's necessary to ``round-trip'' the weight values
so that the starting values for the next block
are the same as the values stored in the sub-block.

\input{wavpack/algorithms/roundtrip_weights}

\clearpage

\subsection{Round-Tripping Correlation Samples}
\label{wavpack:roundtrip_samples}

\input{wavpack/algorithms/roundtrip_samples}
