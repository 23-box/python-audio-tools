%Copyright (C) 2007-2015  Brian Langenberger
%This work is licensed under the
%Creative Commons Attribution-Share Alike 3.0 United States License.
%To view a copy of this license, visit
%http://creativecommons.org/licenses/by-sa/3.0/us/ or send a letter to
%Creative Commons,
%171 Second Street, Suite 300,
%San Francisco, California, 94105, USA.

\chapter{Sun AU}
The AU file format was invented by Sun Microsystems
and also used on NeXT systems.
All values in AU are stored as big-endian.
It supports a wide array of data formats, including \textmu-law logarithmic
encoding, but can also be used as a PCM container.

\section{the Sun AU File Stream}
\begin{figure}[h]
\includegraphics{figures/au_stream.pdf}
\end{figure}

\begin{table}[h]
{\relsize{-2}
\begin{tabular}{|r|l|}
\hline
value & encoding format \\
\hline
1 & 8-bit G.711 \textmu-law \\
2 & 8-bit linear PCM \\
3 & 16-bit linear PCM \\
4 & 24-bit linear PCM \\
5 & 32-bit linear PCM \\
6 & 32-bit IEEE floating point \\
7 & 64-bit IEEE floating point \\
8 & Fragmented sample data \\
9 & DSP program \\
10 & 8-bit fixed point \\
11 & 16-bit fixed point \\
12 & 24-bit fixed point \\
13 & 32-bit fixed point \\
18 & 16-bit linear with emphasis \\
19 & 16-bit linear compressed \\
20 & 16-bit linear with emphasis and compression \\
21 & Music kit DSP commands \\
23 & 4-bit ISDN \textmu-law compressed using \\
& the ITU-T G.721 ADPCM voice data encoding scheme \\
24 & ITU-T G.722 ADPCM \\
25 & ITU-T G.723 3-bit ADPCM \\
26 & ITU-T G.723 5-bit ADPCM \\
27 & 8-bit G.711 A-law \\
\hline
\end{tabular}
}
\end{table}
