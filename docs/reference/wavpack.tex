%This work is licensed under the
%Creative Commons Attribution-Share Alike 3.0 United States License.
%To view a copy of this license, visit
%http://creativecommons.org/licenses/by-sa/3.0/us/ or send a letter to
%Creative Commons,
%171 Second Street, Suite 300,
%San Francisco, California, 94105, USA.

\chapter{WavPack}
WavPack is a format for compressing Wave files, typically in lossless mode.
Notably, it also has a lossy mode and even a hybrid mode which allows
the `correction' file to be separated from a lossy core.

Metadata is stored as an APEv2 tag, which is described on page \pageref{apev2}.

All of its fields are little-endian.

\section{the WavPack file stream}
\begin{figure}[h]
\includegraphics{figures/wavpack_stream.pdf}
\end{figure}

\pagebreak

\section{the WavPack block header}
\begin{figure}[h]
\includegraphics{figures/wavpack_block_header.pdf}
\end{figure}
\parpic[r]{
{\relsize{-1}
\begin{tabular}{|c|r|}
\hline
value & sample rate \\
\hline
\texttt{0000} & 6000 \\
\texttt{0001} & 8000 \\
\texttt{0010} & 9600 \\
\texttt{0011} & 11025 \\
\texttt{0100} & 12000 \\
\texttt{0101} & 16000 \\
\texttt{0110} & 22050 \\
\texttt{0111} & 24000 \\
\texttt{1000} & 32000 \\
\texttt{1001} & 44100 \\
\texttt{1010} & 48000 \\
\texttt{1011} & 64000 \\
\texttt{1100} & 88200 \\
\texttt{1101} & 96000 \\
\texttt{1110} & 192000 \\
\texttt{1111} & reserved \\
\hline
\end{tabular}
}
}
The `flags' field is stored as a little-endian 32-bit integer.
Since some fields cross byte boundaries, their high and low bits
will be far apart when written in this format where the bits are
ordered the way they appear in the file.

`Block Size' is the length of everything past everything past the
block header, minus 24 bytes.

`Bits per Sample' is one of 4 values:

\begin{inparaenum}
\item[\texttt{00} = ] 8 bps,
\item[\texttt{01} = ] 16 bps,
\item[\texttt{10} = ] 24 bps,
\item[\texttt{11} = ] 32 bps
\end{inparaenum}
.

`Mono Output' bit indicates the channel count.
If 1, this block has 1 channel.
If 0, this block has 2 channels.
For an audio stream with more than 2 channels,
check the `Initial Block' and `Final Block' bits to indicate
the start and end of the channels.  As an example:

\begin{tabular}{c|c|c|c}
Initial Block & Final Block & Mono Output & Channels \\
\hline
1 & 0 & 0 & 2 \\
0 & 0 & 1 & 1 \\
0 & 0 & 1 & 1 \\
0 & 1 & 0 & 2 \\
\hline
\multicolumn{3}{r|}{Total} & 6
\end{tabular}

\subsection{WavPack sub-block header}
\begin{figure}[h]
\includegraphics{figures/wavpack_subblock_header.pdf}
\end{figure}
\par
\noindent
If the `Large Block' field is 0, the `Block Size' field is 8 bits long.
If it is 1, the `Block Size' field is 24 bits long.
The `Block Size' field is the length of `Block Data', in 16-bit
words rather than bytes.
If `Actual Size 1 Less' is set, that means `Block Data' doesn't contain
an even number of bytes; it is padded with a single null byte at the
end in order to fit.
If `Nondecoder Data' is set, that means the decoder does not have
to understand the contents of this particular sub-block in
order to decode the audio.
