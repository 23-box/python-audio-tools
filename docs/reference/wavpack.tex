%This work is licensed under the
%Creative Commons Attribution-Share Alike 3.0 United States License.
%To view a copy of this license, visit
%http://creativecommons.org/licenses/by-sa/3.0/us/ or send a letter to
%Creative Commons,
%171 Second Street, Suite 300,
%San Francisco, California, 94105, USA.

\chapter{WavPack}
WavPack is a format for compressing Wave files, typically in lossless mode.
Notably, it also has a lossy mode and even a hybrid mode which allows
the `correction' file to be separated from a lossy core.

Metadata is stored as an APEv2 tag, which is described on page \pageref{apev2}.

Its stream of data is stored little-endian, as described on page
\pageref{bitstreams}.

\section{the WavPack File Stream}
\begin{figure}[h]
\includegraphics{figures/wavpack/stream.pdf}
\end{figure}

\clearpage

\subsection{Block Header}
\begin{figure}[h]
\includegraphics{figures/wavpack/block_header.pdf}
\end{figure}

\begin{table}[h]
\begin{tabular}{rl}
\textbf{block ID} & always \texttt{"wvpk"} or \texttt{0x6B707677} \\
\textbf{block size} & number of bytes between \VAR{block size} field's end and block's end \\
\textbf{version} & the file version, between \texttt{0x0402} and \texttt{0x0410} \\
\textbf{track number} & always 0 \\
\textbf{index number} & always 0 \\
\textbf{total samples} & length of the entire file in PCM frames \\
\textbf{block index} & block's PCM frame offset in the stream \\
\textbf{block samples} & length of the block in PCM frames \\
\textbf{CRC} & checksum of decoded PCM data \\
\end{tabular}
\end{table}

\clearpage

\begin{table}[h]
{\relsize{-1}
\begin{tabular}{rrl}
flag & value & meaning \\
\hline
\hline
\textbf{bits per sample} & \texttt{0} & 8 bits per sample \\
& \texttt{1} & 16 bits per sample \\
& \texttt{2} & 24 bits per sample \\
& \texttt{3} & 32 bits per sample \\
\hline
\textbf{mono output} & \texttt{0} & block has 2 channels of output \\
& \texttt{1} & block has 1 channel of output \\
\hline
\textbf{hybrid mode} & \texttt{0} & data stored lossless \\
& \texttt{1} & data stored lossy, with external correction file \\
\hline
\textbf{joint stereo} & \texttt{0} & channels stored in true stereo \\
& \texttt{1} & channels stored in joint stereo \\
\hline
\textbf{channel decorrelation} & \texttt{0} & channels stored independently \\
& \texttt{1} & cross-channel decorrelation between channels \\
\hline
\textbf{hybrid noise shaping} & \texttt{0} & hybrid has flat noise spectrum \\
& \texttt{1} & hybrid has noise shaping \\
\hline
\textbf{floating point data} & \texttt{0} & samples stored as integers \\
& \texttt{1} & samples stored as floats \\
\hline
\textbf{extended size integers} & \texttt{1} & $\text{bits per sample} > 24$ or left shifted \\
\hline
\textbf{hybrid controls bitrate} & \texttt{0} & hybrid parameters control noise level \\
& \texttt{1} & hybrid parameters control bitrate \\
\hline
\textbf{hybrid noise balanced} & \texttt{1} & hybrid noise balanced between channels \\
\hline
\textbf{initial block} & \texttt{1} & initial block in multichannel sequence \\
\hline
\textbf{final block} & \texttt{1} & final block in multichannel sequence \\
\hline
\textbf{left shift data} & & amount of bits to left-shift decoded data \\
\hline
\textbf{maximum magnitude} & & bit length of largest integer value, minus 1 \\
\hline
\textbf{sample rate} & \texttt{0} & 6000 Hz \\
& \texttt{1} & 8000 Hz \\
& \texttt{2} & 9600 Hz \\
& \texttt{3} & 11025 Hz \\
& \texttt{4} & 12000 Hz \\
& \texttt{5} & 16000 Hz \\
& \texttt{6} & 22050 Hz \\
& \texttt{7} & 24000 Hz \\
& \texttt{8} & 32000 Hz \\
& \texttt{9} & 44100 Hz \\
& \texttt{10} & 48000 Hz \\
& \texttt{11} & 64000 Hz \\
& \texttt{12} & 88200 Hz \\
& \texttt{13} & 96000 Hz \\
& \texttt{14} & 192000 Hz \\
& \texttt{15} & reserved \\
\hline
\textbf{reserved} & & ignore if set \\
\hline
\textbf{use IIR} & & use IIR for negative hybrid noise shaping \\
\hline
\textbf{false stereo} & & mono block which decodes to 2 identical channels \\
\hline
\textbf{reserved} & & always 0 \\
\end{tabular}
}
\end{table}

\clearpage

\subsection{Sub Block}

\begin{figure}[h]
\includegraphics{figures/wavpack/subblock.pdf}
\end{figure}
\par
\noindent
If \VAR{large sub block} is 0, \VAR{sub block size} is an 8 bit field.
Otherwise, it is 24 bits.
If \VAR{actual size 1 less} is 1, the contents of
\VAR{sub block data} are 1 byte shorter than its length;
the block itself is padded with 1 empty byte.
\vskip .15in
\par
\noindent
\begin{table}[h]
\begin{tabular}{rrl}
metadata function & nondecoder data & contents \\
\hline
\texttt{0} & \texttt{0} & padding \\
\texttt{2} & \texttt{0} & decorrelation terms and deltas \\
\texttt{3} & \texttt{0} & initial decorrelation weights \\
\texttt{4} & \texttt{0} & decorrelation samples \\
\texttt{5} & \texttt{0} & initial entropy variables \\
\texttt{6} & \texttt{0} & hybrid entropy variables \\
\texttt{7} & \texttt{0} & hybrid lossless info \\
\texttt{8} & \texttt{0} & floating point info \\
\texttt{9} & \texttt{0} & large integer info \\
\texttt{10} & \texttt{0} & normal compressed audio bitstream \\
\texttt{11} & \texttt{0} & correction file bitstream \\
\texttt{12} & \texttt{0} & large floating point info \\
\texttt{13} & \texttt{0} & channel count and channel mask \\
\hline
\texttt{1} & \texttt{1} & RIFF header \\
\texttt{2} & \texttt{1} & RIFF trailer \\
\texttt{5} & \texttt{1} & encoding info \\
\texttt{6} & \texttt{1} & audio data MD5 sum \\
\texttt{7} & \texttt{1} & non-standard sample rate \\
\end{tabular}
\end{table}

\clearpage

\section{WavPack Decoding}

\ALGORITHM{a WavPack encoded file}{PCM samples}
\SetKwData{INDEX}{index}
\SetKwData{SAMPLES}{samples}
\SetKwData{TOTALSAMPLES}{total samples}
\SetKwData{FINALBLOCK}{final block}
\SetKwData{OUTPUTCHANNEL}{output channel}
\SetKwData{BLOCKCHANNEL}{block channel}
\SetKwData{BLOCKDATASIZE}{block data size}
\Repeat{block header's $\text{\INDEX} + \text{\SAMPLES} >= \text{\TOTALSAMPLES}$}{
  $c \leftarrow 0$\;
  \Repeat{block header's $\text{\FINALBLOCK} = 1$}{
    read block header\;
    read \BLOCKDATASIZE bytes of block data\;
    decode block header and block data to 1 or 2 channels of PCM data\;
    \eIf{2 channels}{
      $\text{\OUTPUTCHANNEL}_c \leftarrow \text{\BLOCKCHANNEL}_0$\;
      $\text{\OUTPUTCHANNEL}_{c + 1} \leftarrow \text{\BLOCKCHANNEL}_1$\;
      $c \leftarrow c + 2$\;
    }{
      $\text{\OUTPUTCHANNEL}_c \leftarrow \text{\BLOCKCHANNEL}_0$\;
      $c \leftarrow c + 1$\;
    }
  }
  update stream MD5 sum with \OUTPUTCHANNEL data\;
  \Return \OUTPUTCHANNEL data\;
}
\BlankLine
attempt to read one additional block for MD5 sub block\;
\If{MD5 sub block found}{
  \ASSERT sub block MD5 sum = stream MD5 sum\;
}
\EALGORITHM
\par
\noindent
For example, four blocks with the following attributes:
\begin{table}[h]
\begin{tabular}{rrr}
channel count & initial block & final block \\
\hline
2 & 1 & 0 \\
1 & 0 & 0 \\
1 & 0 & 0 \\
2 & 0 & 1 \\
\end{tabular}
\end{table}
\par
\noindent
combine into 6 channels of PCM output as follows:
\begin{figure}[h]
\includegraphics{figures/wavpack/block_channels.pdf}
\end{figure}

\clearpage

\subsection{Reading Block Header}
{\relsize{-1}
\ALGORITHM{a WavPack file stream}{block header fields}
\SetKwData{BLOCKID}{block ID}
\SetKwData{BLOCKSIZE}{block data size}
\SetKwData{VERSION}{version}
\SetKwData{TRACKNUMBER}{track number}
\SetKwData{INDEXNUMBER}{index number}
\SetKwData{TOTALSAMPLES}{total samples}
\SetKwData{BLOCKINDEX}{block index}
\SetKwData{BLOCKSAMPLES}{block samples}
\SetKwData{BITSPERSAMPLE}{bits per sample}
\SetKwData{MONOOUTPUT}{mono output}
\SetKwData{HYBRIDMODE}{hybrid mode}
\SetKwData{JOINTSTEREO}{joint stereo}
\SetKwData{CHANNELDECORR}{channel decorrelation}
\SetKwData{HYBRIDNOISESHAPING}{hybrid noise shaping}
\SetKwData{FLOATINGPOINTDATA}{floating point data}
\SetKwData{EXTENDEDSIZEINTEGERS}{extended size integers}
\SetKwData{HYBRIDCONTROLSBITRATE}{hybrid controls bitrate}
\SetKwData{HYBRIDNOISEBALANCED}{hybrid noise balanced}
\SetKwData{INITIALBLOCK}{initial block}
\SetKwData{FINALBLOCK}{final block}
\SetKwData{LEFTSHIFTDATA}{left shift data}
\SetKwData{MAXIMUMMAGNITUDE}{maximum magnitude}
\SetKwData{SAMPLERATE}{sample rate}
\SetKwData{USEIIR}{use IIR}
\SetKwData{FALSESTEREO}{false stereo}
\SetKwData{RESERVED}{reserved}
\SetKwData{CRC}{CRC}
\begin{tabular}{r>{$}c<{$}l}
\BLOCKID & \leftarrow & \READ 4 bytes\; \\
& & \ASSERT $\text{\BLOCKID} = \texttt{"wvpk"}$\; \\
\BLOCKSIZE & \leftarrow & (\READ 32 unsigned bits) - 24\; \\
\VERSION & \leftarrow & \READ 16 unsigned bits\; \\
\TRACKNUMBER & \leftarrow & \READ 8 unsigned bits\; \\
\INDEXNUMBER & \leftarrow & \READ 8 unsigned bits\; \\
\TOTALSAMPLES & \leftarrow & \READ 32 unsigned bits\; \\
\BLOCKINDEX & \leftarrow & \READ 32 unsigned bits\; \\
\BLOCKSAMPLES & \leftarrow & \READ 32 unsigned bits\; \\
\textit{encoded bits per sample} & \leftarrow & \READ 2 unsigned bits\; \\
\MONOOUTPUT & \leftarrow & \READ 1 unsigned bit\; \\
\HYBRIDMODE & \leftarrow & \READ 1 unsigned bit\; \\
\JOINTSTEREO & \leftarrow & \READ 1 unsigned bit\; \\
\CHANNELDECORR & \leftarrow & \READ 1 unsigned bit\; \\
\HYBRIDNOISESHAPING & \leftarrow & \READ 1 unsigned bit\; \\
\FLOATINGPOINTDATA & \leftarrow & \READ 1 unsigned bit\; \\
\EXTENDEDSIZEINTEGERS & \leftarrow & \READ 1 unsigned bit\; \\
\HYBRIDCONTROLSBITRATE & \leftarrow & \READ 1 unsigned bit\; \\
\HYBRIDNOISEBALANCED & \leftarrow & \READ 1 unsigned bit\; \\
\INITIALBLOCK & \leftarrow & \READ 1 unsigned bit\; \\
\FINALBLOCK & \leftarrow & \READ 1 unsigned bit\; \\
\LEFTSHIFTDATA & \leftarrow & \READ 5 unsigned bits\; \\
\MAXIMUMMAGNITUDE & \leftarrow & \READ 5 unsigned bits\; \\
\textit{encoded sample rate} & \leftarrow & \READ 4 unsigned bits\; \\
& & \SKIP 2 bits\; \\
\USEIIR & \leftarrow & \READ 1 unsigned bit\; \\
\FALSESTEREO & \leftarrow & \READ 1 unsigned bit\; \\
\RESERVED & \leftarrow & \READ 1 unsigned bit\; \\
& & \ASSERT $\text{\RESERVED} = 0$\; \\
\CRC & \leftarrow & \READ 32 unsigned bits\; \\
\end{tabular}
\EALGORITHM
}
{\relsize{-1}
\begin{tabular}{rr||rr}
  \textit{encoded bits per sample} & bits per sample &
  \textit{encoded sample rate} & sample rate \\
  \hline
  \texttt{0} & 8 &
  \texttt{0} & 6000 Hz \\
  \texttt{1} & 16 &
  \texttt{1} & 8000 Hz \\
  \texttt{2} & 24 &
  \texttt{2} & 9600 Hz \\
  \texttt{3} & 32 &
  \texttt{3} & 11025 Hz \\
  & & \texttt{4} & 12000 Hz \\
  & & \texttt{5} & 16000 Hz \\
  & & \texttt{6} & 22050 Hz \\
  & & \texttt{7} & 24000 Hz \\
  & & \texttt{8} & 32000 Hz \\
  & & \texttt{9} & 44100 Hz \\
  & & \texttt{10} & 48000 Hz \\
  & & \texttt{11} & 64000 Hz \\
  & & \texttt{12} & 88200 Hz \\
  & & \texttt{13} & 96000 Hz \\
  & & \texttt{14} & 192000 Hz \\
  & & \texttt{15} & reserved \\
\end{tabular}
}

\clearpage

\subsubsection{Reading Block Header Example}
\includegraphics{figures/wavpack/block_header_parse.pdf}

%% \clearpage

%% \begin{table}[h]
%% \begin{tabular}{rrl}
%% field & value & meaning \\
%% \hline
%% \textbf{block ID} & \texttt{0x6B707677} & \texttt{"wvpk"} \\
%% \textbf{block data size} & \texttt{0x000000B2} & $178 - 24 = 154$ bytes \\
%% \textbf{version} & \texttt{0x0407} \\
%% \textbf{track number} & \texttt{0} \\
%% \textbf{index number} & \texttt{0} \\
%% \textbf{total samples} & \texttt{0x00000019} & 25 PCM frames \\
%% \textbf{block index} & \texttt{0x00000000} & 0 PCM frames \\
%% \textbf{block samples} & \texttt{0x00000019} & 25 PCM frames \\
%% \textbf{bits per sample} & \texttt{1} & 16 bits per sample \\
%% \textbf{mono output} & \texttt{0} & 2 channel block \\
%% \textbf{hybrid mode} & \texttt{0} & lossless data \\
%% \textbf{joint stereo} & \texttt{1} & channels stored in joint stereo \\
%% \textbf{channel decorrelation} & \texttt{1} & cross-channel decorrelation used \\
%% \textbf{hybrid noise shaping} & \texttt{0} \\
%% \textbf{floating point data} & \texttt{0} & samples stored as integers \\
%% \textbf{extended size integers} & \texttt{0} \\
%% \textbf{hybrid controls bitrate} & \texttt{0} \\
%% \textbf{hybrid noise balanced} & \texttt{0} \\
%% \textbf{initial block} & \texttt{1} & is initial block in sequence \\
%% \textbf{final block} & \texttt{1} & is final block in sequence \\
%% \textbf{left shift data} & \texttt{0} \\
%% \textbf{maximum magnitude} & \texttt{15} & 16 bits per sample output \\
%% \textbf{sample rate} & \texttt{9} & 44100 Hz \\
%% \textbf{reserved} & \texttt{0} \\
%% \textbf{use IIR} & \texttt{0} \\
%% \textbf{false stereo} & \texttt{0} & both channels are not identical \\
%% \textbf{reserved} & \texttt{0} \\
%% \textbf{CRC} & \texttt{0x22D25AD7} \\
%% \end{tabular}
%% \end{table}

\clearpage

\subsection{Block Decoding}
{\relsize{-1}
\ALGORITHM{block header, block data}{1 or 2 channels of PCM sample data}
\SetKwData{METAFUNC}{metadata function}
\SetKwData{NONDECODER}{nondecoder data}
\SetKwData{ONELESS}{actual size 1 less}
\SetKwData{BLOCKDATA}{block data size}
\SetKwData{SUBBLOCKSIZE}{sub block size}
\SetKwData{TERMS}{decorrelation terms}
\SetKwData{DELTAS}{decorrelation deltas}
\SetKwData{WEIGHTS}{decorrelation weights}
\SetKwData{SAMPLES}{decorrelation samples}
\SetKwData{MEDIANS}{medians}
\SetKwData{RESIDUALS}{residuals}
\SetKwData{ZEROES}{zero bits}
\SetKwData{ONES}{one bits}
\SetKwData{DUPES}{duplicate bits}
\SetKw{AND}{and}
\tcc{read decoding parameters from sub blocks}
\While{$\text{\BLOCKDATA} > 0$}{
  \METAFUNC $\leftarrow$ \READ 5 unsigned bits\;
  \NONDECODER $\leftarrow$ \READ 1 unsigned bit\;
  \ONELESS $\leftarrow$ \READ 1 unsigned bit\;
  \textit{large sub block} $\leftarrow$ \READ 1 unsigned bit\;
  \eIf{$\text{large sub block} = 0$}{
    \SUBBLOCKSIZE $\leftarrow$ \READ 8 unsigned bits\;
  }{
    \SUBBLOCKSIZE $\leftarrow$ \READ 24 unsigned bits\;
  }
  \eIf{$\ONELESS = 0$}{
    read $(\SUBBLOCKSIZE \times 2)$ bytes of sub block data\;
  }{
    read $(\SUBBLOCKSIZE \times 2) - 1$ bytes of sub block data\;
    \SKIP 1 byte\;
  }
  \If{$\text{\NONDECODER} = 0$}{
    \Switch{\METAFUNC}{
      \uCase{2}{
        $\TERMS, \DELTAS \leftarrow$ read decorrelation terms sub block\;
      }
      \uCase{3}{
        \ASSERT \TERMS have been read\;
        $\WEIGHTS \leftarrow$ read decorrelation weights sub block\;
      }
      \uCase{4}{
        \ASSERT \TERMS have been read\;
        $\SAMPLES \leftarrow$ read decorrelation samples sub block\;
      }
      \uCase{5}{
        $\MEDIANS \leftarrow$ read entropy variables sub block\;
      }
      \uCase{9}{
        $\ZEROES, \ONES, \DUPES \leftarrow$ read extended integers sub block\;
      }
      \Case{10}{
        \ASSERT \MEDIANS have been read\;
        $\RESIDUALS \leftarrow$ read WavPack bitstream\;
      }
    }
  }
  \eIf{$\text{large sub block} = 0$}{
    $\BLOCKDATA \leftarrow \BLOCKDATA - (2 + 2 \times \text{\SUBBLOCKSIZE})$
  }{
    $\BLOCKDATA \leftarrow \BLOCKDATA - (4 + 2 \times \text{\SUBBLOCKSIZE})$
  }
}
\If{\TERMS have been read}{
  \ASSERT \WEIGHTS, \SAMPLES have been read\;
}
\ASSERT \RESIDUALS have been read\;
\EALGORITHM
}

\clearpage

\begin{algorithm}[H]
\SetKwData{TERMS}{decorrelation terms}
\SetKwData{MONOOUTPUT}{mono output}
\SetKwData{FALSESTEREO}{false stereo}
\SetKwData{JOINTSTEREO}{joint stereo}
\SetKwData{EXTENDEDINTS}{extended integers}
\SetKwData{RESIDUALS}{residuals}
\SetKwData{DECORRELATED}{decorrelated}
\SetKwData{LEFTRIGHT}{left/right}
\SetKwData{UNSHIFTED}{unshifted}
\SetKwFunction{LEN}{len}
\SetKwFunction{UNDOEXTENDEDINTS}{undo\_extended\_integers}
\SetKw{AND}{and}
\DontPrintSemicolon
\tcc{build PCM data from decoding parameters}
\eIf(\tcc*[f]{2 channels of output}){$\text{\MONOOUTPUT} = 0$ \AND $\text{\FALSESTEREO} = 0$}{
  \eIf{\TERMS have been read \AND $\LEN(\TERMS) > 0$}{
    $\text{\DECORRELATED}_0/\text{\DECORRELATED}_1 \leftarrow$ decorrelate $\text{\RESIDUALS}_0/\text{\RESIDUALS}_1$\;
  }{
    $\text{\DECORRELATED}_0/\text{\DECORRELATED}_1 \leftarrow \text{\RESIDUALS}_0/\text{\RESIDUALS}_1$\;
  }
  \eIf{$\text{\JOINTSTEREO} = 1$}{
    $\LEFTRIGHT \leftarrow$ undo $\text{\DECORRELATED}_0/\text{\DECORRELATED}_1$ channels joint stereo\;
  }{
    $\LEFTRIGHT \leftarrow \text{\DECORRELATED}_0/\text{\DECORRELATED}_1$\;
  }
  verify CRC of $\LEFTRIGHT$ against CRC in block header\;
  \eIf{extended integers have been read}{
    $\text{\UNSHIFTED}_0/\text{\UNSHIFTED}_1 \leftarrow$ undo \LEFTRIGHT channels extended integers\;
  }{
    $\text{\UNSHIFTED}_0/\text{\UNSHIFTED}_1 \leftarrow \LEFTRIGHT$
  }
  \Return $\text{\UNSHIFTED}_0$ and $\text{\UNSHIFTED}_1$\;
}(\tcc*[f]{1 or 2 channels of output}){
  \eIf{\TERMS have been read \AND $\LEN(\TERMS) > 0$}{
    $\text{\DECORRELATED}_0 \leftarrow$ decorrelate $\text{\RESIDUALS}_0$\;
  }{
    $\text{\DECORRELATED}_0 \leftarrow \text{\RESIDUALS}_0$\;
  }
  verify CRC of $\text{\DECORRELATED}_0$ against CRC in block header\;
  \eIf{extended integers have been read}{
    $\text{\UNSHIFTED}_0 \leftarrow$ undo $\text{\DECORRELATED}_0$ channel extended integers\;
  }{
    $\text{\UNSHIFTED}_0 \leftarrow \text{\DECORRELATED}_0$\;
  }
  \eIf{$\text{\FALSESTEREO} = 0$}{
    \Return $\text{\UNSHIFTED}_0$\;
  }{
    \Return $\text{\UNSHIFTED}_0$ and $\text{\UNSHIFTED}_0$\tcc*[r]{duplicate channel}
  }
}
\end{algorithm}


%% \clearpage

%% \subsection{Reading Sub Block Header}
%% \ALGORITHM{block data}{metadata function integer, nondecoder data flag, \VAR{actual size 1 less} flag, sub block size}
%% \SetKwData{METAFUNC}{metadata function}
%% \SetKwData{NONDECODER}{nondecoder data}
%% \SetKwData{SUBBLOCKSIZE}{sub block size}
%% \SetKwData{ACTUALSIZEONELESS}{actual size 1 less}
%% \METAFUNC $\leftarrow$ \READ 5 unsigned bits\;
%% \NONDECODER $\leftarrow$ \READ 1 unsigned bit\;
%% \ACTUALSIZEONELESS $\leftarrow$ \READ 1 unsigned bit\;
%% \textit{large sub block} $\leftarrow$ \READ 1 unsigned bit\;
%% \eIf{$\text{large sub block} = 0$}{
%%   \SUBBLOCKSIZE $\leftarrow$ (\READ 8 unsigned bits) $\times 2$\;
%% }{
%%   \SUBBLOCKSIZE $\leftarrow$ (\READ 24 unsigned bits) $\times 2$\;
%% }
%% \Return (\METAFUNC , \NONDECODER , \ACTUALSIZEONELESS , \SUBBLOCKSIZE)\;
%% \EALGORITHM

%% \subsubsection{Reading Sub Block Header Example}
%% \begin{figure}[h]
%% \includegraphics{figures/wavpack/subblock_parse.pdf}
%% \end{figure}
%% \begin{table}[h]
%% \begin{tabular}{r>{$}c<{$}l}
%% metadata function & \leftarrow & 2 \\
%% nondecoder data & \leftarrow & 0 \\
%% actual size 1 less & \leftarrow & 1 \\
%% large sub block & \leftarrow & 0 \\
%% sub block size & \leftarrow & $3 \times 2 = 6$ bytes\;
%% \end{tabular}
%% \end{table}
%% \par
%% \noindent
%% Note that although the total length of the sub block is
%% 8 bytes (2 bytes for the header plus the 6 bytes indicated by
%% \VAR{sub block size}),
%% \VAR{actual size 1 less} indicates that only the first 5 bytes
%% contain actual data and the final byte is padding.

%% \clearpage

%% \subsection{Reading Decoding Parameters}

%% {\relsize{-1}
%% \ALGORITHM{\VAR{metadata function}, \VAR{nondecoder data}, \VAR{actual size 1 less}, \VAR{sub block size} from sub block header; block header}{parameters required for block decoding}
%% \SetKwData{METAFUNC}{metadata function}
%% \SetKwData{NONDECODER}{nondecoder data}
%% \eIf{$\text{\NONDECODER} = 0$}{
%%   \Switch{\METAFUNC}{
%%     \uCase{2}{
%%       read decorrelation terms and deltas from decorrelation terms sub block\;
%%       \Return list of signed decorrelation terms and list of unsigned deltas\;
%%     }
%%     \uCase{3}{
%%       read decorrelation weights from decorrelation weights sub block\;
%%       \Return signed decorrelation weight per channel per decorrelation term\;
%%     }
%%     \uCase{4}{
%%       read decorrelation samples from decorrelation samples sub block\;
%%       \Return list of signed decorrelation samples per channel per decorrelation term\;
%%     }
%%     \uCase{5}{
%%       read 2 lists of 3 signed entropy variables\;
%%       \Return 2 lists of signed entropy variables\;
%%     }
%%     \uCase{9}{
%%       read zero bits, one bits and duplicate bits from extended integers sub block\;
%%       \Return zero bits, one bits, duplicate bits values\;
%%     }
%%     \uCase{10}{
%%       read WavPack bitstream\;
%%       \Return list of signed residual values per channel\;
%%     }
%%     \Other{
%%       skip sub block\;
%%     }
%%   }
%% }{
%%   skip sub block\;
%% }
%% \EALGORITHM
%% }

%% \subsubsection{The Decoding Parameters}
%% These parameters will be populated by sub blocks during decoding.
%% Once all the sub blocks have been read,
%% they will be used to transform residual data
%% into 1 or 2 channels of PCM data.
%% \begin{description}
%% \item[decorrelation terms] one signed integer from -3 to 18 per decorrelation pass
%% \item[decorrelation deltas] one unsigned integer per decorrelation pass
%% \item[decorrelation weights] one signed integer per decorrelation pass per channel
%% \item[decorrelation samples] one list of signed integers per decorrelation pass per channel
%% \item[entropy variables] two lists of three signed integers
%% \item[zero bits/one bits/duplicate bits] three unsigned integers indicating extended integers
%% \item[residuals] one list of signed integers per channel
%% \end{description}

%% \clearpage

%% \subsubsection{Decorrelation Pass Parameters}
%% The number of terms in the \VAR{decorrelation terms} sub block
%% determines how many decorrelation passes will run over the
%% block's residual data.
%% Decorrelation weight and decorrelation samples parameters
%% for those passes will be delivered in subsequent sub blocks.
%% \par
%% For example, given a 1 channel block
%% containing a sub block with 5 decorrelation terms,
%% decorrelation pass parameters may be laid out as follows:
%% \begin{figure}[h]
%% \includegraphics{figures/wavpack/decoding_params.pdf}
%% \end{figure}
%% \par
%% \noindent
%% Note that although the parameters are delivered in decrementing order
%% ($\text{term}_4$ to $\text{term}_0$),
%% the decorrelation passes are applied in incrementing order
%% ($\text{term}_0$ to $\text{term}_4$).

\clearpage

\subsection{Reading Decorrelation Terms}
\ALGORITHM{\VAR{actual size 1 less} and \VAR{sub block size} values from sub block header, sub block data}{a list of signed decorrelation term integers, a list of unsigned decorrelation delta integers\footnote{$\text{term}_p$ and $\text{delta}_p$ indicate the term and delta values for decorrelation pass $p$}}
\SetKwData{PASSES}{passes}
\SetKwData{SUBBLOCKSIZE}{sub block size}
\SetKwData{ACTUALSIZEONELESS}{actual size 1 less}
\SetKwData{TERM}{term}
\SetKwData{DELTA}{delta}
\SetKw{OR}{or}
\SetKw{KwDownTo}{downto}
\eIf{$\text{\ACTUALSIZEONELESS} = 0$}{
  \PASSES $\leftarrow \text{\SUBBLOCKSIZE} \times 2$\;
}{
  \PASSES $\leftarrow \text{\SUBBLOCKSIZE} \times 2 - 1$\;
}
\ASSERT $\text{\PASSES} \leq 16$\;
\BlankLine
\For(\tcc*[f]{populate in reverse order}){p = \PASSES \emph{\KwDownTo}0}{
  $\text{\TERM}_p \leftarrow$ (\READ 5 unsigned bits) - 5\;
  \ASSERT $\text{\TERM}_p$ \IN \texttt{[-3, -2, -1, 1, 2, 3, 4, 5, 6, 7, 8, 17, 18]}
  \BlankLine
  $\text{\DELTA}_p \leftarrow$ \READ 3 unsigned bits\;
}
\Return decorrelation \TERM and decorrelation \DELTA lists\;
\EALGORITHM

\clearpage

\subsubsection{Reading Decorrelation Terms Example}

\begin{figure}[h]
\includegraphics{figures/wavpack/terms_parse.pdf}
\end{figure}
\begin{center}
{\renewcommand{\arraystretch}{1.25}
\begin{tabular}{>{$}r<{$}>{$}c<{$}>{$}r<{$}|>{$}r<{$}>{$}r<{$}>{$}r<{$}}
\text{decorrelation term}_4 & \leftarrow & 23 - 5 = 18 &
\text{decorrelation delta}_4 & \leftarrow & 2 \\
\text{decorrelation term}_3 & \leftarrow & 23 - 5 = 18 &
\text{decorrelation delta}_3 & \leftarrow & 2 \\
\text{decorrelation term}_2 & \leftarrow & 7 - 5 = 2 &
\text{decorrelation delta}_2 & \leftarrow & 2 \\
\text{decorrelation term}_1 & \leftarrow & 22 - 5 = 17 &
\text{decorrelation delta}_1 & \leftarrow & 2 \\
\text{decorrelation term}_0 & \leftarrow & 8 - 5 = 3 &
\text{decorrelation delta}_0 & \leftarrow & 2 \\
\end{tabular}
\renewcommand{\arraystretch}{1.0}
}
\end{center}

\clearpage

\subsection{Reading Decorrelation Weights}
\ALGORITHM{\VAR{mono output} and \VAR{false stereo} from block header, decorrelation terms count\footnote{from the decorrelation terms sub block, which must be read prior to this sub block}, \VAR{actual size 1 less} and \VAR{sub block size} values from sub block header, sub block data}{a list of signed weight integers per channel\footnote{$\text{weight}_{p~c}$ indicates weight value decorrelation pass $p$, channel $c$}}
\SetKwData{MONO}{mono output}
\SetKwData{FALSESTEREO}{false stereo}
\SetKwData{SUBBLOCKSIZE}{sub block size}
\SetKwData{ACTUALSIZEONELESS}{actual size 1 less}
\SetKwData{WEIGHTCOUNT}{weight count}
\SetKwData{TERMCOUNT}{term count}
\SetKwData{WEIGHTVAL}{weight value}
\SetKwData{WEIGHT}{weight}
\SetKw{AND}{and}
\tcc{read as many 8 bit weight values as possible}
\eIf{$\text{\ACTUALSIZEONELESS} = 0$}{
  \WEIGHTCOUNT $\leftarrow \text{\SUBBLOCKSIZE} \times 2$\;
}{
 \WEIGHTCOUNT $\leftarrow \text{\SUBBLOCKSIZE} \times 2 - 1$\;
}
\For{i = 0 \emph{\KwTo}\WEIGHTCOUNT}{
  $\text{value}_i \leftarrow$ \READ 8 signed bits\;
  $\text{\WEIGHTVAL}_i \leftarrow\begin{cases}
\text{value}_i \times 2 ^ 3 + \left\lfloor\frac{\text{value}_i \times 2 ^ 3 + 2 ^ 6}{2 ^ 7}\right\rfloor & \text{if }\text{value}_i > 0 \\
0 & \text{if }\text{value}_i = 0 \\
\text{value}_i \times 2 ^ 3 & \text{if }\text{value}_i < 0
\end{cases}$\;
}
\BlankLine
\tcc{populate weight values by channel, in reverse order}
\eIf(\tcc*[f]{two channels}){$\text{\MONO} = 0$ \AND $\text{\FALSESTEREO} = 0$}{
  \ASSERT $\lfloor\WEIGHTCOUNT \div 2\rfloor \leq \TERMCOUNT$\;
  \For{i = 0 \emph{\KwTo}$\lfloor\WEIGHTCOUNT \div 2\rfloor$}{
    $\text{\WEIGHT}_{(\TERMCOUNT - i - 1)~0} \leftarrow \text{\WEIGHTVAL}_{i \times 2}$\;
    $\text{\WEIGHT}_{(\TERMCOUNT - i - 1)~1} \leftarrow \text{\WEIGHTVAL}_{i \times 2 + 1}$\;
  }
  \For{i = $\lfloor\WEIGHTCOUNT \div 2\rfloor$ \emph{\KwTo}\TERMCOUNT}{
    $\text{\WEIGHT}_{(\TERMCOUNT - i - 1)~0} \leftarrow 0$\;
    $\text{\WEIGHT}_{(\TERMCOUNT - i - 1)~1} \leftarrow 0$\;
  }
  \Return a \WEIGHT value per pass, per channel\;
}(\tcc*[f]{one channel}){
  \ASSERT $\WEIGHTCOUNT \leq \TERMCOUNT$\;
  \For{i = 0 \emph{\KwTo}\WEIGHTCOUNT}{
    $\text{\WEIGHT}_{(\TERMCOUNT - i - 1)~0} \leftarrow \text{\WEIGHTVAL}_{i}$\;
  }
  \For{i = \WEIGHTCOUNT \emph{\KwTo}\TERMCOUNT}{
    $\text{\WEIGHT}_{(\TERMCOUNT - i - 1)~0} \leftarrow 0$\;
  }
  \Return a \WEIGHT value per pass\;
}
\EALGORITHM

\clearpage

\subsubsection{Reading Decorrelation Weights Example}
Given a 2 channel block containing 5 decorrelation terms:
\begin{figure}[h]
\includegraphics{figures/wavpack/decorr_weights_parse.pdf}
\end{figure}
\begin{center}
{\renewcommand{\arraystretch}{1.25}
\begin{tabular}{r|r|>{$}r<{$}}
$i$ & $\text{value}_i$ & \text{weight value}_i \\
\hline
0 & 6 & 6 \times 2 ^ 3 + \lfloor(6 \times 2 ^ 3 + 2 ^ 6) \div 2 ^ 7\rfloor = 48 \\
1 & 6 & 6 \times 2 ^ 3 + \lfloor(6 \times 2 ^ 3 + 2 ^ 6) \div 2 ^ 7\rfloor = 48 \\
2 & 6 & 6 \times 2 ^ 3 + \lfloor(6 \times 2 ^ 3 + 2 ^ 6) \div 2 ^ 7\rfloor = 48 \\
3 & 6 & 6 \times 2 ^ 3 + \lfloor(6 \times 2 ^ 3 + 2 ^ 6) \div 2 ^ 7\rfloor = 48 \\
4 & 4 & 4 \times 2 ^ 3 + \lfloor(4 \times 2 ^ 3 + 2 ^ 6) \div 2 ^ 7\rfloor = 32 \\
5 & 4 & 4 \times 2 ^ 3 + \lfloor(4 \times 2 ^ 3 + 2 ^ 6) \div 2 ^ 7\rfloor = 32 \\
6 & 6 & 6 \times 2 ^ 3 + \lfloor(6 \times 2 ^ 3 + 2 ^ 6) \div 2 ^ 7\rfloor = 48 \\
7 & 6 & 6 \times 2 ^ 3 + \lfloor(6 \times 2 ^ 3 + 2 ^ 6) \div 2 ^ 7\rfloor = 48 \\
8 & 2 & 2 \times 2 ^ 3 + \lfloor(2 \times 2 ^ 3 + 2 ^ 6) \div 2 ^ 7\rfloor = 16 \\
9 & 3 & 3 \times 2 ^ 3 + \lfloor(3 \times 2 ^ 3 + 2 ^ 6) \div 2 ^ 7\rfloor = 24 \\
\end{tabular}
\renewcommand{\arraystretch}{1.0}
}
\end{center}
\begin{center}
\begin{tabular}{>{$}r<{$}||>{$}r<{$}}
\text{weight}_{4~0} = \text{weight value}_0 = 48 &
\text{weight}_{4~1} = \text{weight value}_1 = 48 \\
\text{weight}_{3~0} = \text{weight value}_2 = 48 &
\text{weight}_{3~1} = \text{weight value}_3 = 48 \\
\text{weight}_{2~0} = \text{weight value}_4 = 32 &
\text{weight}_{2~1} = \text{weight value}_5 = 32 \\
\text{weight}_{1~0} = \text{weight value}_6 = 48 &
\text{weight}_{1~1} = \text{weight value}_7 = 48 \\
\text{weight}_{0~0} = \text{weight value}_8 = 16 &
\text{weight}_{0~1} = \text{weight value}_9 = 24 \\
\end{tabular}
\end{center}

\clearpage

\subsection{Reading Decorrelation Samples}
{\relsize{-2}
\ALGORITHM{\VAR{mono output} and \VAR{false stereo} from block header, decorrelation terms, sub block size and data}{a list of signed decorrelation sample lists per channel per decorrelation term\footnote{\relsize{-1}$\text{sample}_{p~c~s}$ indicates the $s$th sample of decorrelation pass $p$ for channel $c$}}
\SetKwData{MONO}{mono output}
\SetKwData{FALSESTEREO}{false stereo}
\SetKwData{SAMPLE}{sample}
\SetKwData{TOTALSAMPLES}{sample count}
\SetKwData{TERMCOUNT}{term count}
\SetKwData{TERM}{term}
\SetKwFunction{EXP}{wv\_exp2}
\SetKw{KwDownTo}{downto}
\SetKw{AND}{and}
\eIf(\tcc*[f]{2 channels}){$\text{\MONO} = 0$ \AND $\text{\FALSESTEREO} = 0$}{
  \For{p = \TERMCOUNT \emph{\KwDownTo}0}{
    \uIf(\tcc*[f]{2 samples per channel}){$17 \leq \text{\TERM}_p \leq 18$}{
      \eIf{$\text{sub block bytes remaining} \geq 8$}{
        $\text{\SAMPLE}_{p~0~0} \leftarrow \text{read \EXP value}$\;
        $\text{\SAMPLE}_{p~0~1} \leftarrow \text{read \EXP value}$\;
        $\text{\SAMPLE}_{p~1~0} \leftarrow \text{read \EXP value}$\;
        $\text{\SAMPLE}_{p~1~1} \leftarrow \text{read \EXP value}$\;
      }{
        $\text{\SAMPLE}_{p~0} \leftarrow \texttt{[0, 0]}$\;
        $\text{\SAMPLE}_{p~1} \leftarrow \texttt{[0, 0]}$\;
      }
    }
    \uElseIf(\tcc*[f]{"term" samples per channel}){$1 \leq \text{\TERM}_p \leq 8$}{
      \eIf{$\text{sub block bytes remaining} \geq (\text{\TERM}_p \times 4)$}{
        \For{s = 0 \emph{\KwTo}$\text{\TERM}_p$}{
          $\text{\SAMPLE}_{p~0~s} \leftarrow \text{read \EXP value}$\;
          $\text{\SAMPLE}_{p~1~s} \leftarrow \text{read \EXP value}$\;
        }
      }{
        \For{s = 0 \emph{\KwTo}$\text{\TERM}_p$}{
          $\text{\SAMPLE}_{p~0~s} \leftarrow 0$\;
          $\text{\SAMPLE}_{p~1~s} \leftarrow 0$\;
        }
      }
    }
    \ElseIf(\tcc*[f]{1 sample per channel}){$-3 \leq \text{\TERM}_p \leq -1$}{
      \eIf{$\text{sub block bytes remaining} \geq 4$}{
        $\text{\SAMPLE}_{p~0~0} \leftarrow \text{read \EXP value}$\;
        $\text{\SAMPLE}_{p~1~0} \leftarrow \text{read \EXP value}$\;
      }{
        $\text{\SAMPLE}_{p~0~0} \leftarrow 0$\;
        $\text{\SAMPLE}_{p~1~0} \leftarrow 0$\;
      }
    }
  }
  \Return $\text{\SAMPLE}$ lists per pass, per channel
}(\tcc*[f]{1 channel}){
  \For{p = \TERMCOUNT \emph{\KwDownTo}0}{
    \uIf(\tcc*[f]{2 samples per channel}){$17 \leq \text{\TERM}_p \leq 18$}{
      \eIf{$\text{sub block bytes remaining} \geq 4$}{
        $\text{\SAMPLE}_{p~0~0} \leftarrow \text{read \EXP value}$\;
        $\text{\SAMPLE}_{p~0~1} \leftarrow \text{read \EXP value}$\;
      }{
        $\text{\SAMPLE}_{p~0} \leftarrow \texttt{[0, 0]}$\;
      }
    }
    \ElseIf(\tcc*[f]{"term" samples per channel}){$1 \leq \text{\TERM}_p \leq 8$}{
      \eIf{$\text{sub block bytes remaining} \geq (\text{\TERM}_p \times 2)$}{
        \For{s = 0 \emph{\KwTo}$\text{\TERM}_p$}{
          $\text{\SAMPLE}_{p~0~s} \leftarrow \text{read \EXP value}$\;
        }
      }{
        \For{s = 0 \emph{\KwTo}$\text{\TERM}_p$}{
          $\text{\SAMPLE}_{p~0~s} \leftarrow 0$\;
        }
      }
    }
  }
  \Return $\text{\SAMPLE}$ lists per pass\;
}
\EALGORITHM
}

\clearpage

\subsubsection{Reading wv\_exp2 Values}
\label{wavpack_wvexp2}
{\relsize{-1}
\ALGORITHM{2 bytes of sub block data}{a signed value}
\SetKwFunction{EXP}{wexp}
$value \leftarrow$ \READ 16 signed bits\;
\BlankLine
\uIf{$-32768 \leq value < -2304$}{
  \Return $-(\EXP(-value \bmod{256}) \times 2 ^ {\lfloor -value \div 2 ^ 8 \rfloor - 9})$\;
}
\uElseIf{$-2304 \leq value < 0$}{
  \Return $-\lfloor \EXP(-value \bmod{256}) \div 2 ^ {9 - \lfloor -value \div 2 ^ 8 \rfloor} \rfloor$\;
}
\uElseIf{$0 \leq value \leq 2304$}{
  \Return $\lfloor \EXP(value \bmod{256}) \div 2 ^ {9 - \lfloor value \div 2 ^ 8 \rfloor} \rfloor$\;
}
\ElseIf{$2304 < value \leq 32767$}{
  \Return $\EXP(value \bmod{256}) \times 2 ^ {\lfloor value \div 2 ^ 8 \rfloor - 9}$\;
}
\EALGORITHM
}
\par
\noindent
where \texttt{wexp}(\textit{x}) is defined from the following table:
\vskip .10in
\par
\noindent
{\relsize{-3}\ttfamily
\begin{tabular}{| c | c | c | c | c | c | c | c | c | c | c | c | c | c | c | c | c |}
\hline
& 0x?0 & 0x?1 & 0x?2 & 0x?3 & 0x?4 & 0x?5 & 0x?6 & 0x?7 & 0x?8 & 0x?9 & 0x?A & 0x?B & 0x?C & 0x?D & 0x?E & 0x?F \\
\hline
0x0? & 256 & 257 & 257 & 258 & 259 & 259 & 260 & 261 & 262 & 262 & 263 & 264 & 264 & 265 & 266 & 267 \\
0x1? & 267 & 268 & 269 & 270 & 270 & 271 & 272 & 272 & 273 & 274 & 275 & 275 & 276 & 277 & 278 & 278 \\
0x2? & 279 & 280 & 281 & 281 & 282 & 283 & 284 & 285 & 285 & 286 & 287 & 288 & 288 & 289 & 290 & 291 \\
0x3? & 292 & 292 & 293 & 294 & 295 & 296 & 296 & 297 & 298 & 299 & 300 & 300 & 301 & 302 & 303 & 304 \\
0x4? & 304 & 305 & 306 & 307 & 308 & 309 & 309 & 310 & 311 & 312 & 313 & 314 & 314 & 315 & 316 & 317 \\
0x5? & 318 & 319 & 320 & 321 & 321 & 322 & 323 & 324 & 325 & 326 & 327 & 328 & 328 & 329 & 330 & 331 \\
0x6? & 332 & 333 & 334 & 335 & 336 & 337 & 337 & 338 & 339 & 340 & 341 & 342 & 343 & 344 & 345 & 346 \\
0x7? & 347 & 348 & 349 & 350 & 350 & 351 & 352 & 353 & 354 & 355 & 356 & 357 & 358 & 359 & 360 & 361 \\
0x8? & 362 & 363 & 364 & 365 & 366 & 367 & 368 & 369 & 370 & 371 & 372 & 373 & 374 & 375 & 376 & 377 \\
0x9? & 378 & 379 & 380 & 381 & 382 & 383 & 384 & 385 & 386 & 387 & 388 & 389 & 391 & 392 & 393 & 394 \\
0xA? & 395 & 396 & 397 & 398 & 399 & 400 & 401 & 402 & 403 & 405 & 406 & 407 & 408 & 409 & 410 & 411 \\
0xB? & 412 & 413 & 415 & 416 & 417 & 418 & 419 & 420 & 421 & 422 & 424 & 425 & 426 & 427 & 428 & 429 \\
0xC? & 431 & 432 & 433 & 434 & 435 & 436 & 438 & 439 & 440 & 441 & 442 & 444 & 445 & 446 & 447 & 448 \\
0xD? & 450 & 451 & 452 & 453 & 454 & 456 & 457 & 458 & 459 & 461 & 462 & 463 & 464 & 466 & 467 & 468 \\
0xE? & 470 & 471 & 472 & 473 & 475 & 476 & 477 & 478 & 480 & 481 & 482 & 484 & 485 & 486 & 488 & 489 \\
0xF? & 490 & 492 & 493 & 494 & 496 & 497 & 498 & 500 & 501 & 502 & 504 & 505 & 506 & 508 & 509 & 511 \\
\hline
\end{tabular}
}

\subsubsection{Reading Decorrelation Samples Example}
Given a stereo block containing the sub-block:
\begin{figure}[h]
\includegraphics{figures/wavpack/decorr_samples_parse.pdf}
\end{figure}
\begin{center}
{\relsize{-2}
\begin{tabular}{r|r|r|>{$}r<{$}|>{$}r<{$}}
$p$ & $\text{term}_p$ & $s$ &
\text{sample}_{p~0~s} &
\text{sample}_{p~1~s} \\
\hline
4 & 18 & 0 &
-\lfloor \texttt{wexp}(1841 \bmod{256}) \div 2 ^ {9 - \lfloor 1841 \div 2 ^ 8 \rfloor} \rfloor = -73 &
\lfloor \EXP(1487 \bmod{256}) \div 2 ^ {9 - \lfloor 1487 \div 2 ^ 8 \rfloor} \rfloor = 28 \\
& & 1 &
-\lfloor \EXP(1865 \bmod{256}) \div 2 ^ {9 - \lfloor 1865 \div 2 ^ 8 \rfloor} \rfloor = -78 &
\lfloor \EXP(1459 \bmod{256}) \div 2 ^ {9 - \lfloor 1459 \div 2 ^ 8 \rfloor} \rfloor = 26 \\
\hline
3 & 18 & 0 & 0 & 0 \\
& & 1 & 0 & 0 \\
\hline
2 & 2 & 0 & 0 & 0 \\
& & 1 & 0 & 0 \\
\hline
1 & 17 & 0 & 0 & 0 \\
& & 1 & 0 & 0 \\
\hline
0 & 3 & 0 & 0 & 0 \\
& & 1 & 0 & 0 \\
& & 2 & 0 & 0 \\
\hline
\end{tabular}
\renewcommand{\arraystretch}{1.0}
}
\end{center}

\clearpage

\subsection{Reading Entropy Variables}
{\relsize{-2}
\ALGORITHM{\VAR{mono output} and \VAR{false stereo} from block header, \VAR{sub block size}, \VAR{actual size 1 less}, sub block data}{2 lists of 3 signed median value integers\footnote{$\text{median}_{c~i}$ indicates the $i$th median of channel $c$}}
\SetKwData{MEDIAN}{median}
\SetKwFunction{EXP}{wv\_exp2}
\SetKwData{MONO}{mono output}
\SetKwData{FALSESTEREO}{false stereo}
\SetKwData{ONELESS}{actual size 1 less}
\SetKwData{SUBBLOCKSIZE}{sub block size}
\SetKw{AND}{and}
\ASSERT $\text{\ONELESS} = 0$\;
\eIf(\tcc*[f]{2 channels}){$\text{\MONO} = 0$ \AND $\text{\FALSESTEREO} = 0$}{
  \ASSERT $\text{\SUBBLOCKSIZE} = 6$\;
  $\text{\MEDIAN}_{0~0} \leftarrow \text{read \EXP value}$\;
  $\text{\MEDIAN}_{0~1} \leftarrow \text{read \EXP value}$\;
  $\text{\MEDIAN}_{0~2} \leftarrow \text{read \EXP value}$\;
  $\text{\MEDIAN}_{1~0} \leftarrow \text{read \EXP value}$\;
  $\text{\MEDIAN}_{1~1} \leftarrow \text{read \EXP value}$\;
  $\text{\MEDIAN}_{1~2} \leftarrow \text{read \EXP value}$\;
}(\tcc*[f]{1 channel}){
  \ASSERT $\text{\SUBBLOCKSIZE} = 3$\;
  $\text{\MEDIAN}_{0~0} \leftarrow \text{read \EXP value}$\;
  $\text{\MEDIAN}_{0~1} \leftarrow \text{read \EXP value}$\;
  $\text{\MEDIAN}_{0~2} \leftarrow \text{read \EXP value}$\;
  $\text{\MEDIAN}_{1} \leftarrow \texttt{[0, 0, 0]}$\;
}
\Return $\text{\MEDIAN}_0$ list and $\text{\MEDIAN}_1$ list\;
\EALGORITHM
}

\subsubsection{Reading Entropy Variables Example}
Given a 2 channel block containing the subframe:
\begin{figure}[h]
\includegraphics{figures/wavpack/entropy_vars_parse.pdf}
\end{figure}
\begin{center}
{\relsize{-1}
\renewcommand{\arraystretch}{1.25}
\begin{tabular}{r|>{$}r<{$}|>{$}r<{$}}
$i$ & \text{median}_{0~i} & \text{median}_{1~i} \\
\hline
0 &
\lfloor \EXP(2018 \bmod{256}) \div 2 ^ {9 - \lfloor 2018 \div 2 ^ 8 \rfloor} \rfloor = 118 &
\lfloor \EXP(2018 \bmod{256}) \div 2 ^ {9 - \lfloor 2018 \div 2 ^ 8 \rfloor} \rfloor = 118 \\
1 &
\lfloor \EXP(2203 \bmod{256}) \div 2 ^ {9 - \lfloor 2203 \div 2 ^ 8 \rfloor} \rfloor = 194 &
\lfloor \EXP(2166 \bmod{256}) \div 2 ^ {9 - \lfloor 2166 \div 2 ^ 8 \rfloor} \rfloor = 176 \\
2 &
\EXP(2389 \bmod{256}) \times 2 ^ {\lfloor 2389 \div 2 ^ 8 \rfloor - 9} = 322 &
\lfloor \EXP(2234 \bmod{256}) \div 2 ^ {9 - \lfloor 2234 \div 2 ^ 8 \rfloor} \rfloor = 212 \\
\end{tabular}
\renewcommand{\arraystretch}{1.0}
}
\end{center}

\clearpage

\subsection{Reading Bitstream}
{\relsize{-1}
\ALGORITHM{\VAR{mono output}, \VAR{false stereo} and \VAR{block samples} from block header,
median values\footnote{from the entropy variables sub block, which must be read prior to this sub block}, sub block data}{a list of signed residual values per channel\footnote{$\text{residual}_{c~i}$ indicates the $i$th residual of channel $c$}}
\SetKwData{MEDIAN}{median}
\SetKwData{MONO}{mono output}
\SetKwData{FALSESTEREO}{false stereo}
\SetKwData{BLOCKSAMPLES}{block samples}
\SetKwData{CHANNELS}{channel count}
\SetKwData{UNDEFINED}{undefined}
\SetKwData{MEDIAN}{median}
\SetKwData{RESIDUAL}{residual}
\SetKwData{ZEROES}{zeroes}
\SetKw{AND}{and}
\eIf{$\text{\MONO} = 0$ \AND $\text{\FALSESTEREO} = 0$}{
  \CHANNELS $\leftarrow 2$\;
}{
  \CHANNELS $\leftarrow 1$\;
}
$u_{-1} \leftarrow \UNDEFINED$\;
$i \leftarrow 0$\;
\While{$i < (\BLOCKSAMPLES \times \CHANNELS)$}{
  \eIf{$(u_{i - 1} = \UNDEFINED)$ \AND
    $(\text{\MEDIAN}_{0~0} < 2)$ \AND
    $(\text{\MEDIAN}_{1~0} < 2)$}{
    $\ZEROES \leftarrow$ read modified Elias gamma code\;
    \If(\tcc*[f]{handle long run of 0 residuals}){$\text{\ZEROES} > 0$}{
      \For{j = 0 \emph{\KwTo}\ZEROES}{
        $\text{\RESIDUAL}_{(i \bmod \CHANNELS)~\lfloor i \div \CHANNELS \rfloor} \leftarrow $ 0\;
        $i \leftarrow i + 1$\;
      }
      \BlankLine
      $\text{\MEDIAN}_{0~0} \leftarrow 0$\;
      $\text{\MEDIAN}_{0~1} \leftarrow 0$\;
      $\text{\MEDIAN}_{0~2} \leftarrow 0$\;
      $\text{\MEDIAN}_{1~0} \leftarrow 0$\;
      $\text{\MEDIAN}_{1~1} \leftarrow 0$\;
      $\text{\MEDIAN}_{1~2} \leftarrow 0$\;
    }
    \If{$i < (\BLOCKSAMPLES \times \CHANNELS)$}{
      $\text{\RESIDUAL}_{(i \bmod \CHANNELS)~\lfloor i \div \CHANNELS \rfloor} \leftarrow $ read residual value\;
      $i \leftarrow i + 1$\;
    }
  }{
    $\text{\RESIDUAL}_{(i \bmod \CHANNELS)~\lfloor i \div \CHANNELS \rfloor} \leftarrow $ read residual value\;
    $i \leftarrow i + 1$\;
  }
}
\Return a list of signed \RESIDUAL values per channel\;
\EALGORITHM
}

\subsubsection{Reading Modified Elias Gamma Code}
{\relsize{-1}
\ALGORITHM{the sub block bitstream}{an unsigned integer}
    $t \leftarrow$ \UNARY with stop bit 0\;
    \eIf{$t > 1$}{
      $p \leftarrow$ \READ $(t - 1)$ unsigned bits\;
      \Return $2 ^ {(t - 1)} + p$\;
    }{
      \Return $t$\;
    }
\EALGORITHM
}

\clearpage

\subsubsection{Reading Residual Value}
{\relsize{-1}
\ALGORITHM{sub block bitstream, previous residual's unary value $u_{i - 1}$, \VAR{median} values for channel $c$}{a signed residual value; new unary value $u_i$, updated \VAR{median} values for channel $c$}
\SetKwData{UNDEFINED}{undefined}
\SetKwData{MEDIAN}{median}
\SetKwData{RESIDUAL}{residual}
\tcc{determine new "u" and "m" values}
\uIf{$u_{i - 1} = \UNDEFINED$}{
  $u_i \leftarrow$ \UNARY with stop bit 0\;
  \If{$u_i = 16$}{
    $c_i \leftarrow$ read modified Elias gamma code\;
    $u_i \leftarrow u_i + c_i$\;
  }
  $m_i \leftarrow \lfloor u_i \div 2\rfloor$\;
}
\uElseIf{$u_{i - 1} \bmod 2 = 1$}{
  $u_i \leftarrow$ \UNARY with stop bit 0\;
  \If{$u_i = 16$}{
    $c_i \leftarrow$ read modified Elias gamma code\;
    $u_i \leftarrow u_i + c_i$\;
  }
  $m_i \leftarrow \lfloor u_i \div 2\rfloor + 1$\;
}
\ElseIf{$u_{i - 1} \bmod 2 = 0$}{
  $u_i \leftarrow \UNDEFINED$\;
  $m_i \leftarrow 0$\;
}
\BlankLine
\tcc{determine "base", "add" and new median values}
\Switch{$m_i$}{
  \uCase{0}{
    $base \leftarrow 0$\;
    $add \leftarrow \lfloor\text{\MEDIAN}_{c~0} \div 2 ^ 4\rfloor$\;
    $\text{\MEDIAN}_{c~0} \leftarrow \text{\MEDIAN}_{c~0} - \lfloor(\text{\MEDIAN}_{c~0} + 126) \div 2 ^ 7\rfloor \times 2$\;
  }
  \uCase{1}{
    $base \leftarrow \lfloor\text{\MEDIAN}_{c~0} \div 2 ^ 4\rfloor + 1$\;
    $add \leftarrow \lfloor\text{\MEDIAN}_{c~1} \div 2 ^ 4\rfloor$\;
    $\text{\MEDIAN}_{c~0} \leftarrow \text{\MEDIAN}_{c~0} + \lfloor(\text{\MEDIAN}_{c~0} + 128) \div 2 ^ 7\rfloor \times 5$\;
    $\text{\MEDIAN}_{c~1} \leftarrow \text{\MEDIAN}_{c~1} - \lfloor(\text{\MEDIAN}_{c~1} + 62) \div 2 ^ 6\rfloor \times 2$\;
  }
  \uCase{2}{
    $base \leftarrow (\lfloor\text{\MEDIAN}_{c~0} \div 2 ^ 4\rfloor + 1) + (\lfloor\text{\MEDIAN}_{c~1} \div 2 ^ 4\rfloor + 1)$\;
    $add \leftarrow \lfloor\text{\MEDIAN}_{c~2} \div 2 ^ 4\rfloor$\;
    $\text{\MEDIAN}_{c~0} \leftarrow \text{\MEDIAN}_{c~0} + \lfloor(\text{\MEDIAN}_{c~0} + 128) \div 2 ^ 7\rfloor \times 5$\;
    $\text{\MEDIAN}_{c~1} \leftarrow \text{\MEDIAN}_{c~1} + \lfloor(\text{\MEDIAN}_{c~1} + 64) \div 2 ^ 6\rfloor \times 5$\;
    $\text{\MEDIAN}_{c~2} \leftarrow \text{\MEDIAN}_{c~2} - \lfloor(\text{\MEDIAN}_{c~2} + 30) \div 2 ^ 5\rfloor \times 2$\;
  }
  \Other{
    $base \leftarrow (\lfloor\text{\MEDIAN}_{c~0} \div 2 ^ 4\rfloor + 1) + (\lfloor\text{\MEDIAN}_{c~1} \div 2 ^ 4\rfloor + 1) + ((\lfloor\text{\MEDIAN}_{c~2} \div 2 ^ 4\rfloor + 1) \times (m_i - 2))$\;
    $add \leftarrow \lfloor\text{\MEDIAN}_{c~2} \div 2 ^ 4\rfloor$\;
    $\text{\MEDIAN}_{c~0} \leftarrow \text{\MEDIAN}_{c~0} + \lfloor(\text{\MEDIAN}_{c~0} + 128) \div 2 ^ 7\rfloor \times 5$\;
    $\text{\MEDIAN}_{c~1} \leftarrow \text{\MEDIAN}_{c~1} + \lfloor(\text{\MEDIAN}_{c~1} + 64) \div 2 ^ 6\rfloor \times 5$\;
    $\text{\MEDIAN}_{c~2} \leftarrow \text{\MEDIAN}_{c~2} + \lfloor(\text{\MEDIAN}_{c~2} + 32) \div 2 ^ 5\rfloor \times 5$\;
  }
}

\EALGORITHM
}

\clearpage

{\relsize{-1}
\begin{algorithm}[H]
\DontPrintSemicolon
\SetKw{READ}{read}
\tcc{determine final residual value}
\eIf{$add = 0$}{
  $unsigned \leftarrow base$\;
}{
  $p \leftarrow \lfloor\log_2(add)\rfloor$\;
  $e \leftarrow 2 ^ {p + 1} - add - 1$\;
  $r_i \leftarrow$ \READ $p$ unsigned bits\;
  \eIf{$r \geq e$}{
    $b_i \leftarrow$ \READ 1 unsigned bit\;
    $unsigned \leftarrow base + (r_i \times 2) - e + b_i$\;
  }{
    $unsigned \leftarrow base + r_i$\;
  }
}
\BlankLine
$sign \leftarrow$ \READ 1 unsigned bit\;
\eIf{$sign = 1$}{
  \Return $(-unsigned - 1)$ along with unary value $u_i$ and updated medians\;
}{
  \Return $unsigned$ along with unary value $u_i$ and updated medians\;
}
\end{algorithm}
}

\subsubsection{Residual Parsing Example}
\begin{figure}[h]
\includegraphics{figures/wavpack/residuals_parse.pdf}
\end{figure}

\clearpage

\begin{landscape}

Given a 2 channel block with $\text{medians}_0 = \texttt{[118, 194, 322]}$ and
$\text{medians}_1 = \texttt{[118, 176, 212]}$:
\vskip .10in

{\relsize{-2}
\renewcommand{\arraystretch}{1.5}
\begin{tabular}{|>{$}r<{$}||>{$}r<{$}|>{$}r<{$}|>{$}r<{$}|>{$}r<{$}||>{$}r<{$}|>{$}r<{$}|>{$}r<{$}|>{$}r<{$}|>{$}r<{$}|}
i & u_i & m_i &
\text{base} & \text{add} & \text{median}_{c~0} & \text{median}_{c~1} & \text{median}_{c~2} \\
\hline
0 &
7 &
\lfloor 7 \div 2 \rfloor = 3 &
2 + \left\lfloor\frac{118}{2 ^ 4}\right\rfloor + \left\lfloor\frac{194}{2 ^ 4}\right\rfloor + \left(\left\lfloor\frac{322}{2 ^ 4}\right\rfloor \times 1\right) = 42 & \left\lfloor\frac{322}{2 ^ 4}\right\rfloor = 20 & 118 + 5 = 123 & 194 + 20 = 214 & 322 + 55 = 377
\\
1 &
3 &
\lfloor 3 \div 2 \rfloor + 1 = 2 &
2 + \left\lfloor\frac{118}{2 ^ 4}\right\rfloor + \left\lfloor\frac{176}{2 ^ 4}\right\rfloor = 20 & \left\lfloor\frac{212}{2 ^ 4}\right\rfloor = 13 & 118 + 5 = 123 & 176 + 15 = 191 & 212 - 35 = 198
\\
\hline
2 &
3 &
\lfloor 3 \div 2 \rfloor + 1 = 2 &
2 + \left\lfloor\frac{123}{2 ^ 4}\right\rfloor + \left\lfloor\frac{214}{2 ^ 4}\right\rfloor = 22 & \left\lfloor\frac{377}{2 ^ 4}\right\rfloor = 23 & 123 + 5 = 128 & 214 + 20 = 234 & 377 - 60 = 353
\\
3 &
3 &
\lfloor 3 \div 2 \rfloor + 1 = 2 &
2 + \left\lfloor\frac{123}{2 ^ 4}\right\rfloor + \left\lfloor\frac{191}{2 ^ 4}\right\rfloor = 20 & \left\lfloor\frac{198}{2 ^ 4}\right\rfloor = 12 & 123 + 5 = 128 & 191 + 15 = 206 & 198 - 35 = 184
\\
\hline
4 &
1 &
\lfloor 1 \div 2 \rfloor + 1 = 1 &
1 + \left\lfloor\frac{128}{2 ^ 4}\right\rfloor = 9 & \left\lfloor\frac{234}{2 ^ 4}\right\rfloor = 14 & 128 + 10 = 138 & 234 - 8 = 226 & 353
\\
5 &
4 &
\lfloor 4 \div 2 \rfloor + 1 = 3 &
2 + \left\lfloor\frac{128}{2 ^ 4}\right\rfloor + \left\lfloor\frac{206}{2 ^ 4}\right\rfloor + \left(\left\lfloor\frac{184}{2 ^ 4}\right\rfloor \times 1\right) = 34 & \left\lfloor\frac{184}{2 ^ 4}\right\rfloor = 11 & 128 + 10 = 138 & 206 + 20 = 226 & 184 + 30 = 214
\\
\hline
6 &
\textit{undefined} &
0 &
0 & \left\lfloor\frac{138}{2 ^ 4}\right\rfloor = 8 & 138 - 4 = 134 & 226 & 353
\\
7 &
5 &
\lfloor 5 \div 2 \rfloor = 2 &
2 + \left\lfloor\frac{138}{2 ^ 4}\right\rfloor + \left\lfloor\frac{226}{2 ^ 4}\right\rfloor = 24 & \left\lfloor\frac{214}{2 ^ 4}\right\rfloor = 13 & 138 + 10 = 148 & 226 + 20 = 246 & 214 - 35 = 200
\\
\hline
8 &
1 &
\lfloor 1 \div 2 \rfloor + 1 = 1 &
1 + \left\lfloor\frac{134}{2 ^ 4}\right\rfloor = 9 & \left\lfloor\frac{226}{2 ^ 4}\right\rfloor = 14 & 134 + 10 = 144 & 226 - 8 = 218 & 353
\\
9 &
3 &
\lfloor 3 \div 2 \rfloor + 1 = 2 &
2 + \left\lfloor\frac{148}{2 ^ 4}\right\rfloor + \left\lfloor\frac{246}{2 ^ 4}\right\rfloor = 26 & \left\lfloor\frac{200}{2 ^ 4}\right\rfloor = 12 & 148 + 10 = 158 & 246 + 20 = 266 & 200 - 35 = 186
\\
\hline
10 &
3 &
\lfloor 3 \div 2 \rfloor + 1 = 2 &
2 + \left\lfloor\frac{144}{2 ^ 4}\right\rfloor + \left\lfloor\frac{218}{2 ^ 4}\right\rfloor = 24 & \left\lfloor\frac{353}{2 ^ 4}\right\rfloor = 22 & 144 + 10 = 154 & 218 + 20 = 238 & 353 - 55 = 331
\\
11 &
3 &
\lfloor 3 \div 2 \rfloor + 1 = 2 &
2 + \left\lfloor\frac{158}{2 ^ 4}\right\rfloor + \left\lfloor\frac{266}{2 ^ 4}\right\rfloor = 27 & \left\lfloor\frac{186}{2 ^ 4}\right\rfloor = 11 & 158 + 10 = 168 & 266 + 25 = 291 & 186 - 30 = 174
\\
\hline
12 &
5 &
\lfloor 5 \div 2 \rfloor + 1 = 3 &
2 + \left\lfloor\frac{154}{2 ^ 4}\right\rfloor + \left\lfloor\frac{238}{2 ^ 4}\right\rfloor + \left(\left\lfloor\frac{331}{2 ^ 4}\right\rfloor \times 1\right) = 46 & \left\lfloor\frac{331}{2 ^ 4}\right\rfloor = 20 & 154 + 10 = 164 & 238 + 20 = 258 & 331 + 55 = 386
\\
13 &
1 &
\lfloor 1 \div 2 \rfloor + 1 = 1 &
1 + \left\lfloor\frac{168}{2 ^ 4}\right\rfloor = 11 & \left\lfloor\frac{291}{2 ^ 4}\right\rfloor = 18 & 168 + 10 = 178 & 291 - 10 = 281 & 174
\\
\hline
14 &
5 &
\lfloor 5 \div 2 \rfloor + 1 = 3 &
2 + \left\lfloor\frac{164}{2 ^ 4}\right\rfloor + \left\lfloor\frac{258}{2 ^ 4}\right\rfloor + \left(\left\lfloor\frac{386}{2 ^ 4}\right\rfloor \times 1\right) = 53 & \left\lfloor\frac{386}{2 ^ 4}\right\rfloor = 24 & 164 + 10 = 174 & 258 + 25 = 283 & 386 + 65 = 451
\\
15 &
1 &
\lfloor 1 \div 2 \rfloor + 1 = 1 &
1 + \left\lfloor\frac{178}{2 ^ 4}\right\rfloor = 12 & \left\lfloor\frac{281}{2 ^ 4}\right\rfloor = 17 & 178 + 10 = 188 & 281 - 10 = 271 & 174
\\
\hline
16 &
4 &
\lfloor 4 \div 2 \rfloor + 1 = 3 &
2 + \left\lfloor\frac{174}{2 ^ 4}\right\rfloor + \left\lfloor\frac{283}{2 ^ 4}\right\rfloor + \left(\left\lfloor\frac{451}{2 ^ 4}\right\rfloor \times 1\right) = 58 & \left\lfloor\frac{451}{2 ^ 4}\right\rfloor = 28 & 174 + 10 = 184 & 283 + 25 = 308 & 451 + 75 = 526
\\
17 &
\textit{undefined} &
0 &
0 & \left\lfloor\frac{188}{2 ^ 4}\right\rfloor = 11 & 188 - 4 = 184 & 271 & 174
\\
\hline
18 &
6 &
\lfloor 6 \div 2 \rfloor = 3 &
2 + \left\lfloor\frac{184}{2 ^ 4}\right\rfloor + \left\lfloor\frac{308}{2 ^ 4}\right\rfloor + \left(\left\lfloor\frac{526}{2 ^ 4}\right\rfloor \times 1\right) = 65 & \left\lfloor\frac{526}{2 ^ 4}\right\rfloor = 32 & 184 + 10 = 194 & 308 + 25 = 333 & 526 + 85 = 611
\\
19 &
\textit{undefined} &
0 &
0 & \left\lfloor\frac{184}{2 ^ 4}\right\rfloor = 11 & 184 - 4 = 180 & 271 & 174
\\
\hline
\end{tabular}
\renewcommand{\arraystretch}{1.0}
}

\clearpage

\begin{table}[h]
{\relsize{-2}
\renewcommand{\arraystretch}{1.5}
\begin{tabular}{|>{$}r<{$}|>{$}r<{$}|>{$}r<{$}||>{$}r<{$}|>{$}r<{$}|>{$}r<{$}|>{$}r<{$}|>{$}r<{$}|>{$}r<{$}|>{$}r<{$}|}
i & \text{base} & \text{add} & p & e & r_i & b_i & unsigned & sign_i & residual_i \\
\hline
0 &
42 & 20 &
\lfloor\log_2(20)\rfloor = 4 &
2 ^ {4 + 1} - 20 - 1 = 11 &
14 &
1 & 42 + (14 \times 2) - 11 + 1 = 60 &
1 & -60 - 1 = -61
\\
1 &
20 & 13 &
\lfloor\log_2(13)\rfloor = 3 &
2 ^ {3 + 1} - 13 - 1 = 2 &
6 &
1 & 20 + (6 \times 2) - 2 + 1 = 31 &
0 & 31
\\
\hline
2 &
22 & 23 &
\lfloor\log_2(23)\rfloor = 4 &
2 ^ {4 + 1} - 23 - 1 = 8 &
9 &
0 & 22 + (9 \times 2) - 8 + 0 = 32 &
1 & -32 - 1 = -33
\\
3 &
20 & 12 &
\lfloor\log_2(12)\rfloor = 3 &
2 ^ {3 + 1} - 12 - 1 = 3 &
7 &
1 & 20 + (7 \times 2) - 3 + 1 = 32 &
0 & 32
\\
\hline
4 &
9 & 14 &
\lfloor\log_2(14)\rfloor = 3 &
2 ^ {3 + 1} - 14 - 1 = 1 &
4 &
1 & 9 + (4 \times 2) - 1 + 1 = 17 &
1 & -17 - 1 = -18
\\
5 &
34 & 11 &
\lfloor\log_2(11)\rfloor = 3 &
2 ^ {3 + 1} - 11 - 1 = 4 &
2 &
 & 34 + 2 = 36 &
0 & 36
\\
\hline
6 &
0 & 8 &
\lfloor\log_2(8)\rfloor = 3 &
2 ^ {3 + 1} - 8 - 1 = 7 &
1 &
 & 0 + 1 = 1 &
0 & 1
\\
7 &
24 & 13 &
\lfloor\log_2(13)\rfloor = 3 &
2 ^ {3 + 1} - 13 - 1 = 2 &
7 &
1 & 24 + (7 \times 2) - 2 + 1 = 37 &
0 & 37
\\
\hline
8 &
9 & 14 &
\lfloor\log_2(14)\rfloor = 3 &
2 ^ {3 + 1} - 14 - 1 = 1 &
6 &
0 & 9 + (6 \times 2) - 1 + 0 = 20 &
0 & 20
\\
9 &
26 & 12 &
\lfloor\log_2(12)\rfloor = 3 &
2 ^ {3 + 1} - 12 - 1 = 3 &
6 &
0 & 26 + (6 \times 2) - 3 + 0 = 35 &
0 & 35
\\
\hline
10 &
24 & 22 &
\lfloor\log_2(22)\rfloor = 4 &
2 ^ {4 + 1} - 22 - 1 = 9 &
10 &
0 & 24 + (10 \times 2) - 9 + 0 = 35 &
0 & 35
\\
11 &
27 & 11 &
\lfloor\log_2(11)\rfloor = 3 &
2 ^ {3 + 1} - 11 - 1 = 4 &
4 &
0 & 27 + (4 \times 2) - 4 + 0 = 31 &
0 & 31
\\
\hline
12 &
46 & 20 &
\lfloor\log_2(20)\rfloor = 4 &
2 ^ {4 + 1} - 20 - 1 = 11 &
4 &
 & 46 + 4 = 50 &
0 & 50
\\
13 &
11 & 18 &
\lfloor\log_2(18)\rfloor = 4 &
2 ^ {4 + 1} - 18 - 1 = 13 &
13 &
1 & 11 + (13 \times 2) - 13 + 1 = 25 &
0 & 25
\\
\hline
14 &
53 & 24 &
\lfloor\log_2(24)\rfloor = 4 &
2 ^ {4 + 1} - 24 - 1 = 7 &
8 &
0 & 53 + (8 \times 2) - 7 + 0 = 62 &
0 & 62
\\
15 &
12 & 17 &
\lfloor\log_2(17)\rfloor = 4 &
2 ^ {4 + 1} - 17 - 1 = 14 &
6 &
 & 12 + 6 = 18 &
0 & 18
\\
\hline
16 &
58 & 28 &
\lfloor\log_2(28)\rfloor = 4 &
2 ^ {4 + 1} - 28 - 1 = 3 &
6 &
1 & 58 + (6 \times 2) - 3 + 1 = 68 &
0 & 68
\\
17 &
0 & 11 &
\lfloor\log_2(11)\rfloor = 3 &
2 ^ {3 + 1} - 11 - 1 = 4 &
7 &
0 & 0 + (7 \times 2) - 4 + 0 = 10 &
0 & 10
\\
\hline
18 &
65 & 32 &
\lfloor\log_2(32)\rfloor = 5 &
2 ^ {5 + 1} - 32 - 1 = 31 &
6 &
 & 65 + 6 = 71 &
0 & 71
\\
19 &
0 & 11 &
\lfloor\log_2(11)\rfloor = 3 &
2 ^ {3 + 1} - 11 - 1 = 4 &
0 &
 & 0 + 0 = 0 &
0 & 0
\\
\hline
%% end table here
\end{tabular}
}
\end{table}
\par
\noindent
Resulting in:
\newline
\begin{tabular}{rr}
channel 0 residuals : & \texttt{[-61,~-33,~-18,~~1,~20,~35,~50,~62,~68,~71]}\\
channel 1 residuals : & \texttt{[~31,~~32,~~36,~37,~35,~31,~25,~18,~10,~~0]}\\
\end{tabular}

\end{landscape}

\subsection{Channel Decorrelation}
\ALGORITHM{a list of signed residuals per channel, a list of decorrelation terms and deltas, a decorrelation weight per term and channel, a list of decorrelation samples per term and channel}{a list of signed samples per channel}
\SetKwData{TERMCOUNT}{term count}
\SetKwData{PASS}{pass}
\SetKwData{TERM}{term}
\SetKwData{DELTA}{delta}
\SetKwData{WEIGHT}{weight}
\SetKwData{SAMPLES}{samples}
\SetKwData{RESIDUALS}{residuals}
\eIf{$\text{channel count} = 1$}{
  $\text{\PASS}_{(-1)~0} \leftarrow \text{\RESIDUALS}_{0}$\;
  \For{p = 0 \emph{\KwTo}\TERMCOUNT}{
    decorrelate 1 channel $\text{\PASS}_{(p - 1)~0}$ to $\text{\PASS}_{p~0}$
    \newline
    using $\text{\TERM}_p$, $\text{\DELTA}_p$, $\text{\WEIGHT}_{p~0}$
    and $\text{\SAMPLES}_{p~0}$\;
  }
  \Return $\text{\PASS}_{(\text{\TERMCOUNT} - 1)~0}$\;
}{
  $\text{\PASS}_{-1~0} \leftarrow \text{\RESIDUALS}_{0}$\;
  $\text{\PASS}_{-1~1} \leftarrow \text{\RESIDUALS}_{1}$\;
  \For{p = 0 \emph{\KwTo}\TERMCOUNT}{
    decorrelate 2 channel $\text{\PASS}_{(p - 1)~0}/\text{\PASS}_{(p - 1)~1}$ to $\text{\PASS}_{p~0}/\text{\PASS}_{p~1}$
    \newline
    using $\text{\TERM}_p$, $\text{\DELTA}_p$, $\text{\WEIGHT}_{p~0}$,
    $\text{\WEIGHT}_{p~1}$, $\text{\SAMPLES}_{p~0}$ and $\text{\SAMPLES}_{p~1}$\;
  }
  \Return $\text{\PASS}_{(\text{\TERMCOUNT} - 1)~0}$ and $\text{\PASS}_{(\text{\TERMCOUNT} - 1)~1}$\;
}
\EALGORITHM

\clearpage

\subsubsection{Visualizing Decorrelation Passes}

This is an example of a relatively small set of residuals
being transformed back into a set of samples
over the course of 5 decorrelation passes.
What's important to note is that each pass
adjusts the residual values' overall range.
The number of these passes and their decorrelation terms
are what separate high levels of WavPack compression from low levels.

\begin{figure}[h]
  \subfloat{
    \includegraphics{figures/wavpack/decorrelation0.pdf}
  }
  \subfloat{
    \includegraphics{figures/wavpack/decorrelation1.pdf}
  }
  \newline
  \subfloat{
    \includegraphics{figures/wavpack/decorrelation2.pdf}
  }
  \subfloat{
    \includegraphics{figures/wavpack/decorrelation3.pdf}
  }
  \newline
  \subfloat{
    \includegraphics{figures/wavpack/decorrelation4.pdf}
  }
  \subfloat{
    \includegraphics{figures/wavpack/decorrelation5.pdf}
  }
\end{figure}


\clearpage

\subsubsection{1 Channel Decorrelation Pass}
{\relsize{-1}
\ALGORITHM{a list of signed correlated samples, decorrelation term and delta, decorrelation weight, list of decorrelation samples}{a list of signed decorrelated samples}
\SetKwData{CORRELATED}{correlated}
\SetKwData{DECORRELATED}{decorrelated}
\SetKwData{DECORRSAMPLE}{decorrelation sample}
\SetKwData{WEIGHT}{weight}
\SetKwData{TEMP}{temp}
\SetKw{OR}{or}
\SetKw{XOR}{xor}
\SetKwFunction{APPLYWEIGHT}{apply\_weight}
\SetKwFunction{UPDATEWEIGHT}{update\_weight}
$\text{\WEIGHT}_0 \leftarrow$ decorrelation weight\;
\BlankLine
\uIf{$term = 18$}{
  $\text{\DECORRELATED}_0 \leftarrow \text{\DECORRSAMPLE}_1$\;
  $\text{\DECORRELATED}_1 \leftarrow \text{\DECORRSAMPLE}_0$\;
  \For{i = 0 \emph{\KwTo}correlated samples length}{
    $\text{\TEMP}_{i} \leftarrow \lfloor(3 \times \text{\DECORRELATED}_{i + 1} - \text{\DECORRELATED}_{i}) \div 2 \rfloor$\;
    $\text{\DECORRELATED}_{i + 2} \leftarrow \APPLYWEIGHT(\text{\WEIGHT}_i~,~\text{\TEMP}_{i}) + \text{\CORRELATED}_i$\;
    $\text{\WEIGHT}_{i + 1} \leftarrow \text{\WEIGHT}_i + \UPDATEWEIGHT(\text{\TEMP}_{i}~,~\text{\CORRELATED}_i~,~delta)$\;
  }
  \Return \DECORRELATED samples starting from 2\;
}
\uElseIf{$term = 17$}{
  $\text{\DECORRELATED}_0 \leftarrow \text{\DECORRSAMPLE}_1$\;
  $\text{\DECORRELATED}_1 \leftarrow \text{\DECORRSAMPLE}_0$\;
  \For{i = 0 \emph{\KwTo}correlated samples length}{
    $\text{\TEMP}_{i} \leftarrow 2 \times \text{\DECORRELATED}_{i + 1} - \text{\DECORRELATED}_{i}$\;
    $\text{\DECORRELATED}_{i + 2} \leftarrow \APPLYWEIGHT(\text{\WEIGHT}_i~,~\text{\TEMP}_{i}) + \text{\CORRELATED}_i$\;
    $\text{\WEIGHT}_{i + 1} \leftarrow \text{\WEIGHT}_i + \UPDATEWEIGHT(\text{\TEMP}_{i}~,~\text{\CORRELATED}_i~,~delta)$\;
  }
  \Return \DECORRELATED samples starting from 2\;
}
\uElseIf{$1 \leq term \leq 8$}{
  \For{i = 0 \emph{\KwTo}term}{
    $\text{\DECORRELATED}_i \leftarrow \text{\DECORRSAMPLE}_i$\;
  }
  \For{i = 0 \emph{\KwTo}correlated samples length}{
    $\text{\DECORRELATED}_{i + term} \leftarrow \APPLYWEIGHT(\text{\WEIGHT}_i~,~\text{\DECORRELATED}_{i}) + \text{\CORRELATED}_i$\;
    $\text{\WEIGHT}_{i + 1} \leftarrow \text{\WEIGHT}_i + \UPDATEWEIGHT(\text{\DECORRELATED}_{i}~,~\text{\CORRELATED}_i~,~delta)$\;
  }
  \BlankLine
  \Return \DECORRELATED samples starting from $term$\;
}
\Else{
  invalid decorrelation term\;
}
\EALGORITHM
\par
\noindent
\begin{align*}
\intertext{where \texttt{apply\_weight} is defined as:}
\texttt{apply\_weight}(weight~,~sample) &= \left\lfloor\frac{weight \times sample + 2 ^ 9}{2 ^ {10}}\right\rfloor \\
\intertext{and \texttt{update\_weight} is defined as:}
\texttt{update\_weight}(source~,~result~,~delta) &=
\begin{cases}
0 & \text{ if } source = 0 \text{ or } result = 0 \\
delta & \text{ if } (source \textbf{ xor } result ) \geq 0 \\
-delta & \text{ if } (source \textbf{ xor } result) < 0
\end{cases}
\end{align*}

}

\clearpage

\subsubsection{2 Channel Decorrelation Pass}
{\relsize{-1}
\ALGORITHM{2 lists of signed correlated samples, decorrelation term and delta, 2 decorrelation weights, 2 lists of decorrelation samples}{2 lists of signed decorrelated samples}
\SetKwData{CORRELATED}{correlated}
\SetKwData{DECORRELATED}{decorrelated}
\SetKwData{DECORRSAMPLE}{decorrelation sample}
\SetKwData{WEIGHT}{weight}
\SetKw{OR}{or}
\SetKw{XOR}{xor}
\SetKwFunction{MIN}{min}
\SetKwFunction{MAX}{max}
\SetKwFunction{APPLYWEIGHT}{apply\_weight}
\SetKwFunction{UPDATEWEIGHT}{update\_weight}
\uIf{$17 \leq term \leq 18$ \OR $1 \leq term \leq 8$}{
  $\text{\DECORRELATED}_0 \leftarrow$ 1 channel decorrelation pass of $\text{\CORRELATED}_0$\;
  $\text{\DECORRELATED}_1 \leftarrow$ 1 channel decorrelation pass of $\text{\CORRELATED}_1$\;
  \Return $\text{\DECORRELATED}_0$ and $\text{\DECORRELATED}_1$\;
}
\uElseIf{$-3 \leq term \leq -1$}{
  $\text{\WEIGHT}_{0~0} \leftarrow$ decorrelation weight 0\;
  $\text{\WEIGHT}_{1~0} \leftarrow$ decorrelation weight 1\;
  $\text{\DECORRELATED}_{0~0} \leftarrow \text{\DECORRSAMPLE}_{1~0}$\;
  $\text{\DECORRELATED}_{1~0} \leftarrow \text{\DECORRSAMPLE}_{0~0}$\;
  \uIf{$term = -1$}{
    \For{i = 0 \emph{\KwTo}correlated samples length}{
      $\text{\DECORRELATED}_{0~(i + 1)} \leftarrow \APPLYWEIGHT(\text{\WEIGHT}_{0~i}~,~\text{\DECORRELATED}_{1~i}) + \text{\CORRELATED}_{0~i}$\;
      $\text{\DECORRELATED}_{1~(i + 1)} \leftarrow \APPLYWEIGHT(\text{\WEIGHT}_{1~i}~,~\text{\DECORRELATED}_{0~(i + 1)}) + \text{\CORRELATED}_{1~i}$\;
      $\text{\WEIGHT}_{0~(i + 1)} \leftarrow \text{\WEIGHT}_{0~i} + \UPDATEWEIGHT(\text{\DECORRELATED}_{1~i}~,~\text{\CORRELATED}_{0~i}~,~delta)$\;
      $\text{\WEIGHT}_{1~(i + 1)} \leftarrow \text{\WEIGHT}_{1~i} + \UPDATEWEIGHT(\text{\DECORRELATED}_{0~(i + 1)}~,~\text{\CORRELATED}_{1~i}~,~delta)$\;
      $\text{\WEIGHT}_{0~(i + 1)} \leftarrow \MAX(\MIN(\text{\WEIGHT}_{0~(i + 1)}~,~1024)~,~-1024)$\;
      $\text{\WEIGHT}_{1~(i + 1)} \leftarrow \MAX(\MIN(\text{\WEIGHT}_{1~(i + 1)}~,~1024)~,~-1024)$\;
    }
  }
  \uElseIf{$term = -2$}{
    \For{i = 0 \emph{\KwTo}correlated samples length}{
      $\text{\DECORRELATED}_{1~(i + 1)} \leftarrow \APPLYWEIGHT(\text{\WEIGHT}_{1~i}~,~\text{\DECORRELATED}_{0~i}) + \text{\CORRELATED}_{1~i}$\;
      $\text{\DECORRELATED}_{0~(i + 1)} \leftarrow \APPLYWEIGHT(\text{\WEIGHT}_{0~i}~,~\text{\DECORRELATED}_{1~(i + 1)}) + \text{\CORRELATED}_{0~i}$\;
      $\text{\WEIGHT}_{1~(i + 1)} \leftarrow \text{\WEIGHT}_{1~i} + \UPDATEWEIGHT(\text{\DECORRELATED}_{0~i}~,~\text{\CORRELATED}_{1~i}~,~delta)$\;
      $\text{\WEIGHT}_{0~(i + 1)} \leftarrow \text{\WEIGHT}_{0~i} + \UPDATEWEIGHT(\text{\DECORRELATED}_{1~(i + 1)},~\text{\CORRELATED}_{0~i}~,~delta)$\;
      $\text{\WEIGHT}_{1~(i + 1)} \leftarrow \MAX(\MIN(\text{\WEIGHT}_{1~(i + 1)}~,~1024)~,~-1024)$\;
      $\text{\WEIGHT}_{0~(i + 1)} \leftarrow \MAX(\MIN(\text{\WEIGHT}_{0~(i + 1)}~,~1024)~,~-1024)$\;
    }
  }
  \ElseIf{$term = -3$}{
    \For{i = 0 \emph{\KwTo}correlated samples length}{
      $\text{\DECORRELATED}_{0~(i + 1)} \leftarrow \APPLYWEIGHT(\text{\WEIGHT}_{0~i}~,~\text{\DECORRELATED}_{1~i}) + \text{\CORRELATED}_{0~i}$\;
      $\text{\DECORRELATED}_{1~(i + 1)} \leftarrow \APPLYWEIGHT(\text{\WEIGHT}_{1~i}~,~\text{\DECORRELATED}_{0~i}) + \text{\CORRELATED}_{1~i}$\;
      $\text{\WEIGHT}_{0~(i + 1)} \leftarrow \text{\WEIGHT}_{0~i} + \UPDATEWEIGHT(\text{\DECORRELATED}_{1~i}~,~\text{\CORRELATED}_{0~i}~,~delta)$\;
      $\text{\WEIGHT}_{1~(i + 1)} \leftarrow \text{\WEIGHT}_{1~i} + \UPDATEWEIGHT(\text{\DECORRELATED}_{0~i}~,~\text{\CORRELATED}_{1~i}~,~delta)$\;
      $\text{\WEIGHT}_{0~(i + 1)} \leftarrow \MAX(\MIN(\text{\WEIGHT}_{0~(i + 1)}~,~1024)~,~-1024)$\;
      $\text{\WEIGHT}_{1~(i + 1)} \leftarrow \MAX(\MIN(\text{\WEIGHT}_{1~(i + 1)}~,~1024)~,~-1024)$\;
    }
  }
  \Return $\text{\DECORRELATED}_0$ starting from 1 and $\text{\DECORRELATED}_1$ starting from 1\;
}
\Else{
  invalid decorrelation term\;
}
\EALGORITHM
}

\clearpage

\subsubsection{Channel Decorrelation Example}
Given the values from the \VAR{Decorrelation Terms},
\VAR{Decorrelation Weights} and \VAR{Decorrelation Samples}
sub blocks:
\begin{figure}[h]
{\relsize{-1}
  \subfloat{
    \begin{tabular}{|r|r|r|}
      \multicolumn{3}{c}{Decorrelation Terms} \\
      \hline
      $p$ & $\text{term}_p$ & $\text{delta}_p$ \\
      \hline
      4 & 18 & 2 \\
      3 & 18 & 2 \\
      2 & 2 & 2 \\
      1 & 17 & 2 \\
      0 & 3 & 2 \\
      \hline
    \end{tabular}
  }
  \subfloat{
    \begin{tabular}{|r|r|r|}
      \multicolumn{3}{c}{Decorrelation Weights} \\
      \hline
      $p$ & $\text{weight}_{p~0}$ & $\text{weight}_{p~1}$ \\
      \hline
      4 & 48 & 48 \\
      3 & 48 & 48 \\
      2 & 32 & 32 \\
      1 & 48 & 48 \\
      0 & 16 & 24 \\
      \hline
    \end{tabular}
  }
  \subfloat{
    \begin{tabular}{|r|r|r|}
      \multicolumn{3}{c}{Decorrelation Samples} \\
      \hline
      $p$ & $\text{sample}_{p~0~s}$ & $\text{sample}_{p~1~s}$ \\
      \hline
      4 & \texttt{[-73, -78]} & \texttt{[28, 26]} \\
      3 & \texttt{[0, 0]} & \texttt{[0, 0]} \\
      2 & \texttt{[0, 0]} & \texttt{[0, 0]} \\
      1 & \texttt{[0, 0]} & \texttt{[0, 0]} \\
      0 & \texttt{[0, 0, 0]} & \texttt{[0, 0, 0]} \\
      \hline
    \end{tabular}
  }
}
\end{figure}
\par
\noindent
we combine them into a single set of arguments for each decorrelation pass:
\begin{table}[h]
{\relsize{-1}
  \begin{tabular}{|r|r|r|r|r|r|}
    \hline
    & $\textbf{pass}_0$ & $\textbf{pass}_1$ & $\textbf{pass}_2$ &
    $\textbf{pass}_3$ & $\textbf{pass}_4$ \\
    \hline
    $\text{term}_p$ & 3 & 17 & 2 & 18 & 18 \\
    $\text{delta}_p$ & 2 & 2 & 2 & 2 & 2 \\
    $\text{weight}_{p~0}$ & 16 & 48 & 32 & 48 & 48 \\
    $\text{samples}_{p~0~s}$ & \texttt{[0, 0, 0]} & \texttt{[0, 0]} &
    \texttt{[0, 0]} & \texttt{[0, 0]} & \texttt{[-73, -78]} \\
    $\text{weight}_{p~1}$ & 24 & 48 & 32 & 48 & 48 \\
    $\text{samples}_{p~1~s}$ & \texttt{[0, 0, 0]} & \texttt{[0, 0]} &
    \texttt{[0, 0]} & \texttt{[0, 0]} & \texttt{[28, 26]} \\
    \hline
  \end{tabular}
}
\end{table}
\par
\noindent
which we apply to the residuals from the bitstream sub-block:
\par
\noindent
{\relsize{-1}
  \begin{tabular}{|r|r|r|r|r|r|}
    \hline
    $\text{residual}_{0~i}$ &
    after $\textbf{pass}_0$ &
    after $\textbf{pass}_1$ &
    after $\textbf{pass}_2$ &
    after $\textbf{pass}_3$ &
    after $\textbf{pass}_4$ \\
    \hline
    -61 & -61 & -61 & -61 & -61 & -64 \\
    -33 & -33 & -39 & -39 & -43 & -46 \\
    -18 & -18 & -19 & -21 & -23 & -25 \\
    1 & 0 & 0 & -1 & -2 & -3 \\
    20 & 20 & 21 & 20 & 20 & 20 \\
    35 & 35 & 37 & 37 & 39 & 41 \\
    50 & 50 & 53 & 54 & 57 & 60 \\
    62 & 62 & 66 & 67 & 71 & 75 \\
    68 & 68 & 73 & 75 & 80 & 85 \\
    71 & 72 & 77 & 79 & 84 & 90 \\
    \hline
    \hline
    $\text{residual}_{1~i}$ &
    after $\textbf{pass}_0$ &
    after $\textbf{pass}_1$ &
    after $\textbf{pass}_2$ &
    after $\textbf{pass}_3$ &
    after $\textbf{pass}_4$ \\
    \hline
    31 & 31 & 31 & 31 & 31 & 32 \\
    32 & 32 & 35 & 35 & 37 & 39 \\
    36 & 36 & 38 & 39 & 41 & 43 \\
    37 & 38 & 40 & 41 & 43 & 45 \\
    35 & 36 & 38 & 39 & 41 & 44 \\
    31 & 32 & 34 & 36 & 38 & 40 \\
    25 & 26 & 28 & 30 & 32 & 34 \\
    18 & 19 & 20 & 21 & 23 & 25 \\
    10 & 11 & 12 & 13 & 14 & 15 \\
    0 & 1 & 1 & 2 & 3 & 4 \\
    \hline
  \end{tabular}
}
\par
\noindent
Resulting in final decorrelated samples:
\newline
\begin{tabular}{rr}
$\text{channel}_0$ : & \texttt{[-64,~-46,~-25,~-3,~20,~41,~60,~75,~85,~90]} \\
$\text{channel}_1$ : & \texttt{[~32,~~39,~~43,~45,~44,~40,~34,~25,~15,~~4]} \\
\end{tabular}

\clearpage

{\relsize{-2}
\begin{tabular}{r||r|>{$}r<{$}|>{$}r<{$}|>{$}r<{$}|>{$}r<{$}}
%% \multicolumn{5}{c}{\textbf{pass 0} - term 3 - delta 2 - weight 16} \\
& $i$ & \text{correlated}_i & \text{temp}_i & \text{decorrelated}_{i + 3} & \text{weight}_{i + 1} \\
\hline
\multirow{10}{1em}{\begin{sideways}$\textbf{pass}_0$ - term 3\end{sideways}}
& 0 & -61 & &
\lfloor(16 \times 0 + 2 ^ 9) \div 2 ^ {10}\rfloor - 61 = -61 &
16 + 0 = 16
\\
& 1 & -33 & &
\lfloor(16 \times 0 + 2 ^ 9) \div 2 ^ {10}\rfloor - 33 = -33 &
16 + 0 = 16
\\
& 2 & -18 & &
\lfloor(16 \times 0 + 2 ^ 9) \div 2 ^ {10}\rfloor - 18 = -18 &
16 + 0 = 16
\\
& 3 & 1 & &
\lfloor(16 \times -61 + 2 ^ 9) \div 2 ^ {10}\rfloor + 1 = 0 &
16 - 2 = 14
\\
& 4 & 20 & &
\lfloor(14 \times -33 + 2 ^ 9) \div 2 ^ {10}\rfloor + 20 = 20 &
14 - 2 = 12
\\
& 5 & 35 & &
\lfloor(12 \times -18 + 2 ^ 9) \div 2 ^ {10}\rfloor + 35 = 35 &
12 - 2 = 10
\\
& 6 & 50 & &
\lfloor(10 \times 0 + 2 ^ 9) \div 2 ^ {10}\rfloor + 50 = 50 &
10 + 0 = 10
\\
& 7 & 62 & &
\lfloor(10 \times 20 + 2 ^ 9) \div 2 ^ {10}\rfloor + 62 = 62 &
10 + 2 = 12
\\
& 8 & 68 & &
\lfloor(12 \times 35 + 2 ^ 9) \div 2 ^ {10}\rfloor + 68 = 68 &
12 + 2 = 14
\\
& 9 & 71 & &
\lfloor(14 \times 50 + 2 ^ 9) \div 2 ^ {10}\rfloor + 71 = 72 &
14 + 2 = 16
\\
\hline
\hline
%% \multicolumn{5}{c}{\textbf{pass 1 } - term 17 - delta 2 - weight 48} \\
& $i$ & \text{correlated}_i & \text{temp}_i & \text{decorrelated}_{i + 2} & \text{weight}_{i + 1} \\
\hline
\multirow{10}{1em}{\begin{sideways}$\textbf{pass}_1$ - term 17\end{sideways}}
& 0 & -61 &
2 \times 0 - 0 = 0 &
\lfloor(48 \times 0 + 2 ^ 9) \div 2 ^ {10}\rfloor - 61 = -61 &
48 + 0 = 48
\\
& 1 & -33 &
2 \times -61 - 0 = -122 &
\lfloor(48 \times -122 + 2 ^ 9) \div 2 ^ {10}\rfloor - 33 = -39 &
48 + 2 = 50
\\
& 2 & -18 &
2 \times -39 + 61 = -17 &
\lfloor(50 \times -17 + 2 ^ 9) \div 2 ^ {10}\rfloor - 18 = -19 &
50 + 2 = 52
\\
& 3 & 0 &
2 \times -19 + 39 = 1 &
\lfloor(52 \times 1 + 2 ^ 9) \div 2 ^ {10}\rfloor + 0 = 0 &
52 + 0 = 52
\\
& 4 & 20 &
2 \times 0 + 19 = 19 &
\lfloor(52 \times 19 + 2 ^ 9) \div 2 ^ {10}\rfloor + 20 = 21 &
52 + 2 = 54
\\
& 5 & 35 &
2 \times 21 - 0 = 42 &
\lfloor(54 \times 42 + 2 ^ 9) \div 2 ^ {10}\rfloor + 35 = 37 &
54 + 2 = 56
\\
& 6 & 50 &
2 \times 37 - 21 = 53 &
\lfloor(56 \times 53 + 2 ^ 9) \div 2 ^ {10}\rfloor + 50 = 53 &
56 + 2 = 58
\\
& 7 & 62 &
2 \times 53 - 37 = 69 &
\lfloor(58 \times 69 + 2 ^ 9) \div 2 ^ {10}\rfloor + 62 = 66 &
58 + 2 = 60
\\
& 8 & 68 &
2 \times 66 - 53 = 79 &
\lfloor(60 \times 79 + 2 ^ 9) \div 2 ^ {10}\rfloor + 68 = 73 &
60 + 2 = 62
\\
& 9 & 72 &
2 \times 73 - 66 = 80 &
\lfloor(62 \times 80 + 2 ^ 9) \div 2 ^ {10}\rfloor + 72 = 77 &
62 + 2 = 64
\\
\hline
\hline
%% \multicolumn{5}{c}{\textbf{pass 2 } - term 2 - delta 2 - weight 32} \\
& $i$ & \text{correlated}_i & \text{temp}_i & \text{decorrelated}_{i + 2} & \text{weight}_{i + 1} \\
\hline
\multirow{10}{1em}{\begin{sideways}$\textbf{pass}_2$ - term 2\end{sideways}}
& 0 & -61 & &
\lfloor(32 \times 0 + 2 ^ 9) \div 2 ^ {10}\rfloor - 61 = -61 &
32 + 0 = 32
\\
& 1 & -39 & &
\lfloor(32 \times 0 + 2 ^ 9) \div 2 ^ {10}\rfloor - 39 = -39 &
32 + 0 = 32
\\
& 2 & -19 & &
\lfloor(32 \times -61 + 2 ^ 9) \div 2 ^ {10}\rfloor - 19 = -21 &
32 + 2 = 34
\\
& 3 & 0 & &
\lfloor(34 \times -39 + 2 ^ 9) \div 2 ^ {10}\rfloor + 0 = -1 &
34 + 0 = 34
\\
& 4 & 21 & &
\lfloor(34 \times -21 + 2 ^ 9) \div 2 ^ {10}\rfloor + 21 = 20 &
34 - 2 = 32
\\
& 5 & 37 & &
\lfloor(32 \times -1 + 2 ^ 9) \div 2 ^ {10}\rfloor + 37 = 37 &
32 - 2 = 30
\\
& 6 & 53 & &
\lfloor(30 \times 20 + 2 ^ 9) \div 2 ^ {10}\rfloor + 53 = 54 &
30 + 2 = 32
\\
& 7 & 66 & &
\lfloor(32 \times 37 + 2 ^ 9) \div 2 ^ {10}\rfloor + 66 = 67 &
32 + 2 = 34
\\
& 8 & 73 & &
\lfloor(34 \times 54 + 2 ^ 9) \div 2 ^ {10}\rfloor + 73 = 75 &
34 + 2 = 36
\\
& 9 & 77 & &
\lfloor(36 \times 67 + 2 ^ 9) \div 2 ^ {10}\rfloor + 77 = 79 &
36 + 2 = 38
\\
\hline
\hline
%% \multicolumn{5}{c}{\textbf{pass 3 } - term 18 - delta 2 - weight 48} \\
& $i$ & \text{correlated}_i & \text{temp}_i & \text{decorrelated}_{i + 2} & \text{weight}_{i + 1} \\
\hline
\multirow{10}{1em}{\begin{sideways}$\textbf{pass}_3$ - term 18\end{sideways}}
& 0 & -61 &
\lfloor(3 \times 0 - 0) \div 2\rfloor = 0 &
\lfloor(48 \times 0 + 2 ^ 9) \div 2 ^ {10}\rfloor - 61 = -61 &
48 + 0 = 48
\\
& 1 & -39 &
\lfloor(3 \times -61 - 0) \div 2\rfloor = -92 &
\lfloor(48 \times -92 + 2 ^ 9) \div 2 ^ {10}\rfloor - 39 = -43 &
48 + 2 = 50
\\
& 2 & -21 &
\lfloor(3 \times -43 + 61) \div 2\rfloor = -34 &
\lfloor(50 \times -34 + 2 ^ 9) \div 2 ^ {10}\rfloor - 21 = -23 &
50 + 2 = 52
\\
& 3 & -1 &
\lfloor(3 \times -23 + 43) \div 2\rfloor = -13 &
\lfloor(52 \times -13 + 2 ^ 9) \div 2 ^ {10}\rfloor - 1 = -2 &
52 + 2 = 54
\\
& 4 & 20 &
\lfloor(3 \times -2 + 23) \div 2\rfloor = 8 &
\lfloor(54 \times 8 + 2 ^ 9) \div 2 ^ {10}\rfloor + 20 = 20 &
54 + 2 = 56
\\
& 5 & 37 &
\lfloor(3 \times 20 + 2) \div 2\rfloor = 31 &
\lfloor(56 \times 31 + 2 ^ 9) \div 2 ^ {10}\rfloor + 37 = 39 &
56 + 2 = 58
\\
& 6 & 54 &
\lfloor(3 \times 39 - 20) \div 2\rfloor = 48 &
\lfloor(58 \times 48 + 2 ^ 9) \div 2 ^ {10}\rfloor + 54 = 57 &
58 + 2 = 60
\\
& 7 & 67 &
\lfloor(3 \times 57 - 39) \div 2\rfloor = 66 &
\lfloor(60 \times 66 + 2 ^ 9) \div 2 ^ {10}\rfloor + 67 = 71 &
60 + 2 = 62
\\
& 8 & 75 &
\lfloor(3 \times 71 - 57) \div 2\rfloor = 78 &
\lfloor(62 \times 78 + 2 ^ 9) \div 2 ^ {10}\rfloor + 75 = 80 &
62 + 2 = 64
\\
& 9 & 79 &
\lfloor(3 \times 80 - 71) \div 2\rfloor = 84 &
\lfloor(64 \times 84 + 2 ^ 9) \div 2 ^ {10}\rfloor + 79 = 84 &
64 + 2 = 66
\\
\hline
\hline
%% \multicolumn{5}{c}{\textbf{pass 4 } - term 18 - delta 2 - weight 48} \\
& $i$ & \text{correlated}_i & \text{temp}_i & \text{decorrelated}_{i + 2} & \text{weight}_{i + 1} \\
\hline
\multirow{10}{1em}{\begin{sideways}$\textbf{pass}_4$ - term 18\end{sideways}}
& 0 & -61 &
\lfloor(3 \times -73 + 78) \div 2\rfloor = -71 &
\lfloor(48 \times -71 + 2 ^ 9) \div 2 ^ {10}\rfloor - 61 = -64 &
48 + 2 = 50
\\
& 1 & -43 &
\lfloor(3 \times -64 + 73) \div 2\rfloor = -60 &
\lfloor(50 \times -60 + 2 ^ 9) \div 2 ^ {10}\rfloor - 43 = -46 &
50 + 2 = 52
\\
& 2 & -23 &
\lfloor(3 \times -46 + 64) \div 2\rfloor = -37 &
\lfloor(52 \times -37 + 2 ^ 9) \div 2 ^ {10}\rfloor - 23 = -25 &
52 + 2 = 54
\\
& 3 & -2 &
\lfloor(3 \times -25 + 46) \div 2\rfloor = -15 &
\lfloor(54 \times -15 + 2 ^ 9) \div 2 ^ {10}\rfloor - 2 = -3 &
54 + 2 = 56
\\
& 4 & 20 &
\lfloor(3 \times -3 + 25) \div 2\rfloor = 8 &
\lfloor(56 \times 8 + 2 ^ 9) \div 2 ^ {10}\rfloor + 20 = 20 &
56 + 2 = 58
\\
& 5 & 39 &
\lfloor(3 \times 20 + 3) \div 2\rfloor = 31 &
\lfloor(58 \times 31 + 2 ^ 9) \div 2 ^ {10}\rfloor + 39 = 41 &
58 + 2 = 60
\\
& 6 & 57 &
\lfloor(3 \times 41 - 20) \div 2\rfloor = 51 &
\lfloor(60 \times 51 + 2 ^ 9) \div 2 ^ {10}\rfloor + 57 = 60 &
60 + 2 = 62
\\
& 7 & 71 &
\lfloor(3 \times 60 - 41) \div 2\rfloor = 69 &
\lfloor(62 \times 69 + 2 ^ 9) \div 2 ^ {10}\rfloor + 71 = 75 &
62 + 2 = 64
\\
& 8 & 80 &
\lfloor(3 \times 75 - 60) \div 2\rfloor = 82 &
\lfloor(64 \times 82 + 2 ^ 9) \div 2 ^ {10}\rfloor + 80 = 85 &
64 + 2 = 66
\\
& 9 & 84 &
\lfloor(3 \times 85 - 75) \div 2\rfloor = 90 &
\lfloor(66 \times 90 + 2 ^ 9) \div 2 ^ {10}\rfloor + 84 = 90 &
66 + 2 = 68
\\
\end{tabular}
}
\begin{center}
$\text{channel}_0$ decorrelation passes
\end{center}

\clearpage

\subsection{Undo Joint Stereo}

\ALGORITHM{mid and side channels of signed sample data, in that order}{left and right channels of signed sample data, in that order}
\SetKwData{MID}{mid}
\SetKwData{SIDE}{side}
\SetKwData{LEFT}{left}
\SetKwData{RIGHT}{right}
\For{i = 0 \emph{\KwTo}sample count}{
  $\text{\RIGHT}_i \leftarrow \text{\SIDE}_i - \lfloor\text{\MID}_i \div 2\rfloor$\;
  $\text{\LEFT}_i \leftarrow \text{\MID}_i + \text{\RIGHT}_i$\;
}
\Return left and right channels\;
\EALGORITHM

\subsubsection{Joint Stereo Example}
\begin{table}[h]
\begin{tabular}{|r|r|r||>{$}r<{$}|>{$}r<{$}|}
$i$ & $\text{mid}_i$ & $\text{side}_i$ & \text{right}_i & \text{left}_i \\
\hline
0 & -64 & 32 &
32 - \lfloor-64 \div 2\rfloor = 64 &
-64 + 64 = 0 \\
1 & -46 & 39 &
39 - \lfloor-46 \div 2\rfloor = 62 &
-46 + 62 = 16 \\
2 & -25 & 43 &
43 - \lfloor-25 \div 2\rfloor = 56 &
-25 + 56 = 31 \\
3 & -3 & 45 &
45 - \lfloor-3 \div 2\rfloor = 47 &
-3 + 47 = 44 \\
4 & 20 & 44 &
44 - \lfloor20 \div 2\rfloor = 34 &
20 + 34 = 54 \\
5 & 41 & 40 &
40 - \lfloor41 \div 2\rfloor = 20 &
41 + 20 = 61 \\
6 & 60 & 34 &
34 - \lfloor60 \div 2\rfloor = 4 &
60 + 4 = 64 \\
7 & 75 & 25 &
25 - \lfloor75 \div 2\rfloor = -12 &
75 + -12 = 63 \\
8 & 85 & 15 &
15 - \lfloor85 \div 2\rfloor = -27 &
85 + -27 = 58 \\
9 & 90 & 4 &
4 - \lfloor90 \div 2\rfloor = -41 &
90 + -41 = 49 \\
\hline
\end{tabular}
\end{table}

\begin{landscape}

\subsection{Checksum Calculation}

\ALGORITHM{one or two channels of signed audio samples}{an unsigned 32-bit CRC integer}
\SetKwData{MONO}{mono output}
\SetKwData{CRC}{CRC}
\SetKwData{LCRC}{LCRC}
\SetKwData{SCRC}{SCRC}
\SetKwData{CHANNEL}{channel}
$\text{\CRC}_{-1} \leftarrow \texttt{0xFFFFFFFF}$\;
\For{i = 0 \emph{\KwTo}sample count}{
  \eIf{$\text{\MONO} = 0$}{
    $\text{\LCRC}_i \leftarrow (3 \times \text{\CRC}_{i - 1}) + \text{\CHANNEL}_{0~i}$\tcc*[r]{calculate signed CRC of left channel}
    $\text{\SCRC}_i \leftarrow (3 \times \text{\LCRC}_{i - 1}) + \text{\CHANNEL}_{1~i}$\tcc*[r]{calculate signed CRC of right channel}
  }{
    $\text{\SCRC}_i \leftarrow (3 \times \text{\CRC}_{i - 1}) + \text{\CHANNEL}_{0~i}$\tcc*[r]{calculate signed CRC of channel}
  }
  \BlankLine
  \eIf(\tcc*[f]{convert signed CRC to unsigned, 32-bit integer}){$\text{\SCRC}_i \geq 0$}{
    $\text{\CRC}_i \leftarrow \text{\SCRC}_i \bmod \texttt{0x100000000}$\;
  }{
    $\text{\CRC}_i \leftarrow (2 ^ {32} - (-\text{\SCRC}_i)) \bmod \texttt{0x100000000}$\;
  }
}
\Return $\text{\CRC}_{\text{sample count} - 1}$\;
\EALGORITHM

\subsubsection{Checksum Calculation Example}
{\relsize{-1}
\begin{tabular}{|r|r|r||>{$}r<{$}|>{$}r<{$}|>{$}r<{$}|}
$i$ & $\text{channel}_{0~i}$ & $\text{channel}_{1~i}$ & \text{LCRC}_i & \text{SCRC}_i & \text{CRC}_i \\
\hline
0 & 0 & 64 &
(3 \times \texttt{0xFFFFFFFF}) + 0 = \texttt{0x2FFFFFFFD} &
(3 \times \texttt{0x2FFFFFFFD}) + 64 = \texttt{0x900000037} &
\texttt{0x00000037} \\
1 & 16 & 62 &
(3 \times \texttt{0x00000037}) + 16 = \texttt{0x000000B5} &
(3 \times \texttt{0x000000B5}) + 62 = \texttt{0x0000025D} &
\texttt{0x0000025D} \\
2 & 31 & 56 &
(3 \times \texttt{0x0000025D}) + 31 = \texttt{0x00000736} &
(3 \times \texttt{0x00000736}) + 56 = \texttt{0x000015DA} &
\texttt{0x000015DA} \\
3 & 44 & 47 &
(3 \times \texttt{0x000015DA}) + 44 = \texttt{0x000041BA} &
(3 \times \texttt{0x000041BA}) + 47 = \texttt{0x0000C55D} &
\texttt{0x0000C55D} \\
4 & 54 & 34 &
(3 \times \texttt{0x0000C55D}) + 54 = \texttt{0x0002504D} &
(3 \times \texttt{0x0002504D}) + 34 = \texttt{0x0006F109} &
\texttt{0x0006F109} \\
5 & 61 & 20 &
(3 \times \texttt{0x0006F109}) + 61 = \texttt{0x0014D358} &
(3 \times \texttt{0x0014D358}) + 20 = \texttt{0x003E7A1C} &
\texttt{0x003E7A1C} \\
6 & 64 & 4 &
(3 \times \texttt{0x003E7A1C}) + 64 = \texttt{0x00BB6E94} &
(3 \times \texttt{0x00BB6E94}) + 4 = \texttt{0x02324BC0} &
\texttt{0x02324BC0} \\
7 & 63 & -12 &
(3 \times \texttt{0x02324BC0}) + 63 = \texttt{0x0696E37F} &
(3 \times \texttt{0x0696E37F}) - 12 = \texttt{0x13C4AA71} &
\texttt{0x13C4AA71} \\
8 & 58 & -27 &
(3 \times \texttt{0x13C4AA71}) + 58 = \texttt{0x3B4DFF8D} &
(3 \times \texttt{0x3B4DFF8D}) - 27 = \texttt{0xB1E9FE8C} &
\texttt{0xB1E9FE8C} \\
9 & 49 & -41 &
(3 \times \texttt{0xB1E9FE8C}) + 49 = \texttt{0x215BDFBD5} &
(3 \times \texttt{0x215BDFBD5}) - 41 = \texttt{0x64139F356} &
\texttt{0x4139F356} \\
\end{tabular}
}
\vskip 1em
\par
\noindent
Resulting in a final CRC of \texttt{0x4139F356}
\end{landscape}

\clearpage

\subsection{Reading Extended Integers Sub Block}
\ALGORITHM{sub block data}{\VAR{zero bits}, \VAR{one bits}, \VAR{duplicate bits} values as unsigned integers}
\SetKwData{SENTS}{sent bits}
\SetKwData{ZEROES}{zero bits}
\SetKwData{ONES}{one bits}
\SetKwData{DUPLICATES}{duplicate bits}
\SENTS $\leftarrow$ \READ 8 unsigned bits\tcc*[r]{unused}
\ZEROES $\leftarrow$ \READ 8 unsigned bits\;
\ONES $\leftarrow$ \READ 8 unsigned bits\;
\DUPLICATES $\leftarrow$ \READ 8 unsigned bits\;
\Return \ZEROES, \ONES and \DUPLICATES\;
\EALGORITHM

\subsection{Undoing Extended Integers}
\ALGORITHM{\VAR{zero bits}, \VAR{one bits}, \VAR{duplicate bits} values; 1 or 2 channels of shifted PCM data}{1 or 2 channels of un-shifted PCM data}
\SetKwData{ZEROES}{zero bits}
\SetKwData{ONES}{one bits}
\SetKwData{DUPLICATES}{duplicate bits}
\SetKwData{SHIFTED}{shifted channel}
\SetKwData{UNSHIFTED}{un-shifted channel}
\For{c = 0 \emph{\KwTo}channel count}{
  \uIf{$\text{\ZEROES} > 0$}{
    \For{i = 0 \emph{\KwTo}sample count}{
      $\text{\UNSHIFTED}_{c~i} \leftarrow \text{\SHIFTED}_{c~i} \times 2 ^ {\text{\ZEROES}}$\;
    }
  }
  \uElseIf{$\text{\ONES} > 0$}{
    \For{i = 0 \emph{\KwTo}sample count}{
      $\text{\UNSHIFTED}_{c~i} \leftarrow \text{\SHIFTED}_{c~i} \times 2 ^ {\text{\ONES}} + (2 ^ {\text{\ONES}} - 1)$\;
    }
  }
  \uElseIf{$\text{\DUPLICATES} > 0$}{
    \For{i = 0 \emph{\KwTo}sample count}{
      \eIf{$\text{\SHIFTED}_{c~i} \bmod 2 = 0$}{
        $\text{\UNSHIFTED}_{c~i} \leftarrow \text{\SHIFTED}_{c~i} \times 2 ^ {\text{\DUPLICATES}}$\;
      }{
        $\text{\UNSHIFTED}_{c~i} \leftarrow \text{\SHIFTED}_{c~i} \times 2 ^ {\text{\DUPLICATES}} + (2 ^ {\text{\DUPLICATES}} - 1)$\;
      }
    }
  }
  \Else{
    $\text{\UNSHIFTED}_c \leftarrow \text{\SHIFTED}_c$\;
  }
}
\Return \UNSHIFTED data\;
\EALGORITHM

\clearpage

\subsection{MD5 Sum}

The MD5 is the hash of all the PCM sampmes, on a PCM frame basis,
in little-endian format and signed if the bits per sample is greater than 0.

\begin{figure}[h]
\includegraphics{figures/wavpack/md5sum.pdf}
\end{figure}

\subsection{RIFF WAVE Header and Footer}

These sub-blocks are typically found in the first and last
WavPack block, respectively.
The header must always be present in the file while
the footer is optional.

\begin{figure}[h]
\includegraphics{figures/wavpack/pcm_sandwich.pdf}
\end{figure}

\clearpage

\section{WavPack Encoding}

\ALGORITHM{PCM frames, Wave header\footnote{Everything between the file's start and the start of the \texttt{data} chunk's contents.  If one is encoding a WavPack from raw PCM input, this header will need to be generated.}, optional Wave footer\footnote{Everything between the end of the \texttt{data} chunk's contents and the file's end, if anything.}, encoding parameters:
\newline
{\relsize{-1}
\begin{tabular}{rll}
parameter & possible values & typical values \\
\hline
block size & a positive number of PCM frames & 22050 \\
correlation passes & 0, 1, 2, 5, 10 or 16 & 5 \\
\end{tabular}
}
}{an encoded WavPack file}
\SetKwData{BLOCKSIZE}{block size}
\SetKwData{BLOCKINDEX}{block index}
\SetKwData{BLOCKSETCOUNT}{block count}
\SetKwData{BLOCKSETCHANNELS}{block channels}
\SetKwData{PASSES}{correlation passes}
\SetKwData{PARAMS}{block params}
\SetKwData{FIRST}{first}
\SetKwData{LAST}{last}
\SetKwData{BLOCK}{block}
\SetKwData{CHANNELS}{channels}
\SetKwData{CHANNEL}{channel}
$(\text{\BLOCKSETCOUNT}~,~\text{\BLOCKSETCHANNELS}) \leftarrow$ determine block split\;
\For{b = 0 \emph{\KwTo}\BLOCKSETCOUNT}{
  $\text{\PARAMS}_{0~b} \leftarrow$ determine initial correlation parameters and entropy variables from correlation passes and $\text{\BLOCKSETCHANNELS}_b$\;
}
\BlankLine
$\text{\BLOCKINDEX} \leftarrow 0$\;
$s \leftarrow 0$\tcc*[r]{the number of block sets written}
\While{PCM frames remain}{
  $\text{\CHANNELS} \leftarrow$ take up to \BLOCKSIZE PCM frames from the input\;
  update the stream's MD5 sum with that PCM data\;
  $c \leftarrow 0$\;
  \For(\tcc*[f]{blocks in each set}){b = 0 \emph{\KwTo}\BLOCKSETCOUNT}{
    \lIf{$b = 0$}{$\text{\FIRST} = 1$}
    \lElse{$\text{\FIRST} = 0$}\;
    \lIf{$b = \text{\BLOCKSETCOUNT} - 1$}{$\text{\LAST} = 1$}
    \lElse{$\text{\LAST} = 0$}\;
    \uIf{$\text{\BLOCKSETCHANNELS}_b = 1$}{
      $(\text{\BLOCK}_{(s \times \text{\BLOCKSETCHANNELS}) + b}~,~\text{\PARAMS}_{(s + 1)~b}) \leftarrow$ write block\newline
      using $\text{\CHANNEL}_c$, \BLOCKINDEX, \FIRST, \LAST and $\text{\PARAMS}_{s~b}$\;
      $c \leftarrow c + 1$\;
    }
    \ElseIf{$\text{\BLOCKSETCHANNELS}_b = 2$}{
      $(\text{\BLOCK}_{(s \times \text{\BLOCKSETCHANNELS}) + b}~,~\text{\PARAMS}_{(s + 1)~b}) \leftarrow$ write block\newline
      using $\text{\CHANNEL}_c/\text{\CHANNEL}_{c + 1}$, \BLOCKINDEX, \FIRST, \LAST and $\text{\PARAMS}_{s~b}$\;
      $c \leftarrow c + 2$\;
    }
  }
  $s \leftarrow s + 1$\;
  $\text{\BLOCKINDEX} \leftarrow \text{\BLOCKINDEX} + \text{PCM data's frame count}$\;
}
\BlankLine
write final block containing optional Wave footer and MD5 sum sub blocks\;
update Wave header's \texttt{data} chunk size, if generated from scratch\;
update \VAR{total samples} field in all block headers with \BLOCKINDEX\;
\EALGORITHM

\clearpage

\subsection{Determine Block Split}
\ALGORITHM{input stream's channel assignment}{number of blocks per set, list of channel counts per block}
\SetKwData{BLOCKCOUNT}{block count}
\SetKwData{BLOCKCHANNELS}{block channels}
\Switch(\tcc*[f]{split channels by left/right pairs}){channel assignment}{
  \uCase{mono}{
    $\text{\BLOCKCOUNT} \leftarrow 1$\;
    $\text{\BLOCKCHANNELS} \leftarrow \texttt{[1]}$\;
  }
  \uCase{front left, front right}{
    $\text{\BLOCKCOUNT} \leftarrow 1$\;
    $\text{\BLOCKCHANNELS} \leftarrow \texttt{[2]}$\;
  }
  \uCase{front left, front right, front center}{
    $\text{\BLOCKCOUNT} \leftarrow 2$\;
    $\text{\BLOCKCHANNELS} \leftarrow \texttt{[2, 1]}$\;
  }
  \uCase{front left, front right, back left, back right}{
    $\text{\BLOCKCOUNT} \leftarrow 2$\;
    $\text{\BLOCKCHANNELS} \leftarrow \texttt{[2, 2]}$\;
  }
  \uCase{front left, front right, front center, back center}{
    $\text{\BLOCKCOUNT} \leftarrow 3$\;
    $\text{\BLOCKCHANNELS} \leftarrow \texttt{[2, 1, 1]}$\;
  }
  \uCase{front left, front right, front center, back left, back right}{
    $\text{\BLOCKCOUNT} \leftarrow 3$\;
    $\text{\BLOCKCHANNELS} \leftarrow \texttt{[2, 1, 2]}$\;
  }
  \uCase{front left, front right, front center, LFE, back left, back right}{
    $\text{\BLOCKCOUNT} \leftarrow 4$\;
    $\text{\BLOCKCHANNELS} \leftarrow \texttt{[2, 1, 1, 2]}$\;
  }
  \Other(\tcc*[f]{save them independently}){
    $\text{\BLOCKCOUNT} \leftarrow$ channel count\;
    $\text{\BLOCKCHANNELS} \leftarrow$ 1 per channel\;
  }
}
\Return \BLOCKCOUNT and \BLOCKCHANNELS
\EALGORITHM
\vskip 1ex
\par
\noindent
One could invent alternate channel splits for other obscure assignments.
WavPack's only requirement is that all channels must be in
Wave order\footnote{see page \pageref{wave_channel_assignment}}
and each block must contain 1 or 2 channels.

\begin{figure}[h]
\includegraphics{figures/wavpack/block_channels.pdf}
\end{figure}

\begin{landscape}

\subsection{Determine Correlation Parameters and Entropy Variables}
{\relsize{-1}
\begin{description}
\item[$\text{term}_{b~p}$] correlation term for block $b$, correlation pass $p$
\item[$\text{delta}_{b~p}$] correlation delta for block $b$, correlation pass $p$
\item[$\text{weight}_{b~p~c}$] correlation weight for block $b$, correlation pass $p$, channel $c$
\item[$\text{sample}_{b~p~c~s}$] correlation sample $s$ for block $b$, correlation pass $p$, channel $c$
\item[$\text{entropy}_{b~c~m}$] median $m$ for block $b$, channel $c$
\end{description}
\par
\noindent
We'll omit the block $b$ parameter since it will be the same
throughout the block encode, but one must keep it in mind
when transferring parameters from the block of one set of channels
to the next block of those same channels.
}
\vskip .10in
\par
\noindent
\ALGORITHM{correlation pass count, block's channel count of 1 or 2}{correlation term, delta, weights and samples for each pass; 3 entropy variables for each channel}
{\relsize{-2}
$\text{entropy}_0 \leftarrow \texttt{[0, 0, 0]}$\;
$\text{entropy}_1 \leftarrow \texttt{[0, 0, 0]}$\;
\BlankLine
\If{$\text{channel count} = 1$}{
\Switch{correlation pass count}{
\uCase{1}{
\begin{tabular}{r|rrrl}
$\text{pass}~p$ & $\text{term}_p$ & $\text{delta}_p$ & $\text{weight}_{p~0}$ & $\text{samples}_{p~0}$ \\
\hline
0 & 18 & 2 & 0 & \texttt{[0, 0]} \\
\end{tabular}
}
\uCase{2}{
\begin{tabular}{r|rrrl}
$\text{pass}~p$ & $\text{term}_p$ & $\text{delta}_p$ & $\text{weight}_{p~0}$ & $\text{samples}_{p~0}$ \\
\hline
0 & 17 & 2 & 0 & \texttt{[0, 0]} \\
1 & 18 & 2 & 0 & \texttt{[0, 0]} \\
\end{tabular}
}
\uCase(\tcc*[f]{one channel blocks don't use negative terms}){5, 10, or 16}{
\begin{tabular}{r|rrrl}
$\text{pass}~p$ & $\text{term}_p$ & $\text{delta}_p$ & $\text{weight}_{p~0}$ & $\text{samples}_{p~0}$ \\
\hline
0 & 3 & 2 & 0 & \texttt{[0, 0, 0]} \\
1 & 17 & 2 & 0 & \texttt{[0, 0]} \\
2 & 2 & 2 & 0 & \texttt{[0, 0]} \\
3 & 18 & 2 & 0 & \texttt{[0, 0]} \\
4 & 18 & 2 & 0 & \texttt{[0, 0]} \\
\end{tabular}
}}}}
\EALGORITHM

\clearpage

\begin{algorithm}
{\relsize{-2}
\ElseIf{$\text{channel count} = 2$}{
\Switch{correlation pass count}{
\uCase{1}{
\begin{tabular}{r|rrrrll}
$\text{pass}~p$ & $\text{term}_p$ & $\text{delta}_p$ & $\text{weight}_{p~0}$ & $\text{weight}_{p~0}$ & $\text{samples}_{p~0}$ & $\text{samples}_{p~0}$ \\
\hline
0 & 18 & 2 & 0 & 0 & \texttt{[0, 0]} & \texttt{[0, 0]} \\
\end{tabular}
}
\uCase{2}{
\begin{tabular}{r|rrrrll}
$\text{pass}~p$ & $\text{term}_p$ & $\text{delta}_p$ & $\text{weight}_{p~0}$ & $\text{weight}_{p~1}$ & $\text{samples}_{p~0}$ & $\text{samples}_{p~1}$ \\
\hline
0 & 17 & 2 & 0 & 0 & \texttt{[0, 0]} & \texttt{[0, 0]} \\
1 & 18 & 2 & 0 & 0 & \texttt{[0, 0]} & \texttt{[0, 0]} \\
\end{tabular}
}
\uCase{5}{
\begin{tabular}{r|rrrrll}
$\text{pass}~p$ & $\text{term}_p$ & $\text{delta}_p$ & $\text{weight}_{p~0}$ & $\text{weight}_{p~1}$ & $\text{samples}_{p~0}$ & $\text{samples}_{p~1}$ \\
\hline
0 & 3 & 2 & 0 & 0 & \texttt{[0, 0, 0]} & \texttt{[0, 0, 0]} \\
1 & 17 & 2 & 0 & 0 & \texttt{[0, 0]} & \texttt{[0, 0]} \\
2 & 2 & 2 & 0 & 0 & \texttt{[0, 0]} & \texttt{[0, 0]} \\
3 & 18 & 2 & 0 & 0 & \texttt{[0, 0]} & \texttt{[0, 0]} \\
4 & 18 & 2 & 0 & 0 & \texttt{[0, 0]} & \texttt{[0, 0]} \\
\end{tabular}
}
\uCase{10}{
\begin{tabular}{r|rrrrll}
$\text{pass}~p$ & $\text{term}_p$ & $\text{delta}_p$ & $\text{weight}_{p~0}$ & $\text{weight}_{p~1}$ & $\text{samples}_{p~0}$ & $\text{samples}_{p~1}$ \\
\hline
0 & 4 & 2 & 0 & 0 & \texttt{[0, 0, 0, 0]} & \texttt{[0, 0, 0, 0]} \\
1 & 17 & 2 & 0 & 0 & \texttt{[0, 0]} & \texttt{[0, 0]} \\
2 & -1 & 2 & 0 & 0 & \texttt{[0]} & \texttt{[0]} \\
3 & 5 & 2 & 0 & 0 & \texttt{[0, 0, 0, 0, 0]} & \texttt{[0, 0, 0, 0, 0]} \\
4 & 3 & 2 & 0 & 0 & \texttt{[0, 0, 0]} & \texttt{[0, 0, 0]} \\
5 & 2 & 2 & 0 & 0 & \texttt{[0, 0]} & \texttt{[0, 0]} \\
6 & -2 & 2 & 0 & 0 & \texttt{[0]} & \texttt{[0]} \\
7 & 18 & 2 & 0 & 0 & \texttt{[0, 0]} & \texttt{[0, 0]} \\
8 & 18 & 2 & 0 & 0 & \texttt{[0, 0]} & \texttt{[0, 0]} \\
9 & 18 & 2 & 0 & 0 & \texttt{[0, 0]} & \texttt{[0, 0]} \\
\end{tabular}
}
\Case{16}{
\begin{tabular}{r|rrrrll}
$\text{pass}~p$ & $\text{term}_p$ & $\text{delta}_p$ & $\text{weight}_{p~0}$ & $\text{weight}_{p~1}$ & $\text{samples}_{p~0}$ & $\text{samples}_{p~1}$ \\
\hline
0 & 2 & 2 & 0 & 0 & \texttt{[0, 0]} & \texttt{[0, 0]} \\
1 & 18 & 2 & 0 & 0 & \texttt{[0, 0]} & \texttt{[0, 0]} \\
2 & -1 & 2 & 0 & 0 & \texttt{[0]} & \texttt{[0]} \\
3 & 8 & 2 & 0 & 0 & \texttt{[0, 0, 0, 0, 0, 0, 0, 0]} & \texttt{[0, 0, 0, 0, 0, 0, 0, 0]} \\
4 & 6 & 2 & 0 & 0 & \texttt{[0, 0, 0, 0, 0, 0]} & \texttt{[0, 0, 0, 0, 0, 0]} \\
5 & 3 & 2 & 0 & 0 & \texttt{[0, 0, 0]} & \texttt{[0, 0, 0]} \\
6 & 5 & 2 & 0 & 0 & \texttt{[0, 0, 0, 0, 0]} & \texttt{[0, 0, 0, 0, 0]} \\
7 & 7 & 2 & 0 & 0 & \texttt{[0, 0, 0, 0, 0, 0, 0]} & \texttt{[0, 0, 0, 0, 0, 0, 0]} \\
8 & 4 & 2 & 0 & 0 & \texttt{[0, 0, 0, 0]} & \texttt{[0, 0, 0, 0]} \\
9 & 2 & 2 & 0 & 0 & \texttt{[0, 0]} & \texttt{[0, 0]} \\
10 & 18 & 2 & 0 & 0 & \texttt{[0, 0]} & \texttt{[0, 0]} \\
11 & -2 & 2 & 0 & 0 & \texttt{[0]} & \texttt{[0]} \\
12 & 3 & 2 & 0 & 0 & \texttt{[0, 0, 0]} & \texttt{[0, 0, 0]} \\
13 & 2 & 2 & 0 & 0 & \texttt{[0, 0]} & \texttt{[0, 0]} \\
14 & 18 & 2 & 0 & 0 & \texttt{[0, 0]} & \texttt{[0, 0]} \\
15 & 18 & 2 & 0 & 0 & \texttt{[0, 0]} & \texttt{[0, 0]} \\
\end{tabular}
}}}}
\end{algorithm}

\end{landscape}

%% \subsection{Writing Block Set}
%% {\relsize{-1}
%% \ALGORITHM{PCM frames and their channel assignment, block index, encoding parameters}{one or more WavPack blocks}
%% \SetKwData{CHANNEL}{channel}
%% \SetKwData{FIRST}{first}
%% \SetKwData{LAST}{last}
%% \Switch(\tcc*[f]{split channels by left/right pairs}){channel assignment}{
%%   \uCase{mono}{
%%     \begin{tabular}{lrr}
%%       write $\text{\CHANNEL}_0$ & $\text{\FIRST} = 1$ & $\text{\LAST} = 1$ \\
%%     \end{tabular}\;
%%   }
%%   \uCase{front left, front right}{
%%     \begin{tabular}{lrr}
%%       write $\text{\CHANNEL}_0/\text{\CHANNEL}_1$ & $\text{\FIRST} = 1$ & $\text{\LAST} = 1$ \\
%%     \end{tabular}\;
%%   }
%%   \uCase{front left, front right, front center}{
%%     \begin{tabular}{lrr}
%%       write $\text{\CHANNEL}_0/\text{\CHANNEL}_1$ & $\text{\FIRST} = 1$ & $\text{\LAST} = 0$ \\
%%       write $\text{\CHANNEL}_2$ & $\text{\FIRST} = 0$ & $\text{\LAST} = 1$ \\
%%     \end{tabular}\;
%%   }
%%   \uCase{front left, front right, back left, back right}{
%%     \begin{tabular}{lrr}
%%       write $\text{\CHANNEL}_0/\text{\CHANNEL}_1$ & $\text{\FIRST} = 1$ & $\text{\LAST} = 0$ \\
%%       write $\text{\CHANNEL}_2/\text{\CHANNEL}_3$ & $\text{\FIRST} = 0$ & $\text{\LAST} = 1$ \\
%%     \end{tabular}\;
%%   }
%%   \uCase{front left, front right, front center, back center}{
%%     \begin{tabular}{lrr}
%%       write $\text{\CHANNEL}_0/\text{\CHANNEL}_1$ & $\text{\FIRST} = 1$ & $\text{\LAST} = 0$ \\
%%       write $\text{\CHANNEL}_2$ & $\text{\FIRST} = 0$ & $\text{\LAST} = 0$ \\
%%       write $\text{\CHANNEL}_3$ & $\text{\FIRST} = 0$ & $\text{\LAST} = 1$ \\
%%     \end{tabular}\;
%%   }
%%   \uCase{front left, front right, front center, back left, back right}{
%%     \begin{tabular}{lrr}
%%       write $\text{\CHANNEL}_0/\text{\CHANNEL}_1$ & $\text{\FIRST} = 1$ & $\text{\LAST} = 0$ \\
%%       write $\text{\CHANNEL}_2$ & $\text{\FIRST} = 0$ & $\text{\LAST} = 0$ \\
%%       write $\text{\CHANNEL}_3/\text{\CHANNEL}_4$ & $\text{\FIRST} = 0$ & $\text{\LAST} = 1$ \\
%%     \end{tabular}\;
%%   }
%%   \uCase{front left, front right, front center, LFE, back left, back right}{
%%     \begin{tabular}{lrr}
%%       write $\text{\CHANNEL}_0/\text{\CHANNEL}_1$ & $\text{\FIRST} = 1$ & $\text{\LAST} = 0$ \\
%%       write $\text{\CHANNEL}_2$ & $\text{\FIRST} = 0$ & $\text{\LAST} = 0$ \\
%%       write $\text{\CHANNEL}_3$ & $\text{\FIRST} = 0$ & $\text{\LAST} = 0$ \\
%%       write $\text{\CHANNEL}_4/\text{\CHANNEL}_5$ & $\text{\FIRST} = 0$ & $\text{\LAST} = 1$ \\
%%     \end{tabular}\;
%%   }
%%   \Other(\tcc*[f]{save them independently}){
%%     \For{i = 0 \emph{\KwTo}channel count}{
%%       \lIf{$i = 0$}{$\text{\FIRST} = 1$}
%%       \lElse{$\text{\FIRST} = 0$}\;
%%       \lIf{$i = \text{channel count} - 1$}{$\text{\LAST} = 1$}
%%       \lElse{$\text{\LAST} = 0$}\;
%%       write $\text{\CHANNEL}_i$ to block\;
%%     }
%%   }
%% }
%% \Return set of encoded WavPack blocks\;
%% \EALGORITHM


%% \clearpage

\subsection{Writing Block}
{\relsize{-1}
\ALGORITHM{1 or 2 channels of PCM frames, block index, first block, last block, encoding parameters from previous block}{a WavPack block, encoding parameters for next block}
\SetKwData{CHANNEL}{channel}
\SetKwData{MONO}{mono output}
\SetKwData{FALSESTEREO}{false stereo}
\SetKwData{MAGNITUDE}{magnitude}
\SetKwData{WASTEDBPS}{wasted bps}
\SetKwData{SHIFTED}{shifted}
\SetKwData{CRC}{CRC}
\SetKwData{CORRELATED}{correlated}
\SetKwData{MID}{mid}
\SetKwData{SIDE}{side}
\SetKwData{TERMS}{terms}
\SetKwData{DELTAS}{deltas}
\SetKwData{WEIGHTS}{weights}
\SetKwData{SAMPLES}{samples}
\SetKwData{BITSTREAM}{bitstream}
\SetKwData{ENTROPY}{entropy}
\SetKw{OR}{or}
\SetKwFunction{MAX}{max}
\SetKwFunction{MIN}{min}
\eIf(\tcc*[f]{1 channel block}){$\text{channel count} = 1$ \OR $\text{\CHANNEL}_0 = \text{\CHANNEL}_1$}{
  \eIf{$\text{channel count} = 1$}{
    $\text{\MONO} \leftarrow 1$\;
    $\text{\FALSESTEREO} \leftarrow 0$\;
  }{
    $\text{\MONO} \leftarrow 0$\;
    $\text{\FALSESTEREO} \leftarrow 1$\;
  }
  $\text{\MAGNITUDE} \leftarrow$ maximum magnitude of $\text{\CHANNEL}_0$\;
  $\text{\WASTEDBPS} \leftarrow$ wasted bps of $\text{\CHANNEL}_0$\;
  \eIf{$\text{\WASTEDBPS} > 0$}{
    \For{i = 0 \emph{\KwTo}block size}{
      $\text{\SHIFTED}_{0~i} \leftarrow \lfloor\text{\CHANNEL}_{0~i} \div 2 ^ \text{\WASTEDBPS}\rfloor$\;
    }
  }{
    $\text{\SHIFTED}_0 \leftarrow \text{\CHANNEL}_0$\;
  }
  $\text{\CRC} \leftarrow$ calculate CRC of $\text{\SHIFTED}_0$\;
  $(\text{\CORRELATED}_0~,~\text{\WEIGHTS}'~,~\text{\SAMPLES}') \leftarrow$ correlate $\text{\SHIFTED}_0$\newline
  with \TERMS, \DELTAS, \WEIGHTS and \SAMPLES from encoding parameters\;
  $(\text{\BITSTREAM}~,~\text{\ENTROPY}') \leftarrow$ calculate bitstream from $\text{\CORRELATED}_0$\newline
  and $\text{\ENTROPY}$ from encoding parameters\;
}(\tcc*[f]{2 channel block}){
  $\text{\MONO} \leftarrow 0$\;
  $\text{\FALSESTEREO} \leftarrow 0$\;
  $\text{\MAGNITUDE} \leftarrow \MAX(\text{maximum magnitude of \CHANNEL}_0~,~\text{maximum magnitude of \CHANNEL}_1)$\;
  $\text{\WASTEDBPS} \leftarrow \MIN(\text{wasted bps of \CHANNEL}_0~,~\text{wasted bps of \CHANNEL}_1)$\;
  \eIf{$\text{\WASTEDBPS} > 0$}{
    \For{i = 0 \emph{\KwTo}block size}{
      $\text{\SHIFTED}_{0~i} \leftarrow \lfloor\text{\CHANNEL}_{0~i} \div 2 ^ \text{\WASTEDBPS}\rfloor$\;
      $\text{\SHIFTED}_{1~i} \leftarrow \lfloor\text{\CHANNEL}_{1~i} \div 2 ^ \text{\WASTEDBPS}\rfloor$\;
    }
  }{
    $\text{\SHIFTED}_0/\text{\SHIFTED}_1 \leftarrow \text{\CHANNEL}_0/\text{\CHANNEL}_1$\;
  }
  $\text{\CRC} \leftarrow$ calculate CRC of $\text{\SHIFTED}_0/\text{\SHIFTED}_1$\;
  $\text{\MID}/\text{\SIDE} \leftarrow$ convert $\text{\SHIFTED}_0/\text{\SHIFTED}_1$ to joint stereo\;
  $(\text{\CORRELATED}_0/\text{\CORRELATED}_1~,~\text{\WEIGHTS}'~,~\text{\SAMPLES}') \leftarrow$ correlate $\text{\MID}/\text{\SIDE}$\newline
  with \TERMS, \DELTAS, \WEIGHTS and \SAMPLES from encoding parameters\;
  $(\text{\BITSTREAM}~,~\text{\ENTROPY}') \leftarrow$ calculate bitstream from $\text{\CORRELATED}_0/\text{\CORRELATED}_0$\newline
  and $\text{\ENTROPY}$ from encoding parameters\;
}
\EALGORITHM
}

\clearpage
{\relsize{-1}
\begin{algorithm}[H]
\DontPrintSemicolon
\SetKwData{SUBBLOCK}{sub block}
\SetKwData{WASTEDBPS}{wasted bps}
\SetKwData{BITSTREAM}{bitstream}
\SetKwData{TERMS}{terms}
\SetKwData{DELTAS}{deltas}
\SetKwData{WEIGHTS}{weights}
\SetKwData{SAMPLES}{samples}
\SetKwData{ENTROPY}{entropy}
\SetKw{NOT}{not}
\SetKw{IN}{in}
$i \leftarrow 0$\;
\If{first block in file}{
  $\text{\SUBBLOCK}_i \leftarrow$ wave header\;
  $i \leftarrow i + 1$\;
}
\If{$\text{decorrelation passes} > 0$}{
  $\text{\SUBBLOCK}_i \leftarrow$ decorrelation terms sub block from \TERMS and \DELTAS\;
  $\text{\SUBBLOCK}_{i + 1} \leftarrow$ decorrelation weights sub block  from \WEIGHTS\;
  $\text{\SUBBLOCK}_{i + 2} \leftarrow$ decorrelation samples sub block  from \SAMPLES\;
  $i \leftarrow i + 3$\;
}
\If{$\text{\WASTEDBPS} > 0$}{
  $\text{\SUBBLOCK}_i \leftarrow$ extended integers\;
  $i \leftarrow i + 1$\;
}
\If{$\text{total channel count} > 2$}{
  $\text{\SUBBLOCK}_i \leftarrow$ channel info\;
  $i \leftarrow i + 1$\;
}
\If{sample rate not defined in block header}{
  $\text{\SUBBLOCK}_i \leftarrow$ sample rate\;
  $i \leftarrow i + 1$\;
}
$\text{\SUBBLOCK}_i \leftarrow$ entropy variables sub block from \ENTROPY\;
$\text{\SUBBLOCK}_{i + 1} \leftarrow$ bitstream sub block from \BITSTREAM\;
\BlankLine
write block header\;
\For{j = 0 \emph{\KwTo}i}{
  write $\text{\SUBBLOCK}_j$\;
}
\Return block data, $\text{\WEIGHTS}'$, $\text{\SAMPLES}'$ and $\text{\ENTROPY}'$
\end{algorithm}
}

\clearpage

\subsection{Calculating Maximum Magnitude}
{\relsize{-1}
\ALGORITHM{a list of signed PCM samples for a single channel}{an unsigned integer}
\SetKwData{MAXMAGNITUDE}{maximum magnitude}
\SetKwData{SAMPLE}{sample}
\SetKwFunction{MAX}{max}
\SetKwFunction{BITS}{bits}
$\text{\MAXMAGNITUDE} \leftarrow 0$\;
\For{i = 0 \emph{\KwTo}sample count}{
  $\text{\MAXMAGNITUDE} \leftarrow \MAX(\BITS(|\text{\SAMPLE}_i|)~,~\text{\MAXMAGNITUDE})$\;
}
\Return \MAXMAGNITUDE\;
\EALGORITHM
where the \texttt{bits} function is defined as:
\begin{equation*}
\texttt{bits}(x) =
\begin{cases}
0 & \text{if } x = 0 \\
1 + \texttt{bits}(\lfloor x \div 2 \rfloor) & \text{if } x > 0
\end{cases}
\end{equation*}
}
\subsubsection{Maximum Magnitude Example}
\begin{table}[h]
{\relsize{-1}
\begin{tabular}{r|rrrrrrrrrr}
$i$ & 0 & 1 & 2 & 3 & 4 & 5 & 6 & 7 & 8 & 9 \\
\hline
$\text{sample}_i$ & 0 & 16 & 31 & 44 & 54 & 61 & 64 & 63 & 58 & 49 \\
$\texttt{bits}(|\text{sample}_i|)$ & 0 & 5 & 5 & 6 & 6 & 6 & 7 & 6 & 6 & 6
\end{tabular}
}
\end{table}
\par
\noindent
for a maximum magnitude of 7.

\subsection{Calculating Wasted Bits Per Sample}
{\relsize{-1}
\ALGORITHM{a list of signed PCM samples for a single channel}{an unsigned integer}
\SetKwData{WASTEDBPS}{wasted bps}
\SetKwData{SAMPLE}{sample}
\SetKwFunction{MIN}{min}
\SetKwFunction{WASTED}{wasted}
$\text{\WASTEDBPS} \leftarrow \infty$\tcc*[r]{maximum unsigned integer}
\For{i = 0 \emph{\KwTo}sample count}{
  $\text{\WASTEDBPS} \leftarrow \MIN(\WASTED(\text{\SAMPLE}_i)~,~\text{\WASTEDBPS})$\;
}
\eIf(\tcc*[f]{all samples are 0}){$\WASTEDBPS = \infty$}{
  \Return 0\;
}{
  \Return \WASTEDBPS\;
}
\EALGORITHM
where the \texttt{wasted} function is defined as:
\begin{equation*}
\texttt{wasted}(x) =
\begin{cases}
\infty & \text{if } x = 0 \\
0 & \text{if } x \bmod 2 = 1 \\
1 + \texttt{wasted}(x \div 2) & \text{if } x \bmod 2 = 0 \\
\end{cases}
\end{equation*}
}
\subsubsection{Wasted Bits Example}
\begin{table}[h]
{\relsize{-1}
\begin{tabular}{r|rrrrrrrrrr}
$i$ & 0 & 1 & 2 & 3 & 4 & 5 & 6 & 7 & 8 & 9 \\
\hline
$\text{sample}_i$ & 0 & 16 & 31 & 44 & 54 & 61 & 64 & 63 & 58 & 49 \\
$\texttt{wasted}(\text{sample}_i)$ & $\infty$ & 4 & 0 & 2 & 1 & 0 & 6 & 0 & 1 & 0 \\
\end{tabular}
}
\end{table}
\par
\noindent
for a wasted bps of 0 (which is typical).

\begin{landscape}

\subsection{Calculating CRC}
\ALGORITHM{one or two channels of signed audio samples}{an unsigned 32-bit CRC integer}
\SetKwData{MONO}{mono output}
\SetKwData{CRC}{CRC}
\SetKwData{LCRC}{LCRC}
\SetKwData{SCRC}{SCRC}
\SetKwData{CHANNEL}{channel}
$\text{\CRC}_{-1} \leftarrow \texttt{0xFFFFFFFF}$\;
\For{i = 0 \emph{\KwTo}sample count}{
  \eIf{$\text{\MONO} = 0$}{
    $\text{\LCRC}_i \leftarrow (3 \times \text{\CRC}_{i - 1}) + \text{\CHANNEL}_{0~i}$\tcc*[r]{calculate signed CRC of left channel}
    $\text{\SCRC}_i \leftarrow (3 \times \text{\LCRC}_{i - 1}) + \text{\CHANNEL}_{1~i}$\tcc*[r]{calculate signed CRC of right channel}
  }{
    $\text{\SCRC}_i \leftarrow (3 \times \text{\CRC}_{i - 1}) + \text{\CHANNEL}_{0~i}$\tcc*[r]{calculate signed CRC of channel}
  }
  \BlankLine
  \eIf(\tcc*[f]{convert signed CRC to unsigned, 32-bit integer}){$\text{\SCRC}_i \geq 0$}{
    $\text{\CRC}_i \leftarrow \text{\SCRC}_i \bmod \texttt{0x100000000}$\;
  }{
    $\text{\CRC}_i \leftarrow (2 ^ {32} - (-\text{\SCRC}_i)) \bmod \texttt{0x100000000}$\;
  }
}
\Return $\text{\CRC}_{\text{sample count} - 1}$\;
\EALGORITHM

\subsubsection{Checksum Calculation Example}
{\relsize{-1}
\begin{tabular}{|r|r|r||>{$}r<{$}|>{$}r<{$}|>{$}r<{$}|}
$i$ & $\text{channel}_{0~i}$ & $\text{channel}_{1~i}$ & \text{LCRC}_i & \text{SCRC}_i & \text{CRC}_i \\
\hline
0 & 0 & 64 &
(3 \times \texttt{0xFFFFFFFF}) + 0 = \texttt{0x2FFFFFFFD} &
(3 \times \texttt{0x2FFFFFFFD}) + 64 = \texttt{0x900000037} &
\texttt{0x00000037} \\
1 & 16 & 62 &
(3 \times \texttt{0x00000037}) + 16 = \texttt{0x000000B5} &
(3 \times \texttt{0x000000B5}) + 62 = \texttt{0x0000025D} &
\texttt{0x0000025D} \\
2 & 31 & 56 &
(3 \times \texttt{0x0000025D}) + 31 = \texttt{0x00000736} &
(3 \times \texttt{0x00000736}) + 56 = \texttt{0x000015DA} &
\texttt{0x000015DA} \\
3 & 44 & 47 &
(3 \times \texttt{0x000015DA}) + 44 = \texttt{0x000041BA} &
(3 \times \texttt{0x000041BA}) + 47 = \texttt{0x0000C55D} &
\texttt{0x0000C55D} \\
4 & 54 & 34 &
(3 \times \texttt{0x0000C55D}) + 54 = \texttt{0x0002504D} &
(3 \times \texttt{0x0002504D}) + 34 = \texttt{0x0006F109} &
\texttt{0x0006F109} \\
5 & 61 & 20 &
(3 \times \texttt{0x0006F109}) + 61 = \texttt{0x0014D358} &
(3 \times \texttt{0x0014D358}) + 20 = \texttt{0x003E7A1C} &
\texttt{0x003E7A1C} \\
6 & 64 & 4 &
(3 \times \texttt{0x003E7A1C}) + 64 = \texttt{0x00BB6E94} &
(3 \times \texttt{0x00BB6E94}) + 4 = \texttt{0x02324BC0} &
\texttt{0x02324BC0} \\
7 & 63 & -12 &
(3 \times \texttt{0x02324BC0}) + 63 = \texttt{0x0696E37F} &
(3 \times \texttt{0x0696E37F}) - 12 = \texttt{0x13C4AA71} &
\texttt{0x13C4AA71} \\
8 & 58 & -27 &
(3 \times \texttt{0x13C4AA71}) + 58 = \texttt{0x3B4DFF8D} &
(3 \times \texttt{0x3B4DFF8D}) - 27 = \texttt{0xB1E9FE8C} &
\texttt{0xB1E9FE8C} \\
9 & 49 & -41 &
(3 \times \texttt{0xB1E9FE8C}) + 49 = \texttt{0x215BDFBD5} &
(3 \times \texttt{0x215BDFBD5}) - 41 = \texttt{0x64139F356} &
\texttt{0x4139F356} \\
\end{tabular}
}
\vskip 1em
\par
\noindent
Resulting in a final CRC of \texttt{0x4139F356}

\end{landscape}

\subsection{Joint Stereo Conversion}
\ALGORITHM{left and right channels of signed integers}{mid and side channels of signed integers}
\SetKwData{LEFT}{left}
\SetKwData{RIGHT}{right}
\SetKwData{MID}{mid}
\SetKwData{SIDE}{side}
\For{i = 0 \emph{\KwTo}sample count}{
  $\text{\MID}_i \leftarrow \text{\LEFT}_i - \text{\RIGHT}_i$\;
  $\text{\SIDE}_i \leftarrow \lfloor(\text{\LEFT}_i + \text{\RIGHT}_i) \div 2\rfloor$\;
}
\Return \MID and \SIDE channels\;
\EALGORITHM

\subsubsection{Joint Stereo Example}
\begin{table}[h]
{\relsize{-1}
\begin{tabular}{|r|r|r||>{$}r<{$}|>{$}r<{$}|}
$i$ & $\text{left}_i$ & $\text{right}_i$ & \text{mid}_i & \text{side}_i \\
\hline
0 & 0 & 64 & 0 - 64 = -64 & \lfloor(0 + 64) \div 2\rfloor = 32 \\
1 & 16 & 62 & 16 - 62 = -46 & \lfloor(16 + 62) \div 2\rfloor = 39 \\
2 & 31 & 56 & 31 - 56 = -25 & \lfloor(31 + 56) \div 2\rfloor = 43 \\
3 & 44 & 47 & 44 - 47 = -3 & \lfloor(44 + 47) \div 2\rfloor = 45 \\
4 & 54 & 34 & 54 - 34 = 20 & \lfloor(54 + 34) \div 2\rfloor = 44 \\
5 & 61 & 20 & 61 - 20 = 41 & \lfloor(61 + 20) \div 2\rfloor = 40 \\
6 & 64 & 4 & 64 - 4 = 60 & \lfloor(64 + 4) \div 2\rfloor = 34 \\
7 & 63 & -12 & 63 - -12 = 75 & \lfloor(63 + -12) \div 2\rfloor = 25 \\
8 & 58 & -27 & 58 - -27 = 85 & \lfloor(58 + -27) \div 2\rfloor = 15 \\
9 & 49 & -41 & 49 - -41 = 90 & \lfloor(49 + -41) \div 2\rfloor = 4 \\
\end{tabular}
}
\end{table}

\clearpage

\subsection{Correlation Passes}
\ALGORITHM{a list of signed samples per channel; correlation terms, deltas, weights and samples}{a list of signed residuals per channel; correlation weights and samples for the next block}
\SetKwData{PASS}{pass}
\SetKwData{CHANNEL}{channel}
\SetKwData{TERMCOUNT}{term count}
\SetKwData{TERM}{term}
\SetKwData{DELTA}{delta}
\SetKwData{WEIGHT}{weight}
\SetKwData{SAMPLES}{samples}
\SetKw{KwDownTo}{downto}
\eIf{$\text{channel count} = 1$}{
  $\text{\PASS}_{-1~0} \leftarrow \text{\CHANNEL}_0$\;
  \For(\tcc*[f]{perform passes in reverse order}){i = 0 \emph{\KwTo}\TERMCOUNT}{
    $p \leftarrow \TERMCOUNT - i - 1$\;
    $(\text{\PASS}_{i~0}~,~\text{\WEIGHT}'_{p~0}~,~\text{\SAMPLES}'_{p~0}) \leftarrow $correlate $\text{\PASS}_{0~(i - 1)}$\newline
    using $\text{\TERM}_p$, $\text{\DELTA}_p$, $\text{\WEIGHT}_{p~0}$ and
    $\text{\SAMPLES}_{p~0}$\;
  }
  \Return $\text{\PASS}_{(\TERMCOUNT - 1)~0}$, $\text{\WEIGHT}'$ values, $\text{\SAMPLES}'$ values\;
}{
  $\text{\PASS}_{-1~0} \leftarrow \text{\CHANNEL}_0$\;
  $\text{\PASS}_{-1~1} \leftarrow \text{\CHANNEL}_1$\;
  \For(\tcc*[f]{perform passes in reverse order}){i = 0 \emph{\KwTo}\TERMCOUNT}{
    $p \leftarrow \TERMCOUNT - i - 1$\;
    $(\text{\PASS}_{i~0}~/~\text{\PASS}_{i~1}~,~\text{\WEIGHT}'_{p~0}~,~\text{\WEIGHT}'_{p~1}~,~\text{\SAMPLES}'_{p~0}~,~\text{\SAMPLES}'_{p~1}) \leftarrow$ correlate $\text{\PASS}_{(i - 1)~0}~/~\text{\PASS}_{(i - 1)~1}$\newline
    using $\text{\TERM}_p$, $\text{\DELTA}_p$, $\text{\WEIGHT}_{p~0}$,  $\text{\WEIGHT}_{p~1}$, $\text{\SAMPLES}_{p~0}$ and $\text{\SAMPLES}_{p~1}$\;
  }
  \Return $\text{\PASS}_{(\TERMCOUNT - 1)~0}~/~\text{\PASS}_{(\TERMCOUNT - 1)~1}$, $\text{\WEIGHT}'$ values, $\text{\SAMPLES}'$ values\;
}
\EALGORITHM

\clearpage

\subsection{1 Channel Correlation Pass}
{\relsize{-1}
\ALGORITHM{a list of signed uncorrelated samples; correlation term, delta, weight and samples}{a list of signed correlated samples}
\SetKwData{CORRELATED}{correlated}
\SetKwData{DECORRELATED}{uncorrelated}
\SetKwData{DECORRSAMPLE}{correlation sample}
\SetKwData{WEIGHT}{weight}
\SetKwData{TEMP}{temp}
\SetKw{OR}{or}
\SetKw{XOR}{xor}
\SetKwFunction{APPLYWEIGHT}{apply\_weight}
\SetKwFunction{UPDATEWEIGHT}{update\_weight}
$\text{\WEIGHT}_0 \leftarrow$ decorrelation weight\;
\BlankLine
\uIf{$term = 18$}{
  $\text{\DECORRELATED}_{-2} \leftarrow \text{\DECORRSAMPLE}_1$\;
  $\text{\DECORRELATED}_{-1} \leftarrow \text{\DECORRSAMPLE}_0$\;
  \For{i = 0 \emph{\KwTo}uncorrelated samples length}{
    $\text{\TEMP}_{i} \leftarrow \lfloor(3 \times \text{\DECORRELATED}_{i - 1} - \text{\DECORRELATED}_{i - 2}) \div 2 \rfloor$\;
    $\text{\CORRELATED}_i \leftarrow \text{\DECORRELATED}_i - \APPLYWEIGHT(\text{\WEIGHT}_i~,~\text{\TEMP}_{i})$\;
    $\text{\WEIGHT}_{i + 1} \leftarrow \text{\WEIGHT}_i + \UPDATEWEIGHT(\text{\TEMP}_{i}~,~\text{\CORRELATED}_i~,~delta)$\;
  }
  \Return \CORRELATED samples\;
}
\uElseIf{$term = 17$}{
  $\text{\DECORRELATED}_{-2} \leftarrow \text{\DECORRSAMPLE}_1$\;
  $\text{\DECORRELATED}_{-1} \leftarrow \text{\DECORRSAMPLE}_0$\;
  \For{i = 0 \emph{\KwTo}uncorrelated samples length}{
    $\text{\TEMP}_{i} \leftarrow 2 \times \text{\DECORRELATED}_{i - 1} - \text{\DECORRELATED}_{i - 2}$\;
    $\text{\CORRELATED}_i \leftarrow  \text{\DECORRELATED}_i - \APPLYWEIGHT(\text{\WEIGHT}_i~,~\text{\TEMP}_{i})$\;
    $\text{\WEIGHT}_{i + 1} \leftarrow \text{\WEIGHT}_i + \UPDATEWEIGHT(\text{\TEMP}_{i}~,~\text{\CORRELATED}_i~,~delta)$\;
  }
  \Return \CORRELATED samples\;
}
\uElseIf{$1 \leq term \leq 8$}{
  \For{i = 0 \emph{\KwTo}term}{
    $\text{\DECORRELATED}_{i - \text{term}} \leftarrow \text{\DECORRSAMPLE}_i$\;
  }
  \For{i = 0 \emph{\KwTo}correlated samples length}{
    $\text{\CORRELATED}_i \leftarrow  \text{\DECORRELATED}_i - \APPLYWEIGHT(\text{\WEIGHT}_i~,~\text{\DECORRELATED}_{i - \text{term}})$\;
    $\text{\WEIGHT}_{i + 1} \leftarrow \text{\WEIGHT}_i + \UPDATEWEIGHT(\text{\DECORRELATED}_{i - \text{term}}~,~\text{\CORRELATED}_i~,~delta)$\;
  }
  \BlankLine
  \Return \CORRELATED samples\;
}
\Else{
  invalid decorrelation term\;
}
\EALGORITHM
\par
\noindent
\begin{align*}
\intertext{where \texttt{apply\_weight} is defined as:}
\texttt{apply\_weight}(weight~,~sample) &= \left\lfloor\frac{weight \times sample + 2 ^ 9}{2 ^ {10}}\right\rfloor \\
\intertext{and \texttt{update\_weight} is defined as:}
\texttt{update\_weight}(source~,~result~,~delta) &=
\begin{cases}
0 & \text{ if } source = 0 \text{ or } result = 0 \\
delta & \text{ if } (source \textbf{ xor } result ) \geq 0 \\
-delta & \text{ if } (source \textbf{ xor } result) < 0
\end{cases}
\end{align*}
}

\clearpage

\subsection{2 Channel Correlation Pass}
{\relsize{-1}
\ALGORITHM{2 lists of signed uncorrelated samples; correlation term and delta, 2 correlation weights, 2 lists of correlation samples}{2 lists of signed correlated samples}
\SetKwData{CORRELATED}{correlated}
\SetKwData{DECORRELATED}{uncorrelated}
\SetKwData{DECORRSAMPLE}{correlation sample}
\SetKwData{WEIGHT}{weight}
\SetKw{OR}{or}
\SetKw{XOR}{xor}
\SetKwFunction{MIN}{min}
\SetKwFunction{MAX}{max}
\SetKwFunction{APPLYWEIGHT}{apply\_weight}
\SetKwFunction{UPDATEWEIGHT}{update\_weight}
\uIf{$17 \leq term \leq 18$ \OR $1 \leq term \leq 8$}{
  $\text{\CORRELATED}_0 \leftarrow$ 1 channel decorrelation pass of $\text{\DECORRELATED}_0$\;
  $\text{\CORRELATED}_1 \leftarrow$ 1 channel decorrelation pass of $\text{\DECORRELATED}_1$\;
  \Return $\text{\CORRELATED}_0$ and $\text{\CORRELATED}_1$\;
}
\uElseIf{$-3 \leq term \leq -1$}{
  $\text{\WEIGHT}_{0~0} \leftarrow$ correlation weight 0\;
  $\text{\WEIGHT}_{1~0} \leftarrow$ correlation weight 1\;
  $\text{\DECORRELATED}_{0~-1} \leftarrow \text{\DECORRSAMPLE}_{1~0}$\;
  $\text{\DECORRELATED}_{1~-1} \leftarrow \text{\DECORRSAMPLE}_{0~0}$\;
  \uIf{$term = -1$}{
    \For{i = 0 \emph{\KwTo}uncorrelated samples length}{
      $\text{\CORRELATED}_{0~i} \leftarrow  \text{\DECORRELATED}_{0~i} - \APPLYWEIGHT(\text{\WEIGHT}_{0~i}~,~\text{\DECORRELATED}_{1~(i - 1)})$\;
      $\text{\CORRELATED}_{1~i} \leftarrow \text{\DECORRELATED}_{1~i} - \APPLYWEIGHT(\text{\WEIGHT}_{1~i}~,~\text{\DECORRELATED}_{0~i})$\;
      $\text{\WEIGHT}_{0~(i + 1)} \leftarrow \text{\WEIGHT}_{0~i} + \UPDATEWEIGHT(\text{\DECORRELATED}_{1~(i - 1)}~,~\text{\CORRELATED}_{0~i}~,~delta)$\;
      $\text{\WEIGHT}_{1~(i + 1)} \leftarrow \text{\WEIGHT}_{1~i} + \UPDATEWEIGHT(\text{\DECORRELATED}_{0~i}~,~\text{\CORRELATED}_{1~i}~,~delta)$\;
      $\text{\WEIGHT}_{0~(i + 1)} \leftarrow \MAX(\MIN(\text{\WEIGHT}_{0~(i + 1)}~,~1024)~,~-1024)$\;
      $\text{\WEIGHT}_{1~(i + 1)} \leftarrow \MAX(\MIN(\text{\WEIGHT}_{1~(i + 1)}~,~1024)~,~-1024)$\;
    }
  }
  \uElseIf{$term = -2$}{
    \For{i = 0 \emph{\KwTo}uncorrelated samples length}{
      $\text{\CORRELATED}_{0~i} \leftarrow \text{\DECORRELATED}_{0~i} - \APPLYWEIGHT(\text{\WEIGHT}_{0~i}~,~\text{\DECORRELATED}_{1~i})$\;
      $\text{\CORRELATED}_{1~i} \leftarrow \text{\DECORRELATED}_{1~i} - \APPLYWEIGHT(\text{\WEIGHT}_{1~i}~,~\text{\DECORRELATED}_{0~({i - 1})})$\;
      $\text{\WEIGHT}_{0~(i + 1)} \leftarrow \text{\WEIGHT}_{0~i} + \UPDATEWEIGHT(\text{\DECORRELATED}_{1~i}~,~\text{\CORRELATED}_{0~i}~,~delta)$\;
      $\text{\WEIGHT}_{1~(i + 1)} \leftarrow \text{\WEIGHT}_{1~i} + \UPDATEWEIGHT(\text{\DECORRELATED}_{0~(i - 1)}~,~\text{\CORRELATED}_{1~i}~,~delta)$\;
      $\text{\WEIGHT}_{0~(i + 1)} \leftarrow \MAX(\MIN(\text{\WEIGHT}_{0~(i + 1)}~,~1024)~,~-1024)$\;
      $\text{\WEIGHT}_{1~(i + 1)} \leftarrow \MAX(\MIN(\text{\WEIGHT}_{1~(i + 1)}~,~1024)~,~-1024)$\;
    }
  }
  \ElseIf{$term = -3$}{
    \For{i = 0 \emph{\KwTo}uncorrelated samples length}{
      $\text{\CORRELATED}_{0~i} \leftarrow \text{\DECORRELATED}_{0~i} - \APPLYWEIGHT(\text{\WEIGHT}_{0~i}~,~\text{\DECORRELATED}_{1~(i - 1)})$\;
      $\text{\CORRELATED}_{1~i} \leftarrow \text{\DECORRELATED}_{1~i} - \APPLYWEIGHT(\text{\WEIGHT}_{1~i}~,~\text{\DECORRELATED}_{0~(i - 1)})$\;
      $\text{\WEIGHT}_{0~(i + 1)} \leftarrow \text{\WEIGHT}_{0~i} + \UPDATEWEIGHT(\text{\DECORRELATED}_{1~(i - 1)}~,~\text{\CORRELATED}_{0~i}~,~delta)$\;
      $\text{\WEIGHT}_{1~(i + 1)} \leftarrow \text{\WEIGHT}_{1~i} + \UPDATEWEIGHT(\text{\DECORRELATED}_{0~(i - 1)}~,~\text{\CORRELATED}_{1~i}~,~delta)$\;
      $\text{\WEIGHT}_{0~(i + 1)} \leftarrow \MAX(\MIN(\text{\WEIGHT}_{0~(i + 1)}~,~1024)~,~-1024)$\;
      $\text{\WEIGHT}_{1~(i + 1)} \leftarrow \MAX(\MIN(\text{\WEIGHT}_{1~(i + 1)}~,~1024)~,~-1024)$\;
    }
  }
  \Return $\text{\CORRELATED}_0$ and $\text{\CORRELATED}_1$\;
}
\Else{
  invalid decorrelation term\;
}
\EALGORITHM
}

\clearpage

\subsection{Channel Correlation Example}
\begin{figure}[h]
{\relsize{-1}
  \subfloat{
    \begin{tabular}{|r|r|r|}
      \multicolumn{3}{c}{Correlation Terms} \\
      \hline
      $p$ & $\text{term}_p$ & $\text{delta}_p$ \\
      \hline
      0 & 3 & 2 \\
      1 & 17 & 2 \\
      2 & 2 & 2 \\
      3 & 18 & 2 \\
      4 & 18 & 2 \\
      \hline
    \end{tabular}
  }
  \subfloat{
    \begin{tabular}{|r|r|r|}
      \multicolumn{3}{c}{Correlation Weights} \\
      \hline
      $p$ & $\text{weight}_{p~0}$ & $\text{weight}_{p~1}$ \\
      \hline
      0 & 16 & 24 \\
      1 & 48 & 48 \\
      2 & 32 & 32 \\
      3 & 48 & 48 \\
      4 & 48 & 48 \\
      \hline
    \end{tabular}
  }
  \subfloat{
    \begin{tabular}{|r|r|r|}
      \multicolumn{3}{c}{Correlation Samples} \\
      \hline
      $p$ & $\text{sample}_{p~0~s}$ & $\text{sample}_{p~1~s}$ \\
      \hline
      0 & \texttt{[0, 0, 0]} & \texttt{[0, 0, 0]} \\
      1 & \texttt{[0, 0]} & \texttt{[0, 0]} \\
      2 & \texttt{[0, 0]} & \texttt{[0, 0]} \\
      3 & \texttt{[0, 0]} & \texttt{[0, 0]} \\
      4 & \texttt{[-73, -78]} & \texttt{[28, 26]} \\
      \hline
    \end{tabular}
  }
}
\end{figure}
\par
\noindent
we combine them into a single set of arguments for each correlation pass:
\begin{table}[h]
{\relsize{-1}
  \begin{tabular}{|r|r|r|r|r|r|}
    \hline
    & $\textbf{pass}_0$ & $\textbf{pass}_1$ & $\textbf{pass}_2$ &
    $\textbf{pass}_3$ & $\textbf{pass}_3$ \\
    \hline
    $\text{term}_p$ & 18 & 18 & 2 & 17 & 3 \\
    $\text{delta}_p$ & 2 & 2 & 2 & 2 & 2 \\
    $\text{weight}_{p~0}$ & 48 & 48 & 32 & 48 & 16 \\
    $\text{samples}_{p~0~s}$ & \texttt{[-73, -78]} & \texttt{[0, 0]} &
    \texttt{[0, 0]} & \texttt{[0, 0]} & \texttt{[0, 0, 0]} \\
    $\text{weight}_{p~1}$ & 48 & 48 & 32 & 48 & 24 \\
    $\text{samples}_{p~1~s}$ & \texttt{[28, 26]} & \texttt{[0, 0]} &
    \texttt{[0, 0]} & \texttt{[0, 0]} & \texttt{[0, 0, 0]} \\
    \hline
  \end{tabular}
}
\end{table}
\par
\noindent
which we apply to the residuals from the bitstream sub-block:
\par
\noindent
{\relsize{-1}
  \begin{tabular}{|r|r|r|r|r|r|}
    \hline
    $\text{channel}_{0~i}$ &
    after $\textbf{pass}_0$ &
    after $\textbf{pass}_1$ &
    after $\textbf{pass}_2$ &
    after $\textbf{pass}_3$ &
    after $\textbf{pass}_4$ \\
    \hline
    -64 & -61 & -61 & -61 & -61 & -61 \\
    -46 & -43 & -39 & -39 & -33 & -33 \\
    -25 & -23 & -21 & -19 & -18 & -18 \\
    -3 & -2 & -1 & 0 & 0 & 1 \\
    20 & 20 & 20 & 21 & 20 & 20 \\
    41 & 39 & 37 & 37 & 35 & 35 \\
    60 & 57 & 54 & 53 & 50 & 50 \\
    75 & 71 & 67 & 66 & 62 & 62 \\
    85 & 80 & 75 & 73 & 68 & 68 \\
    90 & 84 & 79 & 77 & 72 & 71 \\
    \hline
    \hline
    $\text{channel}_{1~i}$ &
    after $\textbf{pass}_0$ &
    after $\textbf{pass}_1$ &
    after $\textbf{pass}_2$ &
    after $\textbf{pass}_3$ &
    after $\textbf{pass}_4$ \\
    \hline
    32 & 31 & 31 & 31 & 31 & 31 \\
    39 & 37 & 35 & 35 & 32 & 32 \\
    43 & 41 & 39 & 38 & 36 & 36 \\
    45 & 43 & 41 & 40 & 38 & 37 \\
    44 & 41 & 39 & 38 & 36 & 35 \\
    40 & 38 & 36 & 34 & 32 & 31 \\
    34 & 32 & 30 & 28 & 26 & 25 \\
    25 & 23 & 21 & 20 & 19 & 18 \\
    15 & 14 & 13 & 12 & 11 & 10 \\
    4 & 3 & 2 & 1 & 1 & 0 \\
    \hline
  \end{tabular}
}
\par
\noindent
Resulting in final correlated samples:
\newline
\begin{tabular}{rr}
$\text{residual}_0$ : & \texttt{[-61,~-33,~-18,~~1,~20,~35,~50,~62,~68,~71]} \\
$\text{residual}_1$ : & \texttt{[~31,~~32,~~36,~37,~35,~31,~25,~18,~10,~~0]} \\
\end{tabular}

\clearpage

{\relsize{-2}
\begin{tabular}{r||r|>{$}r<{$}|>{$}r<{$}|>{$}r<{$}|>{$}r<{$}}
& $i$ & \text{uncorrelated}_i & \text{temp}_i & \text{correlated}_i & \text{weight}_{i + 1} \\
\hline
%%START
\multirow{10}{1em}{\begin{sideways}$\textbf{pass}_0$ - term 18\end{sideways}}
& 0 & -64 &
\lfloor(3 \times -73 + 78) \div 2\rfloor = -71 &
-64 - \lfloor(48 \times -71 + 2 ^ 9) \div 2 ^ {10}\rfloor = -61 &
48 + 2 = 50
\\
& 1 & -46 &
\lfloor(3 \times -64 + 73) \div 2\rfloor = -60 &
-46 - \lfloor(50 \times -60 + 2 ^ 9) \div 2 ^ {10}\rfloor = -43 &
50 + 2 = 52
\\
& 2 & -25 &
\lfloor(3 \times -46 + 64) \div 2\rfloor = -37 &
-25 - \lfloor(52 \times -37 + 2 ^ 9) \div 2 ^ {10}\rfloor = -23 &
52 + 2 = 54
\\
& 3 & -3 &
\lfloor(3 \times -25 + 46) \div 2\rfloor = -15 &
-3 - \lfloor(54 \times -15 + 2 ^ 9) \div 2 ^ {10}\rfloor = -2 &
54 + 2 = 56
\\
& 4 & 20 &
\lfloor(3 \times -3 + 25) \div 2\rfloor = 8 &
20 - \lfloor(56 \times 8 + 2 ^ 9) \div 2 ^ {10}\rfloor = 20 &
56 + 2 = 58
\\
& 5 & 41 &
\lfloor(3 \times 20 + 3) \div 2\rfloor = 31 &
41 - \lfloor(58 \times 31 + 2 ^ 9) \div 2 ^ {10}\rfloor = 39 &
58 + 2 = 60
\\
& 6 & 60 &
\lfloor(3 \times 41 - 20) \div 2\rfloor = 51 &
60 - \lfloor(60 \times 51 + 2 ^ 9) \div 2 ^ {10}\rfloor = 57 &
60 + 2 = 62
\\
& 7 & 75 &
\lfloor(3 \times 60 - 41) \div 2\rfloor = 69 &
75 - \lfloor(62 \times 69 + 2 ^ 9) \div 2 ^ {10}\rfloor = 71 &
62 + 2 = 64
\\
& 8 & 85 &
\lfloor(3 \times 75 - 60) \div 2\rfloor = 82 &
85 - \lfloor(64 \times 82 + 2 ^ 9) \div 2 ^ {10}\rfloor = 80 &
64 + 2 = 66
\\
& 9 & 90 &
\lfloor(3 \times 85 - 75) \div 2\rfloor = 90 &
90 - \lfloor(66 \times 90 + 2 ^ 9) \div 2 ^ {10}\rfloor = 84 &
66 + 2 = 68
\\
\hline
\hline
\multirow{10}{1em}{\begin{sideways}$\textbf{pass}_1$ - term 18\end{sideways}}
& 0 & -61 &
\lfloor(3 \times 0 - 0) \div 2\rfloor = 0 &
-61 - \lfloor(48 \times 0 + 2 ^ 9) \div 2 ^ {10}\rfloor = -61 &
48 + 0 = 48
\\
& 1 & -43 &
\lfloor(3 \times -61 - 0) \div 2\rfloor = -92 &
-43 - \lfloor(48 \times -92 + 2 ^ 9) \div 2 ^ {10}\rfloor = -39 &
48 + 2 = 50
\\
& 2 & -23 &
\lfloor(3 \times -43 + 61) \div 2\rfloor = -34 &
-23 - \lfloor(50 \times -34 + 2 ^ 9) \div 2 ^ {10}\rfloor = -21 &
50 + 2 = 52
\\
& 3 & -2 &
\lfloor(3 \times -23 + 43) \div 2\rfloor = -13 &
-2 - \lfloor(52 \times -13 + 2 ^ 9) \div 2 ^ {10}\rfloor = -1 &
52 + 2 = 54
\\
& 4 & 20 &
\lfloor(3 \times -2 + 23) \div 2\rfloor = 8 &
20 - \lfloor(54 \times 8 + 2 ^ 9) \div 2 ^ {10}\rfloor = 20 &
54 + 2 = 56
\\
& 5 & 39 &
\lfloor(3 \times 20 + 2) \div 2\rfloor = 31 &
39 - \lfloor(56 \times 31 + 2 ^ 9) \div 2 ^ {10}\rfloor = 37 &
56 + 2 = 58
\\
& 6 & 57 &
\lfloor(3 \times 39 - 20) \div 2\rfloor = 48 &
57 - \lfloor(58 \times 48 + 2 ^ 9) \div 2 ^ {10}\rfloor = 54 &
58 + 2 = 60
\\
& 7 & 71 &
\lfloor(3 \times 57 - 39) \div 2\rfloor = 66 &
71 - \lfloor(60 \times 66 + 2 ^ 9) \div 2 ^ {10}\rfloor = 67 &
60 + 2 = 62
\\
& 8 & 80 &
\lfloor(3 \times 71 - 57) \div 2\rfloor = 78 &
80 - \lfloor(62 \times 78 + 2 ^ 9) \div 2 ^ {10}\rfloor = 75 &
62 + 2 = 64
\\
& 9 & 84 &
\lfloor(3 \times 80 - 71) \div 2\rfloor = 84 &
84 - \lfloor(64 \times 84 + 2 ^ 9) \div 2 ^ {10}\rfloor = 79 &
64 + 2 = 66
\\
\hline
\hline
\multirow{10}{1em}{\begin{sideways}$\textbf{pass}_2$ - term 2\end{sideways}}
& 0 & -61 & &
-61 - \lfloor(32 \times 0 + 2 ^ 9) \div 2 ^ {10}\rfloor = -61 &
32 + 0 = 32
\\
& 1 & -39 & &
-39 - \lfloor(32 \times 0 + 2 ^ 9) \div 2 ^ {10}\rfloor = -39 &
32 + 0 = 32
\\
& 2 & -21 & &
-21 - \lfloor(32 \times -61 + 2 ^ 9) \div 2 ^ {10}\rfloor = -19 &
32 + 2 = 34
\\
& 3 & -1 & &
-1 - \lfloor(34 \times -39 + 2 ^ 9) \div 2 ^ {10}\rfloor = 0 &
34 + 0 = 34
\\
& 4 & 20 & &
20 - \lfloor(34 \times -21 + 2 ^ 9) \div 2 ^ {10}\rfloor = 21 &
34 - 2 = 32
\\
& 5 & 37 & &
37 - \lfloor(32 \times -1 + 2 ^ 9) \div 2 ^ {10}\rfloor = 37 &
32 - 2 = 30
\\
& 6 & 54 & &
54 - \lfloor(30 \times 20 + 2 ^ 9) \div 2 ^ {10}\rfloor = 53 &
30 + 2 = 32
\\
& 7 & 67 & &
67 - \lfloor(32 \times 37 + 2 ^ 9) \div 2 ^ {10}\rfloor = 66 &
32 + 2 = 34
\\
& 8 & 75 & &
75 - \lfloor(34 \times 54 + 2 ^ 9) \div 2 ^ {10}\rfloor = 73 &
34 + 2 = 36
\\
& 9 & 79 & &
79 - \lfloor(36 \times 67 + 2 ^ 9) \div 2 ^ {10}\rfloor = 77 &
36 + 2 = 38
\\
\hline
\hline
\multirow{10}{1em}{\begin{sideways}$\textbf{pass}_3$ - term 17\end{sideways}}
& 0 & -61 &
2 \times 0 - 0 = 0 &
-61 - \lfloor(48 \times 0 + 2 ^ 9) \div 2 ^ {10}\rfloor = -61 &
48 + 0 = 48
\\
& 1 & -39 &
2 \times -61 - 0 = -122 &
-39 - \lfloor(48 \times -122 + 2 ^ 9) \div 2 ^ {10}\rfloor = -33 &
48 + 2 = 50
\\
& 2 & -19 &
2 \times -39 + 61 = -17 &
-19 - \lfloor(50 \times -17 + 2 ^ 9) \div 2 ^ {10}\rfloor = -18 &
50 + 2 = 52
\\
& 3 & 0 &
2 \times -19 + 39 = 1 &
0 - \lfloor(52 \times 1 + 2 ^ 9) \div 2 ^ {10}\rfloor = 0 &
52 + 0 = 52
\\
& 4 & 21 &
2 \times 0 + 19 = 19 &
21 - \lfloor(52 \times 19 + 2 ^ 9) \div 2 ^ {10}\rfloor = 20 &
52 + 2 = 54
\\
& 5 & 37 &
2 \times 21 - 0 = 42 &
37 - \lfloor(54 \times 42 + 2 ^ 9) \div 2 ^ {10}\rfloor = 35 &
54 + 2 = 56
\\
& 6 & 53 &
2 \times 37 - 21 = 53 &
53 - \lfloor(56 \times 53 + 2 ^ 9) \div 2 ^ {10}\rfloor = 50 &
56 + 2 = 58
\\
& 7 & 66 &
2 \times 53 - 37 = 69 &
66 - \lfloor(58 \times 69 + 2 ^ 9) \div 2 ^ {10}\rfloor = 62 &
58 + 2 = 60
\\
& 8 & 73 &
2 \times 66 - 53 = 79 &
73 - \lfloor(60 \times 79 + 2 ^ 9) \div 2 ^ {10}\rfloor = 68 &
60 + 2 = 62
\\
& 9 & 77 &
2 \times 73 - 66 = 80 &
77 - \lfloor(62 \times 80 + 2 ^ 9) \div 2 ^ {10}\rfloor = 72 &
62 + 2 = 64
\\
\hline
\hline
\multirow{10}{1em}{\begin{sideways}$\textbf{pass}_4$ - term 3\end{sideways}}
& 0 & -61 & &
-61 - \lfloor(16 \times 0 + 2 ^ 9) \div 2 ^ {10}\rfloor = -61 &
16 + 0 = 16
\\
& 1 & -33 & &
-33 - \lfloor(16 \times 0 + 2 ^ 9) \div 2 ^ {10}\rfloor = -33 &
16 + 0 = 16
\\
& 2 & -18 & &
-18 - \lfloor(16 \times 0 + 2 ^ 9) \div 2 ^ {10}\rfloor = -18 &
16 + 0 = 16
\\
& 3 & 0 & &
0 - \lfloor(16 \times -61 + 2 ^ 9) \div 2 ^ {10}\rfloor = 1 &
16 - 2 = 14
\\
& 4 & 20 & &
20 - \lfloor(14 \times -33 + 2 ^ 9) \div 2 ^ {10}\rfloor = 20 &
14 - 2 = 12
\\
& 5 & 35 & &
35 - \lfloor(12 \times -18 + 2 ^ 9) \div 2 ^ {10}\rfloor = 35 &
12 - 2 = 10
\\
& 6 & 50 & &
50 - \lfloor(10 \times 0 + 2 ^ 9) \div 2 ^ {10}\rfloor = 50 &
10 + 0 = 10
\\
& 7 & 62 & &
62 - \lfloor(10 \times 20 + 2 ^ 9) \div 2 ^ {10}\rfloor = 62 &
10 + 2 = 12
\\
& 8 & 68 & &
68 - \lfloor(12 \times 35 + 2 ^ 9) \div 2 ^ {10}\rfloor = 68 &
12 + 2 = 14
\\
& 9 & 72 & &
72 - \lfloor(14 \times 50 + 2 ^ 9) \div 2 ^ {10}\rfloor = 71 &
14 + 2 = 16
\\
%%END
\end{tabular}
}
\begin{center}
$\text{channel}_0$ correlation passes
\end{center}

\clearpage

\subsection{Writing Decorrelation Terms}

\ALGORITHM{a list of decorrelation terms, a list of decorrelation deltas}{decorrelation terms sub block data}
\SetKwData{TERM}{term}
\SetKwData{DELTA}{delta}
\SetKwData{KwDownTo}{downto}
\For(\tcc*[f]{populate in reverse order}){p = decorrelation pass count \emph{\KwDownTo}0}{
  \WRITE ($\text{\TERM}_p + 5$) in 5 unsigned bits\;
  \WRITE $\text{\DELTA}_p$ in 3 unsigned bits\;
}
\Return decorrelation terms sub block data\;
\EALGORITHM
\par
\noindent
For example, given decorrelation terms \texttt{[3, 17, 2, 18, 18]}
\newline
and decorrelation deltas \texttt{[2, 2, 2, 2, 2]},
\newline
the decorrelation terms subframe is written as:
\begin{figure}[h]
\includegraphics{figures/wavpack/terms_parse.pdf}
\end{figure}

\clearpage

\subsection{Writing Decorrelation Weights}
\ALGORITHM{a list of decorrelation weights per channel}{decorrelation weights sub block data}
\SetKwData{WEIGHT}{weight}
\SetKwFunction{STOREWEIGHT}{store\_weight}
\SetKwData{KwDownTo}{downto}
\For(\tcc*[f]{populate in reverse order}){p = decorrelation pass count \emph{\KwDownTo}0}{
  \WRITE $\texttt{\STOREWEIGHT}(\text{\WEIGHT}_{p~0})$ in 8 signed bits\;
  \If{$\text{channel count} = 2$}{
    \WRITE $\texttt{\STOREWEIGHT}(\text{\WEIGHT}_{p~1})$ in 8 signed bits\;
  }
}
\Return decorrelation weights sub block data\;
\EALGORITHM
\par
\noindent
where \texttt{store\_weight} is defined as:
\begin{equation*}
\texttt{store\_weight}(w) =
\begin{cases}
\left\lfloor\frac{\texttt{min}(w, 1024) - \lfloor(\texttt{min}(w,1024) + 2 ^ 6) \div 2 ^ 7\rfloor + 4}{2 ^ 3}\right\rfloor & \text{ if } w > 0 \\
0 & \text{ if } w = 0 \\
\left\lfloor \frac{\texttt{max}(w, -1024) + 4}{2 ^ 3} \right\rfloor & \text{ if } w < 0 \\
\end{cases}
\end{equation*}
For example, given the channel 0 weight values: \texttt{[16, 48, 32, 48, 48]}
\newline
and channel 1 weight values: \texttt{[24, 48, 32, 48, 48]}
\newline
the decorrelation weights subframe is written as:
\begin{figure}[h]
\includegraphics{figures/wavpack/decorr_weights_parse.pdf}
\end{figure}

\clearpage

\subsection{Writing Decorrelation Samples}
{\relsize{-1}
\ALGORITHM{a list of decorrelation sample values per pass, per channel}{decorrelation samples sub block data}
\SetKwData{TERM}{term}
\SetKwData{SAMPLE}{sample}
\SetKwFunction{WVLOG}{wv\_log2}
\SetKw{KwDownTo}{downto}
\eIf{$\text{channel count} = 2$}{
  \For{p = decorrelation pass count \emph{\KwDownTo}0}{
    \uIf{$17 \leq \text{\TERM}_p \leq 18$}{
      \WRITE $\WVLOG(\text{\SAMPLE}_{p~0~0})$ in 16 signed bits\;
      \WRITE $\WVLOG(\text{\SAMPLE}_{p~0~1})$ in 16 signed bits\;
      \WRITE $\WVLOG(\text{\SAMPLE}_{p~1~0})$ in 16 signed bits\;
      \WRITE $\WVLOG(\text{\SAMPLE}_{p~1~1})$ in 16 signed bits\;
    }
    \uElseIf{$1 \leq \text{\TERM}_p \leq 8$}{
      \For{s = 0 \emph{\KwTo}$\text{\TERM}_p$}{
        \WRITE $\WVLOG(\text{\SAMPLE}_{p~0~s})$ in 16 signed bits\;
        \WRITE $\WVLOG(\text{\SAMPLE}_{p~1~s})$ in 16 signed bits\;
      }
    }
    \ElseIf{$-3 \leq \text{\TERM}_p \leq -1$}{
      \WRITE $\WVLOG(\text{\SAMPLE}_{p~0~0})$ in 16 signed bits\;
      \WRITE $\WVLOG(\text{\SAMPLE}_{p~1~0})$ in 16 signed bits\;
    }
  }
}{
  \For{p = decorrelation pass count \emph{\KwDownTo}0}{
    \uIf{$17 \leq \text{\TERM}_p \leq 18$}{
      \WRITE $\WVLOG(\text{\SAMPLE}_{p~0~0})$ in 16 signed bits\;
      \WRITE $\WVLOG(\text{\SAMPLE}_{p~0~1})$ in 16 signed bits\;
    }
    \ElseIf{$1 \leq \text{\TERM}_p \leq 8$}{
      \For{s = 0 \emph{\KwTo}$\text{\TERM}_p$}{
        \WRITE $\WVLOG(\text{\SAMPLE}_{p~0~s})$ in 16 signed bits\;
      }
    }
  }
}
\EALGORITHM
}

\subsubsection{The wv\_log2 Function}
{\relsize{-1}
\ALGORITHM{a signed value}{a signed 16 bit value}
\SetKwFunction{LOG}{wlog}
$a \leftarrow |value| + \lfloor |value| \div 2 ^ 9\rfloor$\;
$c \leftarrow $\lIf{$a \neq 0$}{$\lfloor\log_2(a)\rfloor + 1$}
\lElse{$0$}\;
\eIf{$value \geq 0$}{
  \eIf{$0 \leq a < 256$}{
    \Return $c \times 2 ^ 8 + \LOG((a \times 2 ^ {9 - c}) \bmod 256)$\;
  }{
    \Return $c \times 2 ^ 8 + \LOG(\lfloor a \div 2 ^ {c - 9}\rfloor \bmod 256)$\;
  }
}{
    \eIf{$0 \leq a < 256$}{
    \Return $-(c \times 2 ^ 8 + \LOG((a \times 2 ^ {9 - c}) \bmod 256))$\;
  }{
    \Return $-(c \times 2 ^ 8 + \LOG(\lfloor a \div 2 ^ {c - 9}\rfloor \bmod 256))$\;
  }
}
\EALGORITHM
}

\clearpage

where \texttt{wlog} is defined from the following table:
\par
\noindent
{\relsize{-3}\ttfamily
\begin{tabular}{|r|r|r|r|r|r|r|r|r|r|r|r|r|r|r|r|r|}
\hline
& \texttt{0x?0} & \texttt{0x?1} & \texttt{0x?2} & \texttt{0x?3} & \texttt{0x?4} & \texttt{0x?5} & \texttt{0x?6} & \texttt{0x?7} & \texttt{0x?8} & \texttt{0x?9} & \texttt{0x?A} & \texttt{0x?B} & \texttt{0x?C} & \texttt{0x?D} & \texttt{0x?E} & \texttt{0x?F} \\
\hline
\texttt{0x0?} & 0 & 1 & 3 & 4 & 6 & 7 & 9 & 10 & 11 & 13 & 14 & 16 & 17 & 18 & 20 & 21 \\
\texttt{0x1?} & 22 & 24 & 25 & 26 & 28 & 29 & 30 & 32 & 33 & 34 & 36 & 37 & 38 & 40 & 41 & 42 \\
\texttt{0x2?} & 44 & 45 & 46 & 47 & 49 & 50 & 51 & 52 & 54 & 55 & 56 & 57 & 59 & 60 & 61 & 62 \\
\texttt{0x3?} & 63 & 65 & 66 & 67 & 68 & 69 & 71 & 72 & 73 & 74 & 75 & 77 & 78 & 79 & 80 & 81 \\
\texttt{0x4?} & 82 & 84 & 85 & 86 & 87 & 88 & 89 & 90 & 92 & 93 & 94 & 95 & 96 & 97 & 98 & 99 \\
\texttt{0x5?} & 100 & 102 & 103 & 104 & 105 & 106 & 107 & 108 & 109 & 110 & 111 & 112 & 113 & 114 & 116 & 117 \\
\texttt{0x6?} & 118 & 119 & 120 & 121 & 122 & 123 & 124 & 125 & 126 & 127 & 128 & 129 & 130 & 131 & 132 & 133 \\
\texttt{0x7?} & 134 & 135 & 136 & 137 & 138 & 139 & 140 & 141 & 142 & 143 & 144 & 145 & 146 & 147 & 148 & 149 \\
\texttt{0x8?} & 150 & 151 & 152 & 153 & 154 & 155 & 155 & 156 & 157 & 158 & 159 & 160 & 161 & 162 & 163 & 164 \\
\texttt{0x9?} & 165 & 166 & 167 & 168 & 169 & 169 & 170 & 171 & 172 & 173 & 174 & 175 & 176 & 177 & 178 & 178 \\
\texttt{0xA?} & 179 & 180 & 181 & 182 & 183 & 184 & 185 & 185 & 186 & 187 & 188 & 189 & 190 & 191 & 192 & 192 \\
\texttt{0xB?} & 193 & 194 & 195 & 196 & 197 & 198 & 198 & 199 & 200 & 201 & 202 & 203 & 203 & 204 & 205 & 206 \\
\texttt{0xC?} & 207 & 208 & 208 & 209 & 210 & 211 & 212 & 212 & 213 & 214 & 215 & 216 & 216 & 217 & 218 & 219 \\
\texttt{0xD?} & 220 & 220 & 221 & 222 & 223 & 224 & 224 & 225 & 226 & 227 & 228 & 228 & 229 & 230 & 231 & 231 \\
\texttt{0xE?} & 232 & 233 & 234 & 234 & 235 & 236 & 237 & 238 & 238 & 239 & 240 & 241 & 241 & 242 & 243 & 244 \\
\texttt{0xF?} & 244 & 245 & 246 & 247 & 247 & 248 & 249 & 249 & 250 & 251 & 252 & 252 & 253 & 254 & 255 & 255 \\
\hline
\end{tabular}
}

\subsubsection{Writing Decorrelation Samples Example}
Given a 2 channel subframe with 5 correlation passes containing
the following correlation samples:
\begin{table}[h]
\begin{tabular}{rrrr}
pass $p$ & $\text{term}_p$ & $\text{sample}_{p~0~s}$ & $\text{sample}_{p~1~s}$ \\
\hline
4 & 18 & \texttt{[84, 80]} & \texttt{[14, 3]} \\
3 & 18 & \texttt{[79, 75]} & \texttt{[13, 2]} \\
2 & 2 & \texttt{[73, 77]} & \texttt{[1, 12]} \\
1 & 17 & \texttt{[72, 68]} & \texttt{[11, 1]} \\
0 & 3 & \texttt{[62, 68, 71]} & \texttt{[0, 10, 18]} \\
\end{tabular}
\end{table}
\par
\noindent
$\text{sample}_{4~0~0}$ (pass 4, channel 0, sample 0) is encoded as:
\begin{align*}
a &\leftarrow |84| + \lfloor|84| \div 2 ^ 9\rfloor = 84 \\
c &\leftarrow \lfloor\log_2(84)\rfloor + 1 = 7 \\
value &\leftarrow 7 \times 2 ^ 8 + \texttt{wlog}((84 \times 2 ^ 2) \bmod 256) \\
&\leftarrow 1792 + \texttt{wlog}(\texttt{0x50}) = 1892 = \texttt{0x764}
\end{align*}
\par
\noindent
and the entire sub block is written as:
\begin{figure}[h]
\includegraphics{figures/wavpack/decorr_samples_encode.pdf}
\end{figure}

\clearpage

\subsubsection{The wv\_exp2 Function}

\ALGORITHM{a signed 16 bit value}{a signed value}
\SetKwFunction{EXP}{wexp}
\uIf{$-32768 \leq value < -2304$}{
  \Return $-(\EXP(-value \bmod{256}) \times 2 ^ {\lfloor -value \div 2 ^ 8 \rfloor - 9})$\;
}
\uElseIf{$-2304 \leq value < 0$}{
  \Return $-\lfloor \EXP(-value \bmod{256}) \div 2 ^ {9 - \lfloor -value \div 2 ^ 8 \rfloor} \rfloor$\;
}
\uElseIf{$0 \leq value \leq 2304$}{
  \Return $\lfloor \EXP(value \bmod{256}) \div 2 ^ {9 - \lfloor value \div 2 ^ 8 \rfloor} \rfloor$\;
}
\ElseIf{$2304 < value \leq 32767$}{
  \Return $\EXP(value \bmod{256}) \times 2 ^ {\lfloor value \div 2 ^ 8 \rfloor - 9}$\;
}
\EALGORITHM
\par
\noindent
where \texttt{wexp}(\textit{x}) is defined from the following table:
\vskip .10in
\par
\noindent
{\relsize{-3}\ttfamily
\begin{tabular}{|r|r|r|r|r|r|r|r|r|r|r|r|r|r|r|r|r|}
\hline
& \texttt{0x?0} & \texttt{0x?1} & \texttt{0x?2} & \texttt{0x?3} & \texttt{0x?4} & \texttt{0x?5} & \texttt{0x?6} & \texttt{0x?7} & \texttt{0x?8} & \texttt{0x?9} & \texttt{0x?A} & \texttt{0x?B} & \texttt{0x?C} & \texttt{0x?D} & \texttt{0x?E} & \texttt{0x?F} \\
\hline
\texttt{0x0?} & 256 & 257 & 257 & 258 & 259 & 259 & 260 & 261 & 262 & 262 & 263 & 264 & 264 & 265 & 266 & 267 \\
\texttt{0x1?} & 267 & 268 & 269 & 270 & 270 & 271 & 272 & 272 & 273 & 274 & 275 & 275 & 276 & 277 & 278 & 278 \\
\texttt{0x2?} & 279 & 280 & 281 & 281 & 282 & 283 & 284 & 285 & 285 & 286 & 287 & 288 & 288 & 289 & 290 & 291 \\
\texttt{0x3?} & 292 & 292 & 293 & 294 & 295 & 296 & 296 & 297 & 298 & 299 & 300 & 300 & 301 & 302 & 303 & 304 \\
\texttt{0x4?} & 304 & 305 & 306 & 307 & 308 & 309 & 309 & 310 & 311 & 312 & 313 & 314 & 314 & 315 & 316 & 317 \\
\texttt{0x5?} & 318 & 319 & 320 & 321 & 321 & 322 & 323 & 324 & 325 & 326 & 327 & 328 & 328 & 329 & 330 & 331 \\
\texttt{0x6?} & 332 & 333 & 334 & 335 & 336 & 337 & 337 & 338 & 339 & 340 & 341 & 342 & 343 & 344 & 345 & 346 \\
\texttt{0x7?} & 347 & 348 & 349 & 350 & 350 & 351 & 352 & 353 & 354 & 355 & 356 & 357 & 358 & 359 & 360 & 361 \\
\texttt{0x8?} & 362 & 363 & 364 & 365 & 366 & 367 & 368 & 369 & 370 & 371 & 372 & 373 & 374 & 375 & 376 & 377 \\
\texttt{0x9?} & 378 & 379 & 380 & 381 & 382 & 383 & 384 & 385 & 386 & 387 & 388 & 389 & 391 & 392 & 393 & 394 \\
\texttt{0xA?} & 395 & 396 & 397 & 398 & 399 & 400 & 401 & 402 & 403 & 405 & 406 & 407 & 408 & 409 & 410 & 411 \\
\texttt{0xB?} & 412 & 413 & 415 & 416 & 417 & 418 & 419 & 420 & 421 & 422 & 424 & 425 & 426 & 427 & 428 & 429 \\
\texttt{0xC?} & 431 & 432 & 433 & 434 & 435 & 436 & 438 & 439 & 440 & 441 & 442 & 444 & 445 & 446 & 447 & 448 \\
\texttt{0xD?} & 450 & 451 & 452 & 453 & 454 & 456 & 457 & 458 & 459 & 461 & 462 & 463 & 464 & 466 & 467 & 468 \\
\texttt{0xE?} & 470 & 471 & 472 & 473 & 475 & 476 & 477 & 478 & 480 & 481 & 482 & 484 & 485 & 486 & 488 & 489 \\
\texttt{0xF?} & 490 & 492 & 493 & 494 & 496 & 497 & 498 & 500 & 501 & 502 & 504 & 505 & 506 & 508 & 509 & 511 \\
\hline
\end{tabular}
}

\clearpage

\subsection{Encoding Bitstream}
{\relsize{-1}
\ALGORITHM{channel count, entropy values, a list of signed residual values per channel}{bitstream sub block data}
\SetKwData{UNDEFINED}{undefined}
\SetKwData{BLOCKSAMPLES}{block samples}
\SetKwData{CHANNELCOUNT}{channel count}
\SetKwData{ENTROPY}{entropy}
\SetKwData{RESIDUAL}{residual}
\SetKwData{ZEROES}{zeroes}
\SetKwData{SAMPLE}{sample}
\SetKw{AND}{and}
$i \leftarrow 0$\;
$u_{(-2)} \leftarrow \text{\UNDEFINED}$\;
$u_{(-1)} \leftarrow \text{\UNDEFINED}$\;
\While{$i < (\text{\BLOCKSAMPLES} \times \text{\CHANNELCOUNT})$}{
  $\text{\SAMPLE} \leftarrow \text{\RESIDUAL}_{(i \bmod \text{\CHANNELCOUNT})~\lfloor i \div \text{\CHANNELCOUNT}\rfloor}$\;
  \eIf{$(u_{i - 1} = \text{\UNDEFINED})$ \AND $(\text{\ENTROPY}_{0~0} < 2)$ \AND $(\text{\ENTROPY}_{1~0} < 2)$}{
    flush residual ($\text{m}_{i - 1}$, $\text{base}_{i - 1}$, $\text{add}_{i - 1}$, $\text{sign}_{i - 1}$) using $u_{i - 2}$ and $m_i = 0$\;
    $\text{\ZEROES} \leftarrow 0$\;
    \While{$\text{\SAMPLE} = 0$ \AND $i < (\text{\BLOCKSAMPLES} \times \text{\CHANNELCOUNT})$}{
      $\text{\ZEROES} \leftarrow \text{\ZEROES} + 1$\;
      $i \leftarrow i + 1$\;
      $\text{\SAMPLE} \leftarrow \text{\RESIDUAL}_{(i \bmod \text{\CHANNELCOUNT})~\lfloor i \div \text{\CHANNELCOUNT}\rfloor}$\;
    }
    write \ZEROES in modified Elias gamma code\;
    $u_i \leftarrow$ write residual \text{\SAMPLE}\;
  }{
    $u_i \leftarrow$ write residual \text{\SAMPLE}\;
    $i \leftarrow i + 1$\;
  }
}
flush residual ($\text{m}_{i - 1}$, $\text{base}_{i - 1}$, $\text{add}_{i - 1}$, $\text{sign}_{i - 1}$) using $u_{i - 2}$ and $m_i = 0$\;
\Return encoded bitstream sub block\;
\EALGORITHM
}

\subsubsection{Writing Modified Elias Gamma Code}
{\relsize{-1}
\ALGORITHM{a non-negative integer value}{an encoded value}
\eIf{$v \leq 1$}{
  \WUNARY $v$ with stop bit 0\;
}{
  $t \leftarrow \lfloor\log_2(v)\rfloor + 1$\;
  \WUNARY $t$ with stop bit 0\;
  \WRITE $v \bmod 2 ^ {t - 1}$ in ($t - 1$) unsigned bits\;
}
\Return encoded value\;
\EALGORITHM
}

\clearpage

\subsubsection{Writing Residual Value}
{\relsize{-1}
\ALGORITHM{signed sample value, $u_{i - 2}$, entropy variables for channel, previous residual's values}{$u_{i - 1}$, residual's values, updated entropy variables}
\SetKwData{SAMPLE}{sample}
\SetKwData{UNSIGNED}{unsigned}
\SetKwData{SIGN}{sign}
\SetKwData{ENTROPY}{entropy}
\SetKwData{MEDIAN}{median}
\SetKwData{MED}{m}
\SetKwData{ADD}{add}
\SetKwData{OFFSET}{offset}
\eIf{$\SAMPLE \geq 0$}{
  $\UNSIGNED \leftarrow \SAMPLE$\;
  $\text{\SIGN}_i \leftarrow 0$\;
}{
  $\UNSIGNED \leftarrow -\SAMPLE - 1$\;
  $\text{\SIGN}_i \leftarrow 1$\;
}
$\text{\MEDIAN}_{c~0} \leftarrow \lfloor\text{\ENTROPY}_{c~0} \div 2 ^ 4\rfloor + 1$\;
$\text{\MEDIAN}_{c~1} \leftarrow \lfloor\text{\ENTROPY}_{c~1} \div 2 ^ 4\rfloor + 1$\;
$\text{\MEDIAN}_{c~2} \leftarrow \lfloor\text{\ENTROPY}_{c~2} \div 2 ^ 4\rfloor + 1$\;
\uIf{$\UNSIGNED < \text{\MEDIAN}_{c~0}$}{
  $\text{\MED}_i \leftarrow 0$\;
  $\text{\OFFSET}_i \leftarrow \text{\UNSIGNED}$\tcc*[r]{offset is unsigned - base}
  $\text{\ADD}_i \leftarrow \text{\MEDIAN}_{c~0} - 1$\;
  $\text{\ENTROPY}_{c~0} \leftarrow \text{\ENTROPY}_{c~0} - \lfloor(\text{\ENTROPY}_{c~0} + 126) \div 128\rfloor \times 2$\;
}
\uElseIf{$(\UNSIGNED - \text{\MEDIAN}_{c~0}) < \text{\MEDIAN}_{c~1}$}{
  $\text{\MED}_i \leftarrow 1$\;
  $\text{\OFFSET}_i \leftarrow \text{\UNSIGNED} - \text{\MEDIAN}_{c~0}$\;
  $\text{\ADD}_i \leftarrow \text{\MEDIAN}_{c~1} - 1$\;
  $\text{\ENTROPY}_{c~0} \leftarrow \text{\ENTROPY}_{c~0} + \lfloor(\text{\ENTROPY}_{c~0} + 128) \div 128\rfloor \times 5$\;
  $\text{\ENTROPY}_{c~1} \leftarrow \text{\ENTROPY}_{c~1} - \lfloor(\text{\ENTROPY}_{c~1} + 62) \div 64\rfloor \times 2$\;
}
\uElseIf{$(\UNSIGNED - (\text{\ENTROPY}_{c~0} + \text{\ENTROPY}_{c~1})) < \text{\ENTROPY}_{c~2}$}{
  $\text{\MED}_i \leftarrow 2$\;
  $\text{\OFFSET}_i \leftarrow \text{\UNSIGNED} - (\text{\MEDIAN}_{c~0} + \text{\ENTROPY}_{c~1})$\;
  $\text{\ADD}_i \leftarrow \text{\MEDIAN}_{c~2} - 1$\;
  $\text{\ENTROPY}_{c~0} \leftarrow \text{\ENTROPY}_{c~0} + \lfloor(\text{\ENTROPY}_{c~0} + 128) \div 128\rfloor \times 5$\;
  $\text{\ENTROPY}_{c~1} \leftarrow \text{\ENTROPY}_{c~1} + \lfloor(\text{\ENTROPY}_{c~1} + 64) \div 64\rfloor \times 5$\;
  $\text{\ENTROPY}_{c~2} \leftarrow \text{\ENTROPY}_{c~2} - \lfloor(\text{\ENTROPY}_{c~2} + 30) \div 32\rfloor \times 2$\;
}
\Else{
  $\text{\MED}_i \leftarrow \lfloor(\UNSIGNED - (\text{\MEDIAN}_{c~0} + \text{\MEDIAN}_{c~1})) \div \text{\MEDIAN}_{c~2}\rfloor + 2$\;
  $\text{\OFFSET}_i \leftarrow \text{\UNSIGNED} - (\text{\MEDIAN}_{c~0} + \text{\MEDIAN}_{c~1} + ((\text{\MED}_i - 2) \times \text{\MEDIAN}_{c~2}))$\;
  $\text{\ADD}_i \leftarrow \text{\MEDIAN}_{c~2} - 1$\;
  $\text{\ENTROPY}_{c~0} \leftarrow \text{\ENTROPY}_{c~0} + \lfloor(\text{\ENTROPY}_{c~0} + 128) \div 128\rfloor \times 5$\;
  $\text{\ENTROPY}_{c~1} \leftarrow \text{\ENTROPY}_{c~1} + \lfloor(\text{\ENTROPY}_{c~1} + 64) \div 64\rfloor \times 5$\;
  $\text{\ENTROPY}_{c~2} \leftarrow \text{\ENTROPY}_{c~2} + \lfloor(\text{\ENTROPY}_{c~2} + 32) \div 32\rfloor \times 5$\;
}
$u_{i - 1} \leftarrow$ flush previous residual ($\text{\MED}_{i - 1}$, $\text{\OFFSET}_{i - 1}$, $\text{\ADD}_{i - 1}$, $\text{\SIGN}_{i - 1}$) using $u_{i - 2}$ and $\text{\MED}_i$\;
\Return $u_{i - 1}$, $\text{\MED}_i$, $\text{\OFFSET}_i$, $\text{\ADD}_i$, $\text{\SIGN}_i$ and updated $\text{\ENTROPY}_c$\;
\EALGORITHM
}

\clearpage

\subsubsection{Flushing Residual Value}
{\relsize{-1}
%% Residual writing must be handled in this two-step fashion
%% because building the unary ($u$) value for residual $i$,
%% requires the residual's $m$ value,
%% the previous residual's $u$ value
%% and the \textit{next} residual's $m$ value.
%% Therefore, one cannot generate $u_j$ without $m_j$, $u_{j - 1}$ and $m_{j + 1}$,
%% so writing a residual value can't take place until the next residual
%% is calculated.
%% \par
%% \noindent
%% \vskip 1ex
\ALGORITHM{$u_{j - 1}$, $m_j$, $m_{j + 1}$, $\text{offset}_j$, $\text{add}_j$, $\text{sign}_j$}{$u_j$, residual data}
\SetKw{AND}{and}
\SetKw{OR}{or}
\SetKw{IS}{is}
\SetKw{NOT}{not}
\SetKwData{OFFSET}{offset}
\SetKwData{ADD}{add}
\SetKwData{SIGN}{sign}
\SetKwData{UNDEFINED}{undefined}
\tcc{determine $u_j$ from $m_j$, $m_{j + 1}$ and $u_{j - 1}$}
\uIf{$m_j > 0$ \AND $m_{j + 1} > 0$}{
  \eIf{$u_{j - 1}$ \IS \UNDEFINED \OR $u_{j - 1} \bmod 2 = 0$}{
    $u_j \leftarrow (m_j \times 2) + 1$\;
  }{
    $u_j \leftarrow (m_j \times 2) - 1$\;
  }
}
\uElseIf{$m_j = 0$ \AND $m_{j + 1} > 0$}{
  \eIf{$u_{j - 1}$ \IS \UNDEFINED \OR $u_{j - 1} \bmod 2 = 1$}{
    $u_j \leftarrow 1$\;
  }{
    $u_j \leftarrow \UNDEFINED$\;
  }
}
\uElseIf{$m_j > 0$ \AND $m_{j + 1} = 0$}{
  \eIf{$u_{j - 1}$ \IS \UNDEFINED \OR $u_{j - 1} \bmod 2 = 0$}{
    $u_j \leftarrow m_j \times 2$\;
  }{
    $u_j \leftarrow (m_j - 1) \times 2$\;
  }
}
\ElseIf{$m_j = 0$ \AND $m_{j + 1} = 0$}{
  \eIf{$u_{j - 1}$ \IS \UNDEFINED \OR $u_{j - 1} \bmod 2 = 1$}{
    $u_j \leftarrow 0$\;
  }{
    $u_j \leftarrow \UNDEFINED$\;
  }
}
\BlankLine
\tcc{write unary and fixed values to stream, if any}
\If{$u_j$ \IS \NOT \UNDEFINED}{
  \eIf{$u_j < 16$}{
    \WUNARY $u_j$ with stop bit 0\;
  }{
    \WUNARY 16 with stop bit 0\;
    write $(u_j - 16)$ in modified Elias gamma code\;
  }
}
\If{$\text{\ADD}_j > 0$}{
  $p \leftarrow \lfloor\log_2(\text{\ADD}_j)\rfloor$\;
  $e \leftarrow 2 ^ {(p + 1)} - \text{\ADD}_j - 1$\;
  \eIf{$\text{\OFFSET}_j < e$}{
    $r \leftarrow \text{\OFFSET}$\;
    \WRITE $r$ in $p$ unsigned bits\;
  }{
    $r \leftarrow \lfloor(\text{\OFFSET}_j + e) \div 2\rfloor$\;
    $b \leftarrow (\text{\OFFSET}_j + e) \bmod 2$\;
    \WRITE $r$ in $p$ unsigned bits\;
    \WRITE $b$ in 1 unsigned bit\;
  }
}
\WRITE $\text{\SIGN}_j$ in 1 unsigned bit\;
\Return $u_j$, residual data\;
\EALGORITHM
}

\begin{landscape}

\subsubsection{Residual Encoding Example}
{\relsize{-2}
\renewcommand{\arraystretch}{1.75}
\begin{tabular}{|>{$}r<{$}||>{$}r<{$}|>{$}r<{$}|>{$}r<{$}||>{$}r<{$}|>{$}r<{$}|>{$}r<{$}||>{$}r<{$}|>{$}r<{$}|>{$}r<{$}|}
i & \text{residual}_i & \text{unsigned}_i &\text{sign}_i & \text{median}_{c~0} & \text{median}_{c~1} & \text{median}_{c~2} & m_i & \text{offset}_i & \text{add}_i \\
\hline
0 & -61 &
60 & 1 &
\left\lfloor\frac{118}{2 ^ 4}\right\rfloor + 1 = 8 & \left\lfloor\frac{194}{2 ^ 4}\right\rfloor + 1 = 13 & \left\lfloor\frac{322}{2 ^ 4}\right\rfloor + 1 = 21 &
3 & 60 - (8 + 13 + ((3 - 2) \times 21)) = 18 & 21 - 1 = 20
\\
1 & 31 &
31 & 0 &
\left\lfloor\frac{118}{2 ^ 4}\right\rfloor + 1 = 8 & \left\lfloor\frac{176}{2 ^ 4}\right\rfloor + 1 = 12 & \left\lfloor\frac{212}{2 ^ 4}\right\rfloor + 1 = 14 &
2 & 31 - (8 + 12) = 11 & 14 - 1 = 13
\\
\hline
2 & -33 &
32 & 1 &
\left\lfloor\frac{123}{2 ^ 4}\right\rfloor + 1 = 8 & \left\lfloor\frac{214}{2 ^ 4}\right\rfloor + 1 = 14 & \left\lfloor\frac{377}{2 ^ 4}\right\rfloor + 1 = 24 &
2 & 32 - (8 + 14) = 10 & 24 - 1 = 23
\\
3 & 32 &
32 & 0 &
\left\lfloor\frac{123}{2 ^ 4}\right\rfloor + 1 = 8 & \left\lfloor\frac{191}{2 ^ 4}\right\rfloor + 1 = 12 & \left\lfloor\frac{198}{2 ^ 4}\right\rfloor + 1 = 13 &
2 & 32 - (8 + 12) = 12 & 13 - 1 = 12
\\
\hline
4 & -18 &
17 & 1 &
\left\lfloor\frac{128}{2 ^ 4}\right\rfloor + 1 = 9 & \left\lfloor\frac{234}{2 ^ 4}\right\rfloor + 1 = 15 & \left\lfloor\frac{353}{2 ^ 4}\right\rfloor + 1 = 23 &
1 & 17 - 9 = 8 & 15 - 1 = 14
\\
5 & 36 &
36 & 0 &
\left\lfloor\frac{128}{2 ^ 4}\right\rfloor + 1 = 9 & \left\lfloor\frac{206}{2 ^ 4}\right\rfloor + 1 = 13 & \left\lfloor\frac{184}{2 ^ 4}\right\rfloor + 1 = 12 &
3 & 36 - (9 + 13 + ((3 - 2) \times 12)) = 2 & 12 - 1 = 11
\\
\hline
6 & 1 &
1 & 0 &
\left\lfloor\frac{138}{2 ^ 4}\right\rfloor + 1 = 9 & \left\lfloor\frac{226}{2 ^ 4}\right\rfloor + 1 = 15 & \left\lfloor\frac{353}{2 ^ 4}\right\rfloor + 1 = 23 &
0 & 1 & 9 - 1 = 8
\\
7 & 37 &
37 & 0 &
\left\lfloor\frac{138}{2 ^ 4}\right\rfloor + 1 = 9 & \left\lfloor\frac{226}{2 ^ 4}\right\rfloor + 1 = 15 & \left\lfloor\frac{214}{2 ^ 4}\right\rfloor + 1 = 14 &
2 & 37 - (9 + 15) = 13 & 14 - 1 = 13
\\
\hline
8 & 20 &
20 & 0 &
\left\lfloor\frac{134}{2 ^ 4}\right\rfloor + 1 = 9 & \left\lfloor\frac{226}{2 ^ 4}\right\rfloor + 1 = 15 & \left\lfloor\frac{353}{2 ^ 4}\right\rfloor + 1 = 23 &
1 & 20 - 9 = 11 & 15 - 1 = 14
\\
9 & 35 &
35 & 0 &
\left\lfloor\frac{148}{2 ^ 4}\right\rfloor + 1 = 10 & \left\lfloor\frac{246}{2 ^ 4}\right\rfloor + 1 = 16 & \left\lfloor\frac{200}{2 ^ 4}\right\rfloor + 1 = 13 &
2 & 35 - (10 + 16) = 9 & 13 - 1 = 12
\\
\hline
10 & 35 &
35 & 0 &
\left\lfloor\frac{144}{2 ^ 4}\right\rfloor + 1 = 10 & \left\lfloor\frac{218}{2 ^ 4}\right\rfloor + 1 = 14 & \left\lfloor\frac{353}{2 ^ 4}\right\rfloor + 1 = 23 &
2 & 35 - (10 + 14) = 11 & 23 - 1 = 22
\\
11 & 31 &
31 & 0 &
\left\lfloor\frac{158}{2 ^ 4}\right\rfloor + 1 = 10 & \left\lfloor\frac{266}{2 ^ 4}\right\rfloor + 1 = 17 & \left\lfloor\frac{186}{2 ^ 4}\right\rfloor + 1 = 12 &
2 & 31 - (10 + 17) = 4 & 12 - 1 = 11
\\
\hline
12 & 50 &
50 & 0 &
\left\lfloor\frac{154}{2 ^ 4}\right\rfloor + 1 = 10 & \left\lfloor\frac{238}{2 ^ 4}\right\rfloor + 1 = 15 & \left\lfloor\frac{331}{2 ^ 4}\right\rfloor + 1 = 21 &
3 & 50 - (10 + 15 + ((3 - 2) \times 21)) = 4 & 21 - 1 = 20
\\
13 & 25 &
25 & 0 &
\left\lfloor\frac{168}{2 ^ 4}\right\rfloor + 1 = 11 & \left\lfloor\frac{291}{2 ^ 4}\right\rfloor + 1 = 19 & \left\lfloor\frac{174}{2 ^ 4}\right\rfloor + 1 = 11 &
1 & 25 - 11 = 14 & 19 - 1 = 18
\\
\hline
14 & 62 &
62 & 0 &
\left\lfloor\frac{164}{2 ^ 4}\right\rfloor + 1 = 11 & \left\lfloor\frac{258}{2 ^ 4}\right\rfloor + 1 = 17 & \left\lfloor\frac{386}{2 ^ 4}\right\rfloor + 1 = 25 &
3 & 62 - (11 + 17 + ((3 - 2) \times 25)) = 9 & 25 - 1 = 24
\\
15 & 18 &
18 & 0 &
\left\lfloor\frac{178}{2 ^ 4}\right\rfloor + 1 = 12 & \left\lfloor\frac{281}{2 ^ 4}\right\rfloor + 1 = 18 & \left\lfloor\frac{174}{2 ^ 4}\right\rfloor + 1 = 11 &
1 & 18 - 12 = 6 & 18 - 1 = 17
\\
\hline
16 & 68 &
68 & 0 &
\left\lfloor\frac{174}{2 ^ 4}\right\rfloor + 1 = 11 & \left\lfloor\frac{283}{2 ^ 4}\right\rfloor + 1 = 18 & \left\lfloor\frac{451}{2 ^ 4}\right\rfloor + 1 = 29 &
3 & 68 - (11 + 18 + ((3 - 2) \times 29)) = 10 & 29 - 1 = 28
\\
17 & 10 &
10 & 0 &
\left\lfloor\frac{188}{2 ^ 4}\right\rfloor + 1 = 12 & \left\lfloor\frac{271}{2 ^ 4}\right\rfloor + 1 = 17 & \left\lfloor\frac{174}{2 ^ 4}\right\rfloor + 1 = 11 &
0 & 10 & 12 - 1 = 11
\\
\hline
18 & 71 &
71 & 0 &
\left\lfloor\frac{184}{2 ^ 4}\right\rfloor + 1 = 12 & \left\lfloor\frac{308}{2 ^ 4}\right\rfloor + 1 = 20 & \left\lfloor\frac{526}{2 ^ 4}\right\rfloor + 1 = 33 &
3 & 71 - (12 + 20 + ((3 - 2) \times 33)) = 6 & 33 - 1 = 32
\\
19 & 0 &
0 & 0 &
\left\lfloor\frac{184}{2 ^ 4}\right\rfloor + 1 = 12 & \left\lfloor\frac{271}{2 ^ 4}\right\rfloor + 1 = 17 & \left\lfloor\frac{174}{2 ^ 4}\right\rfloor + 1 = 11 &
0 & 0 & 12 - 1 = 11
\\
\hline
\end{tabular}
\renewcommand{\arraystretch}{1.0}
}

\clearpage

{\relsize{-2}
\renewcommand{\arraystretch}{1.75}
\begin{tabular}{|>{$}r<{$}||>{$}r<{$}|>{$}r<{$}|>{$}r<{$}||>{$}r<{$}|>{$}r<{$}|>{$}r<{$}|>{$}r<{$}|>{$}r<{$}|}
i & m_i & \text{offset}_i & \text{add}_i & u_i & p_i & e_i & r_i & b_i \\
\hline
%%START
0 & 3 &
18 & 20 &
(3 \times 2) + 1 = 7 &
\lfloor\log_2(20)\rfloor = 4 &
2 ^ {(4 + 1)} - 20 - 1 = 11 &
\lfloor(18 + 11) \div 2 \rfloor = 14 &
\lfloor(18 + 11) \bmod 2 \rfloor = 1 \\
1 & 2 &
11 & 13 &
(2 \times 2) - 1 = 3 &
\lfloor\log_2(13)\rfloor = 3 &
2 ^ {(3 + 1)} - 13 - 1 = 2 &
\lfloor(11 + 2) \div 2 \rfloor = 6 &
\lfloor(11 + 2) \bmod 2 \rfloor = 1 \\
2 & 2 &
10 & 23 &
(2 \times 2) - 1 = 3 &
\lfloor\log_2(23)\rfloor = 4 &
2 ^ {(4 + 1)} - 23 - 1 = 8 &
\lfloor(10 + 8) \div 2 \rfloor = 9 &
\lfloor(10 + 8) \bmod 2 \rfloor = 0 \\
3 & 2 &
12 & 12 &
(2 \times 2) - 1 = 3 &
\lfloor\log_2(12)\rfloor = 3 &
2 ^ {(3 + 1)} - 12 - 1 = 3 &
\lfloor(12 + 3) \div 2 \rfloor = 7 &
\lfloor(12 + 3) \bmod 2 \rfloor = 1 \\
4 & 1 &
8 & 14 &
(1 \times 2) - 1 = 1 &
\lfloor\log_2(14)\rfloor = 3 &
2 ^ {(3 + 1)} - 14 - 1 = 1 &
\lfloor(8 + 1) \div 2 \rfloor = 4 &
\lfloor(8 + 1) \bmod 2 \rfloor = 1 \\
5 & 3 &
2 & 11 &
(3 - 1) \times 2 = 4 &
\lfloor\log_2(11)\rfloor = 3 &
2 ^ {(3 + 1)} - 11 - 1 = 4 &
2 & \\
6 & 0 &
1 & 8 &
\textit{undefined} &
\lfloor\log_2(8)\rfloor = 3 &
2 ^ {(3 + 1)} - 8 - 1 = 7 &
1 & \\
7 & 2 &
13 & 13 &
(2 \times 2) + 1 = 5 &
\lfloor\log_2(13)\rfloor = 3 &
2 ^ {(3 + 1)} - 13 - 1 = 2 &
\lfloor(13 + 2) \div 2 \rfloor = 7 &
\lfloor(13 + 2) \bmod 2 \rfloor = 1 \\
8 & 1 &
11 & 14 &
(1 \times 2) - 1 = 1 &
\lfloor\log_2(14)\rfloor = 3 &
2 ^ {(3 + 1)} - 14 - 1 = 1 &
\lfloor(11 + 1) \div 2 \rfloor = 6 &
\lfloor(11 + 1) \bmod 2 \rfloor = 0 \\
9 & 2 &
9 & 12 &
(2 \times 2) - 1 = 3 &
\lfloor\log_2(12)\rfloor = 3 &
2 ^ {(3 + 1)} - 12 - 1 = 3 &
\lfloor(9 + 3) \div 2 \rfloor = 6 &
\lfloor(9 + 3) \bmod 2 \rfloor = 0 \\
10 & 2 &
11 & 22 &
(2 \times 2) - 1 = 3 &
\lfloor\log_2(22)\rfloor = 4 &
2 ^ {(4 + 1)} - 22 - 1 = 9 &
\lfloor(11 + 9) \div 2 \rfloor = 10 &
\lfloor(11 + 9) \bmod 2 \rfloor = 0 \\
11 & 2 &
4 & 11 &
(2 \times 2) - 1 = 3 &
\lfloor\log_2(11)\rfloor = 3 &
2 ^ {(3 + 1)} - 11 - 1 = 4 &
\lfloor(4 + 4) \div 2 \rfloor = 4 &
\lfloor(4 + 4) \bmod 2 \rfloor = 0 \\
12 & 3 &
4 & 20 &
(3 \times 2) - 1 = 5 &
\lfloor\log_2(20)\rfloor = 4 &
2 ^ {(4 + 1)} - 20 - 1 = 11 &
4 & \\
13 & 1 &
14 & 18 &
(1 \times 2) - 1 = 1 &
\lfloor\log_2(18)\rfloor = 4 &
2 ^ {(4 + 1)} - 18 - 1 = 13 &
\lfloor(14 + 13) \div 2 \rfloor = 13 &
\lfloor(14 + 13) \bmod 2 \rfloor = 1 \\
14 & 3 &
9 & 24 &
(3 \times 2) - 1 = 5 &
\lfloor\log_2(24)\rfloor = 4 &
2 ^ {(4 + 1)} - 24 - 1 = 7 &
\lfloor(9 + 7) \div 2 \rfloor = 8 &
\lfloor(9 + 7) \bmod 2 \rfloor = 0 \\
15 & 1 &
6 & 17 &
(1 \times 2) - 1 = 1 &
\lfloor\log_2(17)\rfloor = 4 &
2 ^ {(4 + 1)} - 17 - 1 = 14 &
6 & \\
16 & 3 &
10 & 28 &
(3 - 1) \times 2 = 4 &
\lfloor\log_2(28)\rfloor = 4 &
2 ^ {(4 + 1)} - 28 - 1 = 3 &
\lfloor(9 + 3) \div 2 \rfloor = 6 &
\lfloor(9 + 3) \bmod 2 \rfloor = 0 \\
17 & 0 &
10 & 11 &
\textit{undefined} &
\lfloor\log_2(11)\rfloor = 3 &
2 ^ {(3 + 1)} - 11 - 1 = 4 &
\lfloor(10 + 4) \div 2 \rfloor = 7 &
\lfloor(10 + 4) \bmod 2 \rfloor = 0 \\
18 & 3 &
6 & 32 &
3 \times 2 = 6 &
\lfloor\log_2(32)\rfloor = 5 &
2 ^ {(5 + 1)} - 32 - 1 = 31 &
6 & \\
19 & 0 &
0 & 11 &
\textit{undefined} &
\lfloor\log_2(11)\rfloor = 3 &
2 ^ {(3 + 1)} - 11 - 1 = 4 &
0 & \\
%%END
\end{tabular}
\renewcommand{\arraystretch}{1.0}
}

\end{landscape}

%% \begin{table}[h]
%% \begin{tabular}{r||rrrrrr}
%% $i$ & $\text{term}_i$ & $\text{delta}_i$ & $\text{weight A}_i$ & $\text{weight B}_i$ & $\text{samples A}_i$ & $\text{Samples B}_i$ \\
%% \hline
%% 0 & \texttt{18} & \texttt{2} & \texttt{48} & \texttt{48} & \texttt{[-73, -78]} & \texttt{[28, 26]} \\
%% \hline
%% 1 & \texttt{18} & \texttt{2} & \texttt{48} & \texttt{48} & \texttt{[0, 0]} & \texttt{[0, 0]} \\
%% \hline
%% 2 & \texttt{2} & \texttt{2} & \texttt{32} & \texttt{32} & \texttt{[0, 0]} & \texttt{[0, 0]} \\
%% \hline
%% 3 & \texttt{17} & \texttt{2} & \texttt{48} & \texttt{48} & \texttt{[0, 0]} & \texttt{[0, 0]} \\
%% \hline
%% 4 & \texttt{3} & \texttt{2} & \texttt{16} & \texttt{24} & \texttt{[0, 0, 0]} & \texttt{[0, 0, 0]} \\
%% \hline
%% \end{tabular}
%% \end{table}

%% \section{the WavPack Block Header}
%% \begin{figure}[h]
%% \includegraphics{figures/wavpack/block_header.pdf}
%% \end{figure}
%% \begin{wrapfigure}[10]{r}{1.5in}
%% \begin{tabular}{|c|r|}
%% \hline
%% value & sample rate \\
%% \hline
%% \texttt{0000} & 6000 \\
%% \texttt{0001} & 8000 \\
%% \texttt{0010} & 9600 \\
%% \texttt{0011} & 11025 \\
%% \texttt{0100} & 12000 \\
%% \texttt{0101} & 16000 \\
%% \texttt{0110} & 22050 \\
%% \texttt{0111} & 24000 \\
%% \texttt{1000} & 32000 \\
%% \texttt{1001} & 44100 \\
%% \texttt{1010} & 48000 \\
%% \texttt{1011} & 64000 \\
%% \texttt{1100} & 88200 \\
%% \texttt{1101} & 96000 \\
%% \texttt{1110} & 192000 \\
%% \texttt{1111} & reserved \\
%% \hline
%% \end{tabular}
%% \end{wrapfigure}

%% \VAR{Block Size} is the length of everything in the block past
%% the \VAR{Block Size} field itself -
%% or everything in the block past the CRC, minus 24 bytes.

%% \VAR{Bits per Sample} is one of 4 values:

%% \begin{inparaenum}
%% \item[\texttt{00} = ] 8 bps,
%% \item[\texttt{01} = ] 16 bps,
%% \item[\texttt{10} = ] 24 bps,
%% \item[\texttt{11} = ] 32 bps
%% \end{inparaenum}
%% .

%% \VAR{Mono Output} bit indicates the channel count.
%% If 1, this block has 1 channel.
%% If 0, this block has 2 channels.
%% For an audio stream with more than 2 channels,
%% check the \VAR{Initial Block} and \VAR{Final Block} bits to indicate
%% the start and end of the channels.  As an example:

%% \begin{tabular}{c|c|c|c}
%% Initial Block & Final Block & Mono Output & Channels \\
%% \hline
%% 1 & 0 & 0 & 2 \\
%% 0 & 0 & 1 & 1 \\
%% 0 & 0 & 1 & 1 \\
%% 0 & 1 & 0 & 2 \\
%% \hline
%% \multicolumn{3}{r|}{Total} & 6
%% \end{tabular}

%% \clearpage

%% \subsection{WavPack Sub-Block}
%% \begin{figure}[h]
%% \includegraphics{figures/wavpack/subblock_header.pdf}
%% \end{figure}
%% \par
%% \noindent
%% If the \VAR{Large Block} field is 0, the \VAR{Block Size} field is 8 bits long.
%% If it is 1, the \VAR{Block Size} field is 24 bits long.
%% The \VAR{Block Size} field is the length of \VAR{Block Data}, in 16-bit
%% words rather than bytes.
%% If \VAR{Actual Size 1 Less} is set, that means \VAR{Block Data} doesn't contain
%% an even number of bytes; it is padded with a single null byte at the
%% end in order to fit.
%% If \VAR{Nondecoder Data} is set, that means the decoder does not have
%% to understand the contents of this particular sub-block in
%% order to decode the audio.

%% \section{WavPack Decoding}
%% Decoding each WavPack block requires reading its sub-blocks
%% as `arguments' to the decoder.
%% One can envision them like named arguments to a function call
%% since many sub-blocks may be optional or appear in an arbitrary order.
%% As a sort of hypothetical high-level example:
%% \begin{Verbatim}[frame=single]
%% decode_block(decorrelation_terms=sub_block[0],
%%              decorrelation_weights=sub_block[1],
%%              decorrelation_samples=sub_block[2],
%%              entropy_variables=sub_block[3],
%%              bitstream=sub_block[4])
%% \end{Verbatim}
%% Every block containing audio data requires
%% \VAR{Entropy Variables} and \VAR{Bitstream} sub-blocks.
%% The \VAR{Decorrelation Terms}, \VAR{Decorrelation Weights}
%% and \VAR{Decorrelation Samples} sub-blocks are for performing
%% one or more decorrelation passes over the bitstream's samples.

%% Each block will decode to 1 or 2 channels of raw PCM output.
%% Since files may have more than 2 channels, we may need to
%% decode several blocks in order to retrieve all the channels
%% of data so they can be properly combined.

%% \clearpage

%% \subsection{False Stereo}

%% If the \VAR{False Stereo} bit is set in the block header,
%% treat the block as mono for decoding purposes until
%% just before the channel's data is output.

%% \subsection{the Decorrelation Terms Sub-Block}
%% This block contains the decorrelation terms and deltas values.
%% The quantity of those values indicates how many decorrelation passes
%% we'll be performing over the bitstream sub-block's samples.
%% One can presume this sub-block will occur prior to the
%% \VAR{Decorrelation Weights} and \VAR{Decorrelation Samples} sub-blocks.

%% \label{wavpack_decorr_terms}
%% \begin{figure}[h]
%% \includegraphics{figures/wavpack/decorr_terms.pdf}
%% \end{figure}
%% \par
%% \noindent
%% The number of decorrelation terms and deltas
%% equals the \VAR{Block Size} times 2, and minus 1 if
%% \VAR{Actual Size 1 Less} is set.
%% Each term and delta pair is 8 bits and stored in \textit{reverse} order.
%% In addition, one must subtract 5 from the stored value of each
%% unsigned term to get its actual value.

%% For example, given the complete sub-block bytes:
%% \begin{Verbatim}[frame=single]
%% 42 03 57 57 47 56 48 00
%% \end{Verbatim}
%% we have a total of 5 term/delta pairs whose values are as follows:
%% \begin{table}[h]
%% \begin{tabular}{r r | r r}
%% $\text{Decorrelation Term}_5$ & \texttt{0x17} - 5 = 18 & $\text{Decorrelation Delta}_5$ & \texttt{0x2} = 2 \\
%% $\text{Decorrelation Term}_4$ & \texttt{0x17} - 5 = 18 & $\text{Decorrelation Delta}_4$ & \texttt{0x2} = 2 \\
%% $\text{Decorrelation Term}_3$ & \texttt{0x07} - 5 = 2 & $\text{Decorrelation Delta}_3$ & \texttt{0x2} = 2 \\
%% $\text{Decorrelation Term}_2$ & \texttt{0x16} - 5 = 17 & $\text{Decorrelation Delta}_2$ & \texttt{0x2} = 2 \\
%% $\text{Decorrelation Term}_1$ & \texttt{0x08} - 5 = 3 & $\text{Decorrelation Delta}_1$ & \texttt{0x2} = 2 \\
%% \end{tabular}
%% \end{table}
%% \par
%% \noindent
%% Remember that this is a little-endian stream and that the least-significant
%% bits (the \VAR{Decorrelation Delta} value) are on the left side of each byte.

%% \clearpage

%% \subsection{the Decorrelation Weights Sub-Block}
%% \begin{figure}[h]
%% \includegraphics{figures/wavpack/decorr_weights.pdf}
%% \end{figure}
%% \par
%% \noindent
%% As with the decorrelations terms sub-block,
%% the decorrelation weights are stored in reverse order.
%% The number of weights stored can be determined from the sub-block's
%% size.
%% Each is stored in a signed, 8-bit field and interleaved between
%% channels in the case of 2 channel blocks.
%% For example, $\text{Decorrelation Weight}_1$ is for channel \VAR{B},
%% $\text{Decorrelation Weight}_2$ is for channel
%% \VAR{A}\footnote{Why \VAR{A} and \VAR{B}?
%% Since a block may be only two channels out of many,
%% it makes sense not to number them to avoid ambiguity.},
%% $\text{Decorrelation Weight}_3$ is for channel \VAR{B} and so on
%% such that the first weight value in the sub-block will be for the highest
%% \VAR{Decorrelation Weight A}.
%% Converting the 8-bit values to the actual decorrelation weights
%% requires the following formula:
%% \begin{equation*}
%% \text{Decorrelation Weight} =
%% \begin{cases}
%% \text{value} \times 2 ^ 3 + \left\lfloor\frac{\text{value} \times 2 ^ 3 + 2 ^ 6}{2 ^ 7}\right\rfloor & \text{if value} > 0 \\
%% 0 & \text{if value} = 0 \\
%% \text{value} \times 2 ^ 3 & \text{if value} < 0
%% \end{cases}
%% \end{equation*}
%% \par
%% \noindent
%% For example, given a 2 channel block with 5 decorrelation terms and the
%% sub-block bytes:
%% \begin{Verbatim}[frame=single]
%% 03 05 06 06 06 06 04 04 06 06 02 03
%% \end{Verbatim}
%% our \VAR{Decorrelation Weights} are as follows:
%% \begin{table}[h]
%% \begin{tabular}{r r | r r}
%% $\text{Weight A}_5$ & $6 \times 2 ^ 3 + \left\lfloor\frac{6 \times 2 ^ 3 + 2 ^ 6}{2 ^ 7}\right\rfloor$ = 48 & $\text{Weight B}_5$ & $6 \times 2 ^ 3 + \left\lfloor\frac{6 \times 2 ^ 3 + 2 ^ 6}{2 ^ 7}\right\rfloor$ = 48 \\
%% $\text{Weight A}_4$ & $6 \times 2 ^ 3 + \left\lfloor\frac{6 \times 2 ^ 3 + 2 ^ 6}{2 ^ 7}\right\rfloor$ = 48 & $\text{Weight B}_4$ & $6 \times 2 ^ 3 + \left\lfloor\frac{6 \times 2 ^ 3 + 2 ^ 6}{2 ^ 7}\right\rfloor$ = 48 \\
%% $\text{Weight A}_3$ & $4 \times 2 ^ 3 + \left\lfloor\frac{4 \times 2 ^ 3 + 2 ^ 6}{2 ^ 7}\right\rfloor$ = 32 & $\text{Weight B}_3$ & $4 \times 2 ^ 3 + \left\lfloor\frac{4 \times 2 ^ 3 + 2 ^ 6}{2 ^ 7}\right\rfloor$ = 32 \\
%% $\text{Weight A}_2$ & $6 \times 2 ^ 3 + \left\lfloor\frac{6 \times 2 ^ 3 + 2 ^ 6}{2 ^ 7}\right\rfloor$ = 48 & $\text{Weight B}_2$ & $6 \times 2 ^ 3 + \left\lfloor\frac{6 \times 2 ^ 3 + 2 ^ 6}{2 ^ 7}\right\rfloor$ = 48 \\
%% $\text{Weight A}_1$ & $2 \times 2 ^ 3 + \left\lfloor\frac{2 \times 2 ^ 3 + 2 ^ 6}{2 ^ 7}\right\rfloor$ = 16 & $\text{Weight B}_1$ & $3 \times 2 ^ 3 + \left\lfloor\frac{3 \times 2 ^ 3 + 2 ^ 6}{2 ^ 7}\right\rfloor$ = 24 \\
%% \end{tabular}
%% \end{table}
%% \par
%% \noindent
%% Note that decoding a WavPack file requires having the same
%% number of \VAR{Decorrelation Weight} values, per channel, as
%% \VAR{Decorrelation Terms} values.
%% However, this block may contain less.
%% In that event, those low weight values are set to 0.

%% \clearpage

%% \subsection{the Decorrelation Samples Sub-Block}
%% \label{wavpack_decorr_samples}
%% \begin{figure}[h]
%% \includegraphics{figures/wavpack/decorr_samples.pdf}
%% \end{figure}
%% \par
%% \noindent
%% The decorrelation samples values are stored as signed, 16-bit values.
%% Converting them to sample values requires the following formula:
%% \begin{equation*}
%% \text{Sample} =
%% \begin{cases}
%% \lfloor \text{wv\_exp2}(value \bmod{256}) \div 2 ^ {9 - \lfloor value \div 2 ^ 8 \rfloor} \rfloor & \text{if } 0 \leq value \leq 2304 \\
%% \text{wv\_exp2}(value \bmod{256}) \times 2 ^ {\lfloor value \div 2 ^ 8 \rfloor - 9} & \text{if } 2304 < value \leq 32767 \\
%% -\lfloor \text{wv\_exp2}(-value \bmod{256}) \div 2 ^ {9 - \lfloor -value \div 2 ^ 8 \rfloor} \rfloor & \text{if } -2304 \leq value < 0 \\
%% -(\text{wv\_exp2}(-value \bmod{256}) \times 2 ^ {\lfloor -value \div 2 ^ 8 \rfloor - 9}) & \text{if } -32768 \leq value < -2304
%% \end{cases}
%% \end{equation*}
%% \par
%% \noindent

%% \par
%% \noindent
%% For example, given the sub-frame bytes:
%% \begin{Verbatim}[frame=single]
%% 04 04 CF F8 B7 F8 CF 05 B3 05
%% \end{Verbatim}
%% our \VAR{Decorrelation Sample} values are:
%% \begin{align*}
%% \text{Sample}_1 &= \texttt{0xF8CF} = -1841
%%  = -\lfloor \text{wv\_exp2}(1841 \bmod{256}) \div 2 ^ {9 - \lfloor 1841 \div 2 ^ 8 \rfloor} \rfloor \\
%% &= -\lfloor \text{wv\_exp2}(49) \div 2 ^ {9 - 7} \rfloor
%%  = -\lfloor 292 \div 4 \rfloor = \textbf{-73} \\
%% \text{Sample}_2 &= \texttt{0xF8B7} = -1865
%%  = -\lfloor \text{wv\_exp2}(1865 \bmod{256}) \div 2 ^ {9 - \lfloor 1865 \div 2 ^ 8 \rfloor} \rfloor \\
%% &= -\lfloor \text{wv\_exp2}(73) \div 2 ^ {9 - 7} \rfloor
%%  =  -\lfloor 312 \div 4 \rfloor = \textbf{-78} \\
%% \text{Sample}_3 &= \texttt{0x05CF} = 1487
%%  = \lfloor \text{wv\_exp2}(1487 \bmod{256}) \div 2 ^ {9 - \lfloor 1487 \div 2 ^ 8 \rfloor} \rfloor \\
%% &= \lfloor \text{wv\_exp2}(207) \div 2 ^ {9 - 5} \rfloor
%%  = \lfloor 448 \div 16 \rfloor = \textbf{28} \\
%% \text{Sample}_4 &= \texttt{0x05B3} = 1459
%%  = \lfloor \text{wv\_exp2}(1459 \bmod{256}) \div 2 ^ {9 - \lfloor 1459 \div 2 ^ 8 \rfloor} \rfloor \\
%% &= \lfloor \text{wv\_exp2}(179) \div 2 ^ {9 - 5} \rfloor
%%  = \lfloor 416 \div 16 \rfloor = \textbf{26}
%% \end{align*}

%% \clearpage

%% We're not done yet, however.
%% The next step is to determine which set of \VAR{Decorrelation Sample}
%% values correspond to which \VAR{Decorrelation Term}\footnote{As
%% extracted on page \pageref{wavpack_decorr_terms}}.

%% As with \VAR{Decorrelation Weights},
%% \VAR{Decorrelation Sample} values are stored in reverse order,
%% alternate between channels and depend on the corresponding
%% \VAR{Decorrelation Term} values.
%% \begin{figure}[h]
%% \includegraphics{figures/wavpack/decorr_samples2.pdf}
%% \end{figure}
%% \par
%% \noindent
%% It's likely that the decorrelation sample assignment process will
%% request more samples than this sub-block contains.
%% In that event, treat those samples as 0.

%% \clearpage

%% \subsection{the Entropy Variables Sub-Block}
%% Whereas the three preceding sub-blocks are for performing
%% decorrelation passes, this sub-block is required for
%% decoding the \VAR{Bitstream} sub-block's data.
%% These entropy variables are median values which are stored
%% as fractions of integers.
%% \begin{figure}[h]
%% \includegraphics{figures/wavpack/entropy_vars.pdf}
%% \end{figure}
%% \par
%% \noindent
%% If a block is mono, this sub-block contains 3 \VAR{Entropy Variables}.
%% If a block is stereo, this sub-block contains 6.
%% Each is stored as a signed, 16-bit value which is packed in
%% the same fashion as \VAR{Decorrelation Samples}\footnote{As described
%% on page \pageref{wavpack_decorr_samples}.}.
%% For example, given a 2 channel block with the sub-block bytes:
%% \begin{Verbatim}[frame=single]
%% 05 06 e2 07 9b 08 55 09 e2 07 76 08 ba 08
%% \end{Verbatim}
%% our \VAR{Entropy Variables} are:
%% \begin{align*}
%% \text{Entropy Variable A}_1 &= \texttt{0x07E2} = 2018 = \lfloor \text{wv\_exp2}(2018 \bmod{256}) \div 2 ^ {9 - \lfloor 2018 \div 2 ^ 8 \rfloor} \rfloor \\
%% &= \lfloor \text{wv\_exp2}(226) \div 2 ^ {9 - 7} \rfloor
%%  = \lfloor 472 \div 4 \rfloor = \textbf{118} \\
%% \text{Entropy Variable A}_2 &= \texttt{0x089B} = 2203 = \lfloor \text{wv\_exp2}(2203 \bmod{256}) \div 2 ^ {9 - \lfloor 2203 \div 2 ^ 8 \rfloor} \rfloor \\
%% &= \lfloor \text{wv\_exp2}(155) \div 2 ^ {9 - 8} \rfloor
%%  = \lfloor 389 \div 2 \rfloor = \textbf{194} \\
%% \text{Entropy Variable A}_3 &= \texttt{0x0955} = 2389 = \text{wv\_exp2}(2389 \bmod{256}) \times 2 ^ {\lfloor 2389 \div 2 ^ 8 \rfloor - 9} \\
%% &= \text{wv\_exp2}(85) \times 2 ^ {9 - 9} = 322 \times 1 = \textbf{322} \\
%% \text{Entropy Variable B}_1 &= \texttt{0x07E2} = 2018 = \lfloor \text{wv\_exp2}(2018 \bmod{256}) \div 2 ^ {9 - \lfloor 2018 \div 2 ^ 8 \rfloor} \rfloor \\
%% &= \lfloor \text{wv\_exp2}(226) \div 2 ^ {9 - 7} \rfloor
%%  = \lfloor 472 \div 4 \rfloor = \textbf{118} \\
%% \text{Entropy Variable B}_2 &= \texttt{0x0876} = 2166 = \lfloor \text{wv\_exp2}(2166 \bmod{256}) \div 2 ^ {9 - \lfloor 2166 \div 2 ^ 8 \rfloor} \rfloor \\
%% &= \lfloor \text{wv\_exp2}(118) \div 2 ^ {9 - 8} \rfloor = \lfloor 352 \div 2 \rfloor = \textbf{176} \\
%% \text{Entropy Variable B}_3 &= \texttt{0x08BA} = 2234 = \lfloor \text{wv\_exp2}(2234 \bmod{256}) \div 2 ^ {9 - \lfloor 2234 \div 2 ^ 8 \rfloor} \rfloor \\
%% &= \lfloor \text{wv\_exp2}(186) \div 2 ^ {9 - 8} \rfloor
%%  = \lfloor 424 \div 2 \rfloor = \textbf{212}
%% \end{align*}

%% \clearpage
%% \subsection{the Bitstream Sub-Block}
%% This sub-block contains the block's residual values.
%% Once decoded, these will ultimately become our file's raw PCM values.
%% \begin{figure}[h]
%% \includegraphics{figures/wavpack/bitstream.pdf}
%% \end{figure}
%% \par
%% \noindent

%% Decoding the \VAR{Bitstream} sub-block requires the \VAR{Entropy Variables}
%% data which it combines with this sub-block's bitstream to yield a set of signed
%% values totaling \VAR{Block Samples} (times 2 if the block is stereo).

%% Decoding each value is a complex process which I'll divide into
%% three separate steps.

%% As an example, we'll use a 1-channel block with the entropy variables:
%% \begin{itemize}
%% \item $\text{Entropy Variable A}_1$ = 111
%% \item $\text{Entropy Variable A}_1$ = 159
%% \item $\text{Entropy Variable A}_1$ = 299
%% \end{itemize}
%% and the partial bitstream sub-block bytes:
%% \begin{Verbatim}[frame=single]
%% 8a 0e 00 00 a1 77 e9
%% \end{Verbatim}
%% The first four bytes are the header and block size values.
%% The remaining three are as follows as a little-endian stream:
%% \begin{Verbatim}[frame=single]
%% 1 0 0 0 0 1 0 1  1 1 1 0 1 1 1 0  1 0 0 1 0 1 1 1
%% \end{Verbatim}

%% \clearpage
%% \subsubsection{Determining t}
%% \begin{wrapfigure}[37]{r}{3in}
%% \includegraphics{figures/wavpack/read_residual1.pdf}
%% \caption{Step 1: determining t}
%% \end{wrapfigure}
%% The first step is taking two boolean values called
%% \VAR{Holding One} and \VAR{Holding Zero} and determining \VAR{t}.
%% These holding values can be thought of as registers in a sort of
%% WavPack bitstream virtual machine whose values will
%% change over the course of decoding.
%% Their initial values are both false.

%% \VAR{limited unary} means counting the number
%% of \texttt{1} bits until the next \texttt{0} bit, to a maximum
%% of 33, \texttt{1} bits in a row.

%% In our example, to decode the first \VAR{t} value:
%% \begin{description}
%% \item[$\bullet$ is holding\_zero?] no
%% \item[$\bullet$ t = limited\_unary] = `\texttt{1 0}' = 1
%% \item[$\bullet$ is t = 16?] no
%% \item[$\bullet$ is holding\_one?] no
%% \item[$\bullet$ is t odd?] yes
%% \item[$\bullet$ holding\_one = true]
%% \item[$\bullet$ holding\_zero = false]
%% \item[$\bullet$ t = $\lfloor$ t $\div$ 2 $\rfloor$] = $\lfloor 1 \div 2 \rfloor$ = 0
%% \end{description}
%% So our \VAR{t} value is 0, our \VAR{Holding One} value is true,
%% our \VAR{Holding Zero} value is false and we've
%% consumed 2 bits from the sub-block's bitstream.

%% \clearpage

%% \subsubsection{Calculating Base/Add}
%% The next step is taking \VAR{t} and calculating \VAR{Base} and \VAR{Add}
%% from our entropy variables, updating our entropy variables in the process.
%% \begin{figure}[h]
%% \includegraphics{figures/wavpack/read_residual2.pdf}
%% \caption{Step 2: determining base/add}
%% \end{figure}
%% \par
%% \noindent
%% So to continue our example:
%% \begin{description}
%% \setlength{\itemsep}{0pt}
%% \item[$\bullet$ is t = 0?] yes
%% \item[$\bullet$ base] = 0
%% \item[$\bullet$ add = $\lfloor \text{Entropy}_1 \div 16 \rfloor$] = $\lfloor 111 \div 16 \rfloor = 6$
%% \item[$\bullet$ $\text{Entropy}_1$ = $\text{Entropy}_1 - \lfloor(\text{Entropy}_1 + 126) \div 128\rfloor \times 2$] = $111 - 2 = 109$
%% \end{description}
%% %% Note that $\text{Entropy}_2$ remains 159 and $\text{Entropy}_3$ remains 299.

%% \clearpage

%% \subsubsection{Determining Value}
%% \begin{wrapfigure}[30]{r}{3in}
%% \includegraphics{figures/wavpack/read_residual3.pdf}
%% \caption{Step 3: determining value}
%% \end{wrapfigure}
%% Finally, given our \VAR{Base} and \VAR{Add} values, we
%% determine the final residual value as follows:
%% \begin{description}
%% \item[$\bullet$ is $\text{add} < 1$?] no
%% \item[$\bullet$ p = $\text{log}_2(\text{add})$] = $\text{log}_2(6)$ = 2
%% \item[$\bullet$ e = $2 ^ {p + 1} - \text{add} - 1$] = $2 ^ 3 - 6 - 1 = 1$
%% \item[$\bullet$ is $\text{p} > 0$?] yes
%% \item[$\bullet$ result = read 2] = `\texttt{0 0}' = 0
%% \item[$\bullet$ is $\text{result} > \text{e}$?] no
%% \item[$\bullet$ sign = read 1] = `\texttt{0}' = 0
%% \item[$\bullet$ is $\text{sign} = 1$?] no
%% \item[$\bullet$ value = $\text{base} + \text{result}$] = 0 + 0 = 0
%% \end{description}
%% Thus, this stage consumes an additional 3 bits and our first residual
%% value is 0.

%% Determining the second residual value requires going through all
%% three steps again, with our freshly updated \VAR{Holding One},
%% \VAR{Holding Zero} and \VAR{Entropy} values.

%% Note that in a 2 channel (non-mono) block,
%% the \VAR{Entropy} values alternate between residuals.
%% For example,
%% $\text{Residual}_0$ uses $\text{Entropy A}$,
%% $\text{Residual}_1$ uses $\text{Entropy B}$,
%% $\text{Residual}_3$ uses $\text{Entropy A}$, and so forth.
%% However, \VAR{Holding One} and \VAR{Holding Zero} are shared
%% between channels.

%% \clearpage

%% Now, let's run through the next residual on our remaining bits:
%% \begin{Verbatim}[frame=single]
%% 1 0 1  1 1 1 0 1 1 1 0  1 0 0 1 0 1 1 1
%% \end{Verbatim}
%% \begin{description}
%% \item[$\bullet$ is holding\_zero?] no
%% \item[$\bullet$ t = limited\_unary] = `\texttt{1 0}' = 1
%% \item[$\bullet$ is t = 16?] no
%% \item[$\bullet$ is holding\_one?] yes
%% \item[$\bullet$ is t odd?] yes
%% \item[$\bullet$ holding\_one = true]
%% \item[$\bullet$ holding\_zero = false]
%% \item[$\bullet$ t = $\lfloor$ t $\div$ 2 $\rfloor$ + 1] = $\lfloor 1 \div 2 \rfloor$ + 1 = 1
%% \item[$\bullet$ is t = 0?] no
%% \item[$\bullet$ is t = 1?] yes
%% \item[$\bullet$ base = $\lfloor\text{Entropy}_1 \div 16\rfloor + 1$] = $\lfloor 109 \div 16\rfloor + 1 = 7$
%% \item[$\bullet$ add = $\lfloor\text{Entropy}_2 \div 16\rfloor$] = $\lfloor 159 \div 16\rfloor = 9$
%% \item[$\bullet$ $\text{Entropy}_1 = \text{Entropy}_1 + \lfloor (\text{Entropy}_1 + 128) \div 128\rfloor \times 5$] = 114
%% \item[$\bullet$ $\text{Entropy}_2 = \text{Entropy}_2 - \lfloor (\text{Entropy}_2 + 62) \div 64\rfloor \times 2$] = 153
%% \item[$\bullet$ is $\text{add} < 1$?] no
%% \item[$\bullet$ p = $\text{log}_2(\text{add})$] = $\text{log}_2(9)$ = 3
%% \item[$\bullet$ e = $2 ^ {p + 1} - \text{add} - 1$] = $2 ^ 4 - 9 - 1 = 6$
%% \item[$\bullet$ is $\text{p} > 0$?] yes
%% \item[$\bullet$ result = read 3] = `\texttt{1 1 1}' = 7
%% \item[$\bullet$ is $\text{result} > \text{e}$?] yes
%% \item[$\bullet$ result = $(\text{result} \times 2) - \text{e} + $ read 1] = $(7 \times 2) - 6 + \texttt{1} = 9$
%% \item[$\bullet$ sign = read 1] = 0
%% \item[$\bullet$ is sign = 1?] no
%% \item[$\bullet$ value = base + result] = 7 + 9 = 16
%% \end{description}
%% Which returns the value 16 and consumes 7 bits in total.
%% %% \begin{sidewaysfigure}[h]
%% %% \includegraphics{figures/wavpack_read_residual.pdf}
%% %% \caption{the WavPack residual reading sequence}
%% %% \end{sidewaysfigure}

%% \clearpage

%% \subsubsection{Zero Residuals}

%% As with most other lossless codecs, WavPack features a special
%% case to handle a large number of \texttt{0} samples in a row.
%% This is triggered when $\text{Entropy A}_1$ is less than 2,
%% $\text{Entropy B}_1$ is less than 2 (for non-mono blocks),
%% and \VAR{Holding One} and \VAR{Holding Zero} are both false.

%% \begin{wrapfigure}[14]{r}{3in}
%% \includegraphics{figures/wavpack/read_zeroes.pdf}
%% \end{wrapfigure}

%% In that instance, we read a residual-like value to determine
%% how many \texttt{0} values follow.
%% If any, we set the block's six \VAR{Entropy} variables to 0
%% and output the necessary number of \texttt{0} values just as
%% regular residuals.

%% Therefore, for non-mono blocks, these values alternative between channels
%% just as regular residual values do.
%% In addition, they also count against the block's total number of samples.

%% Once all of the \texttt{0} values have been output, if any
%% \VAR{Block Samples} remain, we return to the regular residual
%% reading process.

%% \clearpage

%% \subsection{Sample Decorrelation}

%% Once the bitstream sub-block has been decoded into a set of
%% samples values (alternating between \VAR{Channel A} and \VAR{Channel B}
%% if the block is not mono), we then apply decorrelation passes
%% to those samples - one pass per decorrelation term value,\footnote{As
%% decoded on page \pageref{wavpack_decorr_terms}.} per channel.

%% Each decorrelation pass requires a \VAR{Decorrelation Term},
%% a \VAR{Decorrelation Delta}, one \VAR{Decorrelation Weight} per channel,
%% and one or more \VAR{Decorrelation Sample} values - in addition
%% to our set of input samples we're running the pass over.
%% These passes are applied in \textit{incrementing} order
%% (i.e. $\text{Term}_1$ first, $\text{Term}_2$ next, and so on).
%% The function for each pass depends on its \VAR{Decorrelation Term}:
%% \begin{align*}
%% \intertext{Decorrelation Term = 18:}
%% \text{Temp}_i &= \lfloor ((3 \times \text{Output}_{i - 1}) - \text{Output}_{i - 2}) \div 2 \rfloor \\
%% \text{Output}_i &= \lfloor ((\text{Weight}_{i - 1} \times \text{Temp}_i) + 512) \div 1024 \rfloor + \text{Input}_i \\
%% \text{Weight}_i &=
%% \begin{cases}
%% \text{Weight}_{i - 1} & \text{if } \text{Temp}_i = 0 \text{ or } \text{Input}_i = 0 \\
%% \text{Weight}_{i - 1} + \text{Delta} & \text{if } (\text{Temp}_i \xor \text{Input}_i) \geq 0 \\
%% \text{Weight}_{i - 1} - \text{Delta} & \text{if } (\text{Temp}_i \xor \text{Input}_i) < 0
%% \end{cases}
%% \intertext{Decorrelation Term = 17:}
%% \text{Temp}_i &= (2 \times \text{Output}_{i - 1}) - \text{Output}_{i - 2} \\
%% \text{Output}_i &= \lfloor ((\text{Weight}_{i - 1} \times \text{Temp}_i) + 512) \div 1024 \rfloor + \text{Input}_i \\
%% \text{Weight}_i &=
%% \begin{cases}
%% \text{Weight}_{i - 1} & \text{if } \text{Temp}_i = 0 \text{ or } \text{Input}_i = 0 \\
%% \text{Weight}_{i - 1} + \text{Delta} & \text{if } (\text{Temp}_i \xor \text{Input}_i) \geq 0 \\
%% \text{Weight}_{i - 1} - \text{Delta} & \text{if } (\text{Temp}_i \xor \text{Input}_i) < 0
%% \end{cases}
%% \intertext{1 $\leq$ Decorrelation Term $\leq 8$:}
%% \text{Output}_i &= \lfloor ((\text{Weight}_{i - 1} \times \text{Output}_{i - \text{term}}) + 512) \div 1024 \rfloor + \text{Input}_i \\
%% \text{Weight}_i &=
%% \begin{cases}
%% \text{Weight}_{i - 1} & \text{if } \text{Output}_{i - \text{term}} = 0 \text{ or } \text{Input}_i = 0 \\
%% \text{Weight}_{i - 1} + \text{Delta} & \text{if } (\text{Output}_{i - \text{term}} \xor \text{Input}_i) \geq 0 \\
%% \text{Weight}_{i - 1} - \text{Delta} & \text{if } (\text{Output}_{i - \text{term}} \xor \text{Input}_i) < 0
%% \end{cases}
%% \end{align*}
%% Note that each function uses previously output samples for its calculation.
%% This is where \VAR{Decorrelation Samples} are used;
%% those are our $\text{Output}_{-1}$, $\text{Output}_{-2}$, etc.
%% which are used for decorrelation but not actually output.

%% For 1 or 2 channel blocks, positive decorrelation terms are applied
%% on a per-channel basis with the weight A values being applied
%% to channel A and the weight B values being applied to channel B
%% (if present).
%% However, the three negative correlation terms are only valid
%% for 2 channel blocks.
%% \begin{align*}
%% \intertext{Decorrelation Term = -1:}
%% \text{Output A}_i &= \lfloor ((\text{Weight A}_{i - 1} \times \text{Output B}_{i - 1}) + 512) \div 1024 \rfloor + \text{Input A}_i \\
%% \text{Weight A}_i &=
%% \begin{cases}
%% \text{Weight A}_{i - 1} & \text{if } \text{Output B}_{i - 1} = 0 \text{ or } \text{Input A}_i = 0 \\
%% \text{Weight A}_{i - 1} + \text{Delta} & \text{if } (\text{Output B}_{i - 1} \xor \text{Input A}_i) \geq 0 \\
%% \text{ to a maximum of 1024} \\
%% \text{Weight A}_{i - 1} - \text{Delta} & \text{if } (\text{Output B}_{i - 1} \xor \text{Input A}_i) < 0 \\
%% \text{ to a minimum of -1024}
%% \end{cases} \\
%% \text{Output B}_i &= \lfloor ((\text{Weight B}_{i - 1} \times \text{Output A}_i) + 512) \div 1024 \rfloor + \text{Input B}_i \\
%% \text{Weight B}_i &=
%% \begin{cases}
%% \text{Weight B}_{i - 1} & \text{if } \text{Output A}_i = 0 \text{ or } \text{Input B}_i = 0 \\
%% \text{Weight B}_{i - 1} + \text{Delta} & \text{if } (\text{Output A}_i \xor \text{Input B}_i) \geq 0 \\
%% \text{ to a maximum of 1024} \\
%% \text{Weight B}_{i - 1} - \text{Delta} & \text{if } (\text{Output A}_i \xor \text{Input B}_i) < 0 \\
%% \text{ to a minimum of -1024}
%% \end{cases}
%% \intertext{Decorrelation Term = -2:}
%% \text{Output B}_i &= \lfloor ((\text{Weight B}_{i - 1} \times \text{Output A}_{i - 1}) + 512) \div 1024 \rfloor + \text{Input B}_i \\
%% \text{Weight B}_i &=
%% \begin{cases}
%% \text{Weight B}_{i - 1} & \text{if } \text{Output A}_{i - 1} = 0 \text{ or } \text{Input B}_i = 0 \\
%% \text{Weight B}_{i - 1} + \text{Delta} & \text{if } (\text{Output A}_{i - 1} \xor \text{Input B}_i) \geq 0 \\
%% \text{ to a maximum of 1024} \\
%% \text{Weight B}_{i - 1} - \text{Delta} & \text{if } (\text{Output A}_{i - 1} \xor \text{Input B}_i) < 0 \\
%% \text{ to a minimum of -1024}
%% \end{cases} \\
%% \text{Output A}_i &= \lfloor ((\text{Weight A}_{i - 1} \times \text{Output B}_i) + 512) \div 1024 \rfloor + \text{Input A}_i \\
%% \text{Weight A}_i &=
%% \begin{cases}
%% \text{Weight A}_{i - 1} & \text{if } \text{Output B}_i = 0 \text{ or } \text{Input A}_i = 0 \\
%% \text{Weight A}_{i - 1} + \text{Delta} & \text{if } (\text{Output B}_i \xor \text{Input A}_i) \geq 0 \\
%% \text{ to a maximum of 1024} \\
%% \text{Weight A}_{i - 1} - \text{Delta} & \text{if } (\text{Output B}_i \xor \text{Input A}_i) < 0 \\
%% \text{ to a minimum of -1024}
%% \end{cases}
%% \end{align*}

%% \clearpage

%% \begin{align*}
%% \intertext{Decorrelation Term = -3:}
%% \text{Output A}_i &= \lfloor ((\text{Weight A}_{i - 1} \times \text{Output B}_{i - 1}) + 512) \div 1024 \rfloor + \text{Input A}_i \\
%% \text{Weight A}_i &=
%% \begin{cases}
%% \text{Weight A}_{i - 1} & \text{if } \text{Output B}_{i - 1} = 0 \text{ or } \text{Input A}_i = 0 \\
%% \text{Weight A}_{i - 1} + \text{Delta} & \text{if } (\text{Output B}_{i - 1} \xor \text{Input A}_i) \geq 0 \\
%% \text{ to a maximum of 1024} \\
%% \text{Weight A}_{i - 1} - \text{Delta} & \text{if } (\text{Output B}_{i - 1} \xor \text{Input A}_i) < 0 \\
%% \text{ to a minimum of -1024}
%% \end{cases} \\
%% \text{Output B}_i &= \lfloor ((\text{Weight B}_{i - 1} \times \text{Output A}_{i - 1}) + 512) \div 1024 \rfloor + \text{Input B}_i \\
%% \text{Weight B}_i &=
%% \begin{cases}
%% \text{Weight B}_{i - 1} & \text{if } \text{Output A}_{i - 1} = 0 \text{ or } \text{Input B}_i = 0 \\
%% \text{Weight B}_{i - 1} + \text{Delta} & \text{if } (\text{Output A}_{i - 1} \xor \text{Input B}_i) \geq 0 \\
%% \text{ to a maximum of 1024} \\
%% \text{Weight B}_{i - 1} - \text{Delta} & \text{if } (\text{Output A}_{i - 1} \xor \text{Input B}_i) < 0 \\
%% \text{ to a minimum of -1024}
%% \end{cases}
%% \end{align*}

%% The effect of applying these passes cumulatively is interesting when
%% visualized on a 1 channel sine wave example stream:

%% \begin{figure}[h]
%% \subfloat{
%% \includegraphics{figures/wavpack/decorrelation1.pdf}
%% }
%% \subfloat{
%% \includegraphics{figures/wavpack/decorrelation2.pdf}
%% }
%% \subfloat{
%% \includegraphics{figures/wavpack/decorrelation3.pdf}
%% }
%% \end{figure}

%% \clearpage

%% Now it's time to put all this together into an example.
%% Given a 1 channel block with the sub-block decorrelation values:
%% \begin{table}[h]
%% \begin{tabular}{r r|r r|r r||r r | r r | r r}
%% $\text{Term}_1$ & 3 & $\text{Delta}_1$ & 2 &
%% $\text{Weight}_1$ & 16 &
%% $\text{Sample}_{1~1}$ & 0 &
%% $\text{Sample}_{1~2}$ & 0 &
%% $\text{Sample}_{1~3}$ & 0 \\
%% $\text{Term}_2$ & 17 & $\text{Delta}_2$ & 2 &
%% $\text{Weight}_2$ & 48 &
%% & & $\text{Sample}_{2~1}$ & 0 &
%% $\text{Sample}_{2~2}$ & 0 \\
%% $\text{Term}_3$ & 2 & $\text{Delta}_3$ & 2 &
%% $\text{Weight}_3$ & 32 &
%% & & $\text{Sample}_{3~1}$ & 0 &
%% $\text{Sample}_{3~2}$ & 0 \\
%% $\text{Term}_4$ & 18 & $\text{Delta}_4$ & 2 &
%% $\text{Weight}_4$ & 48 &
%% & & $\text{Sample}_{4~1}$ & 0 &
%% $\text{Sample}_{4~2}$ & 0 \\
%% $\text{Term}_5$ & 18 & $\text{Delta}_5$ & 2 &
%% $\text{Weight}_5$ & 48 &
%% & & $\text{Sample}_{5~1}$ & -78 &
%% $\text{Sample}_{5~2}$ & -73 \\
%% \end{tabular}
%% \end{table}
%% \par
%% \noindent
%% and the residual values:
%% \begin{table}[h]
%% \begin{tabular}{r r}
%% $\text{Residual}_1$ = -61 \\
%% $\text{Residual}_2$ = -33 \\
%% \end{tabular}
%% \end{table}
%% \par
%% \noindent
%% then decorrelation pass 1 applies the $\text{Term}_1$
%% formula 3, $\text{Delta}_1$ value of 2,
%% $\text{Weight}_{1~0}$ value of 16 and initial sample values
%% of 0 ($\text{Sample}_{1~1}$, $\text{Sample}_{1~2}$,
%% $\text{Sample}_{1~3}$) to the residual input values of -61 and -33
%% ($\text{Channel}_1$ and $\text{Channel}_2$).
%% \begin{align*}
%% \text{Output}_1 &= \lfloor ((\text{Weight}_{1~0} \times \text{Output}_{-2}) + 512) \div 1024 \rfloor + \text{Input}_1 \\
%% &= \lfloor ((16 \times 0) + 512) \div 1024 \rfloor - 61 = \textbf{-61} \\
%% \text{Weight}_{1~1} &= \text{Weight}_{1~0} = \textbf{16} \\
%% \text{Output}_2 &= \lfloor ((\text{Weight}_{1~1} \times \text{Output}_{-1}) + 512) \div 1024 \rfloor + \text{Input}_2 \\
%% &= \lfloor ((16 \times 0) + 512) \div 1024 \rfloor - 33 = \textbf{-33} \\
%% \text{Weight}_{1~2} &= \text{Weight}_{1~1} = \textbf{16}
%% \end{align*}
%% Decorrelation pass 2 applies the $\text{Term}_2$ formula 17,
%% $\text{Delta}_2$ value of 2, $\text{Weight}_{2~0}$ value of 48
%% and the initial sample values of 0 ($\text{Sample}_{2~1}$,
%% $\text{Sample}_{2~2}$).  Note that the inputs to pass 2 are the
%% outputs from pass 1.
%% \begin{align*}
%% \text{Temp}_1 &= (2 \times \text{Output}_0) - \text{Output}_{-1} = (2 \times 0) - 0 = \textbf{0} \\
%% \text{Output}_1 &= \lfloor ((\text{Weight}_{2~0} \times \text{Temp}_1) + 512) \div 1024 \rfloor + \text{Input}_1 \\
%% &= \lfloor ((48 \times 0) + 512) \div 1024 \rfloor - 61 = \textbf{-61} \\
%% \text{Weight}_{2~1} &= \text{Weight}_{2~0} = \textbf{48} \\
%% \text{Temp}_2 &= (2 \times \text{Output}_1) - \text{Output}_{0} = (2 \times -61) - 0 = \textbf{-122} \\
%% \text{Output}_2 &= \lfloor ((\text{Weight}_{2~1} \times \text{Temp}_2) + 512) \div 1024 \rfloor + \text{Input}_2 \\
%% &= \lfloor ((48 \times -122) + 512) \div 1024 \rfloor - 33 = -6 - 33 = \textbf{-39} \\
%% \text{Weight}_{2~2} &= \text{Weight}_{2~1} + \text{Delta}_2 = 48 + 2 = \textbf{50}
%% \end{align*}
%% Decorrelation pass 3 applies the $\text{Term}_3$ formula 2,
%% $\text{Delta}_3$ value of 2, $\text{Weight}_{3~0}$ value of 32
%% and initial samples of 0 ($\text{Sample}_{3~1}$, $\text{Sample}_{3~2}$).
%% \begin{align*}
%% \text{Output}_1 &= \lfloor ((\text{Weight}_{3~0} \times \text{Output}_{-1}) + 512) \div 1024 \rfloor + \text{Input}_1 \\
%% &= \lfloor ((32 \times 0) + 512) \div 1024 \rfloor - 61 = \textbf{-61} \\
%% \text{Weight}_{3~1} &= \text{Weight}_{3~0} = \textbf{32} \\
%% \text{Output}_2 &= \lfloor ((\text{Weight}_{3~1} \times \text{Output}_0) + 512) \div 1024 \rfloor + \text{Input}_2 \\
%% &= \lfloor ((32 \times 0) + 512) \div 1024 \rfloor - 39 = \textbf{-39} \\
%% \text{Weight}_{3~2} &= \text{Weight}_{3~1} = \textbf{32}
%% \end{align*}
%% Decorrelation pass 4 applies the $\text{Term}_4$ formula 18,
%% $\text{Delta}_4$ value of 2, $\text{Weight}_{4~0}$ value of 48
%% and initial samples of 0 ($\text{Sample}_{4~1}$, $\text{Sample}_{4~2}$).
%% \begin{align*}
%% \text{Temp}_1 &= \lfloor ((3 \times \text{Output}_0) - \text{Output}_{-1}) \div 2 \rfloor = \lfloor ((3 \times 0) - 0) \div 2 \rfloor = \textbf{0} \\
%% \text{Output}_1 &= \lfloor ((\text{Weight}_{4~0} \times \text{Temp}_1) + 512) \div 1024 \rfloor + \text{Input}_1 \\
%% &= \lfloor ((48 \times 0) + 512) \div 1024 \rfloor - 61 = \textbf{-61} \\
%% \text{Weight}_{4~1} &= \text{Weight}_{4~0} = \textbf{48} \\
%% \text{Temp}_2 &= \lfloor ((3 \times \text{Output}_1) - \text{Output}_0) \div 2 \rfloor = \lfloor ((3 \times -61) - 0) \div 2 \rfloor = \textbf{-92} \\
%% \text{Output}_2 &= \lfloor ((\text{Weight}_{4~1} \times \text{Temp}_2) + 512) \div 1024 \rfloor + \text{Input}_2 \\
%% &= \lfloor ((48 \times -92) + 512) \div 1024 \rfloor - 39 = -4 - 39 = \textbf{-43} \\
%% \text{Weight}_{4~2} &= \text{Weight}_{4~1} + \text{Delta}_4 = 48 + 2 = \textbf{50}
%% \end{align*}
%% Finally, decorrelation pass 5 applies the $\text{Term}_5$ formula 18,
%% $\text{Delta}_5$ value of 2, $\text{Weight}_{5~0}$ value of 48
%% and initial samples of -78, -73 ($\text{Sample}_{5~1}$,
%% $\text{Sample}_{5~2}$).
%% \begin{align*}
%% \text{Temp}_1 &= \lfloor ((3 \times \text{Output}_0) - \text{Output}_{-1}) \div 2 \rfloor = \lfloor ((3 \times -73) + 78) \div 2 \rfloor = \textbf{-71} \\
%% \text{Output}_1 &= \lfloor ((\text{Weight}_{5~0} \times \text{Temp}_1) + 512) \div 1024 \rfloor + \text{Input}_1 \\
%% &= \lfloor ((48 \times -71) + 512) \div 1024 \rfloor - 61 = -3 - 61 = \textbf{-64} \\
%% \text{Weight}_{5~1} &= \text{Weight}_{5~0} + \text{Delta}_5 = 48 + 2 = \textbf{50} \\
%% \text{Temp}_2 &= \lfloor ((3 \times \text{Output}_1) - \text{Output}_0) \div 2 \rfloor = \lfloor ((3 \times -64) + 73) \div 2 \rfloor = \textbf{-60} \\
%% \text{Output}_2 &= \lfloor ((\text{Weight}_{5~1} \times \text{Temp}_2) + 512) \div 1024 \rfloor + \text{Input}_2 \\
%% &= \lfloor ((50 \times -60) + 512) \div 1024 \rfloor - 43 = \textbf{-46} \\
%% \text{Weight}_{5~2} &= \text{Weight}_{5~1} + \text{Delta}_5 = 50 + 2 = \textbf{52}
%% \end{align*}
%% So, after running through all five passes, our samples are now
%% -64 and -46.

%% \clearpage

%% \subsection{Joint Stereo}

%% If the block is not mono and the \VAR{Joint Stereo} bit is set
%% in the block header, our channels require one more stage of processing
%% to transform their mid-side values back into left and right sample
%% values.\footnote{In the case of multi-channel audio, these aren't
%% necessarily \textit{front} left and right; they might be
%% side left and right or rear left and right channels.}
%% \begin{align*}
%% \text{Left}_i &= \left\lceil\frac{\text{Channel A}_i + (\text{Channel B}_i \times 2)}{2}\right\rceil \\
%% \text{Right}_i &= (\text{Channel B}_i \times 2) - \left\lfloor\frac{\text{Channel A}_i + (\text{Channel B}_i \times 2)}{2}\right\rfloor
%% \end{align*}
%% For example, given the \text{Channel A} samples of -64 and -46, and the
%% \text{Channel B} samples of 32 and 39, we convert them to left
%% and right samples as follows:
%% \begin{align*}
%% \text{Left}_1 &= \left\lceil\frac{\text{Channel A}_1 + (\text{Channel B}_1 \times 2)}{2}\right\rceil = \left\lceil\frac{-64 + (32 \times 2)}{2}\right\rceil = \textbf{0} \\
%% \text{Right}_1 &= (\text{Channel B}_1 \times 2) - \left\lfloor\frac{\text{Channel A}_1 + (\text{Channel B}_1 \times 2)}{2}\right\rfloor \\
%% &= (32 \times 2) - \left\lfloor\frac{-64 + (32 \times 2)}{2}\right\rfloor = 64 - 0 = \textbf{64} \\
%% \text{Left}_2 &= \left\lceil\frac{-46 + (39 \times 2)}{2}\right\rceil = \textbf{16} \\
%% \text{Right}_2 &= (\text{Channel B}_2 \times 2) - \left\lfloor\frac{\text{Channel A}_2 + (\text{Channel B}_2 \times 2)}{2}\right\rfloor \\
%% &= (39 \times 2) - \left\lfloor\frac{-46 + (39 \times 2)}{2}\right\rfloor = 78 - 16 = \textbf{62}
%% \end{align*}
%% Thus, our left samples are 0 and 16, and our right samples are 64 and 62.

%% \subsection{the CRC}

%% Verifying the block's CRC is quite simple:
%% \begin{equation*}
%% \text{CRC}_i = (3 \times \text{CRC}_{i - 1}) + \text{Decoded Sample}_i
%% \end{equation*}
%% where $\text{Decoded Sample}_i$ alternates between channels if necessary,
%% $\text{CRC}_{-1}$ begins with a value of \texttt{0xFFFFFFFF}
%% and each $\text{CRC}_i$ value is truncated to 32 bits.

%% The CRC is calculated \textit{after} the joint stereo transformation,
%% but \textit{before} handling extended/shifted integers and false stereo.

%% \clearpage

%% \subsection{Extended/Shifted Integers}
%% \label{wavpack_extended_integers}

%% If \VAR{Extended Size Integers} is set in the block header,
%% there should be an \VAR{Int32 Info} sub-block present whose layout
%% is as follows:

%% \begin{figure}[h]
%% \includegraphics{figures/wavpack/extended_integers.pdf}
%% \end{figure}

%% Curiously, these values are exclusive; if \VAR{Zero Bits} is present,
%% \VAR{One Bits} is ignored and so forth.
%% If \VAR{Zero Bits} is non-zero, we pad each sample's least-significant
%% bits with that many \texttt{0} bits.
%% If \VAR{One Bits} is non-zero, we pad each sample's least-significant
%% bits with that many \texttt{1} bits.
%% If \VAR{Duplicate Bits} is non-zero, we pad each sample's least-significant
%% bits with that sample's own least-significant bit,
%% \VAR{Duplicate Bits} number of times.

%% This can be summarized as follows:
%% \begin{equation*}
%% \text{Extended}_i =
%% \begin{cases}
%% \text{Original}_i \times 2 ^ {\text{Zero Bits}} & \text{if \VAR{Zero Bits} } > 0 \\
%% \text{Original}_i \times 2 ^ {\text{One Bits}} + (2 ^ {\text{One Bits}} - 1) & \text{if \VAR{One Bits} } > 0 \\
%% \text{Original}_i \times 2 ^ {\text{Duplicate Bits}} \\
%% \text{ if \VAR{Duplicate Bits} } > 0 \text{ and }\text{Original}_i \bmod{2} = 0 \\
%% \text{Original}_i \times 2 ^ {\text{Duplicate Bits}} + (2 ^ {\text{Duplicate Bits}} - 1) \\
%% \text{ if \VAR{Duplicate Bits} } > 0 \text{ and }\text{Original}_i \bmod{2} = 1
%% \end{cases}
%% \end{equation*}

%% \subsection{Channel Info}
%% \label{wavpack_channel_info}

%% A WavPack file with more than 2 channels should have a \VAR{Channel Info}
%% sub-block to indicate its channel layout.
%% \begin{figure}[h]
%% \includegraphics{figures/wavpack/channel_info.pdf}
%% \end{figure}
%% \par
%% \noindent
%% \VAR{Channel Mask} is the same as used by Wave, as shown on
%% page \pageref{wave_channel_assignment}.

%% \subsection{False Stereo}

%% If the \VAR{False Stereo} bit is set in the block header,
%% we've been treating the block as being mono thus far.
%% At this point, we duplicate Channel A's values to Channel B
%% just prior to returning the from the block.

%% \clearpage

%% \subsection{RIFF WAVE Header/Footer}

%% These sub-blocks are typically found in the first and last
%% WavPack block, respectively.
%% The header must always be present in the file while
%% the footer is optional.

%% \begin{figure}[h]
%% \includegraphics{figures/wavpack/wave_header.pdf}
%% \vskip .25in
%% \includegraphics{figures/wavpack/wave_footer.pdf}
%% \end{figure}
%% \par
%% \noindent
%% One can think of them as halves of a `PCM sandwich'
%% of which our decoded data comprises the `meat':
%% \begin{figure}[h]
%% \includegraphics{figures/wavpack/pcm_sandwich.pdf}
%% \end{figure}

%% \subsection{MD5}
%% This optional sub-block is typically found in the final WavPack block.
%% \begin{figure}[h]
%% \includegraphics{figures/wavpack/md5sum.pdf}
%% \end{figure}
%% \par
%% \noindent
%% The MD5 is the hash of all the samples over the entire file.
%% It is calculated by running the
%% hashing algorithm\footnote{As described by RFC1321} over
%% the raw input samples in little-endian format
%% and signed if their bits-per-sample are greater than 8.

%% \clearpage

%% \section{WavPack Encoding}

%% For WavPack encoding, one needs a stream of input PCM values
%% along with the stream's sample rate, number of channels, bits per sample
%% and channel mask.

%% We first split our input samples into chunks containing
%% \VAR{Block Size} number of PCM frames.
%% Since WavPack's headers are relatively large and its
%% adaptive algorithm is quite good over long stretches of samples,
%% it makes sense to use a large block size.
%% The reference encoder defaults to 44100 PCM frames.

%% The next step is to split those chunks of PCM frames into WavPack
%% blocks containing 1 or 2 channels each.
%% For a one channel input stream, the blocks are sent as follows:
%% \begin{table}[h]
%% \begin{tabular}{| r | c | c | c | r | r |}
%% Block & First Block Bit & Last Block Bit & Is Mono & Channel A & Channel B \\
%% \hline
%% $\text{Block}_1$ & 1 & 1 & 1 & Front Center &
%% \end{tabular}
%% \end{table}
%% \par
%% \noindent
%% For a two channel input stream, the blocks are sent as follows:
%% \begin{table}[h]
%% \begin{tabular}{| r | c | c | c | r | r |}
%% Block & First Block Bit & Last Block Bit & Is Mono & Channel A & Channel B \\
%% \hline
%% $\text{Block}_1$ & 1 & 1 & 0 & Front Left & Front Right
%% \end{tabular}
%% \end{table}

%% However, for multi-channel input streams, we need to split its
%% channels into a set of blocks with 1 or 2 channels per block.
%% By using the channel mask\footnote{As explained on page
%% \pageref{wave_channel_assignment}.} we can split the stream
%% into 2 channel blocks with left-right channel pairs and
%% 1 channel blocks for everything else.

%% For example, given a 6-channel audio stream with the channel mask
%% \texttt{0x3F}, we have the channels \VAR{Front Left}, \VAR{Front Right},
%% \VAR{Front Center}, \VAR{LFE}, \VAR{Back Left} and \VAR{Back Right} -
%% in that order.
%% So, a good way to split our channels into blocks is as follows:
%% \begin{table}[h]
%% \begin{tabular}{| r | c | c | c | r | r |}
%% Block & First Block Bit & Last Block Bit & Is Mono & Channel A & Channel B \\
%% \hline
%% $\text{Block}_1$ & 1 & 0 & 0 & Front Left & Front Right \\
%% $\text{Block}_2$ & 0 & 0 & 1 & Front Center & \\
%% $\text{Block}_3$ & 0 & 0 & 1 & LFE & \\
%% $\text{Block}_4$ & 0 & 1 & 0 & Back Left & Back Right
%% \end{tabular}
%% \end{table}

%% \subsection{Channel Info}
%% If the stream has more than 2 channels, a \VAR{Channel Info}
%% sub-block should be added to the first block,
%% as illustrated on page \pageref{wavpack_channel_info}.

%% \subsection{False Stereo}

%% If the block is stereo and Channel A's samples are identical to
%% Channel B's samples, one can set the \VAR{False Stereo} bit in
%% the block header and treat the block as having only one channel
%% for the rest of its encoding.
%% Note that the block's \VAR{Is Mono} bit is still \texttt{false}
%% in this case.

%% \subsection{Extended/Shifted Integers}
%% \label{wavpack_encode_extended_integers}
%% If the following condition holds:
%% \begin{equation*}
%% 0 = \overset{\text{block size} - 1}{\underset{i = 0}{\sum}}{\text{Channel}_i \bmod{2 ^ {bits}}}
%% \end{equation*}
%% for Channel A and, if present, Channel B where $bits > 0$, then the
%% highest value of $bits$ if what's used for the \VAR{Zero Bits}
%% field in an \VAR{Extended Size Integers} sub-block, as described
%% on page \pageref{wavpack_extended_integers}.
%% Each channel's samples are then divided by $2 ^ {bits}$ for the
%% remainder of encoding and the \VAR{Extended Size Integers} bit
%% is set in the block header.

%% \subsection{the CRC}

%% After the audio samples have been processed for false stereo
%% and wasted bits, it's best to perform the block header CRC calculation
%% before starting to encode them, as follows:

%% \begin{equation*}
%% \text{CRC}_i = (3 \times \text{CRC}_{i - 1}) + \text{Sample}_i
%% \end{equation*}
%% where $\text{Sample}_i$ alternates between channels if necessary.
%% $\text{CRC}_{-1}$ begins with a value of \texttt{0xFFFFFFFF}
%% and each $\text{CRC}_i$ value is truncated to 32 bits.

%% \subsection{Joint Stereo}

%% Next, for two channel blocks, one typically converts both
%% channels to joint stereo.
%% This involves transforming independent left and right channels
%% to mid and side channels.
%% \begin{align*}
%% \text{Mid}_i &= \text{Channel A}_i - \text{Channel B}_i \\
%% \text{Side}_i &= \left\lfloor\frac{\text{Channel A}_i + \text{Channel B}_i}{2}\right\rfloor
%% \end{align*}
%% Where \VAR{Mid} is the new \VAR{Channel A} and \VAR{Side} is the new
%% \VAR{Channel B}.
%% For example, given the \VAR{Channel A} value of 16 and the
%% \VAR{Channel B} value of 62, our conversion is as follows:
%% \begin{align*}
%% \text{Mid}_0 &= 16 - 62 = \textbf{-46} \\
%% \text{Side}_0 &= \left\lfloor\frac{16 + 62}{2}\right\rfloor = \textbf{39}
%% \end{align*}
%% One must also set the \VAR{Joint Stereo} bit in the block header.

%% \clearpage

%% \subsection{Block Header}

%% Once the \VAR{False Stereo}, \VAR{Extended Size Integers},
%% \VAR{Joint Stereo} and \VAR{CRC} values are decided,
%% we can finally write a block header based on our input:
%% \begin{figure}[h]
%% \includegraphics{figures/wavpack/block_header.pdf}
%% \end{figure}
%% \par
%% \noindent
%% \begin{wrapfigure}[10]{r}{1.75in}
%% \begin{tabular}{|r|c|}
%% \hline
%% sample rate & value \\
%% \hline
%% 6000 & \texttt{0000} \\
%% 8000 & \texttt{0001} \\
%% 9600 & \texttt{0010} \\
%% 11025 & \texttt{0011} \\
%% 12000 & \texttt{0100} \\
%% 16000 & \texttt{0101} \\
%% 22050 & \texttt{0110} \\
%% 24000 & \texttt{0111} \\
%% 32000 & \texttt{1000} \\
%% 44100 & \texttt{1001} \\
%% 48000 & \texttt{1010} \\
%% 64000 & \texttt{1011} \\
%% 88200 & \texttt{1100} \\
%% 96000 & \texttt{1101} \\
%% 192000 & \texttt{1110} \\
%% \hline
%% \hline
%% bits per sample & value \\
%% \hline
%% 8 & \texttt{00} \\
%% 16 & \texttt{01} \\
%% 24 & \texttt{10} \\
%% 32 & \texttt{11} \\
%% \hline
%% \end{tabular}
%% \end{wrapfigure}
%% The remaining fields are as follows:
%% \begin{description}
%% \item[Block Size] 24 + byte length of sub blocks
%% \item[Version] \texttt{0x407}
%% \item[Track Number] \texttt{0}
%% \item[Index Number] \texttt{0}
%% \item[Block Index] total PCM frames written thus far
%% \item[Block Samples] total PCM frames of block
%% \item[Hybrid Mode] \texttt{0}
%% \item[Channel Decorrelation] \texttt{1} if stereo, \texttt{0} if mono
%% \item[Hybrid Noise Shaping] \texttt{0}
%% \item[Floating Point Data] \texttt{0}
%% \item[Hybrid Controls Bitrate] \texttt{0}
%% \item[Hybrid Noise Balanced] \texttt{0}
%% \item[Left Shift Data] \texttt{0}
%% \item[Maximum Magnitude] maximum sample size, in bits
%% \item[Use IIR] \texttt{0}
%% \end{description}
%% Note that the \VAR{Block Size} and \VAR{Total Samples}
%% fields can't be known in advance;
%% all the block's sub-blocks must be generated before we'll know
%% the former, and the entire file must be written before we'll know
%% the latter.

%% \subsection{Decorrelation Terms/Deltas}

%% These are typically defined by the number of decorrelation
%% passes to use:
%% \begin{table}[h]
%% \begin{tabular}{| r | r | r | r | r | r || r | r | r | r | r | r |}
%% \hline
%% Terms & \multicolumn{5}{c||}{Decorrelation Passes} & Deltas &
%% \multicolumn{5}{c|}{Decorrelation Passes} \\
%% & 1 & 2 & 5 & 10 & 16 & & 1 & 2 & 5 & 10 & 16 \\
%% \hline
%% $\text{Term}_1$    & 18 & 17 & 3 & 4 & 2 &
%% $\text{Delta}_1$   & 2  & 2  & 2 & 2 & 2 \\
%% $\text{Term}_2$    & & 18 & 17 & 17 & 18 &
%% $\text{Delta}_2$   & & 2 & 2 & 2 & 2 \\
%% $\text{Term}_3$    & & & 2 & -1 & -1 &
%% $\text{Delta}_3$   & & & 2 & 2 & 2 \\
%% $\text{Term}_4$    & & & 18 & 5 & 8 &
%% $\text{Delta}_4$   & & & 2 & 2 & 2 \\
%% $\text{Term}_5$    & & & 18 & 3 & 6 &
%% $\text{Delta}_5$   & & & 2 & 2 & 2 \\
%% $\text{Term}_6$    & & & & 2 & 3 &
%% $\text{Delta}_6$   & & & & 2 & 2 \\
%% $\text{Term}_7$    & & & & -2 & 5 &
%% $\text{Delta}_7$   & & & & 2 & 2 \\
%% $\text{Term}_8$    & & & & 18 & 7 &
%% $\text{Delta}_8$   & & & & 2 & 2 \\
%% $\text{Term}_9$    & & & & 18 & 4 &
%% $\text{Delta}_9$   & & & & 2 & 2 \\
%% $\text{Term}_{10}$ & & & & 18 & 2 &
%% $\text{Delta}_{10}$ & & & & 2 & 2 \\
%% $\text{Term}_{11}$ & & & & & 18 &
%% $\text{Delta}_{11}$ & & & & & 2 \\
%% $\text{Term}_{12}$ & & & & & -2 &
%% $\text{Delta}_{12}$ & & & & & 2 \\
%% $\text{Term}_{13}$ & & & & & 3 &
%% $\text{Delta}_{13}$ & & & & & 2 \\
%% $\text{Term}_{14}$ & & & & & 2 &
%% $\text{Delta}_{14}$ & & & & & 2 \\
%% $\text{Term}_{15}$ & & & & & 18 &
%% $\text{Delta}_{15}$ & & & & & 2 \\
%% $\text{Term}_{16}$ & & & & & 18 &
%% $\text{Delta}_{16}$ & & & & & 2 \\
%% \hline
%% \end{tabular}
%% \end{table}
%% \par
%% \noindent
%% They are placed in a sub-block as follows:
%% \begin{figure}[h]
%% \includegraphics{figures/wavpack/decorr_terms.pdf}
%% \end{figure}
%% \par
%% \noindent
%% Since each term/delta pair is 8 bits,
%% \VAR{Actual Size 1 Less} is set when the number of terms is odd,
%% \VAR{Large Block} is always going to be 0 and
%% \VAR{Block Size} equals the number of terms, divided by 2.

%% \clearpage

%% \subsection{Decorrelation Passes}

%% Once our number of decorrelation passes is decided,
%% we must also generate decorrelation weights, decorrelation samples
%% and entropy variables sub-blocks before moving on to the residuals sub-block.
%% So where do we get those values?
%% They actually come from the \textit{previous} block.\footnote{More
%% precisely, the previous block covering the same set of channels -
%% in the case of multi-channel audio.}
%% Since encoding will modify decorrelation weights and entropy variables
%% as it progresses, the final values for $\text{Block}_i$ become
%% the initial values for $\text{Block}_{i + 1}$.
%% As for the decorrelation values, the final few decorrelated
%% samples (whose quantity depends on the decorrelation term)
%% are `wrapped' from the previous decorrelation pass
%% into our \VAR{Decorrelation Samples} sub-block as its starting point.

%% However, we can't store the previous block's final values as-is.
%% Remember that the values for decorrelation weights are
%% multiplied by $2 ^ 3$ and the values for decorrelation samples
%% and entropy variables are stored logarithmically.
%% Therefore, we must `round-trip' the previous block's output samples
%% before using them as input samples since they'll be
%% parsed the same way during decoding.
%% This process will be explained in the sub-block sections to follow.

%% For $\text{Block}_0$, we'll set our initial decorrelation weights,
%% decorrelation samples and entropy variables to 0.

%% The application of each pass requires a \VAR{Decorrelation Term},
%% a \VAR{Decorrelation Delta}, one \VAR{Decorrelation Weight} per
%% channel and one or more \VAR{Decorrelation Sample} values -
%% in addition to the set of processed input samples we're running the pass
%% over.

%% \begin{align*}
%% \intertext{Decorrelation Term = 18:}
%% \text{Temp}_i &= \lfloor((3 \times \text{Input}_{i - 1}) - \text{Input}_{i - 2}) \div 2\rfloor \\
%% \text{Output}_i &= \text{Input}_i - \lfloor((\text{Weight}_{i - 1} \times \text{Temp}_i) + 512) \div 1024\rfloor \\
%% \text{Weight}_i &=
%% \begin{cases}
%% \text{Weight}_{i - 1} & \text{if } \text{Temp}_i = 0 \text{ or } \text{Output}_i = 0 \\
%% \text{Weight}_{i - 1} + \text{Delta} & \text{if } (\text{Temp}_i \xor \text{Output}_i) \geq 0 \\
%% \text{Weight}_{i - 1} - \text{Delta} & \text{if } (\text{Temp}_i \xor \text{Output}_i) < 0
%% \end{cases}
%% \intertext{Decorrelation Term = 17:}
%% \text{Temp}_i &= (2 \times \text{Input}_{i - 1}) - \text{Input}_{i - 2} \\
%% \text{Output}_i &= \text{Input}_i - \lfloor((\text{Weight}_{i - 1} \times \text{Temp}_i) + 512) \div 1024\rfloor \\
%% \text{Weight}_i &=
%% \begin{cases}
%% \text{Weight}_{i - 1} & \text{if } \text{Temp}_i = 0 \text{ or } \text{Output}_i = 0 \\
%% \text{Weight}_{i - 1} + \text{Delta} & \text{if } (\text{Temp}_i \xor \text{Output}_i) \geq 0 \\
%% \text{Weight}_{i - 1} - \text{Delta} & \text{if } (\text{Temp}_i \xor \text{Output}_i) < 0
%% \end{cases}
%% \intertext{$1 \leq \text{Decorrelation Term} \leq 8$:}
%% \text{Output}_i &= \text{Input}_i - \lfloor((\text{Weight}_{i - 1} \times \text{Input}_{i - \text{term}}) + 512) \div 1024\rfloor \\
%% \text{Weight}_i &=
%% \begin{cases}
%% \text{Weight}_{i - 1} & \text{if } \text{Input}_{i - \text{term}} = 0 \text{ or } \text{Output}_i = 0 \\
%% \text{Weight}_{i - 1} + \text{Delta} & \text{if } (\text{Input}_{i - \text{term}} \xor \text{Output}_i) \geq 0 \\
%% \text{Weight}_{i - 1} - \text{Delta} & \text{if } (\text{Input}_{i - \text{term}} \xor \text{Output}_i) < 0
%% \end{cases}
%% \end{align*}
%% Similar to decoding, each function uses previous input samples
%% for its calculation.
%% This is where \VAR{Decorrelation Samples} are used;
%% those are our $\text{Input}_{-1}$, $\text{Input}_{-2}$, etc. which
%% are used for decorrelation but not actually output.

%% For 1 or 2 channel blocks, positive decorrelation terms are applied
%% on a per-channel basis with the weight A values being applied to
%% channel A and the weight B values being applied to channel B (if present).
%% However, the three negative correlation terms are only valid for
%% 2 channel blocks:
%% \begin{align*}
%% \intertext{Decorrelation Term = -1:}
%% \text{Temp A}_i &= \text{Input B}_{i - 1} \\
%% \text{Temp B}_i &= \text{Input A}_i \\
%% \text{Output A}_i &= \text{Input A}_i - \lfloor((\text{Weight A}_{i - 1} \times \text{Temp A}_i) + 512) \div 1024\rfloor \\
%% \text{Weight A}_i &=
%% \begin{cases}
%% \text{Weight A}_{i - 1} & \text{if } \text{Temp A}_i \text{ or } \text{Output A}_i = 0 \\
%% \text{Weight A}_{i - 1} + \text{Delta} & \text{if } (\text{Temp A}_i \xor \text{Output A}_i) \geq 0 \\
%% \text{ to a maximum of 1024} \\
%% \text{Weight A}_{i - 1} - \text{Delta} & \text{if } (\text{Temp A}_i \xor \text{Output A}_i) < 0 \\
%% \text{ to a minimum of -1024}
%% \end{cases} \\
%% \text{Output B}_i &= \text{Input B}_i - \lfloor((\text{Weight B}_{i - 1} \times \text{Temp B}_i) + 512) \div 1024\rfloor \\
%% \text{Weight B}_i &=
%% \begin{cases}
%% \text{Weight B}_{i - 1} & \text{if } \text{Temp B}_i \text{ or } \text{Output B}_i = 0 \\
%% \text{Weight B}_{i - 1} + \text{Delta} & \text{if } (\text{Temp B}_i \xor \text{Output B}_i) \geq 0 \\
%% \text{ to a maximum of 1024} \\
%% \text{Weight B}_{i - 1} - \text{Delta} & \text{if } (\text{Temp B}_i \xor \text{Output B}_i) < 0 \\
%% \text{ to a minimum of -1024}
%% \end{cases}
%% \intertext{Decorrelation Term = -2:}
%% \text{Temp A}_i &= \text{Input B}_i \\
%% \text{Temp B}_i &= \text{Input A}_{i - 1} \\
%% \text{Output A}_i &= \text{Input A}_i - \lfloor((\text{Weight A}_{i - 1} \times \text{Temp A}_i) + 512) \div 1024\rfloor \\
%% \text{Weight A}_i &=
%% \begin{cases}
%% \text{Weight A}_{i - 1} & \text{if } \text{Temp A}_i \text{ or } \text{Output A}_i = 0 \\
%% \text{Weight A}_{i - 1} + \text{Delta} & \text{if } (\text{Temp A}_i \xor \text{Output A}_i) \geq 0 \\
%% \text{ to a maximum of 1024} \\
%% \text{Weight A}_{i - 1} - \text{Delta} & \text{if } (\text{Temp A}_i \xor \text{Output A}_i) < 0 \\
%% \text{ to a minimum of -1024}
%% \end{cases} \\
%% \text{Output B}_i &= \text{Input B}_i - \lfloor((\text{Weight B}_{i - 1} \times \text{Temp B}_i) + 512) \div 1024\rfloor \\
%% \text{Weight B}_i &=
%% \begin{cases}
%% \text{Weight B}_{i - 1} & \text{if } \text{Temp B}_i \text{ or } \text{Output B}_i = 0 \\
%% \text{Weight B}_{i - 1} + \text{Delta} & \text{if } (\text{Temp B}_i \xor \text{Output B}_i) \geq 0 \\
%% \text{ to a maximum of 1024} \\
%% \text{Weight B}_{i - 1} - \text{Delta} & \text{if } (\text{Temp B}_i \xor \text{Output B}_i) < 0 \\
%% \text{ to a minimum of -1024}
%% \end{cases}
%% \intertext{Decorrelation Term = -3:}
%% \text{Temp A}_i &= \text{Input B}_{i - 1} \\
%% \text{Temp B}_i &= \text{Input A}_{i - 1} \\
%% \text{Output A}_i &= \text{Input A}_i - \lfloor((\text{Weight A}_{i - 1} \times \text{Temp A}_i) + 512) \div 1024\rfloor \\
%% \text{Weight A}_i &=
%% \begin{cases}
%% \text{Weight A}_{i - 1} & \text{if } \text{Temp A}_i \text{ or } \text{Output A}_i = 0 \\
%% \text{Weight A}_{i - 1} + \text{Delta} & \text{if } (\text{Temp A}_i \xor \text{Output A}_i) \geq 0 \\
%% \text{ to a maximum of 1024} \\
%% \text{Weight A}_{i - 1} - \text{Delta} & \text{if } (\text{Temp A}_i \xor \text{Output A}_i) < 0 \\
%% \text{ to a minimum of -1024}
%% \end{cases} \\
%% \text{Output B}_i &= \text{Input B}_i - \lfloor((\text{Weight B}_{i - 1} \times \text{Temp B}_i) + 512) \div 1024\rfloor \\
%% \text{Weight B}_i &=
%% \begin{cases}
%% \text{Weight B}_{i - 1} & \text{if } \text{Temp B}_i \text{ or } \text{Output B}_i = 0 \\
%% \text{Weight B}_{i - 1} + \text{Delta} & \text{if } (\text{Temp B}_i \xor \text{Output B}_i) \geq 0 \\
%% \text{ to a maximum of 1024} \\
%% \text{Weight B}_{i - 1} - \text{Delta} & \text{if } (\text{Temp B}_i \xor \text{Output B}_i) < 0 \\
%% \text{ to a minimum of -1024}
%% \end{cases}
%% \end{align*}

%% \clearpage

%% \subsection{Decorrelation Weights}

%% Once the decorrelation passes for $\text{Block}_0$ have been completed
%% (with its initial decorrelation weight values of 0),
%% we should store its final updated weight values to be used as the initial
%% decorrelation weights for $\text{Block}_1$, as so on through
%% the rest of the file.

%% There is one decorrelation weight value per decorrelation pass, per channel.
%% Each has a minimum value of -1024 and a maximum value of 1024.
%% Converting their values to 8 bits requires the following formula:
%% \begin{equation*}
%% \text{value} = \begin{cases}
%% \lfloor(\text{Weight} - \left\lfloor\frac{\text{Weight} + 2 ^ 6}{7}\right\rfloor + 4) \div 2 ^ 3\rfloor & \text{ if Weight} > 0 \\
%% 0 & \text{ if Weight} = 0 \\
%% \lfloor(\text{Weight} + 4) \div 2 ^ 3\rfloor & \text{ if Weight} < 0
%% \end{cases}
%% \end{equation*}
%% \par
%% \noindent
%% Weights are placed in a sub-block in reverse order as follows:

%% \begin{figure}[h]
%% \includegraphics{figures/wavpack/decorr_weights.pdf}
%% \end{figure}

%% Since each decorrelation weight value is stored in 8 bits,
%% \VAR{Actual Size 1 Less} is set if the total number of weights is odd,
%% \VAR{Large Block} is always going to be 0
%% and \VAR{Block Size} is the total number of weights divided by 2.

%% After the initial weights for $\text{Block}_i$ have been stored,
%% the `round-trip' formula to retrieve those weight values
%% for $\text{Block}_i$'s decorrelation passes is as follows:

%% \begin{equation*}
%% \text{Decorrelation Weight} =
%% \begin{cases}
%% \text{value} \times 2 ^ 3 + \left\lfloor\frac{\text{value} \times 2 ^ 3 + 2 ^ 6}{2 ^ 7}\right\rfloor & \text{if value} > 0 \\
%% 0 & \text{if value} = 0 \\
%% \text{value} \times 2 ^ 3 & \text{if value} < 0
%% \end{cases}
%% \end{equation*}

%% \clearpage

%% \subsection{Decorrelation Samples}
%% \label{wavpack_encode_decorr_samples}
%% We apply the following formulas to convert our 32-bit, signed
%% decorrelation values to 16-bit signed sub-block values:
%% \begin{align*}
%% asample &= |sample| + \left\lfloor\frac{|sample|}{2^9}\right\rfloor \\
%% bitcount &= \text{count\_bits}(asample) \\
%% \text{Value} &=
%% \begin{cases}
%% (bitcount \times 2^8) + \text{wv\_log2}((asample \times 2^{9 - bitcount}) \bmod{256}) \\
%% \text{ if } 0 \leq asample < 256 \text{ and } sample \geq 0 \\
%% (bitcount \times 2^8) + \text{wv\_log2}(\left\lfloor asample \div 2 ^ {bitcount - 9} \right\rfloor \bmod{256}) \\
%% \text{ if } 256 \leq asample \text{ and } sample \geq 0 \\
%% -((bitcount \times 2^8) + \text{wv\_log2}((asample \times 2^{9 - bitcount}) \bmod{256})) \\
%% \text{ if } 0 \leq asample < 256 \text{ and } sample < 0 \\
%% -((bitcount \times 2^8) + \text{wv\_log2}(\left\lfloor asample \div 2 ^ {bitcount - 9} \right\rfloor \bmod{256})) \\
%% \text{ if } 256 \leq asample \text{ and } sample < 0
%% \end{cases}
%% \intertext{where \VAR{count\_bits} is defined as follows:}
%% \label{wavpackcountbits}
%% \text{count\_bits(x)} &=
%% \begin{cases}
%% 0 & \text{if } x = 0 \\
%% 1 + \text{count\_bits(}\lfloor x \div 2 \rfloor\text{)} & \text{if } x \neq 0
%% \end{cases}
%% \end{align*}
%% \par
%% \noindent
%% and \VAR{wv\_log2} is defined from the following base-16 table:

%% \par
%% \noindent
%% For example, given a sample value of 28:
%% \begin{align*}
%% asample &= |28| + \left\lfloor\frac{|28|}{2^9}\right\rfloor = 28 + 0 = \textbf{28} \\
%% bitcount &= \textbf{5} \\
%% value &= (5 \times 2^8) + \text{wv\_log2}((28 \times 2^{9 - 5}) \bmod{256}) \\
%% &= 1280 + \text{wv\_log2}(448 \bmod{256}) \\
%% &= 1280 + \text{wv\_log2}(192) = \textbf{1487}
%% \end{align*}

%% \clearpage

%% These samples are then placed in a sub-block as follows:
%% \begin{figure}[h]
%% \includegraphics{figures/wavpack/decorr_samples.pdf}
%% \end{figure}
%% \par
%% \noindent
%% where \VAR{Actual Size 1 Less} and \VAR{Large Block} are 0,
%% and \VAR{Block Size} is the total number of decorrelation samples.

%% Writing the values themselves requires traversing the
%% decorrelation samples lists in \textit{reverse} order,
%% from $i = \text{\VAR{Decorrelation Passes}} - 1$ to 0.

%% \subsubsection{For Stereo Block}
%% \begin{itemize}
%% \item If $17 \leq \text{Decorrelation Term}_i \leq 18$
%% \begin{enumerate}
%% \item Write $\text{Sample A}_{i~1}$
%% \item Write $\text{Sample A}_{i~0}$
%% \item Write $\text{Sample B}_{i~1}$
%% \item Write $\text{Sample B}_{i~0}$
%% \end{enumerate}
%% \item If $1 \leq \text{Decorrelation Term}_i \leq 8$
%% \begin{enumerate}
%% \item For $j = 0$ to $\text{Decorrelation Term}_i - 1$
%% \begin{enumerate}
%% \item Write $\text{Sample A}_{i~j}$
%% \item Write $\text{Sample B}_{i~j}$
%% \end{enumerate}
%% \end{enumerate}
%% \item If $-3 \leq \text{Decorrelation Term}_i \leq -1$
%% \begin{enumerate}
%% \item Write $\text{Sample B}_{i~0}$
%% \item Write $\text{Sample A}_{i~0}$
%% \end{enumerate}
%% \end{itemize}

%% \subsubsection{For Mono Block}
%% \begin{itemize}
%% \item If $17 \leq \text{Decorrelation Term}_i \leq 18$
%% \begin{enumerate}
%% \item Write $\text{Sample A}_{i~1}$
%% \item Write $\text{Sample A}_{i~0}$
%% \end{enumerate}
%% \item If $1 \leq \text{Decorrelation Term}_i \leq 8$
%% \begin{enumerate}
%% \item For $j = 0$ to $\text{Decorrelation Term}_i - 1$
%% \begin{enumerate}
%% \item Write $\text{Sample A}_{i~j}$
%% \end{enumerate}
%% \end{enumerate}
%% \end{itemize}

%% Round-tripping these values back to decorrelation samples
%% for the next block requires applying the same formula
%% as decoding:
%% \begin{equation*}
%% \text{Sample} =
%% \begin{cases}
%% \lfloor \text{wv\_exp2}(value \bmod{256}) \div 2 ^ {9 - \lfloor value \div 2 ^ 8 \rfloor} \rfloor & \text{if } 0 \leq value \leq 2304 \\
%% \text{wv\_exp2}(value \bmod{256}) \times 2 ^ {\lfloor value \div 2 ^ 8 \rfloor - 9} & \text{if } 2304 < value \leq 32767 \\
%% -\lfloor \text{wv\_exp2}(-value \bmod{256}) \div 2 ^ {9 - \lfloor -value \div 2 ^ 8 \rfloor} \rfloor & \text{if } -2304 \leq value < 0 \\
%% -(\text{wv\_exp2}(-value \bmod{256}) \times 2 ^ {\lfloor -value \div 2 ^ 8 \rfloor - 9}) & \text{if } -32768 \leq value < -2304
%% \end{cases}
%% \end{equation*}
%% \par
%% \noindent
%% where \VAR{wv\_exp2} is defined from the following base-16 table:
%% \par
%% \noindent
%% {\relsize{-3}\ttfamily
%% \begin{tabular}{| c | c | c | c | c | c | c | c | c | c | c | c | c | c | c | c | c |}
%% \hline
%% & 0x?0 & 0x?1 & 0x?2 & 0x?3 & 0x?4 & 0x?5 & 0x?6 & 0x?7 & 0x?8 & 0x?9 & 0x?A & 0x?B & 0x?C & 0x?D & 0x?E & 0x?F \\
%% \hline
%% 0x0? & 100 & 101 & 101 & 102 & 103 & 103 & 104 & 105 & 106 & 106 & 107 & 108 & 108 & 109 & 10A & 10B \\
%% 0x1? & 10B & 10C & 10D & 10E & 10E & 10F & 110 & 110 & 111 & 112 & 113 & 113 & 114 & 115 & 116 & 116 \\
%% 0x2? & 117 & 118 & 119 & 119 & 11A & 11B & 11C & 11D & 11D & 11E & 11F & 120 & 120 & 121 & 122 & 123 \\
%% 0x3? & 124 & 124 & 125 & 126 & 127 & 128 & 128 & 129 & 12A & 12B & 12C & 12C & 12D & 12E & 12F & 130 \\
%% 0x4? & 130 & 131 & 132 & 133 & 134 & 135 & 135 & 136 & 137 & 138 & 139 & 13A & 13A & 13B & 13C & 13D \\
%% 0x5? & 13E & 13F & 140 & 141 & 141 & 142 & 143 & 144 & 145 & 146 & 147 & 148 & 148 & 149 & 14A & 14B \\
%% 0x6? & 14C & 14D & 14E & 14F & 150 & 151 & 151 & 152 & 153 & 154 & 155 & 156 & 157 & 158 & 159 & 15A \\
%% 0x7? & 15B & 15C & 15D & 15E & 15E & 15F & 160 & 161 & 162 & 163 & 164 & 165 & 166 & 167 & 168 & 169 \\
%% 0x8? & 16A & 16B & 16C & 16D & 16E & 16F & 170 & 171 & 172 & 173 & 174 & 175 & 176 & 177 & 178 & 179 \\
%% 0x9? & 17A & 17B & 17C & 17D & 17E & 17F & 180 & 181 & 182 & 183 & 184 & 185 & 187 & 188 & 189 & 18A \\
%% 0xA? & 18B & 18C & 18D & 18E & 18F & 190 & 191 & 192 & 193 & 195 & 196 & 197 & 198 & 199 & 19A & 19B \\
%% 0xB? & 19C & 19D & 19F & 1A0 & 1A1 & 1A2 & 1A3 & 1A4 & 1A5 & 1A6 & 1A8 & 1A9 & 1AA & 1AB & 1AC & 1AD \\
%% 0xC? & 1AF & 1B0 & 1B1 & 1B2 & 1B3 & 1B4 & 1B6 & 1B7 & 1B8 & 1B9 & 1BA & 1BC & 1BD & 1BE & 1BF & 1C0 \\
%% 0xD? & 1C2 & 1C3 & 1C4 & 1C5 & 1C6 & 1C8 & 1C9 & 1CA & 1CB & 1CD & 1CE & 1CF & 1D0 & 1D2 & 1D3 & 1D4 \\
%% 0xE? & 1D6 & 1D7 & 1D8 & 1D9 & 1DB & 1DC & 1DD & 1DE & 1E0 & 1E1 & 1E2 & 1E4 & 1E5 & 1E6 & 1E8 & 1E9 \\
%% 0xF? & 1EA & 1EC & 1ED & 1EE & 1F0 & 1F1 & 1F2 & 1F4 & 1F5 & 1F6 & 1F8 & 1F9 & 1FA & 1FC & 1FD & 1FF \\
%% \hline
%% \end{tabular}
%% }

%% \subsection{the Entropy Variables Sub-Block}

%% \begin{figure}[h]
%% \includegraphics{figures/wavpack/entropy_vars.pdf}
%% \end{figure}
%% \par
%% \noindent
%% \VAR{Actual Size 1 Less} and \VAR{Large Block} are 0.
%% \VAR{Block Size} is 3 for mono blocks and 6 for stereo blocks.
%% The samples themselves are converted and round-tripped
%% the same way that \VAR{Decorrelation Sample} values are,
%% as explained on pages \pageref{wavpack_encode_decorr_samples}
%% and \pageref{wavpack_decorr_samples}.

%% \clearpage

%% \subsection{the Bitstream Sub-Block}

%% Given a set of residual values and one set of 3 entropy variables per
%% channel, the final encoding step for a WavPack block is generating
%% the bitstream sub-block.
%% Since the subframe header requires a size, we must either
%% write it in advance with a size of 0 and rewrite it once
%% the sub-block is finished, or first write the residuals to temporary space
%% before writing the sub-block header.

%% \begin{figure}[h]
%% \includegraphics{figures/wavpack/bitstream.pdf}
%% \end{figure}

%% As with decoding, writing each residual value is a multi-stage
%% process which involves calculating a unary value,
%% referencing and updating the channel's set of entropy variables, and
%% calculating a fixed collection of bits.

%% However, this procedure is complicated by the \VAR{holding\_one}
%% and \VAR{holding\_zero} boolean values.
%% As you'll recall, when \VAR{holding\_zero} is false, the
%% decoder skips reading a unary value entirely.
%% This means that before we can output $\text{Residual}_{i - 1}$'s values,
%% we must determine the values for $\text{Residual}_i$
%% so that the \VAR{holding\_one} and \VAR{holding\_zero} boolean
%% values can be set properly.
%% Note that $\text{holding\_one}_{-1}$ and $\text{holding\_zero}_{-1}$
%% are both \texttt{false}.

%% In practice, we'll first need to handle the special case of many
%% zero residuals in a row as discussed on page \pageref{wavpack_zero_residuals}.
%% But for clarity, it's best to understand the general case first.

%% \subsubsection{Calculate Unsigned Value and Sign Bit}
%% %% We start by converting our signed residual to an unsigned value and sign bit:
%% \begin{align*}
%% \text{value}_i &=
%% \begin{cases}
%% \text{Residual}_i & \text{if } \text{Residual}_i \geq 0 \\
%% -\text{Residual}_i - 1 & \text{if } \text{Residual}_i < 0
%% \end{cases} \\
%% \text{sign}_i &=
%% \begin{cases}
%% 0 & \text{if } \text{Residual}_i \geq 0 \\
%% 1 & \text{if } \text{Residual}_i < 0
%% \end{cases}
%% \end{align*}
%% \subsubsection{Calculate Unary Value, High, Low and Next Medians}
%% %% We then convert our current channel's entropy variables to
%% %% median values:
%% \begin{align*}
%% \text{Median}_{i~0} &= \lfloor\text{Entropy}_{i~0}\div2 ^ 4\rfloor + 1 \\
%% \text{Median}_{i~1} &= \lfloor\text{Entropy}_{i~1}\div2 ^ 4\rfloor + 1 \\
%% \text{Median}_{i~2} &= \lfloor\text{Entropy}_{i~2}\div2 ^ 4\rfloor + 1 \\
%% \end{align*}
%% %% which allows us to calculate the residual's \VAR{low}, \VAR{high}
%% %% and unary values - along with the next residual's entropy values -
%% %% by determining which set of medians our unsigned value falls between:
%% \begin{align*}
%% \intertext{If $\text{value}_i < \text{Median}_{i~0}$:}
%% \text{unary}_i &= 0 \\
%% \text{low}_i &= 0 \\
%% \text{high}_i &= \text{Median}_{i~0} - 1 \\
%% \text{Entropy}_{(i + 1)~0} &= \text{Entropy}_{i~0} - \left\lfloor\frac{\text{Entropy}_{i~0} + 126}{128}\right\rfloor \times 2 \\
%% \text{Entropy}_{(i + 1)~1} &= \text{Entropy}_{i~1} \\
%% \text{Entropy}_{(i + 1)~2} &= \text{Entropy}_{i~2}
%% \intertext{If $(\text{value}_i - \text{Median}_{i~0}) < \text{Median}_{i~1}$:}
%% \text{unary}_i &= 1 \\
%% \text{low}_i &= \text{Median}_{i~0} \\
%% \text{high}_i &= \text{Median}_{i~0} + \text{Median}_{i~1} - 1 \\
%% \text{Entropy}_{(i + 1)~0} &= \text{Entropy}_{i~0} + \left\lfloor\frac{\text{Entropy}_{i~0} + 128}{128}\right\rfloor \times 5 \\
%% \text{Entropy}_{(i + 1)~1} &= \text{Entropy}_{i~1} - \left\lfloor\frac{\text{Entropy}_{i~1} + 62}{64}\right\rfloor \times 2\\
%% \text{Entropy}_{(i + 1)~2} &= \text{Entropy}_{i~2} \\
%% \intertext{If $(\text{value}_i - (\text{Median}_{i~0} + \text{Median}_{i~1})) < \text{Median}_{i~2}$:}
%% \text{unary}_i &= 2 \\
%% \text{low}_i &= \text{Median}_{i~0} + \text{Median}_{i~1} \\
%% \text{high}_i &= \text{Median}_{i~0} + \text{Median}_{i~1} + \text{Median}_{i~2} - 1 \\
%% \text{Entropy}_{(i + 1)~0} &= \text{Entropy}_{i~0} + \left\lfloor\frac{\text{Entropy}_{i~0} + 128}{128}\right\rfloor \times 5 \\
%% \text{Entropy}_{(i + 1)~1} &= \text{Entropy}_{i~1} + \left\lfloor\frac{\text{Entropy}_{i~1} + 64}{64}\right\rfloor \times 5 \\
%% \text{Entropy}_{(i + 1)~2} &= \text{Entropy}_{i~2} - \left\lfloor\frac{\text{Entropy}_{i~2} + 30}{32}\right\rfloor \times 2 \\
%% \intertext{Otherwise:}
%% \text{unary}_i &= \left\lfloor\frac{\text{value}_i - (\text{Median}_{i~0} + \text{Median}_{i~1})}{\text{Median}_{i~2}}\right\rfloor + 2 \\
%% \text{low}_i &= \text{Median}_{i~0} + \text{Median}_{i~1} + ((\text{unary}_i - 2) \times \text{Median}_{i~2}) \\
%% \text{high}_i &= \text{low}_i + \text{Median}_{i~2} - 1 \\
%% \text{Entropy}_{(i + 1)~0} &= \text{Entropy}_{i~0} + \left\lfloor\frac{\text{Entropy}_{i~0} + 128}{128}\right\rfloor \times 5 \\
%% \text{Entropy}_{(i + 1)~1} &= \text{Entropy}_{i~1} + \left\lfloor\frac{\text{Entropy}_{i~1} + 64}{64}\right\rfloor \times 5 \\
%% \text{Entropy}_{(i + 1)~2} &= \text{Entropy}_{i~2} + \left\lfloor\frac{\text{Entropy}_{i~2} + 32}{32}\right\rfloor \times 5
%% \end{align*}

%% \subsubsection{Calculate Fixed Size, Fixed Value and Optional Extra Bit}
%% \begin{align*}
%% \intertext{If $\text{low}_i = \text{high}_i$:}
%% \text{fixed size}_i &= 0 \\
%% \text{has extra}_i &= \texttt{false}
%% \intertext{Otherwise:}
%% \text{extras}_i &= 2 ^ {\text{count\_bits}(\text{high}_i - \text{low}_i)} - (\text{high}_i - \text{low}_i) - 1 \\
%% \intertext{If $\text{low}_i \neq \text{high}_i$ and $(\text{value}_i - \text{low}_i) < \text{extras}_i$:}
%% \text{fixed size}_i &= \text{value}_i - \text{low}_i \\
%% \text{fixed value}_i &= \text{count\_bits}(\text{high}_i - \text{low}_i) - 1 \\
%% \text{has extra}_i &= \texttt{false} \\
%% \intertext{If $\text{low}_i \neq \text{high}_i$ and $(\text{value}_i - \text{low}_i) \geq \text{extras}_i$:}
%% \text{fixed size}_i &= \lfloor(\text{value}_i - \text{low}_i + \text{extras}_i) \div 2\rfloor \\
%% \text{fixed value}_i &= \text{count\_bits}(\text{high}_i - \text{low}_i) - 1 \\
%% \text{has extra}_i &= \texttt{true} \\
%% \text{extra bit}_i &= (\text{value}_i - \text{low}_i + \text{extras}_i) \bmod{2}
%% \end{align*}

%% \subsubsection{Update Previous Residual Based On Current Residual}
%% \begin{align*}
%% \intertext{If $\text{unary}_{i - 1} > 0$ and $\text{unary}_i > 0$:}
%% \text{unary}_{i - 1} &\gets
%% \begin{cases}
%% (\text{unary}_{i - 1} \times 2) + 1 & \text{if $\text{holding one}_{i - 1} = \texttt{false}$} \\
%% (\text{unary}_{i - 1} \times 2) - 1 & \text{if $\text{holding one}_{i - 1} = \texttt{true}$}
%% \end{cases} \\
%% \text{holding zero}_i &= \texttt{false} \\
%% \text{holding one}_i &= \texttt{true}
%% \intertext{If $\text{unary}_{i - 1} = 0$ and $\text{unary}_i > 0$:}
%% \text{unary}_{i - 1} &\gets
%% \begin{cases}
%% 1 & \text{if $\text{holding zero}_{i - 1} = \texttt{false}$} \\
%% \textit{not output} & \text{if $\text{holding zero}_{i - 1} = \texttt{true}$}
%% \end{cases} \\
%% \text{holding zero}_i &= \texttt{false} \\
%% \text{holding one}_i &= \texttt{not}\text{ holding zero}_{i - 1} \\
%% \intertext{If $\text{unary}_{i - 1} > 0$ and $\text{unary}_i = 0$:}
%% \text{unary}_{i - 1} &\gets
%% \begin{cases}
%% \text{unary}_{i - 1} \times 2 & \text{if $\text{holding one}_{i - 1} = \texttt{false}$} \\
%% (\text{unary}_{i - 1} - 1) \times 2 & \text{if $\text{holding one}_{i - 1} = \texttt{true}$}
%% \end{cases} \\
%% \text{holding zero}_i &= \texttt{true} \\
%% \text{holding one}_i &= \texttt{false} \\
%% \intertext{If $\text{unary}_{i - 1} = 0$ and $\text{unary}_i = 0$:}
%% \text{unary}_{i - 1} &\gets
%% \begin{cases}
%% 0 & \text{if $\text{holding zero}_{i - 1} = \texttt{false}$} \\
%% \textit{not output} & \text{if $\text{holding zero}_{i - 1} = \texttt{true}$}
%% \end{cases} \\
%% \text{holding zero}_i &= \texttt{not}\text{ holding zero}_{i - 1} \\
%% \text{holding one}_i &= \texttt{false}
%% \end{align*}

%% \subsubsection{Output Previous Residual}

%% Once the previous residual's unary value has been determined,
%% its fields can now be output.
%% \begin{itemize}
%% \item If unary output, write $\text{unary}_{i - 1}$ number of \texttt{1} bits followed by a \texttt{0} bit (if $\text{unary}_{i - 1} < 16$)
%% \item Write \VAR{$\text{fixed size}_{i - 1}$} number of bits with the value \VAR{$\text{fixed value}_{i - 1}$}
%% \item If \VAR{$\text{has extra}_{i - 1}$}, write a single bit with the value \VAR{$\text{extra bit}_{i - 1}$}
%% \item Write a single bit with the value \VAR{$\text{sign}_{i - 1}$}
%% \end{itemize}
%% Note that if $\text{unary}_{i - 1} \geq 16$, we write an escape code instead.
%% \begin{itemize}
%% \item If $\text{unary}_{i - 1} = 16$, write 18 bits with the value \texttt{0xFFFF} (unary 16 plus unary 0)
%% \item If $\text{unary}_{i - 1} = 17$, write 19 bits with the value \texttt{0x2FFFF} (unary 16 plus unary 1)
%% \item If $\text{unary}_{i - 1} \geq 18$, write 17 bits with the value \texttt{0xFFFF} (unary 16)
%% \begin{itemize}
%% \item Write $\text{count\_bits}(\text{unary}_{i - 1} - 16)$ number of
%% \texttt{1} bits followed by a \texttt{0} bit
%% \item Write $\text{count\_bits}(\text{unary}_{i - 1} - 16) - 1$ number of bits
%% with the value
%% \linebreak
%% $((\text{unary}_{i - 1} - 16) \bmod{2 ^ {\text{count\_bits}(\text{unary}_{i - 1} - 16) - 1}})$
%% \end{itemize}
%% \end{itemize}

%% \subsubsection{Handle Groups Of Zero Residuals}
%% \label{wavpack_zero_residuals}
%% \begin{wrapfigure}[12]{r}{1.5in}
%% \includegraphics{figures/wavpack/write_zeroes.pdf}
%% \end{wrapfigure}
%% This is necessary when $\text{Entropy A}_{i~0} < 2$ and,
%% for 2 channel blocks, $\text{Entropy B}_{i~0} < 2$,
%% $\text{holding\_zero}_i = \texttt{false}$ and
%% $\text{holding\_one}_i = \texttt{false}$.
%% In that event, whether the current $\text{Residual}_i$ is 0 or not,
%% we must generate a zeroes block after outputting $\text{Residual}_{i - 1}$
%% but before outputting $\text{Residual}_i$.

%% Once the number of zeroes has been determined, their output is
%% quite similar to an escaped unary value.
%% \begin{itemize}
%% \item If $\text{zeroes} = 0$, write a single \texttt{0} bit.
%% \item If $\text{zeroes} > 0$
%% \begin{itemize}
%% \item Write $\text{count\_bits}(\text{zeroes})$ number of \texttt{1} bits
%% followed by a \texttt{0} bit
%% \item Write $\text{count\_bits}(\text{zeroes}) - 1$ number of bits
%% with the value $(\text{zeroes} \bmod{2 ^ {\text{count\_bits}(\text{zeroes}) - 1}})$
%% \end{itemize}
%% \end{itemize}

%% \subsection{Extended Integers}

%% In the rare case that a block has `wasted bits',
%% as explained on page \pageref{wavpack_encode_extended_integers},
%% we generate the following sub-block to store our
%% \VAR{Zero Bits} value:
%% \begin{figure}[h]
%% \includegraphics{figures/wavpack/extended_integers.pdf}
%% \end{figure}

%% \clearpage

%% \subsection{RIFF WAVE Header}

%% WavPack expects to find a RIFF WAVE header sub-block in the
%% first block within the file.
%% This sub-block is laid out as follows:

%% \begin{figure}[h]
%% \includegraphics{figures/wavpack/wave_header.pdf}
%% \end{figure}

%% The RIFF WAVE header is everything from the start of a RIFF WAVE file
%% to the end of its \texttt{data} chunk's header.
%% For non WAVEFORMATEXTENSIBLE files, this is typically the first 36 bytes.
%% For WAVEFORMATEXTENSIBLE files, this is typically the first 60 bytes.

%% \subsection{the Footer Block}

%% Though not required, WavPack files often contain a trailing
%% block after the audio has been exhausted.
%% This block contains only an MD5 sum sub-block and optional
%% RIFF WAVE footer sub-block for wave files with additional chunks of
%% data after the \texttt{data} chunk.

%% \subsubsection{RIFF WAVE Footer}
%% \begin{figure}[h]
%% \includegraphics{figures/wavpack/wave_footer.pdf}
%% \end{figure}

%% \subsubsection{MD5 Sum}
%% \begin{figure}[h]
%% \includegraphics{figures/wavpack/md5sum.pdf}
%% \end{figure}
%% \par
%% \noindent
%% The MD5 is the hash of all the samples over the entire file.
%% It is calculated by running the
%% hashing algorithm\footnote{As described by RFC1321} over
%% the raw input samples in little-endian format
%% and signed if their bits-per-sample are greater than 8.
