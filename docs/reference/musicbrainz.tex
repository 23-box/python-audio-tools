\chapter{MusicBrainz}
MusicBrainz is another CD metadata retrieval service similar to FreeDB,
but designed to eliminate many of FreeDB's limitations.
For example, MusicBrainz has a more robust disc ID calculation mechanism,
it has an easier way to disambiguate database entries in case of collision,
and its XML metadata format is less prone to errors
(track names with `\texttt{/}' characters are a particular problem for
FreeDB).

However, because it is a newer service, it's common to find disc entries
that are on FreeDB but do not yet have a MusicBrainz entry - whereas
the converse is much more rare.
Therefore, a metadata looking program would be wise to check both
services if possible.

\section{Searching releases}
This is analagous to FreeDB's search routine in which one calculates
a CD's disc ID, submits it to MusicBrainz via an HTTP get query
and receives information such as album name, artist name, track names
and so forth as an XML file.

\pagebreak

\subsection{the disc ID}
Calculating a MusicBrainz disc ID requires knowing a CD's first track number,
last track number, track offsets (in CD frames) and lead out track offset
(also in CD frames).
For example, given the following CD:
\begin{table}[h]
\begin{tabular}{|c||r|r|r||r|r|r|}
\hline
Track & \multicolumn{3}{c||}{Length} & \multicolumn{3}{c|}{Offset} \\
Number & in M:SS & in seconds & in frames & in M:SS & in seconds & in frames \\
\hline
1 & 3:37 & 217 & 16340 & 0:02 & 2 & 150 \\
2 & 3:23 & 203 & 15294 & 3:39 & 219 & 16490 \\
3 & 3:37 & 217 & 16340 & 7:03 & 423 & 31784 \\
4 & 3:20 & 200 & 15045 & 10:41 & 641 & 48124 \\
\hline
\end{tabular}
\end{table}
\par
\noindent
The first track number is 1, the last track number is 4, the track offsets
are 150, 16490, 31784 and 48124, and the lead out track offset is
63169 (track 4's offset 48124 plus its length of 15045).

These numbers are then converted to 0-padded, big-endian hexadecimal strings
with the track numbers using 2 digits and the offsets using 8 digits.
In this example, the first track number becomes \texttt{01},
the last track number becomes \texttt{04},
the track offsets become \texttt{00000096}, \texttt{0000406A},
\texttt{00007C28} and \texttt{0000BBFC},
and the lead out track offset becomes \texttt{0000F6C1}.

These individual strings are then combined into a single 804 byte string:
\begin{figure}[h]
\includegraphics{figures/musicbrainz_discid.pdf}
\end{figure}

Excess track offsets are treated as having an offset value of 0,
or a string value of \texttt{00000000}.
Our string starts with \texttt{01040000F6C1000000960000406A00007C280000BBFC}
and is padded with an additional 760 `\texttt{0}' characters
which I'll omit for brevity.

That string is then passed through the SHA-1 hashing algorithm\footnote{This is described in RFC3174}
which results in a 20 byte hash value.
Remember to use the binary hash value, not its 40 byte ASCII hexadecimal one.

In our example, this yields the hash:
\texttt{0xDA3D930462773DD57BBE43B535AD6A457138F079}

The resulting hash value is then encoded to a 28 byte Base64\footnote{This is described in RFC3548 and RFC4648} string.
However, unlike standard Base64, MusicBrainz's disc ID replaces the
characters `\texttt{=}', `\texttt{+}' and `\texttt{/}' with `\texttt{-}', `\texttt{.}' and `\texttt{\_}' respectively to
make the value better suited to HTTP requests.
So to complete our example, the hash value becomes a disc ID of
\texttt{2j2TBGJ3PdV7vkO1Na1qRXE48Hk-}

\pagebreak

\subsection{Server query}
MusicBrainz runs as a service on HTTP port 80.
To retrieve Release information, one can make a GET request to
\texttt{/ws/1/release} using the following fields:
\begin{table}[h]
\begin{tabular}{|r|l|}
\hline
key & value \\
\hline
\texttt{type} & \texttt{xml} \\
\texttt{discid} & \texttt{<disc ID string>} \\
\hline
\end{tabular}
\end{table}
\par
\noindent
For example, to retrieve the Release data for disc ID
\texttt{2jmj7l5rSw0yVb\_vlWAYkK\_YBwk-} one sends the GET query:
\begin{Verbatim}[frame=single]
type=xml&discid=2jmj7l5rSw0yVb_vlWAYkK_YBwk-
\end{Verbatim}
Whether the Release is found in the MusicBrainz database or not,
an XML file will always be generated.

\subsection{Release XML}
\begin{wrapfigure}[21]{r}{1.94in}
\includegraphics{figures/musicbrainz_xml.pdf}
\end{wrapfigure}
All XML files returned by a MusicBrainz query consist of a
\texttt{<metadata>}
tag container.
When making a Release query, it contains a
\texttt{<release-list>}
which is itself a container for zero or more
\texttt{<release>}
tags, depending on how many Release entries match the submitted
disc ID.

The
\texttt{<release>}
tag typically contains a
\texttt{<title>}
which is the album's name,
an
\texttt{<artist>}
tag which is the album's primary artist,
a
\texttt{<release-event-list>}
tag containing information such as the album's
release date and catalog number,
and finally a
\texttt{<track-list>}
which contains all the track data.

The
\texttt{<track>}
tags are always listed in order of their appearance in the album.
Each contains a
\texttt{<title>}
which is the track's name,
a
\texttt{<duration>}
which is the track's length in milliseconds,
and optionally an
\texttt{<artist>}
tag which is information about a track-specific artist,
for instances where the track's artist differs from the album's artist.

In addition, the
\texttt{<release>}
,
\texttt{<artist>}
,
and
\texttt{<track>}
tags all contain an `id' attribute with 32 hex digits in the format
`12345678-9abc-def1-2345-6789abcdef1234'.
These uniquely identify the Release, Artist and Track information
in the MusicBrainz database and can be used for direct lookups.

