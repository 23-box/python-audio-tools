\chapter{Shorten}
Shorten is one of the earliest lossless audio compression formats.
Though superceded by FLAC and other formats, it remains interesting
from a historical perspective.

\section{Shorten data types}
Notably, almost nothing in the Shorten file format is byte-aligned.
Instead, it uses its own set of variable-length types which I'll
refer to as \texttt{unsigned}, \texttt{signed} and \texttt{long}.

\begin{wrapfigure}[16]{r}{2in}
\includegraphics{figures/shorten_unsigned.pdf}
\caption{Unsigned}
\includegraphics{figures/shorten_signed.pdf}
\caption{Signed}
\end{wrapfigure}
An \texttt{unsigned} field of a certain \VAR{size} means we first
take a unary-encoded\footnote{In this instance, unary-encoding is a simple
matter of counting the number of 0 bits before the next 1 bit.
The resulting sum is the value.}, number of high bits and combine
the resulting value with \VAR{size} number of low bits.
For example, given a \VAR{size} of 2 and the bits `\texttt{0 0 1 1 1}',
the high unary value of `\texttt{0 0 1}' combines with the low
raw value of `\texttt{1 1}' resulting in a decimal value of 11.

A \texttt{signed} field is similar, but its low value contains
one additional trailing bit for the sign value.
{\relsize{-2}
\begin{equation*}
\text{signed value} =
\begin{cases}
\text{unsigned value} & \text{if sign bit} = 0 \\
-\text{unsigned value} - 1 & \text{if sign bit} = 1
\end{cases}
\end{equation*}
}
For example, given a \VAR{size} of 3 and the bits `\texttt{0 1 1 0 1 1}',
the high unary value of `\texttt{0 1}' combines with the low
raw value of `\texttt{1 0 1}' and the sign bit `\texttt{1}'
resulting in a decimal value of -14.
Note that the sign bit is counted seperately, so we're
actually reading 4 additional bits after the unary value in this case.

Lastly, and most confusingly, a \texttt{long} field is the combination
of two seperate \texttt{unsigned} fields.
The first, of size 2, determines the size value of the second.
For example, given the bits `\texttt{1 1 1 1 1 0 1}',
the first \texttt{unsigned} field of `\texttt{1 1 1}' has the value
of 3 (unary 0 combined with a raw value of 3) - which is the size
of the next \texttt{unsigned} field.
That field, in turn, consists of the bits `\texttt{1 1 0 1}'
which is 5 (unary 0 combined with a raw value of 5).
So, the value of the entire \texttt{long} field is 5.

A Shorten file consists almost entirely of these three types
in various sizes.
Therefore, when one reads ``\texttt{unsigned(3)}'' in a Shorten field
description, it means an \texttt{unsigned} field of size 3.

\pagebreak

\section{the Shorten file stream}
\label{shorten_stream}
\begin{figure}[h]
\includegraphics{figures/shorten_stream.pdf}
\end{figure}
\begin{table}[h]
\begin{tabular}{|r|l|}
\hline
file type & format \\
\hline
0 & lossless \textmu-Law \\
1 & signed 8 bit \\
2 & unsigned 8 bit \\
3 & signed 16 bit, big-endian \\
4 & unsigned 16 bit, big-endian \\
5 & signed 16 bit, little-endian \\
6 & unsigned 16 bit, little-endian \\
7 & lossy \textmu-Law \\
8 & new \textmu-Law with zero mapping \\
9 & lossless a-Law \\
10 & lossy a-Law \\
11 & Microsoft .wav \\
12 & Apple .aiff \\
\hline
\end{tabular}
\end{table}
\par
\noindent
\VAR{Channels} is the number of channels in the audio stream.
\VAR{Block Length} is the length of each command block, in samples.
\VAR{Max LPC} is the maximum LPC value a block may have.
\VAR{Samples to Wrap} is the number of samples to be wrapped around
from the top of an output block to the bottom.
This will be explained in more detail in the decoding section.

\pagebreak

\section{Shorten Decoding}
\begin{wrapfigure}[4]{r}{2.75in}
\begin{tabular}{|r|l||r|l|}
\hline
value & command & value & command \\
\hline
0 & \texttt{DIFF0} & 5 & \texttt{BLOCKSIZE} \\
1 & \texttt{DIFF1} & 6 & \texttt{BITSHIFT} \\
2 & \texttt{DIFF2} & 7 & \texttt{QLPC} \\
3 & \texttt{DIFF3} & 8 & \texttt{ZERO} \\
4 & \texttt{QUIT} & 9 & \texttt{VERBATIM} \\
\hline
\end{tabular}
\end{wrapfigure}
Internally,
a Shorten file acts as a list of commands to be executed by a tiny
virtual machine.\footnote{Interestingly, although
Shorten's successor, FLAC, presents its input as frames and subframes,
references to a FLAC virtual machine are still present in its source code.}
Each command is a \texttt{unsigned(2)} field followed by zero or more
arguments.

\subsection{the DIFF commands}
All four \texttt{DIFF} commands are structured the same:
\begin{figure}[h]
\includegraphics{figures/shorten_diff.pdf}
\end{figure}
\par
\noindent
There are \VAR{Block Size} number of residuals per \texttt{DIFF}
(whose initial value is determined by the Shorten header)
and each one's size is determined by \VAR{Energy Size}.
The process of transforming these residuals into samples
depends on the \texttt{DIFF} command and the values of
previously decoded samples.

\begin{minipage}{\linewidth}
\renewcommand\thefootnote{\thempfootnote}
\begin{tabular}{|c| >{$}l<{$} |}
\hline
Command & \text{Calculation} \\
\hline
\texttt{DIFF0} & Sample_i = Residual_i + Coffset\footnote{See page \pageref{shorten_coffset}} \\
\texttt{DIFF1} & Sample_i = Sample_{i - 1} + Residual_i  \\
\texttt{DIFF2} & Sample_i = (2 \times Sample_{i - 1}) - Sample_{i - 2} + Residual_i \\
\texttt{DIFF3} & Sample_i = (3 \times (Sample_{i - 1} - Sample_{i - 2})) + Sample_{i - 3} + Residual_i \\
\hline
\end{tabular}
\end{minipage}
\par
\noindent
For example, given a \texttt{DIFF1} command at the stream's beginning
and the residual values 10, 1, 2, -2, 1 and -1, samples are
calculated as follows:
\begin{table}[h]
\begin{tabular}{|c|r|>{$}r<{$}|}
\hline
Index & Residual & \text{Sample} \\
\hline
-1 & (before stream) & \text{(not output) } \bf0 \\
\hline
0 & 10 & 0 + 10 = \bf10 \\
1 & 1 & 10 + 1 = \bf11 \\
2 & 2 & 11 + 2 = \bf13 \\
3 & -2 & 13 - 2 = \bf11 \\
4 & 1 & 11 + 1 = \bf12 \\
5 & -1 & 12 - 1 = \bf11 \\
\hline
\end{tabular}
\end{table}

\pagebreak

\subsection{Channels and wrapping}
The audio commands \texttt{DIFF}, \texttt{QLPC} and \texttt{ZERO} send
their samples to channels in order.
For example, a stream of \texttt{DIFF} commands in a 2 channel stereo
stream (a very typical configuration) sends $\texttt{DIFF}_1$ to
the left channel, $\texttt{DIFF}_2$ to the right channel,
$\texttt{DIFF}_3$ to left channel, $\texttt{DIFF}_4$ to the right channel
and so on.

However, recall that most of the \texttt{DIFF} commands require
previously decoded samples as part of their calculation.
What this means is that $\texttt{DIFF}_3$ takes the last
few samples from $\texttt{DIFF}_1$ in order to apply its residuals
(since both are on the left channel) and $\texttt{DIFF}_4$
takes the last few samples from $\texttt{DIFF}_2$.

This is where the header's \VAR{Samples to Wrap} field comes into play.
Its value is the number of samples to be wrapped from the top of the buffer
to its pre-zero values.
For example, if \VAR{Sample Count} is 256 and \VAR{Samples to Wrap} is 3
(another typical configuration),
$\text{Buffer}_{-1}$ takes the value of $\text{Buffer}_{255}$,
$\text{Buffer}_{-2}$ takes the value of $\text{Buffer}_{254}$, and
$\text{Buffer}_{-3}$ takes the value of $\text{Buffer}_{253}$.
However, these pre-zero starting-point values are obviously not
re-output when the buffer is finally completed and returned.

\subsection{the QUIT command}

This command takes no arguments.
It indicates the Shorten stream is finished and decoding is completed.

\subsection{the BLOCKSIZE command}

This command takes a single \texttt{long} argument
whose value is the new \VAR{Block Size}.
In effect, it modifies that variable in the Shorten virtual machine.

\subsection{the ZERO command}

This command takes no arguments.
It simply generates \VAR{Block Size} number of zero samples
into the current channel's output buffer.

\subsection{the BITSHIFT command}

This commands takes a single \texttt{unsigned(2)} value
and modifies the \VAR{bitshift} variable in the Shorten virtual machine.
This value is how many bits to left-shift all output samples
prior to returning them.

For example, imagine a scenario in which all the samples in a set of
blocks have 0 for their rightmost (least significant) bit.
Setting a bitshift of 1 allows us to ignore that bit during
calculation which, in turn, allows us to store those samples more
efficiently.

Note that bit shifting is applied \textit{after} channel wrapping
and \VAR{coffset} calculation\footnote{See page \pageref{shorten_coffset}}.

\pagebreak

\subsection{the QLPC command}

The \texttt{QLPC} command is structured as follows:

\begin{figure}[h]
\includegraphics{figures/shorten_qlpc.pdf}
\end{figure}
\par
\noindent
So, given a set of LPC coefficients and a set of residuals,
samples are calculated using the following formula:
\begin{equation}
\text{Sample}_i = \left\lfloor \frac{2 ^ 5 + \overset{Count - 1}{\underset{j = 0}{\sum}}
  \text{LPC Coefficient}_j \times \text{Sample}_{i - j - 1} } {2 ^ 5}\right\rfloor + \text{Residual}_i
\end{equation}
This simply means we're taking the sum of the calculated values from
0 to LPC Count - 1, bit-shifting that sum down and added the residual
when determining the current sample.
As with the \texttt{DIFF} commands, previously encoded samples
(possibly from previous commands) are used to calculate the current
sample.

For example, given the LPC Coefficients and previously encoded samples:
\begin{table}[h]
\begin{tabular}{>{$}r<{$} r || >{$}r<{$} r}
\text{LPC Coefficient}_0 & 21 & \text{Sample}_1 & -2 \\
\text{LPC Coefficient}_1 & 2 & \text{Sample}_2 & -3 \\
\text{LPC Coefficient}_2 & 7 & \text{Sample}_3 & -2
\end{tabular}
\end{table}
\begin{figure}[h]
\begin{tabular}{|c|r|>{$}r<{$}|}
\hline
Index & Residual & \text{Sample} \\
\hline
1 & & \bf-2 \\
2 & & \bf-3 \\
3 & & \bf-2 \\
\hline
4 & 1 & \left \lfloor \frac{2 ^ 5 + (21 \times -2) + (2 \times -3) + (7 \times -2)}{2 ^ 5} \right \rfloor + 1 = \left \lfloor \frac{32 - 62}{32} \right \rfloor + 1 = -1 + 1 = \bf0 \\
5 & -2 & \left \lfloor \frac{2 ^ 5 + (21 \times \textbf{0}) + (2 \times -2) + (7 \times -3)}{2 ^ 5} \right \rfloor - 2 = \left \lfloor \frac{32 - 25}{32} \right \rfloor - 2 = 0 - 2 = \bf-2 \\
6 & -1 & \left \lfloor \frac{2 ^ 5 + (21 \times \textbf{-2}) + (2 \times \textbf{0}) + (7 \times -2)}{2 ^ 5} \right \rfloor - 1 = \left \lfloor \frac{32 - 56}{32} \right \rfloor - 1 = -1 - 1 = \bf-2 \\
\hline
\end{tabular}
\end{figure}
\par
Unfortunately, there's one more wrinkle to consider for proper
\texttt{QLPC} command decoding: the \VAR{coffset}.
How to calculate this value will be covered in the next section.
But when a \texttt{QLPC} command is encountered, the coffset value
is subtracted from the \texttt{QLPC}'s warm-up samples
(taken from the top of the previous command, for the current channel).
Then that coffset value is re-added to our output samples after
calculation.

For example, given a \VAR{coffset} value of 5, one would subtract 5 from
$\text{Sample}_{-3}$, $\text{Sample}_{-2}$ and $\text{Sample}_{-1}$,
perform the QLPC calculation and then add 5 to our
$\text{Sample}_{0}$, $\text{Sample}_{1}$, $\text{Sample}_{2}$, ... ,
$\text{Sample}_{255}$
before returning those values.

\pagebreak

\label{shorten_coffset}
\subsection{the coffset}

Calcuating the \VAR{coffset} value for a given command on a given channel
requires a set of \VAR{offset} values
(whose count equals the \VAR{Number of Means}, from the Shorten header)
and the \VAR{Number of Means} value itself.
\begin{equation}
\text{coffset} = \frac{\frac{\text{nmeans}}{2} +
\overset{\text{nmeans} - 1}{\underset{i = 0}{\sum}} \text{offset}_i }{\text{nmeans}}
\end{equation}
For example, given a \VAR{Number of Means} value of 4 and offsets of:
\begin{table}[h]
\begin{tabular}{>{$}r<{$} r}
\text{offset}_0 & 32 \\
\text{offset}_1 & 28 \\
\text{offset}_2 & 17 \\
\text{offset}_3 & 14 \\
\end{tabular}
\end{table}
\par
\noindent
\begin{equation}
\text{coffset} = \frac{\frac{4}{2} + (32 + 28 + 17 + 14)}{4} = \frac{93}{4} = \bf23
\end{equation}
\par
The next obvious question is where to those \VAR{offset} values come from?
They're actually a queue of (mostly) sample value averages on the
given channel.
So once we've decoded offset for $\text{command}_5$ on channel 0,
$\text{offset}_0$ takes the value of $\text{offset}_1$,
$\text{offset}_1$ takes the value of $\text{offset}_2$,
$\text{offset}_2$ takes the value of $\text{offset}_3$,
and $\text{command}_5$'s offset becomes the new $\text{offset}_3$.

However, the offset is not entirely a sample average.
Its actual formula is as follows:
\begin{equation}
\text{offset} = \frac{\frac{\text{block size}}{2} +
\overset{\text{block size} - 1}{\underset{i = 0}{\sum}} \text{sample}_i }{\text{block size}}
\end{equation}
\par
\noindent
For example, if a command with a \VAR{block size} of 256 has samples
that total 1056, its offset value is:
\begin{equation}
\text{offset} = \frac{\frac{\text{256}}{2} + 1056}{\text{256}} = \frac{1184}{256} = \bf4
\end{equation}

\subsection{the VERBATIM command}

This command is for generating raw, non-audio data such
as .wav or .aiff chunks and is structured as follows:
\begin{figure}[h]
\includegraphics{figures/shorten_verbatim.pdf}
\end{figure}
\par
\noindent
These chunks of raw data are expected to be written in the order
they appear in the Shorten file.

\pagebreak

\section{Shorten encoding}

For the purposes of Shorten encoding,
one needs an entire PCM container file and its
PCM values, number of channels and bits per sample.
Recall that nearly the entire Shorten file format is made up
of three variable-length data types with no byte-alignment of any sort.
Because of that, there's no way to ``rewind'' the Shorten stream
and replace values.
Therefore, everything encoded to Shorten must be written in order.

\begin{figure}[h]
\includegraphics{figures/shorten_stream.pdf}
\end{figure}
\par
\noindent
Remember that \texttt{long} variables contain a size \texttt{unsigned(2)}
field followed by its value.
So what's the best size to use for a given value?
Quite frankly, in my opinion, any small size will do.
Most of a Shorten file is encoded residuals, so wasting a handful
of bytes in the header won't make any difference.

A good set of header values to use are as follows:
\begin{table}[h]
\begin{tabular}{|r|l|}
\hline
field & value \\
\hline
& \textbf{2} for unsigned 8 bit input \\
Field Type & \textbf{5} for signed, little-endian, 16 bit input \\
& \textbf{3} for signed, big-endian, 16 bit input \\
Channels & from input \\
Block Length & \textbf{256} \\
Max LPC & \textbf{0} \\
Number of Means & \textbf{0} \\
Bytes to Skip & \textbf{0} \\
Samples to Wrap & \textbf{3} \\
\hline
\end{tabular}
\end{table}
\par
\noindent
The remainder of the stream is various Shorten commands.
I'll be limiting this encoding documentation to the \texttt{DIFF1},
\texttt{DIFF2}, \texttt{DIFF3}, \texttt{QUIT}, \texttt{BLOCKSIZE},
\texttt{ZERO} and \texttt{VERBATIM} commands.
\texttt{QLPC}'s implementation is actually broken in the reference
decoder\footnote{By limiting its accumulator to a 32-bit integer,
it's prone to overflow at high LPC counts.} and tends not to
produce smaller files.
And, by omitting \texttt{DIFF0} which is rarely used,
we can avoid \VAR{coffset} calculation during encoding.

\pagebreak

\subsection{the VERBATIM commands}

PCM containers such as Wave and AIFF consist of
several blocks of data, one of which contains a large chunk of
PCM data.
Shorten encodes these files by turning non-audio data at
the beginning and end of the container into \texttt{VERBATIM} commands,
often in the following format:
\begin{figure}[h]
\includegraphics{figures/shorten_sandwich.pdf}
\end{figure}
\par
\noindent
These \texttt{VERBATIM} commands \textit{must} appear in
the Shorten stream in the same order as they appear in the
PCM container itself.

\subsection{the BLOCKSIZE command}

Most \texttt{DIFF} and \texttt{ZERO} commands should contain
\VAR{Block Length} number of samples, as indicated in the header.
But when the end of the input stream is reached and a different
number of samples are all that remain, a \texttt{BLOCKSIZE}
is required.
This is simply an \texttt{unsigned(2)} variable with a value of 5
(indicating the \texttt{BLOCKSIZE} command)
followed by a \texttt{long} variable containing the new \VAR{Block Length}.

\subsection{the QUIT command}

When no more samples remain and all \texttt{VERBATIM} commands have
been delivered (a Wave file may contain additional chunks after
the `data' chunk, for example), the \texttt{QUIT} command indicates
the end of the Shorten stream.
This is an \texttt{unsigned(2)} variable with the value 4.

However, the stream \textit{must} be padded such that the total
stream length, minus 5 bytes for the magic number and version,
is a multiple of 4 bytes.
If this padding is not performed, the reference decoder's bit stream
implementation will exit with an error.

\subsection{the ZERO command}

When a block's entire set of samples are 0, we can generate
a \texttt{ZERO} command to represent them, which is
simply the \texttt{unsigned(2)} command variable of 8.

\pagebreak

\subsection{the DIFF commands}

The bulk of a Shorten file is \texttt{DIFF} commands,
which are \texttt{unsigned(2)} command variables 1, 2, and 3
for \texttt{DIFF1}, \texttt{DIFF2} and \texttt{DIFF3}, respectively,
followed by the same argument syntax:
\begin{figure}[h]
\includegraphics{figures/shorten_diff.pdf}
\end{figure}
\par
Given a set of input samples, we decide which \texttt{DIFF} command
to use by calculating their minimum delta sum, which is best
explained by example:
\par
\begin{table}[h]
{\relsize{-2}
\begin{tabular}{|c|r|r|r|r|}
\hline
index & sample & $\Delta^1$ & $\Delta^2$ & $\Delta^3$ \\
\hline
-1 & 0 & & & \\
0 & 18 & -18 & & \\
1 & 20 & -2 & -16 & \\
2 & 26 & -6 & 4 & -20 \\
3 & 24 & 2 & -8 & 12 \\
4 & 24 & 0 & 2 & -10 \\
5 & 23 & 1 & -1 & 3 \\
6 & 21 & 2 & -1 & 0 \\
7 & 24 & -3 & 5 & -6 \\
8 & 23 & 1 & -4 & 9 \\
9 & 20 & 3 & -2 & -2 \\
10 & 18 & 2 & 1 & -3 \\
11 & 18 & 0 & 2 & -1 \\
12 & 17 & 1 & -1 & 3 \\
13 & 17 & 0 & 1 & -2 \\
14 & 20 & -3 & 3 & -2 \\
15 & 23 & -3 & 0 & 3 \\
16 & 21 & 2 & -5 & 5 \\
17 & 23 & -2 & 4 & -9 \\
18 & 22 & 1 & -3 & 7 \\
19 & 18 & 4 & -3 & 0 \\
\hline
\multicolumn{2}{|r|}{$| sum |$} & 56 & 66 & 97 \\
\hline
\end{tabular}
}
\end{table}
\par
\noindent
In this example, the $|sum|$ of $\Delta^1$ is the smallest value.
Therefore, the best command to use for this set of samples is \texttt{DIFF1}.
Once we know which \texttt{DIFF} command to use for a given set of
input samples, calculating the command's set of residual values
can be done automatically:
\par
\noindent
\begin{table}[h]
\begin{tabular}{|c|>{$}l<{$}|}
\hline
command & \text{calculation} \\
\hline
\texttt{DIFF1} & \text{Residual}_i = \text{Sample}_i - \text{Sample}_{i - 1} \\
\texttt{DIFF2} & \text{Residual}_i = \text{Sample}_i - ((2 * \text{Sample}_{i - 1}) - \text{Sample}_{i - 2}) \\
\texttt{DIFF3} & \text{Residual}_i = \text{Sample}_i - ((3 \times (Sample_{i - 1} - Sample_{i - 2})) + Sample_{i - 3}) \\
\hline
\end{tabular}
\end{table}
\par
\noindent
In this example, our residuals for \texttt{DIFF1} are:
18, 2, 6, -2, 0, -1, -2, 3, -1, -3, -2, 0, -1, 0, 3, 3, -2, 2, -1, -4.

\pagebreak

Finally, given a set of residual values, we need to determine
their \VAR{Energy Size}.
This is done by selecting the smallest value of `x' such that:
\begin{equation}
\text{sample count} \times 2 ^ x > \overset{\text{residual count} - 1}{\underset{i = 0}{\sum}} |\text{residual}_i|
\end{equation}
\par
\noindent
To finish our example, given we have 20 residuals:
\begin{table}[h]
\begin{tabular}{|c|>{$}r<{$}|>{$}r<{$}|}
\hline
index & \text{residual}_i & | \text{residual}_i | \\
\hline
0 & 18 & 18 \\
1 & 2 & 2 \\
2 & 6 & 6 \\
3 & -2 & 2 \\
4 & 0 & 0 \\
5 & -1 & 1 \\
6 & -2 & 2 \\
7 & 3 & 3 \\
8 & -1 & 1 \\
9 & -3 & 3 \\
10 & -2 & 2 \\
11 & 0 & 0 \\
12 & -1 & 1 \\
13 & 0 & 0 \\
14 & 3 & 3 \\
15 & 3 & 3 \\
16 & -2 & 2 \\
17 & 2 & 2 \\
18 & -1 & 1 \\
19 & -4 & 4 \\
\hline
\multicolumn{2}{|r|}{$|sum|$} & 56 \\
\hline
\end{tabular}
\end{table}
\par
\noindent
\begin{align*}
20 \times 2^0 & \leq 56 \\
20 \times 2^1 & \leq 56 \\
20 \times 2^2 & > 56
\end{align*}
Therefore, the best \VAR{Energy Size} for this set of residuals is 2.

