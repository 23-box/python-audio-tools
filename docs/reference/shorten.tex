\chapter{Shorten}
Shorten is one of the earliest lossless audio compression formats.
Though superceded by FLAC and other formats, it remains interesting
from a historical perspective.

\section{Shorten data types}
Notably, almost nothing in the Shorten file format is byte-aligned.
Instead, it uses its own set of variable-length types which I'll
refer to as \texttt{unsigned}, \texttt{signed} and \texttt{long}.

\begin{wrapfigure}[16]{r}{2in}
\includegraphics{figures/shorten_unsigned.pdf}
\caption{Unsigned}
\includegraphics{figures/shorten_signed.pdf}
\caption{Signed}
\end{wrapfigure}
An \texttt{unsigned} field of a certain ``size'' means we first
take a unary-encoded\footnote{In this instance, unary-encoding is a simple
matter of counting the number of 0 bits before the next 1 bit.
The resulting sum is the value.}, number of high bits and combine
the resulting value with ``size'' number of low bits.
For example, given a ``size'' of 2 and the bits `\texttt{0 0 1 1 1}',
the high unary value of `\texttt{0 0 1}' combines with the low
raw value of `\texttt{1 1}' resulting in a decimal value of 11.

A \texttt{signed} field is similar, but its low value contains
one additional trailing bit for the sign value.
{\relsize{-2}
\begin{equation*}
\text{signed value} =
\begin{cases}
\text{unsigned value} & \text{if sign bit} = 0 \\
-\text{unsigned value} - 1 & \text{if sign bit} = 1
\end{cases}
\end{equation*}
}
For example, given a ``size'' of 3 and the bits `\texttt{0 1 1 0 1 1}',
the high unary value of `\texttt{0 1}' combines with the low
raw value of `\texttt{1 0 1}' and the sign bit `\texttt{1}'
resulting in a decimal value of -14.
Note that the sign bit is counted seperately, so we're
actually reading 4 additional bits after the unary value in this case.

Lastly, and most confusingly, a \texttt{long} field is the combination
of two seperate \texttt{unsigned} fields.
The first, of size 2, determines the size value of the second.
For example, given the bits `\texttt{1 1 1 1 1 0 1}',
the first \texttt{unsigned} field of `\texttt{1 1 1}' has the value
of 3 (unary 0 combined with a raw value of 3) - which is the size
of the next \texttt{unsigned} field.
That field, in turn, consists of the bits `\texttt{1 1 0 1}'
which is 5 (unary 0 combined with a raw value of 5).
So, the value of the entire \texttt{long} field is 5.

A Shorten file consists almost entirely of these three types
in various sizes.
Therefore, when one reads ``\texttt{unsigned(3)}'' in a Shorten field
description, it means an \texttt{unsigned} field of size 3.

\pagebreak

\section{the Shorten file stream}
\begin{figure}[h]
\includegraphics{figures/shorten_stream.pdf}
\end{figure}
\par
\noindent

